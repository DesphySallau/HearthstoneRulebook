\chapter{基本概念}
\label{basic-concept}

\section{实体}
\label{entity}

\term{实体}包括:
\begin{itemize}
    \item 游戏本身。一些和游戏进程相关的扳机是在它上面触发的。
    \item 玩家。即使你替换了英雄,有的游戏数据仍然是保留的(例如你本局游戏中使用英雄技能的次数)。这些数据都保留在对应的玩家实体上。
    \item 英雄。
    \item 随从。
    \item 法术。
    \item 武器。
    \item 英雄技能。
    \item 状态。
\end{itemize}

每个实体都有一个\term{实体ID},它是按照实体的创建顺序依次分配的数字。因此,各个实体的ID各不相同。按照顺序,游戏中首先创建的几个实体依次为:
\begin{itemize}
    \item \texttt{1} 游戏本身
    \item \texttt{2} 主玩家
    \item \texttt{3} 副玩家
    \item \texttt{4-33} 主玩家的套牌
    \item \texttt{34-63} 副玩家的套牌
    \item \texttt{64} 主玩家的英雄
    \item \texttt{65} 主玩家的英雄技能
    \item \texttt{66} 副玩家的英雄
    \item \texttt{67} 副玩家的英雄技能
    \item \texttt{68} 幸运币
\end{itemize}
在不同的游戏模式中,套牌占据的编号数量可能不为30张。此外,幸运币可能随着玩家的偏好设置而有所不同。

\notice 所谓的「随从与随从牌、法术与法术牌、武器与武器牌、英雄与英雄牌」仅仅是叫法与视觉效果上的不同,牌在进行区域移动时仍然是同一实体,并不会因为外形的改变而变成不同的实体。

在战场上的英雄和随从合称为\term{角色}。

\section{状态}
\label{enchantment}

当你将鼠标放在一个随从或武器上时,详情大图下方显示的增益列表中的各项就是\term{状态},这些状态\term{结附}于该实体。此外,状态也可以结附于英雄、玩家、手牌或牌库中的牌,但是这些状态通常是不可见的。

状态可以:
\begin{itemize}
    \item 表示实体所受到的buff/debuff。
    \item 包含延时触发的扳机,例如\card{力量的代价}所添加的。
    \item 包含光环。例如\card{冰霜女巫吉安娜}的战吼为你的玩家结附一个状态。
    \item 实现光环。大多数光环的实际效果为,在适当的时间(称为\term{光环更新})将一个特定的状态结附于受该光环影响的实体上。例如上面的例子中,光环更新时会为你的所有元素添加「该随从具有吸血」的状态。
    \item 实现冒险模式、乱斗模式、酒馆战棋等模式中的特殊游戏规则。
\end{itemize}

\section{光环}
\label{aura}

具有\term{光环}的实体可以在满足一定的条件时对自身、其它实体或游戏规则施加持续性的影响。光环包括以下几种类型:
\begin{itemize}
    \item 改变实体生命值/攻击力的,例如\card{暴风城勇士}、\card{鱼人领军}等。
    \item 改变实体法力值消耗的,例如\card{召唤传送门}、\card{机械跃迁者}等。
    \item 使实体获得某种状态的,例如\card{冰霜女巫吉安娜}的元素吸血光环、\card{凯恩·日怒}等。
    \item 改变效果数值的,例如\nameref{spell-damage}、\card{先知维伦}、\card{水晶工匠坎格尔}等。
    \item 改变效果结算次数的,例如\card{布莱恩·铜须}、\card{瑞文戴尔男爵}、\card{达卡莱附魔师}等。
    \item 改变效果结算规则的,例如\card{奥金尼灵魂祭司}、\card{伊格诺斯}等。
\end{itemize}

改变实体生命值/攻击力/法力值消耗的光环通常通过使实体获得一个状态实现。
\example \card{暴风城勇士}为其它随从结附状态暴风城之力,该状态使随从+1/+1。
\notice 由光环附加的改变实体生命值/攻击力的状态具有低优先级,总是晚于其他状态。因此计算一个实体的身材时,首先计算它具有的状态,然后计算它享有的光环。
\example 你操控\card{暴风城勇士}和\card{小精灵}。小精灵当前身材是2/2。你对小精灵使用\card{黑暗裁决}。小精灵首先因黑暗裁决变为3/3,再因暴风城勇士光环变为4/4。
\exception 改变随从原始身材(白字身材)的光环所附加的状态不具有特殊优先级,如\card{水晶核心}和\card{黑暗法老塔卡恒}。
\notice 与之相对的是,光环产生的改变实体法力值消耗的状态通常没有特殊优先级。因此计算一个实体的法力值消耗时,光环和状态通常按添加顺序结算。详见\nameref{cost}。
\example 你依次使用\card{艾维娜}和\card{索瑞森大帝}。回合结束时你手中的\card{摩天龙}被大帝减为0费。如果你持有一个被大帝减为9费的摩天龙并使用艾维娜,则摩天龙变为1费。

战吼、亡语和回合结束扳机在触发前决定要触发的次数,这意味着通过战吼召唤\card{布莱恩·铜须}不能使战吼再次结算。\card{瑞文戴尔男爵}和\card{达卡莱附魔师}也类似。
\notice 如果你通过亡语召唤瑞文戴尔男爵或通过回合结束扳机召唤达卡莱附魔师,你接下来的亡语/回合结束扳机会阶段两次。
\example 你操控一个瑞文戴尔男爵并对其使用\card{先祖之魂}和\card{终极巫毒}。对手杀死瑞文戴尔男爵。先祖之魂只会召唤一个瑞文戴尔男爵,但接下来的终极巫毒可以触发两次。
\example 你操控\card{亚煞极}和\card{炎魔之王拉格纳罗斯}。回合结束时亚煞极首先召唤达卡莱附魔师,亚煞极不会再召唤一个随从,但接下来你的炎魔之王可以开出两炮。

光环并非入场即生效、离场即失效。事实上只有每次光环更新时,游戏才会检查是否有新光环生效、是否有旧光环失效。详见\nameref{aura-update}。

\card{天启四骑士}的消灭效果虽然是一次性效果,但它是在每次光环更新时进行检测。这称为\term{光环更新扳机}。

\section{区域}
\label{zone}

游戏中每个实体都位于某个\term{区域}。区域包括:

\begin{itemize}[itemsep=\parsep]
    \item 战场\texttt{PLAY}:游戏、玩家、当前的英雄和英雄技能都在战场上。被使用或召唤的随从或武器、被使用或施放且未结算完毕的法术(奥秘和任务除外)都在战场上。所有的状态都在战场上,无论它结附于哪里的实体。
        \notice 由于状态都在场上,因此状态附带的扳机是场上扳机。
    \item 奥秘区\texttt{SECRET}:使用或施放后的奥秘和任务位于此区域。
        \notice 奥秘区的奥秘和任务均被视为场上扳机。
    \item 手牌\texttt{HAND}:玩家可以从手牌中使用牌。
    \item 牌库\texttt{DECK}:在对战开始时,玩家套牌中的牌都位于牌库中。
    \item 墓地\texttt{GRAVEYARD}:死去的实体、弃掉或烧毁的牌、被反制或结算完的法术、已触发的奥秘或已完成的任务前往此区域。
    \item 除外区\texttt{SETASIDE}:存放一些临时实体和双方被移除的部分实体,如\card{追踪术}和\nameref{discover}呈现的卡牌、替换英雄后的旧\card{加拉克苏斯大王}随从、亡语生效后的\card{玛洛恩}等。
        \notice 大多数亡语的区域移动效果实质上是伪区域移动。伪区域移动会将实体本身移动到除外区。详见\nameref{move}。
        \notice 部分移除牌库卡牌的效果将卡牌移动到除外区,例如\card{“丛林猎人”赫米特}、\card{吞噬者阿扎里}等。将它们与烧毁卡牌的效果(如\card{侏儒吸血鬼}、\card{哀泣女妖}等)区分的一个方式是观察有没有卡牌被撕毁的动画。
        \notice 同理,一些效果移除的手牌直接前往除外区,例如\card{黄金猿}、\card{菲里克·飞刺}等。它们与弃牌效果也可以通过观察动画区分。此外,\card{神奇的雷诺}将所有场上的随从移动到除外区,它也不播放随从的死亡动画。
    \item 失效区\texttt{REMOVEDFROMGAME}:存放已失效的状态。
        \notice 状态失效一般的原因可能是:过时(如\card{嗜血}等)、所属的实体被移除、状态来源的实体被移除、被沉默等。
    \item 佣兵技能区\texttt{LETTUCE\_ABILITY}:存放佣兵技能。
\end{itemize}

实体还具有\texttt{.ZONE\_POSITION}属性。这个属性称作\term{位置},它表示实体在其区域中的位置。
\begin{itemize}
    \item 手牌中的所有牌都具有非零位置。
    \item 战场上的随从牌具有非零位置,但其它实体的位置为0。
    \item 墓地、除外区和失效区的任何实体的位置都为0。
    \item 尽管牌库是有序的,牌库中的牌的位置都为0。这显然是为了避免玩家从客户端获得牌库中牌的顺序。
    \item 尽管奥秘看起来有顺序,奥秘区中的牌位置为0。
\end{itemize}

\section{事件与扳机}
\label{event-and-trigger}

\term{事件}对应游戏中发生的事情。
\example 「造成6点伤害」、「抽一张牌」都对应事件。前者对应伤害事件,后者对应抽牌事件。
\notice 游戏中发生的事情和事件并不一定紧挨着出现的。例如在一个AOE伤害发生,所有伤害都产生之后,才会结算对于各个实体的伤害事件。

\term{结算}事件指处理该事件所对应的所有扳机。
\notice 在下文中,我们偶尔会使用事件\term{产生}这样的描述。这实际上指的是该事件对应的游戏操作结束。

\term{扳机}是响应特定的事件而触发并结算的效果。它通常描述为「在/每当A时,E。」
\example \card{飞刀杂耍者}具有「在你召唤一个随从后,对一个随机敌方角色造成1点伤害。」这是一个召唤后扳机。
\example \card{北郡牧师}具有「每当一个随从获得治疗时,抽一张牌。」这是一个治疗扳机。
\example \nameref{lifesteal}为「每当此实体造成伤害时,为你的英雄恢复等量生命值。」这是一个伤害扳机。

当游戏结算一个事件时,首先确定所有响应该事件触发的扳机(称为\term{列队}),然后这些扳机按照顺序结算。不满足条件的扳机,即使后来条件改变也不能再触发。
\notice 即使列队时合法的扳机,也有可能无法结算。详见\nameref{trigger-cond}。

如果在事件A的结算中又产生了另一个事件B,则首先结算事件B,等待B的结算完成后才会继续A的结算。这种\emph{深度优先}的结算规则称为\term{嵌套结算}。
\example 对手操控\card{狙击}和\card{镜像实体}。你使用\card{古拉巴什狂暴者}。首先狙击触发,对古拉巴什狂暴者造成4点伤害,此伤害立即触发它的扳机,把它变成一个5/4。然后镜像实体触发,对手也获得一个5/4的古拉巴什狂暴者。
\example 你操控\card{镜像实体}、\card{飞刀杂耍者}、\card{卡德加}、\card{公正之剑}和\card{勇敢的记者}。对手操控\card{暴乱狂战士}、\card{苦痛侍僧}、\card{铸甲师}和\card{重型攻城战车}并使用\card{小精灵}。可能发生以下情况:

\begin{itemize}
    \item 作为响应使用后事件的扳机,镜像实体触发并召唤一个小精灵的复制(记为小精灵1)。
    \begin{itemize}
        \item 小精灵1的召唤事件触发飞刀,对对手的苦痛侍僧造成1点伤害。
        \begin{itemize}
            \item 首先暴乱狂战士响应苦痛侍僧的受伤并获得+1攻击力。
            \item 其次苦痛侍僧自身扳机触发抽一张牌。
            \begin{itemize}
                \item 你的勇敢的记者响应此次抽牌,获得+1/+1。
            \end{itemize}
            \item 最后铸甲师响应苦痛侍僧的受伤使对手获得1点护甲。
            \begin{itemize}
                \item 重型攻城战车响应护甲获得事件并+1攻击力。
            \end{itemize}
        \end{itemize}
        \item 小精灵1的召唤事件触发卡德加,召唤另一只小精灵2。
        \begin{itemize}
            \item 小精灵2的召唤触发飞刀,对对手英雄造成1点伤害。
            \item 你的公正之剑响应小精灵2的召唤使其+1/+1。
        \end{itemize}
        \item 小精灵1的召唤事件触发公正之剑,并获得+1/+1。
    \end{itemize}
\end{itemize}

结算某事件触发的所有扳机时,各扳机的先后顺序主要按照入场顺序。
\notice 入场顺序指实体进入战场的顺序。判断两个扳机间的入场顺序,就是判断两个扳机进入战场时间点的先后。对于随从自带的扳机而言,其入场时间点为随从的入场时间点;对于由状态添加的扳机而言,其入场时间点为状态附加的时间点。随从的操控权发生改变不影响其入场时间点。
\example 你使用\card{小鬼召唤师}A和B,然后对A使用\card{力量的代价}。回合结束时,首先A触发效果,然后B触发效果,最后A所带的力代触发效果消灭A。
\example 对手使用\card{法力之潮图腾},然后你使用\card{霍格},再使用\card{精神控制}目标法力之潮图腾。回合结束时,首先法力之潮图腾触发,再霍格触发。
\notice 除了入场顺序之外,还有很多顺序也决定着扳机的结算顺序,例如\nameref{trigger-order-in-different-zone},以及某些效果具有的\nameref{special-priority}等。
\notice 当你在一回合打出多张\card{秘密通道}时,回合结束的归还手牌效果将按照使用顺序的相反顺序进行。实际上,为了保证秘密通道能正确地归还手牌,多次使用的秘密通道将在一个扳机中结算全部的归还效果。这个扳机与其它回合结束扳机仍然按照入场顺序结算。\card{裂心者伊露希亚}与之类似。
\example 你在一回合中依次使用秘密通道A、\card{唤醒}、秘密通道B、秘密通道C。回合结束时依次结算秘密通道C、B、A的归还效果,再结算唤醒的弃牌。
\example 你在一回合中依次使用秘密通道A、裂心者A、秘密通道B、裂心者B、秘密通道C、裂心者C。回合结束依次结算秘密通道C、B、A,再结算裂心者C、B、A的效果。

群体伤害和群体治疗的伤害/治疗事件通常在全部伤害/治疗产生之后再结算。
\example 你操控七个\card{暴乱狂战士},对手使用\card{烈焰风暴}。首先所有的暴乱狂受到伤害减为0血,然后它们依次响应每个伤害事件触发(共49次)。
\notice \card{陨石术}对三个随从造成的伤害属于群体伤害。
\notice \card{燃烧}对两侧随从造成的伤害也属于群体伤害。
\notice 伤害/治疗值不定的 AOE 通常不适用此条规则,它们以「第一个随从受到伤害 - 对应的伤害事件 - 第二个随从受到伤害 - 对应的伤害事件……」依次交替。这样的例子有\card{闪电风暴}、\card{元素毁灭}、\card{圣光炸弹}、\card{生命之树}和\card{黑铁潜藏者}等等。
\exception 即使闪电风暴已被改动,它造成的伤害仍然不属于群体伤害。
\example 你操控\card{血沼迅猛龙}和\card{暴乱狂战士}。对手使用圣光炸弹。首先迅猛龙受到3点伤害,然后暴乱狂触发变成3攻,然后它受到3点伤害。
\notice 对双方英雄造成的伤害通常不适用此条规则。例如\card{翡翠掠夺者}首先对先入场的英雄造成伤害,再对后入场的英雄造成伤害,中间可以插入\card{以眼还眼}。
\notice 伤害和治疗以外的「复合效果」通常并不适用此规则。例如\card{冰霜新星}和\card{莫拉比}是交替冰冻-触发;一次抽多张牌中间可以插入\card{勇敢的记者}或\card{古董收藏家}等。
\notice \nameref{lifesteal}和\nameref{overkill}并不触发多次;它们只在最后触发一次。

此外,战斗中的两个伤害事件也是在伤害全部产生之后再结算。详见\nameref{battle}。

卡牌具有的「使用时」「使用后」「召唤时」「召唤后」扳机(如伊利丹·怒风或\card{飞刀杂耍者})不能在该牌的对应事件中触发。
\notice 这可以视为这些扳机的隐藏条件。但由于这种情况过于普遍,我们将它提出为一条规则。
\example 你使用具有\card{瓦兰奈尔}亡语增益的\card{灵魂歌者安布拉}。安布拉并不会触发自己的亡语为你装备一把瓦兰奈尔。
\exception 你使用具有突袭的\card{团伙核心},团伙核心会触发自己的效果为你召唤一个剩余生命值为1的团伙核心。
\exception 法力值消耗增加了的\card{寻求指引}可以触发自己的扳机。

扳机不能在自身的结算之中再次触发。部分扳机在这次结算完成之后,将补偿进行等同于跳过次数的结算。详见\nameref{self-triggering-protection}。

\section{序列}
\label{sequence}

炉石传说的回合由若干个\term{序列}构成。序列由使用牌、使用英雄技能及战斗产生。它也会由特定的游戏时刻(回合开始、回合抽牌、回合结束)产生。序列的末尾会进行\term{胜负裁定}。

一些玩家操作也有与之对应的非玩家操作,例如使用随从对应召唤随从,使用法术对应施放法术,战斗对应强迫战斗(由卡牌效果产生的战斗)等。这些操作也会产生类似序列的东西,但它们内部不包含死亡结算与胜负裁定。这种序列类似物称作\term{退化序列}。退化序列将在对应玩家操作的章节中进行描述。
\notice 在规则集的旧版本中,无论是玩家操作还是非玩家操作都产生序列,而玩家操作产生的被强调为\emph{最外层序列}。现版本采取了不同的定义。

序列的内部不包含胜负裁定。
\notice 有的时候,你可能会以为是在序列没有进行完的时候游戏就结束了;实际上是剩余的所有游戏动作被略过。
\example 你的生命值为2。你攻击敌方英雄,触发敌方\card{爆炸陷阱}。因为你濒死,伤害步骤和攻击后步骤被略过,你输掉游戏。
\example 你的生命值为3。你的牌库中只有\card{炸弹}和\card{阿兰纳丝蛛后}。回合开始你抽到炸弹,受到5点伤害。因为你濒死,炸弹不抽牌。然后你输掉游戏,没有机会抽到蛛后把你救回来。请与下面的例子作比较。
\example 你的生命值为3。你的牌库中只有\card{炸弹}和\card{阿兰纳丝蛛后}。你操控两个\card{战利品贮藏者},然后使用\card{扭曲虚空}。第一个战利品的亡语结算,你抽到炸弹,受到5点伤害。因为你濒死,炸弹不抽牌。然后第二个战利品的亡语结算,你抽到蛛后,救回了你自己。

\section{胜负裁定}
\label{winner-check}

胜负裁定是一个特殊的步骤(它只在序列结尾出现)。它负责结束游戏。

\begin{itemize}
    \item 如果一个玩家的\texttt{.PLAYSTATE = LOSING},则游戏以该玩家输,其对手赢结束;
    \item 如果两个玩家的\texttt{.PLAYSTATE = LOSING},则游戏以平局结束,没有赢家。
        \notice 虽然平局时双方玩家的英雄都爆炸,但是这不代表其中的某一方输掉了游戏——你的天梯星数不会减少(但是会中止连胜);你的竞技场负场也不会增加。
    \item 如果没有玩家的\texttt{.PLAYSTATE = LOSING},游戏继续。
\end{itemize}

在客户端UI中,动画上显示一方的英雄爆炸(或因平局双方英雄均爆炸)代表胜负裁定判定游戏结束;英雄发出死亡音效代表死亡检索判定一个英雄死亡。

在游戏的结果产生之后,游戏不会立即结束。游戏会将整个序列继续执行完毕。如果其中需要玩家进行操作,这些操作将被随机选择。
\example 你的对手的生命值为1,你使用\card{符文之矛}攻击敌方英雄。在发现界面时,如果此时你选择投降,那么本局游戏将以平局结束;对手投降则为你获胜。
\example 你的生命值为5,使用了\card{追踪术}看到了两张\card{炸弹}和一张其它牌。此时对手投降,你有可能随机选择到炸弹而导致游戏平局。

\section{阶段}
\label{pahse}

以每一次死亡结算为分界线,序列又可以分成若干\term{阶段}。在\nameref{player-action}章节中详细阐述了每个序列中包含的阶段。阶段的末尾会进行\term{阶段间步骤}。
\notice 按照这个划分标准,一般情况下阶段的内部不会发生死亡结算。
\example 你在满场的情况下使用\card{火焰之地传送门}杀死一个友方随从。事实上,火焰之地传送门「造成伤害」和「召唤随从」的效果同属于法术的结算阶段,被杀死的随从不立即离场,而是等到法术结算阶段结束才离场。因此格子会卡住,火门不能召唤出一个5费随从。

一个序列会包含一系列固定的阶段,称为\term{固有阶段}。如使用法术牌的序列包括:
\begin{itemize}
    \item 使用阶段,使得你的\card{紫罗兰教师}能在法术生效前召唤出1/1的学徒。
    \item 结算阶段,使得法术卡牌描述中的效果能够结算。
    \item 完成阶段,使得你的\card{狂野炎术师}能在法术生效后对所有随从造成过1点伤害。
\end{itemize}

此外,一个序列还会包含若干个死亡阶段,取决于是否有实体在序列中死亡。详见\nameref{death-creation-step}和\nameref{death-phase}。

\section{阶段间步骤}

\term{阶段间步骤}是在阶段结束时依次执行的步骤的统称,按先后顺序依次包括:
\begin{itemize}
    \item \term{任务奖励步骤}:如果在前一个阶段你完成了任务,则在此步骤中获得任务奖励。
    \item \term{死亡检索步骤}:移除你当前濒死的随从。详见\nameref{dying}、\nameref{death-creation-step}和\nameref{death-phase}。
    \item \term{光环更新步骤}:更新你所拥有的光环。详见\nameref{aura-update}。
\end{itemize}

\section{濒死}
\label{dying}

\term{濒死}指一个场上的实体被消灭或其当前生命值/耐久度被降至0或更低。\\
被消灭指一个实体受到消灭效果的作用,其\texttt{.TO\_BE\_DESTROYED := 1}。我们也会用\term{濒死状态}和\term{待摧毁状态}来描述这种情况。

\notice 一个实体濒死并不意味着它会被立即送去墓地。它要等到随后的\term{死亡检索步骤}才会被送入墓地。这意味着,一个濒死的实体可能会在后续结算中被救回来。
\example 你操控先入场的\card{炎魔之王拉格纳罗斯},你的对手操控8/8后入场的\card{格鲁尔}。你的回合结束,螺丝对格鲁尔造成8点伤害,然后格鲁尔把自己变成 9/1救了回来。
\example 你操控两个\card{资深档案管理员}。回合结束,管理员A施放\card{扭曲虚空},然后管理员B施放\card{消失}。所有随从被移回手牌,它们不会在场上死亡也不会在手牌中被弃掉。
\notice 沉默不能清除待摧毁状态。因此,把上面例子中的\card{消失}改为\card{群体驱散}不能救回对手的随从。

\section{死亡检索步骤}
\label{death-creation-step}

这是一个阶段间步骤。它按照以下顺序进行:

\begin{itemize}
    \item 游戏检查所有的英雄、随从、武器,并标记濒死的实体。
    \item \term{死亡时扳机}触发,例如\card{火焰狂人}。
    \item 游戏按照入场顺序将所有的濒死实体移入墓地。如果一个英雄被移除,且它是其操控者的英雄,则这个玩家的\texttt{.PLAYSTATE := LOSING}。
    \item 如果游戏检测到并移除了濒死的实体,则进入死亡阶段结算它们的死亡事件;否则进入序列中设定的下一个阶段。
\end{itemize}

\notice 也就是说,实体移除的顺序并不是它们进入濒死的顺序。
\notice 第一步中被标记的实体必定会被移除。也就是说,即使火焰狂人抽一张牌使得某个濒死的随从不再濒死,它也仍然会被移除。
\notice 在极为特殊的情况下,死亡的英雄可能同时为双方的英雄。此时只有它当前的操控者会输掉游戏。

\card{火焰狂人}可以产生一些令人困惑的结算。这是一个例子:、
\example 你操控火焰狂人和\card{小精灵}。你的牌库里只有一张\card{惊奇卡牌}。你使用\card{火焰冲击}消灭小精灵。接下来发生这些事情:

\begin{itemize}
    \item 死亡检索步骤 I.1:标记小精灵为濒死。
    \item 死亡检索步骤 I.2:火焰狂人抽一张牌。抽到惊奇卡牌。
    \begin{itemize}
        \item 惊奇卡牌的抽牌扳机:施放(无关的)法术,进行一次强制死亡。
        \begin{itemize}
            \item 死亡检索步骤 II.1:标记小精灵为濒死。
            \item 死亡检索步骤 II.2:火焰狂人\emph{再}抽一张牌。
        \end{itemize}
        \item 扳机继续结算。你抽一张牌。
    \end{itemize}
\end{itemize}

\noindent 最后一共抽了两张牌。如果你操控更多狂人,会抽更多的牌(而且由于\nameref{self-triggering-protection},其数目不是一个容易计算的数字)。

\section{死亡阶段}
\label{death-phase}

\term{死亡}指一个实体从场上被送去墓地。如果在前一个阶段及其阶段间步骤中有死亡的实体,那么在阶段间步骤结束之后会插入一个死亡阶段。按照前一个\term{阶段}和其后的\term{死亡检索步骤}中被移除的实体的入场顺序,依次结算它们的死亡事件。
\notice 通常情况下实体只在死亡检索步骤中被移除。但也有少数情况下实体在阶段中直接被移除,详见\nameref{move-to-full-zone}。

对于每一个死亡实体的死亡事件,其自身具有的亡语和其他死亡扳机列队结算。
\example 对手操控依次入场的\card{麻风侏儒}、\card{诅咒教派领袖}、\card{发条侏儒}和\card{诈死}。你使用\card{魔爆术}。首先结算麻风侏儒的死亡事件:其亡语对你的英雄造成2点伤害,诅咒教派领袖抽一张牌,然后诈死将一个麻风侏儒加入对手手牌。然后结算发条侏儒的死亡事件:诅咒教派领袖先抽一张牌,然后其亡语将一张零件牌置入对手手牌。
\notice \card{救赎}和\nameref{reborn}在死亡事件中具有特殊优先级。

死亡阶段结束后也会进行阶段间步骤,这可能引发另一个死亡阶段。如此重复直到没有任何实体死亡,才会进入序列设定的下一个阶段。
\example 对手操控两个\card{自爆绵羊}和\card{萨隆铁矿监工},你使用1/1的\card{机械克苏恩}并\card{刺杀}敌方自爆绵羊A。此时你场上没有其他随从且你的手牌和牌库均为空。情况如下:

\begin{itemize}
    \item 法术结算阶段:刺杀将自爆绵羊A标记为待摧毁。
    \item 死亡检索步骤I:移除自爆绵羊A。
    \item 死亡阶段I:自爆绵羊A亡语结算,对自爆绵羊B、监工和机械克苏恩造成2点伤害。
    \item 死亡检索步骤II:移除自爆绵羊B和机械克苏恩。
    \item 死亡阶段II:自爆绵羊B亡语结算,对监工造成2点伤害。然后机械克苏恩消灭敌方英雄。
    \item 死亡检索步骤III:移除敌方英雄和监工。
    \item 死亡阶段III:监工给你召唤\card{自由的矿工}。由于没有实体死亡,进入序列中的下一个阶段,即完成阶段。
    \item 完成阶段:无事发生,序列结束。
    \item 进行胜负裁定,你赢得游戏。
\end{itemize}