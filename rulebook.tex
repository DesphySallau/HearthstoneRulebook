\documentclass[ebook,10pt,oneside,openany,final]{memoir}

\usepackage{ctex}
\usepackage{geometry}
\usepackage{titlesec}
\usepackage{imakeidx}
\usepackage[colorlinks,linkcolor=black]{hyperref}
\usepackage{ifthen}
\usepackage{xifthen}
\usepackage{enumitem}
\usepackage{indentfirst}
\usepackage{pifont}
\usepackage{tabularx}
\usepackage[toc,nopostdot]{glossaries}

\geometry{left=1cm, right=1cm, top=2cm, bottom=2cm}

\titleformat{\chapter}{\raggedright\Huge\bfseries}{\thechapter\quad}{0pt}{}

\newpagestyle{main}{\sethead{\chaptertitle}{}{\thepage}}
\pagestyle{main}

\setlength{\parskip}{1em}

\setenumerate[1]{itemsep=0pt,partopsep=0pt,parsep=0pt,topsep=0pt}
\setitemize[1]{itemsep=0pt,partopsep=0pt,parsep=0pt,topsep=0pt}
\setdescription{itemsep=0pt,partopsep=0pt,parsep=0pt,topsep=0pt}

\makeatletter
\def\imki@progdefault{zhmakeindex}
\makeatother

\makeindex[name=termindex,title=术语表,intoc,options={-s zh.ist}]
% \makeindex[name=termindex,title=术语表,intoc]

\newcommand{\termindex}[1]{\index[termindex]{#1}}

\makeatletter
\newcommand{\term}{\@ifstar{\@termpartial}{\@termfull}}
\newcommand{\@termpartial}[1]{\emph{#1}}
\newcommand{\@termfull}[1]{\emph{#1}\termindex{#1}}

\renewcommand*{\glstextformat}[1]{\textcolor{black}{#1}}
\makeglossaries
\newwrite\missingcard

\newcommand{\card@desc}[2]{%
    \ifthenelse{\isempty{#1}}{#2}{\ifthenelse{\isempty{#2}}{#1}{#1\quad#2}}
}

\newcommand{\card@def}[5][]{%
    \ifthenelse{\isempty{#1}}{%
        \newglossaryentry{#2}{name={#3},description={\card@desc{#4}{#5}}}%
    }{%
        \newglossaryentry{#2}{name={#3},description={\card@desc{#4}{#5}},parent={#1}}%
    }%
    \@namedef{card@nametoid@#3}{#2}%
}

% 怀旧
\card@def{VAN-EX1-323}{加拉克苏斯大王}{术士9/3/15恶魔}{战吼:消灭你的英雄,并用加拉克苏斯大王替换之。}

% 基本
\card@def{CS1-042}{闪金镇步兵}{中立1/1/2随从}{嘲讽}
\card@def{CS2-171}{石牙野猪}{中立1/1/1野兽}{冲锋}
\card@def{CS2-tk1}{绵羊}{中立1/1/1野兽}{}
\card@def{CS2-120}{淡水鳄}{中立2/2/3野兽}{}
\card@def{CS2-172}{血沼迅猛龙}{中立2/3/2野兽}{}
\card@def{CS2-125}{铁鬃灰熊}{中立3/3/3野兽}{嘲讽}
\card@def{CS2-131}{暴风城骑士}{中立4/2/5随从}{冲锋}
\card@def{CS2-182}{冰风雪人}{中立4/4/5随从}{}
\card@def{CS2-197}{食人魔法师}{中立4/4/4随从}{法术伤害+1}
\card@def{EX1-062}{老瞎眼}{中立4/2/4鱼人}{冲锋,在战场上每有一个其他鱼人便获得+1攻击力。}
\card@def{EX1-399}{古拉巴什狂暴者}{中立5/2/8随从}{每当该随从受到伤害,便获得 +3攻击力。}
\card@def{EX1-593}{夜刃刺客}{中立5/4/4随从}{战吼:对敌方英雄造成3点伤害。}
\card@def{CS2-222}{暴风城勇士}{中立7/7/7随从}{你的其他随从获得+1/+1。}
\card@def{EX1-277}{奥术飞弹}{法师1费法术}{造成3点伤害,随机分配到所有敌人身上。}
\card@def{CS2-024}{寒冰箭}{法师2费法术}{对一个角色造成3点伤害,并使其冻结。}
\card@def{CS2-025}{魔爆术}{法师2费法术}{对所有敌方随从造成1点伤害。}
\card@def{HERO-08bp}{火焰冲击}{法师2费英雄技能}{英雄技能 造成1点伤害。}
\card@def{CS2-026}{冰霜新星}{法师3费法术}{冻结所有敌方随从。}
\card@def{CS2-022}{变形术}{法师4费法术}{使一个随从变形成为1/1的绵羊。}
\card@def{CS2-029}{火球术}{法师4费法术}{造成6点伤害。}
\card@def{CS2-032}{烈焰风暴}{法师7费法术}{对所有敌方随从造成5点伤害。}
\card@def{EX1-169}{激活}{德鲁伊0费法术}{在本回合中,获得一个 法力水晶。}
\card@def{CS2-013}{野性成长}{德鲁伊3费法术}{获得一个空的法力水晶。}
\card@def[CS2-013]{CS2-013t}{法力过剩}{德鲁伊0费法术}{抽一张牌。(你最多可以拥有 十个法力水晶。)}
\card@def{CS2-012}{横扫}{德鲁伊4费法术}{对一个敌人造成4点伤害,并对所有其他敌人 造成1点伤害。}
\card@def{DS1-184}{追踪术}{猎人1费法术}{从你的牌库中发现一张牌。}
\card@def{DS1-185}{奥术射击}{猎人1费法术}{造成2点伤害。}
\card@def{CS2-237}{饥饿的秃鹫}{猎人2/2/1野兽}{每当你召唤一个野兽,抽一张牌。}
\card@def{CS2-087}{力量祝福}{圣骑士1费法术}{使一个随从获得+3攻击力。}
\card@def{CS2-091}{圣光的正义}{圣骑士1/1/4武器}{}
\card@def{HERO-04bp}{援军}{圣骑士2费英雄技能}{英雄技能 召唤一个1/1的白银之手新兵。}
\card@def{CS2-097}{真银圣剑}{圣骑士4/4/2武器}{每当你的英雄进攻,便为其恢复2点生命值。}
\card@def{CS2-235}{北郡牧师}{牧师1/1/3随从}{每当一个随从获得治疗时,抽一张牌。}
\card@def{CS2-004}{真言术:盾}{牧师1费法术}{使一个随从获得+2生命值。 抽一张牌。}
\card@def{CS2-236}{神圣之灵}{牧师2费法术}{使一个随从的生命值翻倍。}
\card@def{EX1-622}{暗言术:灭}{牧师2费法术}{消灭一个攻击力大于或等于5的随从。}
\card@def{CS1-113}{精神控制}{牧师10费法术}{获得一个敌方随从的控制权。}
\card@def{HERO-03bp}{匕首精通}{潜行者2费英雄技能}{英雄技能 装备一把1/2的 匕首。}
\card@def{CS2-076}{刺杀}{潜行者4费法术}{消灭一个敌方随从。}
\card@def{NEW1-004}{消失}{潜行者6费法术}{将所有随从移回其拥有者的 手牌。}
\card@def{HERO-02bp}{图腾召唤}{萨满祭司2费英雄技能}{英雄技能 随机召唤一个 图腾。}
\card@def{EX1-246}{妖术}{萨满祭司4费法术}{使一个随从变形成为一只0/1并具有嘲讽的青蛙。}
\card@def{CS2-046}{嗜血}{萨满祭司5费法术}{在本回合中,使你的所有随从获得+3攻击力。}
\card@def{CS2-042}{火元素}{萨满祭司6/6/5元素}{战吼:造成4点伤害。}
\card@def{CS3-002}{末日仪式}{术士0费法术}{消灭一个友方随从。如果你拥有5个或更多随从,召唤一个5/5的恶魔。}
\card@def{CS2-063}{腐蚀术}{术士1费法术}{选择一个敌方随从,在你的回合开始时,消灭该随从。}
\card@def{EX1-302}{死亡缠绕}{术士1费法术}{对一个随从造成1点伤害。如果“死亡缠绕”消灭该随从,抽一张牌。}
\card@def{CS2-062}{地狱烈焰}{术士4费法术}{对所有角色造成3点伤害。}
\card@def{CS2-064}{恐惧地狱火}{术士6/6/6恶魔}{战吼:对所有其他角色造成1点伤害。}
\card@def{CS2-105}{英勇打击}{战士2费法术}{在本回合中,使你的英雄获得+4攻击力。}

% 经典
\card@def{DREAM-05}{梦魇}{dream0费法术}{使一个随从获得+4/+4,在你的下个回合开始时,消灭该随从。}
\card@def{CS2-231}{小精灵}{中立0/1/1随从}{}
\card@def{CS2-146}{南海船工}{中立1/2/1海盗}{如果你装备一把武器,该随从具有 冲锋。}
\card@def{CS2-188}{叫嚣的中士}{中立1/1/1随从}{战吼:在本回合中,使一个随从获得+2攻击力。}
\card@def{EX1-009}{愤怒的小鸡}{中立1/1/1野兽}{受伤时具有+5攻 击力。}
\card@def{EX1-029}{麻风侏儒}{中立1/2/1随从}{亡语:对敌方英雄造成2点伤害。}
\card@def{EX1-509}{鱼人招潮者}{中立1/1/2鱼人}{每当你召唤一个鱼人,便获得 +1攻击力。}
\card@def{NEW1-025}{血帆海盗}{中立1/1/2海盗}{战吼:使对手的武器失去1点耐久度。}
\card@def{EX1-049}{年轻的酒仙}{中立2/3/2随从}{战吼:使一个友方随从从战场上移回你的手牌。}
\card@def{EX1-059}{疯狂的炼金师}{中立2/2/2随从}{战吼: 使一个随从的攻击力和生命值互换。}
\card@def{EX1-076}{小个子召唤师}{中立2/2/2随从}{你每个回合使用的第一张随从牌的法力值消耗减少(1)点。}
\card@def{EX1-096}{战利品贮藏者}{中立2/2/1随从}{亡语:抽一张牌。}
\card@def{NEW1-019}{飞刀杂耍者}{中立2/3/2随从}{在你召唤一个随从后,随机对一个敌人造成1点伤害。}
\card@def{NEW1-020}{狂野炎术师}{中立2/3/2随从}{在你施放一个法术后,对所有随从造成1点伤害。}
\card@def{NEW1-029}{米尔豪斯·法力风暴}{中立2/4/4随从}{战吼:下个回合敌方法术的法力值消耗为(0)点。}
\card@def{CS2-181}{负伤剑圣}{中立3/4/7随从}{战吼:对自身造成4点伤害。}
\card@def{EX1-006}{报警机器人}{中立3/0/3机械}{在你的回合开始时,随机将你的手牌中的一张随从牌与该随从 交换。}
\card@def{EX1-007}{苦痛侍僧}{中立3/1/3随从}{每当该随从受到伤害,抽一张牌。}
\card@def{EX1-044}{任务达人}{中立3/2/2随从}{每当你使用一张牌时,便获得+1/+1。}
\card@def{EX1-050}{寒光智者}{中立3/2/2鱼人}{战吼:每个玩家抽两张牌。}
\card@def{EX1-085}{精神控制技师}{中立3/3/3随从}{战吼:如果你的对手拥有4个或者更多随从,随机获得其中一个的控制权。}
\card@def{EX1-507}{鱼人领军}{中立3/3/3鱼人}{你的其他鱼人获得+2攻击力。}
\card@def{EX1-597}{小鬼召唤师}{中立3/1/5随从}{在你的回合结束时,对该随从造成1点伤害,并召唤一个1/1的 小鬼。}
\card@def{EX1-043}{暮光幼龙}{中立4/4/1龙}{战吼: 你每有一张手牌,便获得+1生命值。}
\card@def{EX1-093}{阿古斯防御者}{中立4/3/3随从}{战吼:使相邻的随从获得+1/+1和嘲讽。}
\card@def{EX1-595}{诅咒教派领袖}{中立4/4/2随从}{在一个友方随从死亡后,抽一张牌。}
\card@def{NEW1-026}{紫罗兰教师}{中立4/3/5随从}{每当你施放一个法术,召唤一个1/1的紫罗兰学徒。}
\card@def{EX1-564}{无面操纵者}{中立5/3/3随从}{战吼:选择一个随从,成为它的复制。}
\card@def{NEW1-041}{狂奔科多兽}{中立5/3/5野兽}{战吼:随机消灭一个攻击力小于或等于2的敌方随从。}
\card@def{EX1-016}{希尔瓦娜斯·风行者}{中立6/5/5随从}{亡语:随机获得一个敌方随从的控制权。}
\card@def{EX1-067}{银色指挥官}{中立6/4/2随从}{冲锋 圣盾}
\card@def{EX1-614}{萨维斯}{中立6/7/5恶魔}{在你使用一张牌后,召唤一个2/1的萨特。}
\card@def[EX1-614]{EX1-614t}{萨维亚萨特}{中立1/2/1恶魔}{}
\card@def{NEW1-040}{霍格}{中立6/4/4随从}{在你的回合结束时,召唤一个2/2并具有嘲讽的豺狼人。}
\card@def{EX1-298}{炎魔之王拉格纳罗斯}{中立8/8/8元素}{无法攻击。在你的回合结束时,随机对一个敌人造成8点伤害。}
\card@def{NEW1-038}{格鲁尔}{中立8/7/7随从}{在每个回合结束时,获得+1/+1。}
\card@def{EX1-560}{诺兹多姆}{中立9/8/8龙}{所有玩家 只有15秒的时间来进行他们的回合。}
\card@def{EX1-561}{阿莱克丝塔萨}{中立9/8/8龙}{战吼: 将一方英雄的剩余生命值变为15。}
\card@def{EX1-563}{玛里苟斯}{中立9/4/12龙}{法术伤害+5}
\card@def{EX1-572}{伊瑟拉}{中立9/4/12龙}{在你的回合结束时,将一张梦境牌置入你的手牌。}
\card@def{NEW1-030}{死亡之翼}{中立10/12/12龙}{战吼: 消灭所有其他随从,并弃掉你的手牌。}
\card@def{EX1-105}{山岭巨人}{中立12/8/8元素}{你每有一张其他手牌,该牌的法力值消耗便减少(1)点。}
\card@def{EX1-620}{熔核巨人}{中立20/8/8元素}{你的英雄 每受到1点伤害,该牌的法力值消耗便减少(1)点。}
\card@def{EX1-295o}{寒冰屏障}{}{在本回合中,你的英雄获得免疫。}
\card@def{NEW1-012}{法力浮龙}{法师1/1/2随从}{每当你施放一个法术,便获得 +1攻击力。}
\card@def{tt-010a}{扰咒师}{法师1/1/3随从}{}
\card@def{EX1-612}{肯瑞托法师}{法师3/4/3随从}{战吼: 在本回合中,你使用的下一个奥秘的法力值消耗为(0)点。}
\card@def{EX1-275}{冰锥术}{法师3费法术}{冻结一个随从和其相邻的随从,并对它们造成1点伤害。}
\card@def{EX1-287}{法术反制}{法师3费法术}{奥秘:当你的对手施放一个法术时,反制该法术。}
\card@def{EX1-289}{寒冰护体}{法师3费法术}{奥秘:当你的英雄受到攻击时,获得8点护甲值。}
\card@def{EX1-294}{镜像实体}{法师3费法术}{奥秘:在你的对手使用一张随从牌后,召唤一个该随从的复制。}
\card@def{EX1-295}{寒冰屏障}{法师3费法术}{奥秘:当你的英雄将要承受致命伤害时,防止这些伤害,并使其在本回合中获得免疫。}
\card@def{EX1-594}{蒸发}{法师3费法术}{奥秘:当一个随从攻击你的英雄,将其消灭。}
\card@def{tt-010}{扰咒术}{法师3费法术}{奥秘:当一个敌方法术以一个随从为目标时,召唤一个1/3的随从并使其成为新的目标。}
\card@def{CS2-028}{暴风雪}{法师6费法术}{对所有敌方随从造成2点伤害,并使其冻结。}
\card@def{EX1-279}{炎爆术}{法师10费法术}{造成10点伤害。}
\card@def{EX1-158}{丛林之魂}{德鲁伊4费法术}{使你的所有随从获得“亡语:召唤一个2/2的树人”。}
\card@def[EX1-158]{EX1-158t}{树人}{德鲁伊2/2/2随从}{}
\card@def{EX1-164}{滋养}{德鲁伊5费法术}{抉择:获得两个法力水晶;或者抽三张牌。}
\card@def{EX1-544}{照明弹}{猎人1费法术}{所有随从失去潜行,摧毁所有敌方奥秘,抽一张牌。}
\card@def{EX1-531}{食腐土狼}{猎人2/2/2野兽}{每当一个友方野兽死亡,便获得+2/+1。}
\card@def{EX1-533}{误导}{猎人2费法术}{奥秘:当一个敌人攻击你的英雄时,改为该敌人随机攻击另一个角色。}
\card@def{EX1-554}{毒蛇陷阱}{猎人2费法术}{奥秘:当你的随从受到攻击时,召唤三条1/1的蛇。}
\card@def{EX1-609}{狙击}{猎人2费法术}{奥秘:在你的对手使用一张随从牌后,对该随从造成4点伤害。}
\card@def{EX1-610}{爆炸陷阱}{猎人2费法术}{奥秘:当你的英雄受到攻击,对所有敌人造成2点伤害。}
\card@def{EX1-611}{冰冻陷阱}{猎人2费法术}{奥秘:当一个敌方随从攻击时,将其移回拥有者的手牌,并且法力值消耗增加(2)点。}
\card@def{EX1-537}{爆炸射击}{猎人5费法术}{对一个随从造成5点伤害,并对其相邻的随从造成 2点伤害。}
\card@def{DS1-188}{角斗士的长弓}{猎人7/5/2武器}{你的英雄在攻击时具有免疫。}
\card@def{EX1-130a}{防御者}{圣骑士1/2/1随从}{}
\card@def{EX1-130}{崇高牺牲}{圣骑士1费法术}{奥秘:当一个敌人攻击时,召唤一个2/1的防御者,并使其成为攻击的目标。}
\card@def{EX1-132}{以眼还眼}{圣骑士1费法术}{奥秘: 当你的英雄受到伤害时,对敌方英雄造成等量伤害。}
\card@def{EX1-136}{救赎}{圣骑士1费法术}{奥秘:当一个友方随从死亡时,使其回到战场,并具有1点生命值。}
\card@def{EX1-363}{智慧祝福}{圣骑士1费法术}{选择一个随从,每当其进行攻击,便抽一张牌。}
\card@def{EX1-379}{忏悔}{圣骑士1费法术}{奥秘: 在你的对手使用一张随从牌后,使该随从的生命值降为1。}
\card@def{EX1-382}{奥尔多卫士}{圣骑士3/3/3随从}{战吼:使一个敌方随从的攻击力变为1。}
\card@def{EX1-366}{公正之剑}{圣骑士3/1/5武器}{在你召唤一个随从后,使其获得+1/+1,这把武器失去1点耐久度。}
\card@def{EX1-355}{受祝福的勇士}{圣骑士5费法术}{使一个随从的攻击力翻倍。}
\card@def{EX1-365}{神圣愤怒}{圣骑士5费法术}{抽一张牌,并造成等同于其法力值消耗的伤害。}
\card@def{EX1-384}{复仇之怒}{圣骑士6费法术}{造成8点伤害,随机分配到所有敌人身上。}
\card@def{EX1-383}{提里奥·弗丁}{圣骑士8/6/6随从}{圣盾,嘲讽,亡语:装备一把5/3的 灰烬使者。}
\card@def[EX1-383]{EX1-383t}{灰烬使者}{圣骑士5/5/3武器}{}
\card@def{CS1-129}{心灵之火}{牧师1费法术}{使一个随从的攻击力等同于其生命值。}
\card@def{EX1-334}{暗影狂乱}{牧师3费法术}{直到回合结束,获得一个攻击力小于或等于3的敌方随从的控制权。}
\card@def{EX1-591}{奥金尼灵魂祭司}{牧师4/3/5随从}{你的恢复生命值的牌和技能改为造成等量的伤害。}
\card@def{EX1-345}{控心术}{牧师4费法术}{随机将你对手的牌库中的一张随从牌的复制置入战场。}
\card@def{EX1-626}{群体驱散}{牧师4费法术}{沉默所有敌方随从,抽一张牌。}
\card@def{EX1-623}{圣殿执行者}{牧师5/5/6随从}{战吼:使一个友方随从获得+3生命值。}
\card@def{EX1-350}{先知维伦}{牧师7/7/7随从}{使你的法术和英雄技能的伤害和治疗效果翻倍。}
\card@def{EX1-145o}{伺机待发}{}{在本回合中,你所施放的下一个法术的法力值消耗减少(2)点。}
\card@def{EX1-145}{伺机待发}{潜行者0费法术}{在本回合中,你所施放的下一个法术的法力值消耗减少(2)点。}
\card@def{EX1-126}{背叛}{潜行者2费法术}{使一个敌方随从对其相邻的随从 造成等同于其攻击力的伤害。}
\card@def{EX1-613}{艾德温·范克里夫}{潜行者3/2/2随从}{连击:在本回合中,你每使用一张其他牌,便获得+2/+2。}
\card@def{EX1-133}{毁灭之刃}{潜行者3/2/2武器}{战吼:造成1点伤害。连击:改为造成2点伤害。}
\card@def{EX1-238}{闪电箭}{萨满祭司1费法术}{造成3点伤害,过载:(1)}
\card@def{CS2-038}{先祖之魂}{萨满祭司2费法术}{使一个随从获得“亡语:再次召唤该随从。”}
\card@def{EX1-258}{无羁元素}{萨满祭司3/3/4元素}{在你使用一张具有过载的牌后,便获得+1/+1。}
\card@def{EX1-575}{法力之潮图腾}{萨满祭司3/0/3图腾}{在你的回合结束时,抽一张牌。}
\card@def{EX1-259}{闪电风暴}{萨满祭司3费法术}{对所有敌方随从造成3点伤害,过载:(2)}
\card@def[VAN-EX1-323]{EX1-323h}{加拉克苏斯大王}{术士0费恶魔}{}
\card@def{EX1-316}{力量的代价}{术士1费法术}{使一个友方随从获得+4/+4,该随从会在回合结束时死亡。}
\card@def{EX1-596}{恶魔之火}{术士2费法术}{对一个随从造成2点伤害,如果该随从是友方恶魔,则改为使其获得+2/+2。}
\card@def{EX1-301}{恶魔卫士}{术士3/3/5恶魔}{嘲讽,战吼:摧毁你的一个法力水晶。}
\card@def{EX1-315}{召唤传送门}{术士4/0/4随从}{你的随从牌的法力值消耗减少(2)点,但不能少于(1)点。}
\card@def{EX1-303}{暗影烈焰}{术士4费法术}{消灭一个友方随从,对所有敌方随从造成等同于其攻击力的伤害。}
\card@def{EX1-310}{末日守卫}{术士5/5/7恶魔}{冲锋,战吼:随机弃两张牌。}
\card@def{EX1-309}{灵魂虹吸}{术士5费法术}{消灭一个随从,为你的英雄恢复3点生命值。}
\card@def{EX1-312}{扭曲虚空}{术士8费法术}{消灭所有随从。}
\card@def{EX1-323}{加拉克苏斯大王}{术士9费英雄}{战吼:装备一把3/8的血怒。}
\card@def{EX1-409}{升级}{战士1费法术}{如果你装备一把武器,使它获得+1/+1。否则,装备一把1/3的武器。}
\card@def{EX1-402}{铸甲师}{战士2/1/4随从}{每当一个友方随从受到伤害,便获得1点护甲值。}
\card@def{NEW1-036}{命令怒吼}{战士2费法术}{在本回合中,你的随从的生命值无法被降到1点以下。抽一张牌。}
\card@def{EX1-604}{暴乱狂战士}{战士3/2/4随从}{每当一个随从 受到伤害,便获得+1攻击力。}
\card@def{EX1-407}{绝命乱斗}{战士5费法术}{随机选择一个随从,消灭除了该随从外的所有其他随从。}
\card@def{EX1-411}{血吼}{战士7/7/1武器}{攻击随从不会消耗耐久度,改为降低1点攻击力。}
\card@def[EX1-411]{EX1-411e}{血性狂暴}{}{不会消耗耐久度。}

% 核心2021
\card@def{GAME-005}{幸运币}{中立0费法术}{在本回合中,获得一个 法力水晶。}
\card@def{CS3-028}{暗中生长}{牧师2费法术}{从你的牌库中发现一张法术牌。}

% 纳克萨玛斯
\card@def{FP1-028}{送葬者}{中立1/1/2随从}{每当你召唤一个具有亡语的随从,便获得+1/+1。}
\card@def{FP1-002}{鬼灵爬行者}{中立2/1/2野兽}{亡语:召唤两只1/1的鬼灵蜘蛛。}
\card@def[FP1-002]{FP1-002t}{鬼灵蜘蛛}{中立1/1/1随从}{}
\card@def{FP1-004}{疯狂的科学家}{中立2/2/2随从}{亡语: 将一个奥秘从你的牌库中置入战场。}
\card@def{FP1-009}{死亡领主}{中立3/2/8随从}{嘲讽,亡语:你的对手将一个随从从其牌库置入战场。}
\card@def{FP1-029}{舞动之剑}{中立3/4/4随从}{亡语:你的对手抽一张牌。}
\card@def{FP1-031}{瑞文戴尔男爵}{中立4/1/7随从}{你的随从的亡语将触发两次。}
\card@def{FP1-030}{洛欧塞布}{中立5/5/5随从}{战吼:下个回合敌方法术的法力值消耗增加(5)点。}
\card@def{FP1-013}{克尔苏加德}{中立8/6/8随从}{在每个回合结束时,召唤所有在本回合中死亡的友方随从。}
\card@def{FP1-018}{复制}{法师3费法术}{奥秘:当一个友方随从死亡时,将两张该随从的复制置入你的手牌。}
\card@def{FP1-019}{剧毒之种}{德鲁伊4费法术}{消灭所有随从,并召唤等量的2/2树人代替他们。}
\card@def{FP1-020}{复仇}{圣骑士1费法术}{奥秘:当你的随从死亡时,随机使一个友方随从获得+3/+2。}
\card@def{FP1-026}{阿努巴尔伏击者}{潜行者4/5/5随从}{亡语: 随机将一个友方随从移回你的手牌。}
\card@def{FP1-025}{转生}{萨满祭司2费法术}{消灭一个随从,然后将其复活,并具有所有生命值。}
\card@def{FP1-022}{空灵召唤者}{术士4/3/4恶魔}{亡语: 随机将一张恶魔牌从你的手牌置入战场。}

% 地精大战侏儒
\card@def{GVG-093}{活动假人}{中立0/0/2机械}{嘲讽}
\card@def{GVG-082}{发条侏儒}{中立1/2/1机械}{亡语:将一张零件牌置入你的手牌。}
\card@def{GVG-006}{机械跃迁者}{中立2/2/3机械}{你的机械的法力值消耗减少(1)点。}
\card@def{GVG-075}{船载火炮}{中立2/2/3随从}{在你召唤一个海盗后,随机对一个敌人造成2点伤害。}
\card@def{GVG-076}{自爆绵羊}{中立2/1/1机械}{亡语:对所有随从造成2点伤害。}
\card@def{GVG-108}{侏儒变形师}{中立2/3/2随从}{战吼: 将一个友方随从随机变形成为一个法力值消耗相同的随从。}
\card@def{GVG-044}{蜘蛛坦克}{中立3/3/4机械}{}
\card@def{GVG-065}{食人魔步兵}{中立3/4/4随从}{50\%几率攻击错误的敌人。}
\card@def{GVG-092}{侏儒实验技师}{中立3/3/2随从}{战吼: 抽一张牌,如果该牌是随从牌,则将其变形成为一只小鸡。}
\card@def[GVG-092]{GVG-092t}{小鸡}{中立1/1/1野兽}{}
\card@def{GVG-104}{大胖}{中立3/2/3随从}{每当你使用一张攻击力为1的随从牌,便使其获得+2/+2。}
\card@def{GVG-074}{科赞秘术师}{中立4/4/3随从}{战吼:随机获得一个敌方奥秘的控制权。}
\card@def{GVG-094}{基维斯}{中立4/1/4机械}{在每个玩家的回合结束时,该玩家抽若干牌,直至其手牌数量达到3张。}
\card@def{GVG-016}{魔能机甲}{中立5/8/8机械}{每当你的对手使用一张卡牌时,便移除你的牌库顶的三张牌。}
\card@def{GVG-111}{米米尔隆的头部}{中立5/4/5机械}{在你的回合开始时,如果你控制至少三个机械,则消灭这些机械,并将其组合成V-07-TR-0N。}
\card@def[GVG-111]{GVG-111t}{V-07-TR-0N}{中立8/4/8机械}{冲锋 超级风怒}
\card@def{GVG-112}{食人魔勇士穆戈尔}{中立6/7/6随从}{所有随从有50\%几率攻击错误的敌人。}
\card@def{GVG-110}{砰砰博士}{中立7/7/7随从}{战吼: 召唤两个1/1的砰砰机器人。警告:该机器人随时可能爆炸。}
\card@def{GVG-116}{瑟玛普拉格}{中立9/9/7机械}{每当一个敌方随从死亡,召唤一个 麻风侏儒。}
\card@def{GVG-005}{麦迪文的残影}{法师4费法术}{复制你的所有随从,并将其置入你的手牌。}
\card@def{GVG-007}{烈焰巨兽}{法师7/7/7机械}{当你抽到该牌时,对所有角色造成 2点伤害。}
\card@def{GVG-031}{回收}{德鲁伊6费法术}{将一个敌方随从洗入你对手的 牌库。}
\card@def{GVG-035}{玛洛恩}{德鲁伊7/9/7野兽}{亡语:将该随从洗入你的牌库。}
\card@def{GVG-033}{生命之树}{德鲁伊9费法术}{为所有角色恢复所有生命值。}
\card@def{GVG-026}{假死}{猎人2费法术}{触发所有友方随从的亡语。}
\card@def{GVG-049}{加兹瑞拉}{猎人7/6/9野兽}{每当该随从受到伤害,便使其攻击力 翻倍。}
\card@def{GVG-061}{作战动员}{圣骑士3费法术}{召唤三个1/1的白银之手新兵,装备一把1/4的武器。}
\card@def{GVG-063}{伯瓦尔·弗塔根}{圣骑士5/1/7随从}{如果这张牌在你的手牌中,每当一个友方随从死亡,便获得+1攻击力。}
\card@def{GVG-011}{缩小射线工程师}{牧师2/3/2随从}{战吼:在本回合中,使一个随从获得-2攻击力。}
\card@def{GVG-014}{沃金}{牧师5/6/2随从}{战吼:与另一个随从交换生命值。}
\card@def{GVG-008}{圣光炸弹}{牧师6费法术}{对所有随从造成等同于其攻击力的伤害。}
\card@def{GVG-088}{食人魔忍者}{潜行者5/6/6随从}{潜行,50\%几率攻击错误的敌人。}
\card@def{GVG-066}{砂槌萨满祭司}{萨满祭司4/5/4随从}{风怒,过载:(1) 50\%几率攻击错误的敌人。}
\card@def{GVG-029}{先祖召唤}{萨满祭司4费法术}{每个玩家从手牌中随机将一个随从置入战场。}
\card@def{GVG-045}{小鬼爆破}{术士4费法术}{对一个随从造成2-4点伤害。每造成1点伤害,便召唤一个1/1的小鬼。}
\card@def[GVG-045]{GVG-045t}{小鬼}{术士1/1/1恶魔}{}
\card@def{GVG-100}{漂浮观察者}{术士5/4/4恶魔}{每当你的英雄在你的回合受到伤害,便获得+2/+2。}
\card@def{GVG-021}{玛尔加尼斯}{术士9/9/7恶魔}{你的其他恶魔获得+2/+2。你的英雄获得免疫。}
\card@def{GVG-054}{食人魔战槌}{战士3/4/2武器}{50\%几率攻击错误的敌人。}
\card@def{GVG-086}{重型攻城战车}{战士5/5/5机械}{每当你获得护甲,该随从便获得 +1攻击力。}
\card@def{GVG-056}{钢铁战蝎}{战士6/6/5机械}{战吼:将一张“地雷” 牌洗入你对手的牌库。当抽到“地雷”时,便会受到10点伤害。}
\card@def[GVG-056]{GVG-056t}{地雷}{战士6费法术}{抽到时施放 你受到10点伤害。}

% 黑石山
\card@def{BRM-022}{龙蛋}{中立1/0/2随从}{每当该随从受到伤害,便召唤一条2/1的雏龙。}
\card@def[BRM-022]{BRM-022t}{黑色雏龙}{中立1/2/1龙}{}
\card@def{BRM-027p}{死吧,虫子!}{中立2费英雄技能}{英雄技能 随机对一个敌人造成8点伤害。}
\card@def{BRM-019}{恐怖的奴隶主}{中立5/3/3随从}{在该随从受到伤害并没有死亡后,召唤另一个恐怖的奴隶主。}
\card@def{BRM-025}{火山幼龙}{中立6/6/4龙}{在本回合中每有一个随从死亡,该牌的 法力值消耗就减少(1)点。}
\card@def{BRM-028}{索瑞森大帝}{中立6/5/5随从}{在你的回合结束时,你所有手牌的法力值消耗减少(1)点。}
\card@def{BRM-031}{克洛玛古斯}{中立8/6/8龙}{每当你抽一张牌时,将该牌的另一张复制置入你的手牌。}
\card@def{BRM-027}{管理者埃克索图斯}{中立9/9/7随从}{亡语: 用炎魔之王拉格纳罗斯替换你的英雄。}
\card@def[BRM-027]{BRM-027h}{炎魔之王拉格纳罗斯}{中立0费英雄}{}
\card@def{BRM-030}{奈法利安}{中立9/8/8龙}{战吼:随机将两张(你对手职业的)法术牌置入你的手牌。}
\card@def[BRM-030]{BRM-030t}{扫尾}{中立4费法术}{造成4点伤害。}
\card@def{BRM-002}{火妖}{法师3/2/4随从}{在你施放一个法术后,造成2点伤害,随机分配到所有敌人身上。}
\card@def{BRM-017}{复活术}{牧师2费法术}{随机召唤一个在本局对战中死亡的友方随从。}
\card@def{BRM-008}{黑铁潜藏者}{潜行者5/4/3随从}{战吼: 对所有未受伤的敌方随从造成2点伤害。}
\card@def{BRM-006}{小鬼首领}{术士3/2/4恶魔}{每当该随从受到伤害,召唤一个1/1的 小鬼。}
\card@def{BRM-016}{掷斧者}{战士4/2/5随从}{每当该随从受到伤害,对敌方英雄造成 2点伤害。}

% 冠军的试炼
\card@def{AT-082}{低阶侍从}{中立1/1/2随从}{激励: 获得+1攻击力。}
\card@def{AT-080}{要塞指挥官}{中立2/2/3随从}{每个回合你可以使用两次英雄技能。}
\card@def{AT-086}{破坏者}{中立3/4/3随从}{战吼:下个回合敌方英雄技能的法力值消耗增加(5)点。}
\card@def{AT-110}{角斗场主管}{中立3/2/5随从}{激励:将该随从移回你的手牌。}
\card@def{AT-121}{人气选手}{中立4/4/4随从}{每当你使用一张具有战吼的牌,便获得+1/+1。}
\card@def{AT-122}{穿刺者戈莫克}{中立4/4/4随从}{战吼:如果你拥有至少四个其他随从,则造成4点伤害。}
\card@def{AT-088}{穆戈尔的勇士}{中立6/8/5随从}{50\%几率攻击错误的敌人。}
\card@def{AT-098}{杂耍吞法者}{中立6/6/5随从}{战吼:复制对手的英雄技能。}
\card@def{AT-124}{博尔夫·碎盾}{中立6/3/9随从}{每当你的英雄受到伤害,便会由该随从来承担。}
\card@def{AT-128}{骷髅骑士}{中立6/7/4随从}{亡语:揭示双方牌库里的一张随从牌。如果你的牌法力值消耗较大,则将骷髅骑士移回你的手牌。}
\card@def{AT-132}{裁决者图哈特}{中立6/6/3随从}{战吼:以更强的英雄技能来替换你的初始英雄技能。}
\card@def{AT-123}{冰喉}{中立7/6/6龙}{嘲讽,亡语: 如果你的手牌中有龙牌,则对所有随从造成3点伤害。}
\card@def{AT-003}{英雄之魂}{法师2/3/2随从}{你的英雄技能会额外造成1点伤害。}
\card@def{AT-002}{轮回}{法师3费法术}{奥秘:当一个友方随从死亡时,随机召唤一个法力值消耗相同的随从。}
\card@def{AT-008}{考达拉幼龙}{法师6/6/7龙}{你可以使用任意次数的英雄技能。}
\card@def{AT-043}{星界沟通}{德鲁伊4费法术}{获得十个法力水晶。弃掉 你的手牌。}
\card@def{AT-041}{荒野骑士}{德鲁伊7/6/6随从}{每当你召唤一个野兽,该随从牌的法力值消耗减少(1)点。}
\card@def{AT-045}{艾维娜}{德鲁伊10/5/5随从}{你的随从牌的法力值消耗为(1)点。}
\card@def{AT-060}{捕熊陷阱}{猎人2费法术}{奥秘:在你的英雄受到攻击后,召唤一个3/3并具有嘲讽的灰熊。}
\card@def{AT-056}{强风射击}{猎人3费法术}{对一个随从及其相邻的随从造成2点伤害。}
\card@def{AT-073}{争强好胜}{圣骑士1费法术}{奥秘:在你的回合开始时,使你的所有随从获得+1/+1。}
\card@def{AT-013}{真言术:耀}{牧师1费法术}{选择一个随从。每当其进行攻击,为你的英雄恢复 4点生命值。}
\card@def{AT-012}{暗影子嗣}{牧师4/5/4随从}{激励:对每个英雄造成4点伤害。}
\card@def{AT-029}{锈水海盗}{潜行者1/2/1海盗}{每当你装备一把武器,使武器获得+1攻击力。}
\card@def{AT-031}{窃贼}{潜行者2/2/2随从}{每当该随从攻击一方英雄,会将幸运币置入你的手牌。}
\card@def{AT-033}{剽窃}{潜行者3费法术}{随机将两张(你对手职业的)卡牌置入你的手牌。}
\card@def{AT-036}{阿努巴拉克}{潜行者9/8/4随从}{亡语:将该随从移回你的手牌,召唤一个4/4的蛛魔。}
\card@def[AT-036]{AT-036t}{蛛魔}{潜行者4/4/4随从}{}
\card@def{AT-051}{元素毁灭}{萨满祭司3费法术}{对所有随从造成4到5点伤害。过载:(5)}
\card@def{AT-021}{小鬼骑士}{术士2/3/2恶魔}{每当你弃掉一张牌时,便获得+1/+1。}
\card@def{AT-023}{虚空碾压者}{术士6/5/4恶魔}{激励:随机消灭每个玩家的一个随从。}
\card@def{AT-130}{破海者}{战士6/6/7随从}{当你抽到该牌时,对你的随从造成 1点伤害。}
\card@def{AT-072}{瓦里安·乌瑞恩}{战士10/7/7随从}{战吼:抽三张牌。将抽到的随从牌直接置入战场。}

% 探险者协会
\card@def{LOE-076}{芬利·莫格顿爵士}{中立1/1/3鱼人}{战吼:发现一个新的基础英雄技能。}
\card@def{LOE-039}{A3型机械金刚}{中立3/3/4机械}{战吼:如果你控制其他任何机械,则发现一张机械牌。}
\card@def{LOE-077}{布莱恩·铜须}{中立3/2/4随从}{你的战吼会触发 两次。}
\card@def{LOE-079}{伊莉斯·逐星}{中立4/3/5随从}{战吼:将“黄金猿藏宝图”洗入你的牌库。}
\card@def[LOE-079]{LOE-019t2}{黄金猿}{中立4/6/6随从}{嘲讽 战吼:将你的手牌和牌库里的卡牌替换成传说随从。}
\card@def{LOE-110}{远古暗影}{中立4/7/4随从}{战吼:将一张“远古诅咒”牌洗入你的牌库。当你抽到该牌,便会受到7点伤害。}
\card@def[LOE-110]{LOE-110t}{远古诅咒}{中立4费法术}{抽到时施放 你受到7点伤害。}
\card@def{LOE-053}{西风灯神}{中立5/4/6元素}{在你对一个其他友方随从施放法术后,将法术效果复制在此随从身上。}
\card@def{LOE-086}{召唤石}{中立5/0/6随从}{每当你施放一个法术,随机召唤一个法力值消耗相同的随从。}
\card@def{LOE-038}{纳迦海巫}{中立8/5/5naga}{你的卡牌法力值消耗为(5)点。}
\card@def{LOEA16-4}{恐怖丧钟}{中立10费法术}{造成10点伤害,随机分配到所有敌人身上。}
\card@def{LOE-119}{复活的铠甲}{法师4/4/4随从}{你的英雄每次只会受到1点伤害。}
\card@def{LOE-003}{虚灵巫师}{法师5/6/4随从}{战吼: 发现一张法术牌。}
\card@def{LOE-021}{毒镖陷阱}{猎人2费法术}{奥秘: 在对方使用英雄技能后,随机对一个敌人造成5点伤害。}
\card@def{LOE-027}{审判}{圣骑士1费法术}{奥秘:在你的对手使用一张随从牌后,如果他控制至少三个其他随从,则将其消灭。}
\card@def{LOE-111}{极恶之咒}{牧师5费法术}{对所有随从造成3点伤害。将该牌洗入你对手的牌库。}
\card@def{LOE-104}{埋葬}{牧师6费法术}{选择一个敌方随从。将该随从洗入你的牌库。}
\card@def{LOE-019}{石化迅猛龙}{潜行者3/3/4随从}{战吼:选择一个友方随从,获得其亡语的复制。}
\card@def{LOE-016}{顽石元素}{萨满祭司4/2/6元素}{在你使用一张具有 战吼的随从牌后,随机对一个敌人造成2点伤害。}
\card@def{LOE-007}{拉法姆的诅咒}{术士2费法术}{使你的对手获得一张“诅咒”。在对手的回合开始时,如果它在对手的手牌中,则造成2点伤害。}
\card@def[LOE-007]{LOE-007t}{诅咒}{术士2费法术}{如果这张牌在你的手牌中,在你的回合开始时,你的英雄受到2点伤害。}
\card@def{LOE-118}{诅咒之刃}{战士1/2/3武器}{你的英雄受到的所有伤害效果翻倍。}

% 古神
\card@def{OG-123}{百变泽鲁斯}{中立1/1/1随从}{如果这张牌在你的手牌中,每个回合都会随机变成一张随从牌。}
\card@def[OG-123]{OG-123e}{变形}{}{随机变成一张随从牌。}
\card@def{OG-256}{恩佐斯的子嗣}{中立3/2/2随从}{亡语:使你的所有随从获得+1/+1。}
\card@def{OG-138}{蛛魔先知}{中立6/4/4随从}{在你的回合开始时,该随从牌的法力值消耗减少(1)点。}
\card@def{OG-255}{厄运召唤者}{中立8/7/9随从}{战吼:使你的克苏恩获得+2/+2(无论它在哪里)。如果克苏恩死亡,将其洗入你的牌库。}
\card@def{OG-300}{波戈蒙斯塔}{中立8/6/7随从}{每当波戈蒙斯塔攻击并消灭一个随从,便获得+2/+2。}
\card@def{OG-042}{亚煞极}{中立10/10/10随从}{在你的回合结束时,将一个随从从你的牌库置入战场。}
\card@def{OG-133}{恩佐斯}{中立10/5/7随从}{战吼:召唤所有你在本局对战中死亡的,并具有亡语的随从。}
\card@def{OG-134}{尤格-萨隆}{中立10/7/5随从}{战吼: 在本局对战中,你每施放过一个法术,便随机施放一个法术(目标随机而定)。}
\card@def{OG-279}{克苏恩}{中立10/6/6随从}{战吼: 造成等同于该随从攻击力的伤害,随机分配到所有敌人身上。}
\card@def{OG-087}{尤格-萨隆的仆从}{法师5/5/4随从}{战吼:随机施放一个法力值消耗小于或等于(5)点的法术(目标随机而定)。}
\card@def{OG-120}{阿诺玛鲁斯}{法师8/8/6元素}{亡语:对所有随从造成8点伤害。}
\card@def{OG-313}{腐化灰熊}{德鲁伊3/2/2野兽}{在你召唤一个随从后,使其获得+1/+1。}
\card@def{OG-308}{巨型沙虫}{猎人8/8/8野兽}{每当该随从攻击并消灭一个随从,便可再次攻击。}
\card@def{OG-310}{夜色镇执法官}{圣骑士3/3/3随从}{每当你召唤一个生命值为1的随从,便使其获得圣盾。}
\card@def{OG-229}{光耀之主拉格纳罗斯}{圣骑士8/8/8元素}{在你的回合结束时,为一个受伤的友方角色恢复8点生命值。}
\card@def{OG-072}{深渊探险}{潜行者1费法术}{发现一张亡语牌。}
\card@def{OG-330}{幽暗城商贩}{潜行者2/2/2随从}{亡语:随机将一张(你对手职业的)卡牌置入你的手牌。}
\card@def{OG-267}{南海畸变船长}{潜行者4/4/4海盗}{亡语:使你的武器获得+2攻击力。}
\card@def{OG-027}{异变}{萨满祭司1费法术}{随机将你的 所有随从变形成为法力值消耗增加(1)点的随从。}
\card@def{OG-328}{异变之主}{萨满祭司4/4/5随从}{战吼:将一个友方随从随机变形成为一个法力值消耗增加(1)点的随从。}
\card@def{OG-113}{夜色镇议员}{术士3/1/5随从}{在你召唤一个随从后,获得+1攻击力。}
\card@def{OG-116}{狂乱传染}{术士3费法术}{造成9点伤害,随机分配到所有角色身上。}
\card@def{OG-121}{古加尔}{术士7/7/7随从}{战吼:在本回合中,你施放的下一个法术不再消耗法力值,转而消耗生命值。}
\card@def{OG-312}{恩佐斯的副官}{战士1/1/1海盗}{战吼:装备一把1/3的锈蚀铁钩。}

% 卡拉赞
\card@def{KAR-096}{玛克扎尔王子}{中立5/5/6恶魔}{对战开始时:额外将五张传说随从牌置入你的牌库。}
\card@def{KAR-114}{巴内斯}{中立5/3/4随从}{战吼:随机挑选你牌库里的一个随从,召唤一个1/1的复制。}
\card@def{KAR-041}{沟渠潜伏者}{中立6/3/3随从}{战吼:消灭一个随从。亡语:再次召唤被消灭的随从。}
\card@def{KAR-711}{奥术巨人}{中立12/8/8随从}{在本局对战中,你每施放一个法术就会使 法力值消耗减少(1)点。}
\card@def{KAR-092}{麦迪文的男仆}{法师2/2/3随从}{战吼: 如果你控制一个奥秘,则造成3点伤害。}
\card@def{KAR-076}{火焰之地传送门}{法师7费法术}{造成5点伤害。随机召唤一个法力值消耗为(5)的随从。}
\card@def{KAR-077}{银月城传送门}{圣骑士4费法术}{使一个随从获得+2/+2。随机召唤一个法力值消耗为(2)的随从。}
\card@def{KAR-069}{吹嘘海盗}{潜行者1/1/1海盗}{战吼:随机将一张另一职业的卡牌置入你的手牌。}
\card@def{KAR-070}{虚灵商人}{潜行者5/5/6随从}{战吼:如果你的手牌中有另一职业的卡牌,则其法力值消耗减少(2)点。}
\card@def{KAR-089}{玛克扎尔的小鬼}{术士1/1/3恶魔}{每当你弃掉一张牌时,抽一张牌。}
\card@def{KAR-205}{镀银魔像}{术士3/3/3随从}{如果你弃掉了这张随从牌,则会召唤它。}
\card@def{KAR-028}{愚者之灾}{战士5/3/4武器}{每个回合攻击次数不限,但无法攻击英雄。}

% 加基森
\card@def{CFM-095}{鼬鼠挖掘工}{中立1/1/1野兽}{亡语:将该随从洗入你对手的牌库。}
\card@def{CFM-120}{亡灵药剂师}{中立1/2/2随从}{亡语:为每个英雄恢复4点生命值。}
\card@def{CFM-637}{海盗帕奇斯}{中立1/1/1海盗}{在你使用一张海盗牌后,从你的牌库中召唤该随从。}
\card@def{CFM-712-t01}{青玉魔像}{中立1/1/1随从}{}
\card@def{CFM-790}{卑劣的脏鼠}{中立2/2/6随从}{嘲讽,战吼:使你的对手随机从手牌中召唤一个随从。}
\card@def{CFM-807}{大富翁比尔杜}{中立3/3/4随从}{在你施放一个法术后,复原你的 英雄技能。}
\card@def{CFM-808}{“鲨鱼”加佐}{中立4/5/4随从}{每当该随从进行攻击时,双方玩家抽若干数量的牌,直到拥有三张手牌。}
\card@def{CFM-851}{勇敢的记者}{中立4/3/3随从}{每当你的对手抽一张牌时,便获得+1/+1。}
\card@def{CFM-025}{发条强盗机器人}{中立6/5/5机械}{每当该随从攻击另一个随从并存活时,抽一张牌。}
\card@def{CFM-670}{诺格弗格市长}{中立9/5/4随从}{所有角色都会随机选择目标。}
\card@def{CFM-066}{暗金教侍从}{法师1/2/1随从}{战吼: 在本回合中,你使用的下一个奥秘的法力值消耗为(0)点。}
\card@def{CFM-620}{变形药水}{法师3费法术}{奥秘:在你的对手使用一张随从牌后,将其变形成为1/1的绵羊。}
\card@def{CFM-760}{暗金教水晶侍女}{法师6/5/5随从}{在本局对战中,你每使用一张奥秘牌 就会使法力值消耗减少(2)点。}
\card@def{CFM-687}{墨水大师索莉娅}{法师7/5/5随从}{战吼:在本回合中,如果你的牌库里没有相同的牌,你所施放的下一个法术的法力值消耗为(0)点。}
\card@def{CFM-623}{强能奥术飞弹}{法师7费法术}{随机对敌人发射三枚飞弹,每枚飞弹造成3点伤害。}
\card@def{CFM-616}{妙手空空}{德鲁伊3费法术}{每控制一个友方随从,便获得一个空的法力水晶。}
\card@def{CFM-308}{遗忘之王库恩}{德鲁伊10/7/7随从}{抉择:获得10点护甲值;或者复原你的法力水晶。}
\card@def{CFM-026}{军备宝箱}{猎人2费法术}{奥秘:在你的对手使用一张随从牌后,随机使你手牌中的一张随从牌获得+2/+2。}
\card@def{CFM-305}{风驰电掣}{圣骑士1费法术}{使你手牌中的所有随从牌获得+1/+1。}
\card@def{CFM-800}{战术撤离}{圣骑士1费法术}{奥秘:当一个友方随从死亡时,将其移回你的手牌。}
\card@def{CFM-603}{疯狂药水}{牧师1费法术}{直到回合结束,获得一个攻击力小于或等于2的敌方随从的控制权。}
\card@def{CFM-781}{收集者沙库尔}{潜行者3/2/3随从}{潜行。每当该随从攻击时,随机将一张(你对手职业的)卡牌置入你的手牌。}
\card@def{CFM-313}{先到先得}{萨满祭司1费法术}{发现一张具有过载的牌。 过载: (1)}
\card@def{CFM-696}{衰变}{萨满祭司2费法术}{随机将所有 敌方随从变形成为法力值消耗减少(1)点的随从。}
\card@def{CFM-717}{青玉之爪}{萨满祭司2/2/2武器}{战吼:召唤一个{0}的青玉魔像。 过载:(1)@战吼:召唤一个青玉魔像。 过载:(1)}
\card@def{CFM-900}{无证药剂师}{术士3/5/5恶魔}{在你召唤一个随从后,对你的英雄造成5点伤害。}
\card@def{CFM-699}{海魔钉刺者}{术士4/4/2鱼人}{战吼:在本回合中,你使用的下一张鱼人牌不再消耗法力值,转而消耗生命值。}
\card@def{CFM-094}{邪火药水}{术士6费法术}{对所有角色造成5点伤害。}
\card@def{CFM-751}{渊狱惩击者}{术士7/6/6恶魔}{战吼:对所有其他角色造成3点伤害。}
\card@def{CFM-750}{唤魔者克鲁尔}{术士9/7/9恶魔}{战吼:如果你的牌库里没有相同的牌,则召唤你手牌中的所有 恶魔。}
\card@def{CFM-756}{兽人铸甲师}{战士5/2/7随从}{嘲讽 每当该随从造成伤害时,获得等量的护甲值。}
\card@def{CFM-621}{卡扎库斯}{暗金教4/3/3随从}{战吼:如果你的牌库里没有相同的牌,则为你创建一个自定义 法术。}

% 安戈洛
\card@def{UNG-803}{翡翠掠夺者}{中立1/2/1野兽}{战吼:对每个英雄造成1点伤害。}
\card@def{UNG-809}{火羽精灵}{中立1/1/2元素}{战吼:将一张1/2的元素牌置入你的手牌。}
\card@def{UNG-075}{凶恶的翼龙}{中立3/3/3野兽}{在该随从攻击一方英雄后,获得进化。}
\card@def{UNG-113}{热情的探险家}{中立4/3/4随从}{战吼:抽一张牌,使其法力值消耗变为(5)点。}
\card@def{UNG-843}{沃拉斯}{中立4/3/3随从}{在你对该随从施放一个法术后,召唤一个1/1的植物,并对其施放相同的法术。}
\card@def{UNG-900}{灵魂歌者安布拉}{中立4/3/4随从}{在你召唤一个随从后,触发其亡语。}
\card@def{UNG-840}{“丛林猎人”赫米特}{中立6/6/6随从}{战吼: 摧毁你牌库中所有法力值消耗小于或等于(3)点的卡牌。}
\card@def{UNG-847}{火焰使者}{中立7/6/6元素}{战吼:如果你在上个回合使用过元素牌,则造成5点伤害。}
\card@def{UNG-806}{摩天龙}{中立10/7/14野兽}{}
\card@def{UNG-028}{打开时空之门}{法师1费法术}{任务:施放8个你的套牌之外的 法术。  奖励:时空扭曲。}
\card@def[UNG-028]{UNG-028t}{时空扭曲}{法师5费法术}{获得一个额外回合。}
\card@def{UNG-027}{派烙斯}{法师2/2/2元素}{亡语:将该随从移回你的手牌,并变为法力值消耗为(6)点的6/6随从牌。}
\card@def{UNG-024}{法术共鸣}{法师3费法术}{奥秘:当你的对手施放一个法术时,将该法术的一张复制置入你的手牌,其法力值消耗变为(0)点。}
\card@def{UNG-955}{陨石术}{法师6费法术}{对一个随从造成15点伤害,并对其相邻的随从造成 3点伤害。}
\card@def{UNG-116}{丛林巨兽}{德鲁伊1费法术}{任务:召唤5个攻击力大于或等于5的随从。 奖励:班纳布斯。}
\card@def[UNG-116]{UNG-116t}{“践踏者”班纳布斯}{德鲁伊5/8/8野兽}{战吼:使你牌库中所有随从的法力值消耗减为(0)点。}
\card@def{UNG-920}{湿地女王}{猎人1费法术}{任务:使用七张法力值消耗为(1)的随从牌。 奖励:卡纳莎女王。}
\card@def{UNG-919}{沼泽之王爵德}{猎人7/9/9野兽}{在你的对手使用一张随从牌后,攻击该随从。}
\card@def{UNG-954}{最后的水晶龙}{圣骑士1费法术}{任务:对你的随从施放6个法术。 奖励:嘉沃顿。}
\card@def{UNG-011}{水文学家}{圣骑士2/2/2鱼人}{战吼: 发现一张奥秘牌。}
\card@def{UNG-952}{剑龙骑术}{圣骑士6费法术}{使一个随从获得+2/+6和嘲讽。当该随从死亡时,召唤一只剑龙。}
\card@def{UNG-940}{唤醒造物者}{牧师1费法术}{任务:召唤7个具有亡语的随从。 奖励:希望守护者阿玛拉。}
\card@def{UNG-940t8}{希望守护者阿玛拉}{牧师5/8/8随从}{嘲讽,战吼: 将你英雄的生命值变为40。}
\card@def{UNG-067}{探索地下洞穴}{潜行者1费法术}{任务:使用四张名称相同的随从牌。 奖励:水晶核心。}
\card@def[UNG-067]{UNG-067t1e2}{晶化}{}{5/5。}
\card@def{UNG-856}{幻觉}{潜行者1费法术}{发现一张你对手职业的卡牌。}
\card@def{UNG-065}{“尸魔花”瑟拉金}{潜行者4/5/3随从}{亡语:进入休眠状态。在一回合中使用4张牌可复活该随从。}
\card@def[UNG-065]{UNG-065t}{瑟拉金之种}{潜行者11/0/1随从}{休眠 在一回合中使用4张牌可复活该随从。}
\card@def{UNG-061}{黑曜石碎片}{潜行者4/3/3武器}{每使用一张 另一职业的卡牌,该牌的法力值消耗便减少(1)点。}
\card@def{UNG-067t1}{水晶核心}{潜行者5费法术}{在本局对战的剩余时间内,你的所有随从变为5/5。}
\card@def{UNG-942}{鱼人总动员}{萨满祭司1费法术}{任务:召唤10个鱼人。 奖励:老鲨嘴。}
\card@def{UNG-956}{灵魂回响}{萨满祭司3费法术}{使你的所有随从获得“亡语:将该随从移回你的手牌”。}
\card@def{UNG-025}{火山喷发}{萨满祭司5费法术}{造成15点伤害,随机分配到所有随从身上。 过载:(2)}
\card@def{UNG-829}{拉卡利献祭}{术士1费法术}{任务:弃掉六张牌。 奖励:虚空传送门。}
\card@def{UNG-047}{饥饿的翼手龙}{术士4/4/4野兽}{战吼: 消灭一个友方随从,并连续进化两次。}
\card@def{UNG-832}{血色绽放}{术士4费法术}{在本回合中,你施放的下一个法术不再消耗法力值,转而消耗生命值。}
\card@def{UNG-934}{火羽之心}{战士1费法术}{任务:使用七张具有嘲讽的随从牌。 奖励:萨弗拉斯。}
\card@def{UNG-929}{熔岩之刃}{战士1/1/1武器}{如果这张牌在你的手牌中,每个回合都会变成一张新的武器牌。}

% 冰封王座
\card@def{ICC-023}{雪鳍企鹅}{中立0/1/1野兽}{}
\card@def{ICC-468}{失心农夫}{中立1/1/1随从}{每当该随从进行攻击,对敌方英雄造成2点伤害。}
\card@def{ICC-481p}{灵体转化}{中立2费英雄技能}{英雄技能 将一个友方随从随机变形成为一个法力值消耗增加(1)点的随从。}
\card@def{ICC-828p}{合成僵尸兽}{中立2费英雄技能}{英雄技能 制造一个自定义的僵尸兽。}
\card@def{ICC-829p}{天启四骑士}{中立2费英雄技能}{英雄技能 召唤一个2/2的天启骑士。如果你控制所有四个天启骑士,消灭敌方英雄。}
\card@def{ICC-833h}{冰冷触摸}{中立2费英雄技能}{英雄技能 造成1点伤害。如果该英雄技能消灭了一个随从,则召唤一个水元素。}
\card@def{ICC-700}{开心的食尸鬼}{中立3/3/3随从}{在本回合中,如果你的英雄受到治疗,则 法力值消耗为(0)点。}
\card@def{ICC-852}{塔达拉姆王子}{中立3/3/3随从}{战吼:如果你的牌库里没有法力值消耗为(3)的牌,则该随从变形成为一个随从的3/3的复制。}
\card@def{ICC-901}{达卡莱附魔师}{中立3/1/5随从}{你的回合结束效果会生效两次。}
\card@def{ICC-902}{摧心者}{中立3/2/5随从}{双方英雄技能均无法使用。}
\card@def{ICC-810}{亡斧惩罚者}{中立4/3/3随从}{战吼:随机使你手牌中一个具有吸血的随从获得+2/+2。}
\card@def[ICC-810]{ICC-810e}{嗜血渴望}{}{亡斧惩罚者使其获得+2/+2。}
\card@def{ICC-911}{哀泣女妖}{中立4/5/5随从}{每当你使用一张牌,便移除你的牌库顶的三张牌。}
\card@def{ICC-257}{唤尸者}{中立5/3/3随从}{战吼:使一个友方随从获得“亡语:再次召唤该随从。”}
\card@def{ICC-905}{血虫}{中立5/4/4野兽}{吸血}
\card@def{ICC-701}{游荡恶鬼}{中立6/4/6随从}{战吼:摧毁双方手牌中和牌库中所有法力值消耗为(1)的 法术牌。}
\card@def{ICC-706}{蛛魔拆解者}{中立6/5/5随从}{法术的法力值消耗增加(2)点。}
\card@def{ICC-314}{巫妖王}{中立8/8/8随从}{嘲讽 在你的回合结束时,随机将一张死亡骑士牌置入你的手牌。}
\card@def[ICC-314]{ICC-314t2}{亡者大军}{死亡骑士6费法术}{移除你的牌库顶的五张牌。召唤其中所有被移除的随从。}
\card@def{ICC-068}{寒冰行者}{法师2/1/3元素}{你的英雄技能还会 冻结目标。}
\card@def{ICC-082}{寒冰克隆}{法师3费法术}{奥秘:在你的对手使用一张随从牌后,将两张该随从的复制置入你的手牌。}
\card@def{ICC-823}{模拟幻影}{法师3费法术}{复制你手牌中法力值消耗最低的随从牌。}
\card@def{ICC-833}{冰霜女巫吉安娜}{法师9费英雄}{战吼:召唤一个3/6的水元素。在本局对战中,你的所有元素具有吸血。}
\card@def{ICC-808}{地穴领主}{德鲁伊3/1/6随从}{嘲讽 在你召唤一个随从后,获得+1生命值。}
\card@def{ICC-047}{命运织网蛛}{德鲁伊5/5/3随从}{秘密亡语: 抉择:对所有随从造成3点伤害;或者使所有随从获得+2/+2。}
\card@def[ICC-047]{ICC-047t}{命运织网蛛}{德鲁伊5/5/3随从}{秘密亡语:对所有随从造成3点伤害; 或者使所有随从获得+2/+2。@秘密亡语:使所有随从获得+2/+2。@秘密亡语:对所有随从造成3点伤害。}
\card@def[ICC-047]{ICC-047t2}{命运织网蛛}{德鲁伊5/5/3随从}{亡语:对所有随从造成3点伤害,并使所有随从获得+2/+2。}
\card@def{ICC-054}{传播瘟疫}{德鲁伊6费法术}{召唤一只1/5并具有嘲讽的甲虫。如果你的对手拥有的随从更多,则再次施放该法术。}
\card@def{ICC-052}{装死}{猎人1费法术}{触发一个友方随从的亡语。}
\card@def{ICC-021}{自爆肿胀蝠}{猎人4/2/1野兽}{亡语:对所有敌方随从造成2点伤害。}
\card@def{ICC-204}{普崔塞德教授}{猎人4/5/4随从}{在你使用一个奥秘后,随机将一个猎人的奥秘置入战场。}
\card@def{ICC-039}{黑暗裁决}{圣骑士2费法术}{将一个随从的攻击力和生命值 变为3。}
\card@def{ICC-244}{殊死一搏}{圣骑士2费法术}{使一个随从获得“亡语:回到战场,并具有1点生命值。”}
\card@def{ICC-802}{灵魂鞭笞}{牧师2费法术}{吸血 对所有随从造成 1点伤害。}
\card@def{ICC-830p}{虚空形态}{牧师2费英雄技能}{英雄技能 造成2点伤害。在你使用一张牌后,复原此技能。}
\card@def{ICC-213}{永恒奴役}{牧师4费法术}{发现一个在本局对战中死亡的友方随从,并召唤该随从。}
\card@def{ICC-830}{暗影收割者安度因}{牧师8费英雄}{战吼:消灭所有攻击力大于或等于5的随从。}
\card@def{ICC-827p}{死亡暗影}{潜行者0费英雄技能}{被动英雄技能 在你的回合时,将一张“暗影映像”置入你的手牌。}
\card@def{ICC-233}{末日回旋镖}{潜行者1费法术}{对一个随从投掷你的武器,对该随从造成等同于该武器攻击力的伤害,随后该武器返回你的手牌。}
\card@def{ICC-201}{命运骨骰}{潜行者2费法术}{抽一张牌。如果这张牌具有亡语,则再次施放该法术。}
\card@def{ICC-850}{暗影之刃}{潜行者3/3/2武器}{战吼:在本回合中,你的英雄获得免疫。}
\card@def{ICC-811}{莉莉安·沃斯}{潜行者4/4/5随从}{战吼:随机将你手牌中所有的法术牌替换成(你对手职业的)法术牌。}
\card@def{ICC-827}{虚空之影瓦莉拉}{潜行者9费英雄}{战吼:获得潜行直到你的下个回合。}
\card@def[ICC-827]{ICC-827t}{暗影映像}{潜行者0费法术}{每当你使用一张牌,变形成为该卡牌的复制。}
\card@def{ICC-481}{死亡先知萨尔}{萨满祭司5费英雄}{战吼:随机将你的所有随从变形成为法力值消耗增加(2)点的随从。}
\card@def{ICC-289}{莫拉比}{萨满祭司6/4/4随从}{每当有其他随从被冻结,将一张被冻结随从的复制置入你的 手牌。}
\card@def{ICC-090}{雪怒巨人}{萨满祭司11/8/8元素}{在本局对战中,你每过载一个法力水晶,该牌的法力值消耗便减少(1)点。}
\card@def{ICC-903}{血色狂欢者}{术士1/1/1随从}{战吼:消灭一个友方随从,并获得+2/+2。}
\card@def[ICC-903]{ICC-903t}{嗜血渴望}{}{属性值提高。}
\card@def{ICC-407}{侏儒吸血鬼}{术士2/2/3随从}{战吼:移除你对手的牌库顶的一张牌。}
\card@def{ICC-041}{亵渎}{术士2费法术}{对所有随从造成1点伤害,如果有随从死亡,则再次施放该法术。}
\card@def{ICC-218}{咆哮魔}{术士3/3/6恶魔}{每当该随从受到伤害,随机弃掉 一张牌。}
\card@def{ICC-206}{变节}{术士3费法术}{选择一个友方随从,交给你的 对手。}
\card@def{ICC-450}{死亡幽魂}{战士5/3/3随从}{战吼:每有一个受伤的随从,便获得+1/+1。}
\card@def[ICC-450]{ICC-450e}{嗜血渴望}{}{属性值提高。}
\card@def{ICC-405}{腐面}{战士8/4/6随从}{在该随从受到伤害并没有死亡后,随机召唤一个传说随从。}
\card@def{ICC-314t4}{死亡之握}{死亡骑士2费法术}{从你对手的牌库中偷取一张随从牌,并置入你的手牌。}
\card@def{ICC-314t6}{湮灭}{死亡骑士2费法术}{消灭一个随从。你的英雄受到等同于该随从生命值的 伤害。}
\card@def{ICCA08-020}{霜之哀伤}{死亡骑士7/5/3武器}{战吼:移除你所有 的随从。 亡语:再次召唤之前移除的随从。}

% 狗头人
\card@def{LOOT-111}{机械异种蝎}{中立2/1/2机械}{战吼: 消灭一个攻击力小于或等于1的随从。}
\card@def{LOOT-998k}{黄金狗头人}{中立3/6/6随从}{嘲讽,战吼:将你的手牌替换成传说 随从。}
\card@def{LOOT-526}{黑暗之主}{中立4/20/20随从}{起始休眠状态。 战吼:将三张蜡烛牌洗入对手的牌库。抽到三张蜡烛牌后唤醒该随从。}
\card@def{LOOT-130}{奥术统御者}{中立5/4/4元素}{在本回合中,如果你施放过法力值消耗大于或等于(5)的法术,则这张牌的法力值消耗为(0)点。}
\card@def{LOOT-161}{食肉魔块}{中立5/4/6随从}{战吼: 消灭一个友方随从。 亡语:召唤两个被消灭随从的复制。}
\card@def{LOOT-149}{通道爬行者}{中立7/2/5野兽}{如果这张牌在你的手牌中,每当一个随从死亡,法力值消耗就减少(1)点。}
\card@def{LOOT-414}{资深档案管理员}{中立8/4/7随从}{在你的回合结束时,从你的牌库中施放一张法术牌(目标随机而定)。}
\card@def{LOOT-521}{欧克哈特大师}{中立9/5/5随从}{战吼: 招募攻击力为1,2,3的随从各一个。}
\card@def{LOOTA-822}{燃烧权杖}{中立10费法术}{随机施放炎爆术直到一方英雄 死亡。}
\card@def{LOOT-104}{变形卷轴}{法师0费法术}{如果这张牌在你的手牌中,每个回合都会变成一张随机法师法术牌。}
\card@def{LOOT-103}{小型法术红宝石}{法师2费法术}{随机将一张法师法术牌置入你的手牌。@(使用两张元素牌后升级。)}
\card@def{LOOT-101}{爆炸符文}{法师3费法术}{奥秘:在你的对手使用一张随从牌后,对该随从造成6点伤害,超过其生命值的伤害将由对方英雄 承受。}
\card@def{LOOT-537}{魔网操控者}{法师4/4/5元素}{战吼:如果你的手牌中有你的套牌之外的牌,则这些牌的法力值消耗减少(2)点。}
\card@def{LOOT-106}{惊奇套牌}{法师5费法术}{将五张惊奇卡牌洗入你的牌库。抽到时随机施放一个 法术。}
\card@def[LOOT-106]{LOOT-106t}{惊奇卡牌}{法师5费法术}{抽到时施放 随机施放一个 法术。}
\card@def{LOOT-535}{巨龙召唤者奥兰纳}{法师9/3/3随从}{战吼:在本局对战中,你每施放过一个法力值消耗大于或等于(5)点的法术,便召唤一个5/5的龙。}
\card@def{LOOT-051}{小型法术玉石}{德鲁伊1费法术}{对一个随从造成2点伤害。@(获得3点护甲值后升级。)}
\card@def{LOOT-351}{贪婪的林精}{德鲁伊3/3/1随从}{亡语:获得一个空的法力水晶。}
\card@def{LOOT-329}{伊克斯里德,真菌之王}{德鲁伊5/2/4随从}{在你使用一张随从牌后,召唤一个它的复制。}
\card@def{LOOT-314}{灰熊守护者}{德鲁伊8/3/5野兽}{嘲讽,亡语:招募两个法力值消耗小于或等于(4)点的随从。}
\card@def{LOOT-222}{蜡烛弓}{猎人1/1/3武器}{你的英雄在攻击时具有免疫。}
\card@def{LOOT-079}{游荡怪物}{猎人2费法术}{奥秘: 当一个敌人攻击你的英雄时,随机召唤一个法力值消耗为(3)的随从,并使其成为攻击的目标。}
\card@def{LOOT-080}{小型法术翡翠}{猎人5费法术}{召唤两只3/3的狼。(使用一个奥秘后升级。)}
\card@def{LOOT-091}{小型法术珍珠}{圣骑士2费法术}{召唤一个2/2并具有嘲讽的灵魂。@(恢复3点生命值后升级。)}
\card@def{LOOT-093}{战斗号角}{圣骑士4费法术}{招募三个法力值消耗小于或等于(2)点的随从。}
\card@def{LOOT-500}{瓦兰奈尔}{圣骑士6/4/2武器}{亡语:使你手牌中的一个随从获得+4/+2。当此随从死亡时,重新装备这把武器。}
\card@def{LOOT-216}{莱妮莎·炎伤}{圣骑士7/1/1随从}{战吼:将你在本局对战中对友方随从施放过的所有法术施放在此随从身上。}
\card@def{LOOT-187}{暮光召唤}{牧师3费法术}{召唤两个在本局对战中死亡,并具有亡语的友方随从的1/1复制。}
\card@def{LOOT-538}{坦普卢斯}{牧师7/6/6龙}{战吼:在本回合结束后,你的对手连续行动两个回合。然后你行动两个回合。}
\card@def{LOOT-507}{小型法术钻石}{牧师7费法术}{复活两个不同的友方随从。@(施放四个法术后升级。)}
\card@def{LOOT-542}{弑君}{潜行者1/1/3武器}{始终保留所有额外效果。亡语:将这把武器洗入你的牌库。}
\card@def{LOOT-204}{诈死}{潜行者2费法术}{奥秘:当一个友方随从死亡时,将其移回你的手牌,它的法力值消耗减少(2)点。}
\card@def{LOOT-210}{叛变}{潜行者2费法术}{奥秘:当一个随从攻击你的英雄时,改为该随从攻击与其相邻的一个随从。}
\card@def{LOOT-165}{影舞者索尼娅}{潜行者3/2/2随从}{在一个友方随从死亡后,将它的1/1复制置入你的手牌,其法力值消耗变为(1)点。}
\card@def{LOOT-503}{小型法术黑曜石}{潜行者5费法术}{随机消灭一个敌方随从。@(使用三张亡语牌后升级。)}
\card@def{LOOT-504}{不稳定的异变}{萨满祭司1费法术}{将一个友方随从随机变形成为一个法力值消耗增加(1)点的随从。在本回合可以重复使用。}
\card@def{LOOT-517}{低语元素}{萨满祭司2/1/1元素}{战吼:你在本回合中的下一个战吼将触发两次。}
\card@def{LOOT-373}{治疗之雨}{萨满祭司3费法术}{恢复12点生命值,随机分配到所有友方角色上。}
\card@def{LOOT-064}{小型法术蓝宝石}{萨满祭司7费法术}{选择一个友方随从,召唤一个它的复制。@(过载三个法力水晶后升级。)}
\card@def{LOOT-506}{符文之矛}{萨满祭司8/3/3武器}{在你的英雄攻击后,发现一张法术牌,并向随机目标施放。}
\card@def{LOOT-014}{狗头人图书管理员}{术士1/2/1随从}{战吼: 抽一张牌。对你的英雄造成2点伤害。}
\card@def{LOOT-043}{小型法术紫水晶}{术士4费法术}{吸血 对一个随从造成3点伤害。(受到来自你的卡牌的伤害后升级。)}
\card@def{LOOT-415t6}{吞噬者阿扎里}{术士10/10/10恶魔}{战吼: 摧毁你对手的牌库。}
\card@def{LOOT-041}{狗头人蛮兵}{战士3/4/4随从}{在你的回合开始时,随机攻击一名敌人。}
\card@def{LOOT-203}{小型法术秘银石}{战士7费法术}{召唤一个5/5的秘银魔像。 (装备一把武器后升级。)}

% 女巫森林
\card@def{GILA-BOSS-35e2}{嗜血渴望}{}{吸血。}
\card@def{GILA-BOSS-35p}{嗜血渴望}{中立1费英雄技能}{英雄技能 使一个友方随从 获得吸血。}
\card@def{GIL-620}{人偶大师多里安}{中立5/2/6随从}{每当你抽到一张随从牌,召唤一个它的1/1复制。}
\card@def{GIL-646}{发条机器人}{中立5/4/4机械}{使你的英雄技能的伤害和治疗效果翻倍。}
\card@def{GIL-198}{窃魂者阿扎莉娜}{中立7/3/3随从}{战吼:将你的手牌替换成对手手牌的 复制。}
\card@def{GIL-548}{怨灵之书}{法师2费法术}{抽三张牌。 弃掉抽到的所有法术牌。}
\card@def{GIL-147}{燃烬风暴}{法师3费法术}{造成5点伤害,随机分配到所有敌人身上。}
\card@def{GIL-640}{古董收藏家}{法师5/4/4随从}{每当你抽一张牌时,便获得+1/+1。}
\card@def{GIL-553}{精灵之森}{德鲁伊4费法术}{你每有一张手牌,便召唤一个1/1的小精灵。}
\card@def{GIL-800}{暮陨者艾维娜}{德鲁伊5/3/7随从}{在每个玩家的回合中,使用的第一张牌法力值消耗为(0)点。}
\card@def{GIL-658}{碎枝}{德鲁伊8/8/8随从}{战吼:选择一个友方随从。将它的一张10/10复制置入你的手牌,其法力值消耗为(10)点。}
\card@def{GIL-577}{捕鼠陷阱}{猎人2费法术}{奥秘:当你的对手在一回合中使用三张牌后,召唤一只6/6的老鼠。}
\card@def{GIL-607}{毒药贩子}{猎人4/2/4随从}{每当你使用一张法力值消耗为(1)的随从牌,使其获得剧毒。}
\card@def{GIL-828}{凶猛狂暴}{猎人4费法术}{使一个野兽获得+3/+3。将它的三张复制洗入你的牌库,且这些复制都具有+3/+3。}
\card@def{GIL-903}{隐秘的智慧}{圣骑士1费法术}{奥秘:当你的对手在一回合中使用三张牌后,抽两张牌。}
\card@def{GIL-203}{责难}{圣骑士2费法术}{下个回合敌方法术的法力值消耗增加(5)点。}
\card@def{GIL-142}{变色龙卡米洛斯}{牧师1/1/1野兽}{如果这张牌在你的手牌中,每个回合都会变成你对手手牌中的一张牌。}
\card@def{GIL-813}{鲜活梦魇}{牧师3费法术}{选择一个友方随从,召唤一个该随从的复制,且剩余生命值为1点。}
\card@def{GIL-805}{破棺者}{牧师6/6/5随从}{亡语:从你的手牌中召唤一个亡语随从。}
\card@def{GIL-840}{白衣幽魂}{牧师6/5/5随从}{战吼:对你牌库中的所有随从施放“心灵之火”(使其攻击力等同于生命值)。}
\card@def{GIL-696}{搜索}{潜行者2费法术}{回响 随机将一张(你对手职业的)卡牌置入你的手牌。}
\card@def{GIL-827}{闪狐}{潜行者3/3/3野兽}{战吼:随机将一张(你对手职业的)卡牌置入你的手牌。}
\card@def{GIL-672}{幽灵弯刀}{潜行者4/2/2武器}{吸血 每当你使用一张另一职业的卡牌时,获得+1耐久度。}
\card@def{GIL-598}{苔丝·格雷迈恩}{潜行者8/6/6随从}{战吼:重新使用在本局对战中你所使用过的另一职业的卡牌(目标随机而定)。}
\card@def{GILA-500h4}{苔丝·格雷迈恩}{潜行者8费英雄}{战吼:对所有随从造成8点伤害,并选择一个被动宝藏。}
\card@def{GIL-530}{阴燃电鳗}{萨满祭司2/2/3野兽}{战吼: 如果你的牌库中只有法力值消耗为偶数的牌,造成2点伤害。}
\card@def{GIL-820}{沙德沃克}{萨满祭司9/6/6随从}{战吼:重复在本局对战中你所使用过的所有其他卡牌的战吼效果(目标随机而定)。}
\card@def{GIL-543}{黑暗附体}{术士1费法术}{对一个友方角色造成2点伤害。发现一张恶魔牌。}
\card@def{GIL-665}{虚弱诅咒}{术士2费法术}{回响 直到你的下个回合,使所有敌方随从获得-2攻击力。}
\card@def{GIL-693}{鲜血女巫}{术士4/3/6随从}{在你的回合开始时,对你的英雄造成 1点伤害。}
\card@def{GIL-618}{格林达·鸦羽}{术士6/3/7随从}{你手牌中的所有随从牌获得回响。}
\card@def{GIL-825}{高弗雷勋爵}{术士7/4/4随从}{战吼:对所有其他随从造成2点伤害。如果有随从死亡,则重复此战吼效果。}
\card@def{GIL-113}{狂暴的狼人}{战士3/3/3随从}{突袭}
\card@def{GIL-547}{达利乌斯·克罗雷}{战士5/4/5随从}{突袭 在克罗雷攻击并消灭一个随从后,获得+2/+2。}
\card@def{GIL-655}{腐树巨人}{战士5/2/7随从}{在一个友方随从攻击后,获得+1攻击力。}

% 砰砰计划
\card@def{BOT-020}{滑板机器人}{中立1/1/1机械}{磁力 突袭}
\card@def{BOT-079}{可靠的灯泡}{中立1/1/1机械}{战吼:使一个友方机械获得+1/+1。}
\card@def{BOT-098}{没电的铁皮人}{中立2/2/4机械}{在本回合中,除非你施放过法术,否则无法进行攻击。}
\card@def{BOT-309}{可升级机器人}{中立2/1/5机械}{}
\card@def{BOT-559}{强能雷象}{中立3/3/4野兽}{每当你将一张牌洗入牌库,额外洗入一张相同的牌。}
\card@def{BOT-280}{全息术士}{中立5/3/3随从}{在你的对手使用一张随从牌后,召唤一个它的1/1的复制。}
\card@def{BOT-511}{爆盐投弹手}{中立5/5/5随从}{战吼:将一张“炸弹” 牌洗入你对手的牌库。当抽到“炸弹”时,便会受到5点伤害。}
\card@def[BOT-511]{BOT-511t}{炸弹}{中立5费法术}{抽到时施放 你受到5点伤害。}
\card@def{BOT-563}{战争机兵}{中立5/5/5机械}{磁力}
\card@def{BOT-539}{奥能水母}{中立6/3/4随从}{战吼:发现一张法力值消耗大于或等于(5)点的法术牌。}
\card@def{BOT-424}{机械克苏恩}{中立10/10/10机械}{亡语: 如果你的牌库、手牌和战场没有任何牌,消灭敌方英雄。}
\card@def{BOT-531}{星界密使}{法师2/2/1元素}{战吼:在本回合中,你的下一个法术将获得法术伤害+2。}
\card@def{BOT-254}{鲁莽试验}{法师3费法术}{随机召唤两个法力值消耗为(2)的随从(受法术伤害加成影响)。}
\card@def{BOT-256}{星术师}{法师7/5/5随从}{战吼:随机召唤一个法力值消耗等同于你手牌数量的随从。}
\card@def{BOT-257}{露娜的口袋银河}{法师7费法术}{使你牌库中所有随从牌的法力值消耗变为(1)点。}
\card@def{BOT-054}{生物计划}{德鲁伊1费法术}{每个玩家获得两个法力水晶。}
\card@def{BOT-434}{软泥教授弗洛普}{德鲁伊4/3/4随从}{如果这张牌在你的手牌中,变成你使用的最后一张随从牌的3/4复制。}
\card@def{BOT-402}{奥秘图纸}{猎人1费法术}{发现一张奥秘牌。}
\card@def{BOT-908}{自动防御矩阵}{圣骑士1费法术}{奥秘:当你的随从受到攻击时,使其获得圣盾。}
\card@def{BOT-236}{水晶工匠坎格尔}{圣骑士2/1/2随从}{圣盾,吸血 你的治疗效果翻倍。}
\card@def{BOT-087}{学术剽窃}{潜行者4费法术}{将十张你对手职业的卡牌洗入你的牌库,其法力值消耗为(1)点。}
\card@def{BOT-243}{迈拉·腐泉}{潜行者5/4/2随从}{战吼: 发现一张亡语随从牌,并获得其亡语。}
\card@def{BOT-508}{死金药剂}{潜行者5费法术}{触发一个友方随从的亡语两次。}
\card@def{BOT-543}{欧米茄灵能者}{萨满祭司2/2/3随从}{战吼:如果你有十个法力水晶,在本回合中你的所有法术具有 吸血。}
\card@def{BOT-411}{伊莱克特拉·风潮}{萨满祭司3/3/3元素}{战吼:在本回合中,你的下一个法术将施放两次。}
\card@def{BOT-433}{莫瑞甘博士}{术士6/5/5随从}{亡语: 将该随从与你牌库中的一个随从互换。}
\card@def{BOT-067}{火箭靴}{战士2费法术}{使一个随从获得突袭。抽 一张牌。}
\card@def{BOT-238p}{红色按钮}{战士2费英雄技能}{英雄技能 每回合切换动力 装置!}
\card@def{BOT-406}{超级对撞器}{战士5/1/3武器}{在你攻击一个随从后,迫使其攻击相邻的一个 随从。}

% 奥丹姆
\card@def{ULD-COIN}{幸运币}{中立0费法术}{在本回合中,获得一个 法力水晶。}
\card@def{ULD-723}{鱼人木乃伊}{中立1/1/1鱼人}{复生}
\card@def{ULD-433}{洗劫天空殿}{法师1费法术}{任务:施放10个法术。 奖励:升格卷轴。}
\card@def{ULD-239}{火焰结界}{法师3费法术}{奥秘:在一个随从攻击你的英雄后,对所有敌方随从造成3点伤害。}
\card@def{ULD-216}{尤格-萨隆的谜之匣}{法师10费法术}{随机施放10个法术(目标随机而定)。}
\card@def{ULD-131}{发掘潜力}{德鲁伊1费法术}{任务:在有未使用的法力水晶的情况下结束4个回合。 奖励:奥斯里安之泪。}
\card@def{ULD-134}{蜂群来袭}{德鲁伊3费法术}{选择一个随从。召唤四只1/1的蜜蜂攻击该随从。}
\card@def{ULD-155}{打开宝库}{猎人1费法术}{任务:召唤20个随从。奖励:法老的面盔。}
\card@def{ULD-152}{压感陷阱}{猎人2费法术}{奥秘:在你的对手施放一个法术后,随机消灭一个敌方 随从。}
\card@def{ULD-431}{制作木乃伊}{圣骑士1费法术}{任务:使用5张复生随从牌。奖励:帝王裹布。}
\card@def{ULD-500}{沙漠爵士芬利}{圣骑士2/2/3鱼人}{战吼:如果你的牌库里没有相同的牌,则发现一个升级过的英雄技能。}
\card@def{ULD-724}{激活方尖碑}{牧师1费法术}{任务:恢复15点生命值。奖励:方尖碑之眼。}
\card@def{ULD-269}{卑劣的回收者}{牧师3/3/3随从}{战吼:消灭一个友方随从,然后将其复活,并具有所有生命值。}
\card@def{ULD-326}{劫掠集市}{潜行者1费法术}{任务:将4张其他职业的卡牌置入你的手牌。 奖励:远古刀锋。}
\card@def{ULD-288}{被埋葬的安卡}{潜行者5/5/5随从}{战吼:使你手牌中所有具有亡语的随从牌变为1/1,且法力值消耗为(1)点。}
\card@def{ULD-291}{腐化水源}{萨满祭司1费法术}{任务:使用6张战吼牌。奖励:维尔纳尔之心。}
\card@def{ULD-181}{地震术}{萨满祭司7费法术}{对所有随从造成5点伤害,再对所有随从造成2点伤害。}
\card@def{ULD-140}{最最伟大的考古学}{术士1费法术}{任务:抽20张牌。 奖励:源生魔典。}
\card@def{ULD-167}{染病的兀鹫}{术士4/3/5野兽}{每当你的英雄在自己的回合受到伤害,随机召唤一个法力值消耗为(3)的随从。}
\card@def{ULD-168}{黑暗法老塔卡恒}{术士5/4/4随从}{战吼:在本局对战的剩余时间内,你的跟班变为4/4。}
\card@def{ULDA-BOSS-40p1e1}{命令咆哮}{}{在本回合中,生命值无法被降到1点以下。}
\card@def{ULD-711}{侵入系统}{战士1费法术}{任务:用你的英雄攻击5次。奖励:安拉斐特之核。}
\card@def{ULDA-BOSS-40p1}{命令咆哮}{战士1费英雄技能}{英雄技能 在本回合中,你的随从的生命值无法被降到1点以下。}

% 拉斯塔哈
\card@def{TRL-363}{萨隆铁矿监工}{中立1/2/3随从}{亡语:为你的对手召唤一个0/3并具有嘲讽的自由的矿工。}
\card@def[TRL-363]{TRL-363t}{自由的矿工}{中立1/0/3随从}{嘲讽}
\card@def{TRL-015}{黑心票贩}{中立4/5/3海盗}{超杀:抽两张牌。}
\card@def{TRL-521}{竞技场奴隶主}{中立5/3/3随从}{超杀:召唤另一个竞技场奴隶主。}
\card@def{TRL-532}{莫什奥格播报员}{中立5/6/5随从}{攻击该随从的敌人有50\%几率攻击其他角色。}
\card@def{TRL-535}{钳嘴龟盾卫}{中立5/3/8随从}{每当相邻的随从受到伤害,便会由该随从来承担。}
\card@def{TRL-528}{阵线破坏者}{中立7/5/10随从}{超杀:使该随从的攻击力翻倍。}
\card@def{TRL-537}{送葬者安德提卡}{中立8/8/5随从}{战吼:获得在本局对战中三个死亡的友方随从的亡语。}
\card@def{TRL-541}{夺灵者哈卡}{中立10/9/6随从}{亡语:将一张“堕落之血”分别洗入双方玩家的牌库。}
\card@def[TRL-541]{TRL-541t}{堕落之血}{中立1费法术}{抽到时施放 受到3点伤害。在你抽牌后,将此牌的两张复制洗入你的牌库。}
\card@def{TRL-390}{大胆的吞火者}{法师1/1/1随从}{战吼:在本回合中,你的下一个英雄技能会额外造成2点伤害。}
\card@def{TRL-319}{龙鹰之灵}{法师2/0/3随从}{潜行一回合。你的英雄技能会以选中的随从及其相邻随从作为 目标。}
\card@def{TRL-315}{火焰狂人}{法师3/3/4随从}{每当你的英雄技能消灭一个随从,抽 一张牌。}
\card@def{TRL-400}{裂魂残像}{法师3费法术}{奥秘:当你的随从受到攻击时,召唤一个该随从的复制。}
\card@def{TRL-317}{冲击波}{法师5费法术}{对所有随从造成2点伤害。超杀:随机将一张法师法术牌置入你的手牌。}
\card@def{TRL-241}{贡克,迅猛龙之神}{德鲁伊7/4/9野兽}{在你的英雄攻击并消灭一个随从后,便可再次攻击。}
\card@def{TRL-119}{野兽之心}{猎人1费法术}{使一个友方野兽获得+1/+1,使其随机攻击一个敌方随从。}
\card@def{TRL-065}{祖尔金}{猎人10费英雄}{战吼: 施放你在本局对战中使用过的所有法术(目标随机而定)。}
\card@def{TRL-306}{永恒祭司}{圣骑士2/1/3随从}{亡语:将该随从洗入你的牌库。保留所有额外效果。}
\card@def{TRL-309}{猛虎之灵}{圣骑士4/0/3随从}{潜行一回合。在你施放一个法术后,召唤一只属性值 等于其法力值消耗的老虎。}
\card@def[TRL-309]{TRL-309t}{老虎}{圣骑士1/1/1野兽}{}
\card@def{TRL-258}{群体狂乱}{牧师5费法术}{使每个随从随机攻击其他随从。}
\card@def{TRL-408}{墓园恐魔}{牧师12/7/8随从}{嘲讽 在本局对战中,你每施放一个法术就会使法力值消耗减少(1)点。}
\card@def{TRL-156}{盗取武器}{潜行者2费法术}{发现一张(另一职业的) 武器牌。}
\card@def{TRL-092}{鲨鱼之灵}{潜行者4/0/3随从}{潜行一回合。你的随从的战吼和连击触发两次。}
\card@def{TRL-522}{疾疫使者}{萨满祭司1/2/1随从}{战吼:如果你在本回合施放了两个法术,则造成2点伤害。}
\card@def{TRL-082}{终极巫毒}{萨满祭司2费法术}{使一个友方随从获得“亡语:随机召唤一个法力值消耗增加(1)点的随从。”}
\card@def{TRL-060}{青蛙之灵}{萨满祭司3/0/3随从}{潜行一回合。每当你施放一个法术,从你的牌库中抽取一张法力值消耗增加(1)点的法术牌。}
\card@def{TRL-085}{泽蒂摩}{萨满祭司3/1/3随从}{每当你以一个随从为目标施放法术时,对该随从相邻的随从再次施放。}
\card@def{TRL-058}{亡鬼幻象}{萨满祭司3费法术}{在本回合中,你所施放的下一个法术的法力值消耗减少(3)点。发现一张法术牌。}
\card@def{TRL-345}{卡格瓦,青蛙之神}{萨满祭司6/4/6野兽}{战吼:将你上回合使用的所有法术牌移回你的手牌。}
\card@def{TRL-555}{恶魔之箭}{术士8费法术}{消灭一个随从。你每有一个随从,该牌的法力值消耗便减少(1)点。}
\card@def{TRL-326}{燃棘枪兵}{战士3/3/2随从}{战吼:如果你的手牌中有龙牌,则消灭一个受伤的敌方随从。}
\card@def{TRLA-176}{火焰传令官}{战士4/7/7龙}{当你使用一张龙牌时,使其获得突袭,并抽一张龙牌。}
\card@def{TRL-325}{鞭笞者苏萨斯}{战士6/4/4武器}{超杀:你可以再次攻击。}

% 暗影崛起
\card@def{DAL-615}{女巫跟班}{中立1/1/1随从}{战吼:将一个友方随从变形成为一个法力值消耗增加(1)点的随从。}
\card@def{DALA-720}{沼泽女王的召唤}{中立2费法术}{随机将你的所有随从变形成为传说随从。在本回合可以重复使用。}
\card@def{DAL-558}{大法师瓦格斯}{中立4/2/6随从}{在你的回合结束时,施放你在本回合中施放过的一个法术(目标随机而定)。}
\card@def{DAL-538}{隐秘破坏者}{中立6/5/6随从}{战吼: 随机使你的对手从手牌中施放一个法术(目标随机而定)。}
\card@def{DAL-575}{卡德加}{法师2/2/2随从}{你的召唤随从的卡牌召唤数量翻倍。}
\card@def{DAL-603}{法力飓风}{法师2/2/2元素}{战吼:你在本回合中每施放过一个法术,便随机将一张法师法术牌置入你的手牌。}
\card@def{DAL-177}{咒术师的召唤}{法师3费法术}{双生法术 消灭一个随从。召唤两个法力值消耗相同的随从来替换它。}
\card@def{DAL-177ts}{咒术师的召唤}{法师3费法术}{消灭一个随从。召唤两个法力值消耗相同的随从来 替换它。}
\card@def{DAL-609}{卡雷苟斯}{法师9/4/12龙}{你每个回合使用的第一张法术牌的法力值消耗为(0)点。战吼:发现一张法术牌。}
\card@def{DAL-357}{卢森巴克}{德鲁伊8/4/8随从}{嘲讽,亡语:进入休眠状态。累计恢复5点生命值可唤醒该随从。}
\card@def{DAL-377}{九命兽魂}{猎人3费法术}{发现一个在本局对战中死亡的友方亡语随从,并触发其 亡语。}
\card@def{DAL-731}{决斗}{圣骑士5费法术}{从双方玩家的牌库中各召唤一个随从,并使其互相 攻击!}
\card@def{DAL-030}{阴暗的人影}{牧师2/2/2随从}{战吼: 变形成为一个友方亡语随从的2/2复制。}
\card@def{DAL-729}{拉祖尔女士}{牧师3/3/2随从}{战吼:发现一张你的对手手牌的复制。}
\card@def{DAL-416}{荆棘帮蟊贼}{潜行者4/4/3海盗}{战吼:发现一张另一职业的法术牌。}
\card@def{DAL-071}{突变}{萨满祭司0费法术}{将一个友方随从随机变形成为一个法力值消耗增加(1)点的随从。}
\card@def{DAL-049}{下水道渔人}{萨满祭司2/2/3鱼人}{在你使用一张鱼人牌后,随机将一张鱼人牌置入你的手牌。}
\card@def{DAL-432}{女巫杂酿}{萨满祭司2费法术}{恢复4点生命值。在本回合可以重复使用。}
\card@def{DAL-431}{沼泽女王哈加莎}{萨满祭司7/5/5随从}{战吼: 将一个5/5的恐魔置入你的手牌,并教会它两个萨满祭司法术。}
\card@def[DAL-431]{DAL-431t}{德鲁斯瓦恐魔}{萨满祭司5/5/5随从}{战吼:施放{0}和{1}。}
\card@def{DAL-602}{情势反转}{术士2费法术}{将你的手牌洗入牌库。抽取相同数量的牌。}
\card@def{DAL-185}{阿兰纳丝蛛后}{术士6/4/6恶魔}{嘲讽 当你抽到该牌时,为你的英雄恢复4点生命值。}
\card@def{DAL-173}{至暗时刻}{术士6费法术}{消灭所有友方随从。每消灭一个随从,便随机从你的牌库中召唤一个随从。}

% 巨龙降临
\card@def{DRG-056}{空降歹徒}{中立2/2/2海盗}{在你使用一张海盗牌后,从你的手牌中召唤该随从。}
\card@def{DRG-062}{龙眠净化者}{中立2/3/2随从}{战吼: 将你牌库中的所有中立卡牌随机变形成为你的职业的卡牌。}
\card@def{DRG-075}{深蓝系咒师}{中立5/3/5龙}{战吼:随机将两张你职业的法力值消耗为(1)的法术牌置入你的手牌。}
\card@def{DRG-323}{学习龙语}{法师1费法术}{支线任务: 消耗8点法力值用于法术牌上。奖励:召唤一条6/6的龙。}
\card@def{DRG-324}{元素盟军}{法师1费法术}{支线任务: 连续两个回合使用元素牌。奖励:从你的牌库中抽三张法术牌。}
\card@def{DRG-270}{织法巨龙玛里苟斯}{法师5/2/8龙}{战吼: 如果你的手牌中有龙牌,便发现一张升级过的法师法术牌。}
\card@def{DRG-321}{火球滚滚}{法师5费法术}{对一个随从造成8点伤害,超过其生命值的伤害将由左侧或右侧的随从承担。}
\card@def{DRG-051}{人多势众}{德鲁伊1费法术}{支线任务: 消耗10点法力值用于随从牌上。奖励:从你的牌库中召唤一个随从。}
\card@def{DRG-317}{保护甲板}{德鲁伊1费法术}{支线任务: 用你的英雄攻击两次。奖励:将三张“爪击”法术牌置入你的手牌。}
\card@def{DRG-251}{扫清道路}{猎人1费法术}{支线任务: 召唤三个突袭随从。奖励:召唤一头4/4并具有突袭的狮鹫。}
\card@def{DRG-255}{病毒增援}{猎人1费法术}{支线任务: 使用你的英雄技能三次。奖励:召唤三个2/1的麻风侏儒。}
\card@def{DRG-252}{相位追猎者}{猎人2/2/3野兽}{在你使用你的英雄技能后,从你的牌库中施放一个奥秘。}
\card@def{DRG-008}{正义感召}{圣骑士1费法术}{支线任务: 召唤五个随从。奖励:使你的所有随从获得+1/+1。}
\card@def{DRG-258}{庇护}{圣骑士2费法术}{支线任务: 在一回合内不受伤害。奖励:召唤一个3/6并具有嘲讽的随从。}
\card@def{DRG-232}{光铸狂热者}{圣骑士4/4/2随从}{战吼: 如果你的牌库中没有中立卡牌,便装备一把4/2的真银圣剑。}
\card@def{DRG-090}{永恒巨龙姆诺兹多}{牧师8/8/8龙}{战吼:使用你的对手上个回合使用的所有卡牌。}
\card@def{DRG-035}{血帆飞贼}{潜行者1/1/1海盗}{战吼:将两张1/1的海盗牌置入你的手牌。}
\card@def[DRG-035]{DRG-035t}{空中海盗}{潜行者1/1/1海盗}{}
\card@def{DRG-036}{蜡烛巨龙}{潜行者5/7/5龙}{亡语:将一支蜡烛洗入你的牌库。抽到蜡烛时,重新召唤蜡烛巨龙。}
\card@def[DRG-036]{DRG-036t}{巨龙的蜡烛}{潜行者5费法术}{抽到时施放 召唤蜡烛巨龙。}
\card@def{DRG-037}{菲里克·飞刺}{潜行者6/4/4随从}{战吼:消灭一个随从及其所有的复制(无论它们在哪)。}
\card@def{DRG-096}{班德斯莫什}{萨满祭司5/5/5随从}{如果这张牌在你的手牌中,每个回合都会随机变成一张传说随从牌的5/5的复制。}
\card@def{DRG-204}{黑暗天际}{术士3费法术}{随机对一个随从造成1点伤害。你每有一张手牌,就重复 一次。}
\card@def{DRG-208}{瓦迪瑞斯·邪噬}{术士7/4/4随从}{战吼:将你的手牌上限提高至12张。抽四张牌。}
\card@def{DRG-209}{扭曲巨龙泽拉库}{术士8/4/12龙}{每当你的英雄受到伤害,召唤一条6/6的虚空幼龙。}
\card@def{DRG-024}{空中悍匪}{战士1/1/2海盗}{战吼:随机将一张海盗牌置入你的手牌。}
\card@def{DRG-022}{横冲直撞}{战士3费法术}{迫使一个随从攻击相邻的一个 随从。}
\card@def{DRG-026}{疯狂巨龙死亡之翼}{战士8/12/12龙}{战吼:攻击所有其他随从。}

% 迦拉克隆的觉醒
\card@def{YOD-033}{持枪恶霸}{中立5/5/5随从}{战吼:下个回合敌方战吼牌的法力值消耗增加(5)点。}
\card@def{DRGA-001}{神奇的雷诺}{法师6/4/12随从}{在你使用一张法术牌后,获得法术伤害+1。}
\card@def{YOD-009}{神奇的雷诺}{法师10费英雄}{战吼:使所有随从消失。*咻!*}
\card@def{YOD-015}{黑暗预兆}{牧师3费法术}{发现一张法力值消耗为(2)的随从牌。召唤该随从并使其获得+3生命值。}
\card@def{YOD-025}{扭曲学识}{术士2费法术}{发现两张术士牌。}
\card@def{YOD-027}{混乱凝视者}{术士3/4/3恶魔}{战吼:诅咒你对手的一张可用手牌。对手将有一个回合的机会来使用那张牌!}
\card@def{YOD-022}{冒进的艇长}{战士1/1/3海盗}{在你使用一张随从牌后,对所有随从造成1点伤害。}

% 恶魔猎手新兵
\card@def{BT-801}{眼棱}{恶魔猎手3费法术}{吸血。 对一个随从造成3点伤害。流放:法力值消耗为(1)点。}

% 外域
\card@def{BT-733}{莫尔葛工匠}{中立2/2/4恶魔}{所有随从受到的法术伤害翻倍。}
\card@def{BT-737}{玛维·影歌}{中立4/4/3随从}{战吼: 选择一个随从。使其休眠2回合。}
\card@def{BT-255}{凯尔萨斯·逐日者}{中立6/4/7随从}{在每回合中,你每施放三个法术,第三个法术的法力值消耗为(0)点。}
\card@def{BT-006}{唤醒}{法师1费法术}{随机将法师法术牌置入你的手牌,直到你的手牌数量达到上限。在你的回合结束时,弃掉它们。}
\card@def{BT-129}{萌芽分裂}{德鲁伊4费法术}{召唤一个友方随从的复制。使复制获得嘲讽。}
\card@def{BT-252}{复苏}{牧师1费法术}{恢复3点 生命值。发现一张法术牌。}
\card@def{BT-781}{埃辛诺斯壁垒}{战士3/1/4武器}{每当你的英雄即将受到伤害,改为埃辛诺斯壁垒失去1点耐久度。}
\card@def{BT-491}{幽灵视觉}{恶魔猎手2费法术}{抽一张牌。流放:再抽一张。}
\card@def{BT-514}{献祭光环}{恶魔猎手2费法术}{对所有随从造成1点伤害两次。}
\card@def{BT-187}{凯恩·日怒}{恶魔猎手4/3/5随从}{冲锋 所有友方攻击无视 嘲讽。}
\card@def{BT-601}{古尔丹之颅}{恶魔猎手6费法术}{抽三张牌。流放:这些牌的法力值消耗减少(3)点。}

% 通灵学园
\card@def{SCH-248}{甩笔侏儒}{中立1/1/1随从}{战吼:对一个随从造成1点伤害。法术迸发:将本随从移回你的 手牌。}
\card@def{SCH-713}{异教低阶牧师}{中立2/3/2随从}{战吼:下个回合你的对手法术的法力值消耗增加(1)点。}
\card@def{SCH-224}{校长克尔苏加德}{中立5/4/6随从}{法术迸发:如果法术消灭了任意随从,召唤被消灭的随从。}
\card@def{SCH-230}{黑岩法术抄写员}{中立6/4/9龙}{法术迸发:随机将两张你职业的法术牌置入你的手牌。}
\card@def{SCH-348}{燃烧}{法师3费法术}{对一个随从造成4点伤害,相邻的随从均会受到超过其生命值的伤害。}
\card@def{SCH-614}{林地守护者欧穆}{德鲁伊6/5/4随从}{法术迸发:复原你的法力水晶。}
\card@def{SCH-539}{斯雷特教授}{猎人3/3/4随从}{你的法术具有剧毒。}
\card@def{SCH-159}{裂心者伊露希亚}{牧师3/1/3随从}{战吼:直到回合结束,将你的手牌替换成对手手牌的复制。}
\card@def{SCH-305}{秘密通道}{潜行者1费法术}{将你的手牌替换为你牌库中的4张牌。下回合换回。}
\card@def{SCH-507}{导师火心}{萨满祭司3/3/3随从}{战吼:发现一张法力值消耗大于或等于(1)点的法术牌。如果你在本回合使用这张法术牌,重复此效果。}
\card@def{SCH-271}{岩浆爆裂}{萨满祭司3费法术}{造成2点伤害,召唤相同数量的1/1的元素。}
\card@def[SCH-271]{SCH-271t}{熔岩元素}{萨满祭司1/1/1元素}{}
\card@def{SCH-158}{恶魔研习}{术士1费法术}{发现一张恶魔牌。你的下一张恶魔牌法力值消耗减少(1)点。}
\card@def{SCH-307}{校园精魂}{术士3费法术}{对所有随从造成2点伤害。将两张灵魂残片洗入你的 牌库。}
\card@def[SCH-307]{SCH-307t}{灵魂残片}{术瞎双职业0费法术}{抽到时施放 为你的英雄恢复2点生命值。}
\card@def{SCH-317}{团伙核心}{战士3/4/3随从}{在你使用一张突袭随从牌后,召唤一个剩余生命值为1的复制。}
\card@def{SCH-621}{血骨傀儡}{战士9/9/9随从}{亡语:再次召唤该随从并获得-1/-1。}
\card@def{SCH-253}{仇恨之轮}{恶魔猎手7费法术}{对所有随从造成3点伤害。每消灭一个随从,召唤一个3/3的怨魂。}
\card@def{SCH-350}{魔杖窃贼}{法贼双职业1/1/2随从}{连击:发现一张法师法术牌。}
\card@def{SCH-352}{幻觉药水}{法贼双职业4费法术}{将你的所有随从的1/1的复制置入你的手牌,并使其法力值消耗变为(1)点。}
\card@def{SCH-427}{雷霆绽放}{德萨双职业0费法术}{在本回合中,获得两个法力水晶。过载:(2)}
\card@def{SCH-514}{亡者复生}{牧术双职业0费法术}{对你的英雄造成3点伤害。将两个在本局对战中死亡的友方随从移回你的手牌。}
\card@def{SCH-126}{教导主任加丁}{牧术双职业4/3/6随从}{在你使用一张随从牌后,将其消灭并召唤一个4/4的挂掉的 学生。}
\card@def{SCH-235}{衰变飞弹}{萨法双职业1费法术}{随机向敌方随从发射三枚飞弹,使其变形成为法力值消耗减少(1)点的随从。}
\card@def{SCH-279}{引月长弓}{瞎猎双职业1/1/4武器}{在你的英雄攻击一个随从后,你的所有随从也会攻击该随从。}

% 马戏团
\card@def{DMF-004t1}{神秘魔盒}{中立0费法术}{在本局对战中,你每施放过一个法术,便随机施放一个法术(目标随机而定)。}
\card@def{DMF-004t6}{燃烧权杖}{中立0费法术}{随机施放炎爆术直到一方英雄 死亡。}
\card@def{DMF-124}{恐怖增生体}{中立2/2/2随从}{腐蚀:获得+1/+1。可以被无限腐蚀。}
\card@def{DMF-068}{乐观的食人魔}{中立5/6/7随从}{50\%几率攻击正确的敌人。}
\card@def{DMF-080}{迅蹄珠齿象}{中立5/4/4野兽}{突袭 腐蚀:获得+4/+4。}
\card@def{YOP-035}{月牙}{中立5/6/3野兽}{每次只能受到1点伤害。}
\card@def{DMF-254t3}{克苏恩之眼}{中立5费法术}{克苏恩碎片(@/4) 造成7点伤害,随机分配到所有敌人身上。}
\card@def{DMF-074}{希拉斯·暗月}{中立7/4/4随从}{战吼:选择一个方向,让所有随从转 起来。}
\card@def{DMF-188}{亚煞极,污染之源}{中立10/10/10随从}{战吼:将你在本局对战中使用过的每张已腐蚀牌的复制置入你的手牌。在本回合中,这些已腐蚀牌的法力值消耗为(0)点。}
\card@def{DMF-254}{克苏恩,破碎之劫}{中立10/6/6随从}{对战开始时:破碎成片。战吼:造成30点伤害,随机分配到所有敌人身上。}
\card@def{DMF-105}{套圈圈}{法师4费法术}{发现一张奥秘牌并将其施放。腐蚀:改为发现两张。}
\card@def{DMF-103}{克苏恩面具}{法师7费法术}{造成10点伤害,随机分配到所有敌人身上。}
\card@def{DMF-058}{日蚀}{德鲁伊2费法术}{在本回合中,你施放的下一个法术将施放两次。}
\card@def{DMF-236}{古神在上}{圣骑士1费法术}{奥秘:在你的对手施放一个法术时,使其改为随机施放一个法力值消耗相同的法术。}
\card@def{DMF-056}{戈霍恩,鲜血之神}{牧师8/8/8随从}{战吼:抽两张牌,使其消耗生命值,而非法力值。}
\card@def{YOP-023}{大地崩陷}{萨满祭司2费法术}{对所有敌方随从造成1点伤害。如果你有过载的法力水晶,再次造成1点伤害。}
\card@def{DMF-701}{深水炸弹}{萨满祭司4费法术}{造成4点伤害。腐蚀:再对所有敌方随从造成2点伤害。}
\card@def{DMF-117}{连环灾难}{术士4费法术}{随机消灭一个敌方随从。腐蚀:消灭两个。再次腐蚀:消灭三个。}
\card@def{DMF-118}{提克特斯}{术士6/8/8恶魔}{战吼:移除你的牌库顶的五张牌。腐蚀:改为对手的牌库。}
\card@def{DMF-522}{雷区挑战}{战士2费法术}{造成5点伤害,随机分配到所有随从身上。}
\card@def{DMF-248}{魔钢处决者}{恶魔猎手3/4/3元素}{腐蚀:变为武器牌。}
\card@def{DMF-230}{伊格诺斯}{恶魔猎手4/2/6随从}{吸血 你的吸血会对敌方英雄造成伤害,而非治疗你。}
\card@def{YOP-018}{钥匙守护者艾芙瑞}{法贼双职业5/4/5随从}{战吼:发现一张任意职业的双职业法术牌。法术迸发:获得发现的法术牌的另一张复制。}

% 贫瘠之地
\card@def{WC-030}{吞噬者穆坦努斯}{中立7/4/4鱼人}{战吼:吃掉你对手手牌中的一张随从牌,获得其属性值。}
\card@def{BAR-042}{始生保护者}{中立8/6/6元素}{战吼:抽取你法力值消耗最高的法术牌,随机召唤一个法力值消耗相同的随从。}
\card@def{WC-006}{安娜科德拉}{德鲁伊6/3/7随从}{你的自然法术的法力值消耗减少(2)点。}
\card@def{BAR-037}{战歌驯兽师}{猎人4/3/4随从}{战吼:从你的牌库中发现一张野兽牌。使其所有的复制获得+2/+1(无论它们在哪)。}
\card@def{BAR-032}{穿刺射击}{猎人4费法术}{对一个随从造成6点伤害,超过目标生命值的伤害会命中敌方英雄。}
\card@def{BAR-311}{噬灵疫病}{牧师3费法术}{吸血 造成4点伤害,随机分配到所有敌方随从 身上。}
\card@def{BAR-323}{偷师学艺}{潜行者1费法术}{发现一个英雄技能,并使其法力值消耗变为(0)点。使用两次后换回。}
\card@def{BAR-860}{火焰术士弗洛格尔}{萨满祭司2/2/3鱼人}{在你使用一张鱼人牌后,对所有敌人造成1点伤害。}
\card@def{WC-020}{永恒之火}{萨满祭司2费法术}{随机对一个敌方随从造成3点伤害。如果该随从死亡,则再次施放此法术。过载:(1)}
\card@def{WC-021}{不稳定的暗影震爆}{术士2费法术}{对一个随从造成6点伤害,超过目标生命值的伤害会命中你的英雄。}

% 暴风城
\card@def{DED-525}{哥利亚,斯尼德的杰作}{中立8/8/8机械}{战吼:对敌方随从发射五枚火箭,每枚火箭造成2点伤害。(目标由你选定!)}
\card@def{SW-450}{巫师的计策}{法师1费法术}{任务线:施放火焰,冰霜和奥术法术各一个。奖励:抽一张法术牌。}
\card@def{DED-517}{奥术溢爆}{法师5费法术}{对一个敌方随从造成8点伤害。召唤一滩残渣,属性值等同于超过目标生命值的伤害。}
\card@def{DED-515}{灰贤鹦鹉}{法师6/4/5野兽}{战吼: 重复你施放的上一个法力值消耗大于或等于(5)点的法术。}
\card@def{SW-113}{大魔导师安东尼达斯}{法师8/6/6随从}{战吼:如果你在上三个回合中都施放过火焰法术,随机对敌人施放三个火球术。@(@/3)}
\card@def{SW-428}{游园迷梦}{德鲁伊1费法术}{任务线:使你的英雄获得4点攻击力。奖励:获得5点 护甲值。}
\card@def{SW-447}{沙德拉斯·月树}{德鲁伊8/5/5随从}{战吼:使你接下来抽到的三张法术牌获得抽到时施放效果。}
\card@def{SW-322}{保卫矮人区}{猎人1费法术}{任务线:使用两张法术牌造成伤害。奖励:你的英雄技能能够以随从为目标。}
\card@def{SW-313}{挺身而出}{圣骑士1费法术}{任务线:使用三张不同的法力值消耗为(1)的牌。奖励:装备一把1/4的圣光的正义。}
\card@def{SW-446}{虚触侍从}{牧师1/1/3随从}{双方英雄受到的所有伤害提高一点。}
\card@def{SW-433}{寻求指引}{牧师1费法术}{任务线:使用法力值消耗为(2),(3),(4)的牌各一张。奖励:从你的牌库中发现一张牌。}
\card@def{SW-440}{墓园召唤}{牧师1费法术}{发现一张亡语随从牌。如果你有足够的法力值使用这张随从牌,触发其亡语。}
\card@def{SW-448}{黑暗主教本尼迪塔斯}{牧师5/5/6随从}{对战开始时:如果你的套牌中所有法术牌都是暗影法术牌,则进入暗影形态。}
\card@def{SW-052}{探查内鬼}{潜行者1费法术}{任务线:使用两张军情七处牌。奖励:将一张间谍小工具置入你的手牌。}
\card@def{SW-311}{锁喉}{潜行者2费法术}{对敌方英雄造成2点伤害,将两张流血洗入你的牌库。抽到流血时,再造成2点伤害。}
\card@def[SW-311]{SW-311t}{流血}{潜行者1费法术}{抽到时施放 对敌方英雄造成2点伤害。}
\card@def{SW-031}{号令元素}{萨满祭司1费法术}{任务线:使用三张过载牌。 奖励:解锁你过载的法力水晶。}
\card@def{SW-091}{恶魔之种}{术士1费法术}{任务线:在你的回合中受到8点伤害。奖励:吸血。对敌方英雄造成3点伤害。}
\card@def{SW-091t4}{枯萎化身塔姆辛}{术士5/7/7随从}{战吼:在本局对战的剩余时间内,你在你的回合受到的伤害改为伤害你的对手。}
\card@def{SW-028}{开进码头}{战士1费法术}{任务线:使用三张海盗牌。奖励:抽一张武器牌。}
\card@def{SW-024}{洛萨}{战士7/7/7随从}{在你的回合结束时,随机攻击一个敌方随从。如果目标随从死亡,获得+3/+3。}
\card@def{SW-039}{一决胜负}{恶魔猎手1费法术}{任务线:在一回合中抽四张牌。奖励:使抽到的牌法力值消耗减少(1)点。}
\card@def{SW-044}{杰斯·织暗}{恶魔猎手8/7/5随从}{战吼: 施放你在本局对战中使用过的所有邪能法术(尽可能以敌人为目标)。}

% 奥特兰克
\card@def{AV-222}{话痨奥术师}{中立5/3/4随从}{战吼:对所有其他随从造成1点伤害。如果有随从死亡,则重复此效果。}
\card@def{AV-200}{魔导师晨拥}{法师7费英雄}{战吼:再次施放你在本局对战中施放过的每个法术派系的一个法术。}
\card@def{AV-283}{大法师的符文}{法师9费法术}{对敌人施放总计消耗20点法力值的法师法术。}
\card@def{AV-405}{珍藏私货}{潜行者5费法术}{重新使用五张在本局对战中你所使用过的其他职业的卡牌。}
\card@def{ONY-011}{别站在火里!}{萨满祭司5费法术}{造成10点伤害,随机分配到所有敌方随从身上。过载:(1)}
\card@def{ONY-034}{痛苦诅咒}{术士1费法术}{将三张痛苦洗入你对手的牌库。抽到痛苦时会造成疲劳伤害。}
\card@def[ONY-034]{ONY-034t}{痛苦}{术士1费法术}{抽到时施放 受到@点疲劳伤害。}
\card@def{AV-312}{献祭召唤者}{术士3/3/3随从}{战吼:消灭一个友方随从。从你的牌库中召唤一个法力值消耗增加(1)点的随从。}
\card@def{AV-285}{邪恶入骨}{术士3费法术}{造成5点伤害,随机分配到所有敌方随从身上。在本回合可以重复使用。}
\card@def{AV-657}{被亵渎的墓园}{术士3费法术}{在你的回合结束时,消灭你的攻击力最低的随从,召唤一个4/4的影魔。持续3回合。}
\card@def{AV-313}{可怕的憎恶}{术士5/2/8随从}{战吼:对所有敌方随从造成1点伤害。荣誉消灭:获得目标随从的攻击力。}

% 沉没之城
\card@def{TSC-908}{海中向导芬利爵士}{中立1/1/3鱼人}{战吼:将你的手牌和牌库底的牌交换。}
\card@def{TSC-032}{剑圣奥卡尼}{中立4/2/6随从}{战吼:秘密选择一项,当本随从存活时,反制你对手使用的下一张随从牌或法术牌。}
\card@def{TSC-055}{海床传送口}{法师3费法术}{抽一张机械牌。使你手牌中所有机械牌的法力值消耗减少(1)点。}
\card@def{TSC-023}{倒刺捕网}{猎人1费法术}{对一个敌人造成2点伤害。如果你在此牌在你手中时使用过纳迦牌,则再选择一个目标。}
\card@def{TSC-211}{深海低语}{牧师1费法术}{沉默一个友方随从,然后造成等同于其攻击力的伤害,随机分配到所有敌方随从身上。}
\card@def{TSC-630}{怒脊附魔师}{萨满祭司7/5/4naga}{战吼:施放你手牌中火焰,冰霜和自然法术各一张的复制(目标随机而定)。}
\card@def{TSC-955}{希拉柯丝教徒}{术士3/2/3naga}{战吼:使你的对手获得一张深渊诅咒。}
\card@def[TSC-955]{TSC-955t}{深渊诅咒}{术士2费法术}{在你的回合开始时,受到{0}点伤害。每一次诅咒都会比上一次更严重。(还剩{1}回合。)}

% 酒馆战棋
\card@def{BGS-113}{无面酒客}{中立4/4/4随从}{战吼:选择一个鲍勃的酒馆中的随从,变形成为它的原始版 复制。}
\card@def{BGS-041}{奥术守护者卡雷苟斯}{中立8/4/12龙}{在你使用一张战吼随从牌后,使你的龙获得+1/+1。}
\card@def{BGS-044}{小鬼妈妈}{术士8/6/10恶魔}{每当该随从受到伤害,随机召唤一个恶魔并使其获得嘲讽。}

\newcommand{\card}[2][]{%
    \ifthenelse{\isempty{#1}}{%
        \@ifundefined{card@nametoid@#2}{%
            \immediate\write\missingcard{#2}%
        }{%
            \gls{\@nameuse{card@nametoid@#2}}%
        }%
    }{%
        \ifglsentryexists{#1}{%
            \gls{#1}%
        }{%
            \immediate\write\missingcard{!#1}%
        }
    }%
}

\makeatother

\newcommand{\example}{\par\vspace{-\parskip}\noindent\emph{例如}\quad}
\newcommand{\notice}{\par\vspace{-\parskip}\noindent\emph{注意}\quad}
\newcommand{\exception}{\par\vspace{-\parskip}\noindent\emph{例外}\quad}
\newcommand{\history}{\par\vspace{-\parskip}\noindent\emph{历史}\quad}

\newcommand{\version}[2]{
    \ifthenelse{\isempty{#1}}{
        \ifthenelse{\isempty{#2}}{\emph{未知版本}}{\emph{版本$<$#2}}
    }{
        \ifthenelse{\isempty{#2}}{\emph{版本$\geq$#1}}{\emph{版本#1$\sim$#2}}
    }
}

\newcommand{\cmark}{\ding{51}}%
\newcommand{\xmark}{\ding{55}}%

\title{炉石传说进阶规则集}
\author{}

\begin{document}

\immediate\openout\missingcard=missing-cards.txt

\maketitle
\clearpage

\frontmatter
\tableofcontents
\clearpage

\mainmatter
\hypersetup{linkcolor=blue}

\setcounter{secnumdepth}{-2}
\chapter{前言}

虽然炉石传说被设计得简单且易上手,但随着游戏的不断发展、大量的卡牌和各种各样的全新效果的加入游戏,它不可避免地变得复杂起来。然而,炉石传说官方并没有提供一份可靠的规则,这就使得对游戏本身的了解变得困难起来。本规则集试图提供一种对炉石传说游戏的一个彻底解释,玩家可以通过本规则集了解并预测所有游戏进行中可能遇到的事情以及其后果。但实际上,我们距离达到这个目标仍然有非常遥远的距离。如果你对本规则集有任何意见或建议,请参与它的维护。

需要强调的是,本规则集并不追求与暴雪设计师或程序员保持一致,而可能采取某个更易理解但不同的解释;这主要是由于暴雪缺乏对游戏机制的透露。仍然有一些可靠的内容:设计师访谈、游戏log以及描述清晰的更新说明都为我们确认游戏机制提供了证据。

此外,游戏机制不是一成不变的。前一个版本的某个游戏机制在更改后并不能称为「修复bug」。重大的机制更新将在\nameref{rule-update}一节予以总结。

\chapter{约定}

\begin{itemize}
    \item 在不产生歧义的情况下,「你受到伤害」「你恢复生命值」「你的生命值/攻击力」等描述中的你均指你的英雄。
    \item 使用\version{X}{}\version{}{Y}\version{X}{Y}表示某条目只在特定版本的游戏中生效。由于测试的不完善,X和Y通常不是准确值。如果X或Y是某个扩展包的缩写,则代表它是这个扩展包的某个未知版本。如果X或Y是「早期版本」,则代表它是某个较早但未知的版本。\version{X}{Y}不包括Y这个版本。
    \item 使用\texttt{e.t}表示实体\texttt{e}上的标签\texttt{t}。使用\texttt{e.t := v}表示「将实体\texttt{e}的标签\texttt{t}设置为\texttt{v}。如果\texttt{e}被省略,则表示实体是唯一的或者显然的。此外,还有一些特殊词语表示特殊的实体:
        \begin{center}
            \begin{tabular}{|c|c|}
                \hline
                \texttt{Game} & 游戏实体 \\
                \hline
                \texttt{Player} & 玩家实体 \\
                \hline
                \texttt{Hero} & 英雄实体 \\
                \hline
            \end{tabular}
        \end{center}
\end{itemize}

在例子中,若无说明,遵循如下约定:

\begin{itemize}
    \item 所有随从和武器都是初始身材和费用。
    \item 双方英雄的生命值足够高。
    \item 没有额外的随从、武器、任务和奥秘。
    \item 所有人的手牌与牌库中的卡牌都对例子无影响。
    \item 所有人的手牌与牌库的数量无关紧要。手牌的上限为十张。
    \item 如果没有描述你操控的随从具体信息,则这些随从的效果与攻击力无关紧要,并且生命值足够高。
\end{itemize}

对于描述「某玩家操控A、B、C」而言,若无说明,遵循如下约定:

\begin{itemize}
    \item A、B、C 的入场顺序与描述顺序相同。
    \item A、B、C 的场上位置无关紧要。
    \item 不同玩家之间实体的入场顺序无关紧要。
\end{itemize}


\setcounter{secnumdepth}{1}
\chapter{基本概念}
\label{basic-concept}

\section{实体}
\label{entity}

\term{实体}包括:
\begin{itemize}
    \item 游戏本身。一些和游戏进程相关的扳机是在它上面触发的。
    \item 玩家。即使你替换了英雄,有的游戏数据仍然是保留的(例如你本局游戏中使用英雄技能的次数)。这些数据都保留在对应的玩家实体上。
    \item 英雄。
    \item 随从。
    \item 法术。
    \item 武器。
    \item 英雄技能。
    \item 状态。
\end{itemize}

每个实体都有一个\term{实体ID},它是按照实体的创建顺序依次分配的数字。因此,各个实体的ID各不相同。按照顺序,游戏中首先创建的几个实体依次为:
\begin{itemize}
    \item \texttt{1} 游戏本身
    \item \texttt{2} 主玩家
    \item \texttt{3} 副玩家
    \item \texttt{4-33} 主玩家的套牌
    \item \texttt{34-63} 副玩家的套牌
    \item \texttt{64} 主玩家的英雄
    \item \texttt{65} 主玩家的英雄技能
    \item \texttt{66} 副玩家的英雄
    \item \texttt{67} 副玩家的英雄技能
    \item \texttt{68} \card{幸运币}
\end{itemize}

\notice 所谓的「随从与随从牌、法术与法术牌、武器与武器牌、英雄与英雄牌」仅仅是叫法与视觉效果上的不同,牌在进行区域移动时仍然是同一实体,并不会因为外形的改变而变成不同的实体。

在战场上的英雄和随从合称为\term{角色}。

\section{状态}
\label{enchantment}

当你将鼠标放在一个随从或武器上时,详情大图下方显示的增益列表中的各项就是\term{状态},这些状态\term{结附}于该实体。此外,状态也可以结附于英雄、玩家、手牌或牌库中的牌,但是这些状态通常是不可见的。

状态可以:
\begin{itemize}
    \item 表示实体所受到的buff/debuff。
    \item 包含延时触发的扳机,例如\card{力量的代价}所添加的。
    \item 包含光环。例如\card{冰霜女巫吉安娜}的战吼为你的玩家结附一个状态。
    \item 实现光环。大多数光环表现为将一个状态结附于受该光环所影响的实体上。例如上面的例子中,你玩家具有的状态会为你的所有元素添加「该随从具有吸血」的状态。
    \item 实现冒险模式、乱斗模式、酒馆战棋等模式中的特殊游戏规则。
\end{itemize}

\section{光环}
\label{aura}

当一个实体在场时,对某些实体或某类效果产生持续性的改变,我们称这个实体具有\term{光环}。光环包括:
\begin{itemize}
    \item 改变实体生命值/攻击力的,例如\card{暴风城勇士}、\card{鱼人领军}等。
    \item 改变实体法力值消耗的,例如\card{召唤传送门}、\card{机械跃迁者}等。
    \item 使实体获得某种状态的,例如\card{冰霜女巫吉安娜}的元素吸血光环、\card{凯恩·日怒}等。
    \item 改变效果数值的,例如\nameref{spell-damage}、\card{先知维伦}、\card{水晶工匠坎格尔}等。
    \item 改变效果结算次数的,例如\card{布莱恩·铜须}、\card{瑞文戴尔男爵}、\card{达卡莱附魔师}等。
    \item 改变效果结算规则的,例如\card{奥金尼灵魂祭司}、\card{伊格诺斯}等。
    \item 其他特殊光环,例如\card{天启四骑士}的消灭效果。
\end{itemize}

改变实体生命值/攻击力/法力值消耗的光环本质上也是通过使实体获得一个状态实现的。如暴风城勇士为随从附加一个名为暴风城之力\texttt{CS2\_222o}的状态使随从+1/+1。
\notice 由光环附加的改变实体生命值/攻击力的状态具有低优先级,总是晚于其他状态。因此计算一个实体的身材时,首先计算它具有的状态,然后计算它享有的光环。
\example 你操控一个\card{暴风城勇士}和\card{小精灵}。小精灵当前身材是2/2。你对小精灵使用\card{黑暗裁决}。小精灵首先因黑暗裁决变为3/3,再因暴风城勇士光环变为4/4。
\exception 改变随从默认身材(白字身材)的光环所附加的状态不具有特殊优先级,如\card{水晶核心}和\card{黑暗法老塔卡恒}。
\notice 与之相对的是,由光环附加的改变实体法力状态的状态没有特殊优先级。因此计算一个实体的法力值消耗时,光环和状态通常按添加顺序结算。详见\nameref{cost}。
\example 你依次使用\card{艾维娜}和\card{索瑞森大帝}。回合结束时你手中的\card{摩天龙}被大帝减为0费。如果你持有一个被大帝减为9费的摩天龙并使用艾维娜,则摩天龙变为1费。

战吼、亡语和回合结束扳机在触发前决定要触发的次数,这意味着通过战吼召唤\card{布莱恩·铜须}不能使战吼再次结算。\card{瑞文戴尔男爵}和\card{达卡莱附魔师}也类似。
\notice 如果你通过亡语召唤瑞文戴尔男爵或通过回合结束扳机召唤达卡莱附魔师,你接下来的亡语/回合结束扳机会发动两次。
\example 你操控一个瑞文戴尔男爵并对其使用\card{先祖之魂}和\card{终极巫毒}。对手杀死瑞文戴尔男爵。先祖之魂只会召唤一个瑞文戴尔男爵,但接下来的终极巫毒可以触发两次。
\example 你操控\card{亚煞极}和\card{炎魔之王拉格纳罗斯}。回合结束时亚煞极首先召唤达卡莱附魔师,亚煞极不会再召唤一个随从,但接下来你的炎魔之王可以开出两炮。

光环并非入场即生效、离场即失效。事实上只有每次光环更新时,游戏才会检查是否有新光环入场、是否有旧光环离场。在绝大多数的结算中,这与「入场即生效、离场即失效」的结算方式没有区别。详见\nameref{aura-update}。

\section{区域}
\label{zone}

游戏中每个实体都位于对应的\term{区域}。区域包括:

\begin{itemize}[itemsep=\parsep]
    \item 战场\texttt{PLAY}:游戏、玩家、当前的英雄和英雄技能都在战场上。被使用或召唤的随从或武器、被使用或施放且未结算完毕的法术(奥秘和任务除外)都在战场上。所有的状态都在战场上,无论它结附于哪里的实体。
        \notice 由于状态都在场上,因此状态附带的扳机是场上扳机。
    \item 奥秘区\texttt{SECRET}:使用或施放后的奥秘和任务位于此区域。
        \notice 奥秘区的奥秘和任务均被视为场上扳机。
    \item 手牌\texttt{HAND}:玩家可以从手牌中使用牌。
    \item 牌库\texttt{DECK}:在对战开始时,玩家套牌中的牌都位于牌库中。
    \item 墓地\texttt{GRAVEYARD}:死去的实体、弃掉或烧毁的牌、被反制或结算完的法术、已触发的奥秘或已完成的任务前往此区域。
    \item 除外区\texttt{SETASIDE}:存放一些临时实体和双方被移除的部分实体,如\card{追踪术}和\nameref{discover}呈现的卡牌、替换英雄后的旧\card{加拉克苏斯大王}随从、亡语生效后的\card{玛洛恩}等。
        \notice 大多数亡语的区域移动效果实质上是伪区域移动。伪区域移动会将实体本身移动到除外区。详见\nameref{move}。
        \notice 部分移除牌库卡牌的效果将卡牌移动到除外区,例如\card{“丛林猎人”赫米特}、\card{吞噬者阿扎里}等。将它们与烧毁卡牌的效果(如\card{侏儒吸血鬼}、\card{哀泣女妖}等)区分的一个方式是观察有没有卡牌被撕毁的动画。
        \notice 同理,一些效果移除的手牌直接前往除外区,例如\card{黄金猿}、\card{菲里克·飞刺}等。它们与弃牌效果也可以通过观察动画区分。同理,\card{神奇的雷诺}将所有场上的随从移动到除外区,它也不播放随从的死亡动画。
    \item 失效区\texttt{REMOVEDFROMGAME}:存放已失效的状态。
        \notice 状态失效一般的原因可能是:过时(如\card{嗜血}等)、所属的实体被移除、状态来源的实体被移除、被沉默等。
\end{itemize}

实体还具有\texttt{.ZONE\_POSITION}属性。这个属性称作\term{位置},它表示实体在其区域中的位置。
\begin{itemize}
    \item 手牌中的所有牌都具有非零位置。
    \item 战场上的随从牌具有非零位置,但其它实体的位置为0。
    \item 墓地、除外区和失效区的任何实体的位置都为0。
    \item 尽管牌库是有序的,牌库中的牌的位置都为0。这显然是为了避免玩家从客户端获得牌库中牌的顺序。
    \item 尽管奥秘看起来有顺序,奥秘区中的牌位置为0。
\end{itemize}

\section{事件与扳机}
\label{event-and-trigger}

\term{事件}对应游戏中发生的事情。
\example 「造成6点伤害」、「抽一张牌」都对应事件。前者对应伤害事件,后者对应抽牌事件。
\notice 游戏中发生的事情和事件并不一定紧挨着出现的。例如在一个AOE伤害发生,所有伤害都产生之后,才会结算对于各个实体的伤害事件。

\term{结算}事件指处理该事件所对应的所有扳机。
\notice 在下文中,我们偶尔会使用事件\term{产生}这样的描述。这实际上指的是该事件对应的游戏操作结束。

\term{扳机}是响应特定的事件而触发并结算的效果。它通常描述为「在/每当A时,E。」
\example \card{飞刀杂耍者}具有「在你召唤一个随从后,对一个随机敌方角色造成1点伤害。」这是一个召唤后扳机。
\example \card{北郡牧师}具有「每当一个随从获得治疗时,抽一张牌。」这是一个治疗扳机。
\example \nameref{lifesteal}为「每当此实体造成伤害时,为你的英雄恢复等量生命值。」这是一个伤害扳机。

当游戏结算一个事件时,首先确定所有响应该事件触发的扳机(称为\term{列队}),然后这些扳机按照顺序结算。不满足条件的扳机,即使后来条件改变也不能再触发。
\notice 即使列队时合法的扳机,也有可能无法结算。详见\nameref{trigger-cond}。

如果在事件A的结算中又产生了另一个事件B,则首先结算事件B,等待B的结算完成后才会继续A的结算。这种\term{深度优先}的结算规则称为\term{嵌套结算}。
\example 对手操控\card{狙击}和\card{镜像实体}。你使用\card{古拉巴什狂暴者}。首先狙击触发,对古拉巴什狂暴者造成4点伤害,此伤害立即触发它的扳机,把它变成一个5/4。然后镜像实体触发,对手也获得一个5/4的古拉巴什狂暴者。
\example 你操控\card{镜像实体}、\card{飞刀杂耍者}、\card{卡德加}、\card{公正之剑}和\card{勇敢的记者}。对手操控\card{暴乱狂战士}、\card{苦痛侍僧}、\card{铸甲师}和\card{重型攻城战车}并使用\card{小精灵}。可能发生以下情况:

\begin{itemize}
    \item 作为响应使用后事件的扳机,镜像实体触发并召唤一个小精灵的复制(记为小精灵1)。
    \begin{itemize}
        \item 小精灵1的召唤事件触发飞刀,对对手的苦痛侍僧造成1点伤害。
        \begin{itemize}
            \item 首先暴乱狂战士响应苦痛侍僧的受伤并获得+1攻击力。
            \item 其次苦痛侍僧自身扳机触发抽一张牌。
            \begin{itemize}
                \item 你的勇敢的记者响应此次抽牌,获得+1/+1。
            \end{itemize}
            \item 最后铸甲师响应苦痛侍僧的受伤使对手获得1点护甲。
            \begin{itemize}
                \item 重型攻城战车响应护甲获得事件并+1攻击力。
            \end{itemize}
        \end{itemize}
        \item 小精灵1的召唤事件触发卡德加,召唤另一只小精灵2。
        \begin{itemize}
            \item 小精灵2的召唤触发飞刀,对对手英雄造成1点伤害。
            \item 你的公正之剑响应小精灵2的召唤使其+1/+1。
        \end{itemize}
        \item 小精灵1的召唤事件触发公正之剑,并获得+1/+1。
    \end{itemize}
\end{itemize}


结算某事件触发的所有扳机时,各扳机的先后顺序主要按照入场顺序。
\notice 入场顺序指实体进入战场的顺序。判断两个扳机间的入场顺序,就是判断两个扳机进入战场时间点的先后。对于随从自带的扳机而言,其入场时间点为随从的入场时间点;对于由状态添加的扳机而言,其入场时间点为状态附加的时间点。随从的操控权发生改变不影响其入场时间点。
\example 你使用\card{小鬼召唤师}A和B,然后对A使用\card{力量的代价}。回合结束时,首先A触发效果,然后B触发效果,最后A所带的力代触发效果消灭A。
\example 对手使用\card{法力之潮图腾},然后你使用\card{霍格},再使用\card{精神控制}目标法力之潮图腾。回合结束时,首先法力之潮图腾触发,再霍格触发。
\notice 除了入场顺序之外,还有很多顺序也决定着扳机的结算顺序,例如\nameref{trigger-order-in-different-zone},以及某些效果具有的\nameref{special-priority}等。


群体伤害和群体治疗的伤害/治疗事件通常在全部伤害/治疗产生之后再结算。
\example 你操控七个\card{暴乱狂战士},对手使用\card{烈焰风暴}。首先所有的暴乱狂受到伤害减为0血,然后它们依次响应每个伤害事件触发(共49次)。
\notice \card{陨石术}对三个随从造成的伤害属于群体伤害。
\notice \card{燃烧}对两侧随从造成的伤害也属于群体伤害。
\notice 伤害/治疗值不定的 AOE 通常不适用此条规则,它们以「第一个随从受到伤害 - 对应的伤害事件 - 第二个随从受到伤害 - 对应的伤害事件……」依次交替。这样的例子有\card{闪电风暴}、\card{元素毁灭}、\card{圣光炸弹}、\card{生命之树}和\card{黑铁潜藏者}等等。
\exception 即使闪电风暴已被改动,它造成的伤害仍然不属于群体伤害。
\example 你操控\card{血沼迅猛龙}和\card{暴乱狂战士}。对手使用圣光炸弹。首先迅猛龙受到3点伤害,然后暴乱狂触发变成3攻,然后它受到3点伤害。
\notice 对双方英雄造成的伤害通常不适用此条规则。例如\card{翡翠掠夺者}首先对先入场的英雄造成伤害,再对后入场的英雄造成伤害,中间可以插入\card{以眼还眼}。
\notice 伤害和治疗以外的「复合效果」通常并不适用此规则。例如\card{冰霜新星}和\card{莫拉比}是交替冰冻-触发;一次抽多张牌中间可以插入\card{勇敢的记者}或\card{古董收藏家}等。
\notice \nameref{lifesteal}和\nameref{overkill}并不触发多次;它们只在最后触发一次。

此外,战斗中的两个伤害事件也是在伤害全部产生之后再结算。详见\nameref{battle}。

具有「使用时」「使用后」「召唤时」「召唤后」(如\card{伊利丹·怒风}或\card{飞刀杂耍者})的扳机不能在该随从的对应事件中触发。这称为\term{随从扳机自触发保护}。
\notice 我们也可以解释为这些扳机的条件没有完全写明。如飞刀的扳机应写为「每当你召唤一个其他随从后」。为不过由于这是一个极为普遍的情况,我们将其提出为一条规则。
\example 你使用具有\card{瓦兰奈尔}亡语增益的\card{灵魂歌者安布拉}。安布拉并不会触发自己的亡语为你装备一把瓦兰奈尔。
\exception 你使用具有突袭的\card{团伙核心},团伙核心会触发自己的效果为你召唤一个剩余生命值为1的团伙核心。

扳机不能在自身的结算之中再次触发。部分扳机在这次结算完成之后,将补偿进行等同于跳过次数的结算。详见\nameref{self-triggering-protection}。

\section{序列}
\label{sequence}

炉石传说的回合由若干个\term{序列}构成。序列由使用牌、使用英雄技能及战斗产生。它也会由特定的游戏时刻(回合开始、回合抽牌、回合结束)产生。序列的末尾会进行\term{胜负裁定}。

一些玩家操作也有与之对应的非玩家操作,例如使用随从对应召唤随从,使用法术对应施放法术,战斗对应强迫战斗(由卡牌效果产生的战斗)等。这些操作也会产生类似序列的东西,但它们内部不包含死亡结算与胜负裁定。这种序列类似物称作\term{退化序列}。退化序列将在对应玩家操作的章节中进行描述。
\notice 在规则集的旧版本中,无论是玩家操作还是非玩家操作都产生序列,而玩家操作产生的被强调为\emph{最外层序列}。现版本采取了不同的定义。

序列的内部不包含胜负裁定。
\notice 有的时候,你可能会以为是在序列没有进行完的时候游戏就结束了;实际上是剩余的所有游戏动作被略过。
\example 你的生命值为2。你攻击敌方英雄,触发敌方\card{爆炸陷阱}。因为你濒死,伤害步骤和攻击后步骤被略过,你输掉游戏。
\example 你的生命值为3。你的牌库中只有\card{炸弹}和\card{阿兰纳丝蛛后}。回合开始你抽到炸弹,受到5点伤害。因为你濒死,炸弹不抽牌。然后你输掉游戏,没有机会抽到蛛后把你救回来。请与下面的例子作比较。
\example 你的生命值为3。你的牌库中只有\card{炸弹}和\card{阿兰纳丝蛛后}。你操控两个\card{战利品贮藏者},然后使用\card{扭曲虚空}。第一个战利品的亡语结算,你抽到炸弹,受到5点伤害。因为你濒死,炸弹不抽牌。然后第二个战利品的亡语结算,你抽到蛛后,救回了你自己。

\section{胜负裁定}
\label{winner-check}

胜负裁定是一个特殊的步骤(它只在序列结尾出现)。它负责结束游戏。

\begin{itemize}
    \item 如果一个玩家的\texttt{.PLAYSTATE = LOSING},则游戏以该玩家输,其对手赢结束;
    \item 如果两个玩家的\texttt{.PLAYSTATE = LOSING},则游戏以平局结束,没有赢家。
        \notice 虽然平局时双方玩家的英雄都爆炸,但是这不代表其中的某一方输掉了游戏——你的天梯星数不会减少(但是会中止连胜);你的竞技场负场也不会增加。
    \item 如果没有玩家的\texttt{.PLAYSTATE = LOSING},游戏继续。
\end{itemize}

在客户端UI中,动画上显示一方的英雄爆炸(或因平局双方英雄均爆炸)代表胜负裁定判定游戏结束;英雄发出死亡音效代表死亡检索判定一个英雄死亡。

在游戏的结果产生之后,游戏不会立即结束。游戏会将整个序列继续执行完毕。如果其中需要玩家进行操作,这些操作将被随机选择。
\example 你的对手的生命值为1,你使用\card{符文之矛}攻击敌方英雄。在发现界面时,如果此时你选择投降,那么本局游戏将以平局结束;对手投降则为你获胜。
\example 你的生命值为5,使用了\card{追踪术}看到了两张\card{炸弹}和一张其它牌。此时对手投降,你有可能随机选择到炸弹而导致游戏平局。

\section{阶段}
\label{pahse}

以每一次死亡结算为分界线,序列又可以分成若干\term{阶段}。在\nameref{player-action}章节中详细阐述了每个序列中包含的阶段。阶段的末尾会进行\term{阶段间步骤}。
\notice 按照这个划分标准,一般情况下阶段的内部不会发生死亡结算。
\example 你在满场的情况下使用\card{火焰之地传送门}杀死一个友方随从。事实上,火焰之地传送门「造成伤害」和「召唤随从」的效果同属于法术的结算阶段,被杀死的随从不立即离场,而是等到法术结算阶段结束才离场。因此格子会卡住,火门不能召唤出一个5费随从。

一个序列会包含一系列固定的阶段,称为\term{固有阶段}。如使用法术牌的序列包括:
\begin{itemize}
    \item 使用阶段,使得你的\card{紫罗兰教师}能在法术生效前召唤出1/1的学徒。
    \item 结算阶段,使得法术卡牌描述中的效果能够结算。
    \item 完成阶段,使得你的\card{狂野炎术师}能在法术生效后对所有随从造成过1点伤害。
\end{itemize}

此外,一个序列还会包含若干个死亡阶段,取决于是否有实体在序列中死亡。详见\nameref{death-creation-step}和\nameref{death-phase}。

\section{阶段间步骤}

\term{阶段间步骤}是在阶段结束时依次执行的步骤的统称,按先后顺序依次包括:
\begin{itemize}
    \item \term{任务奖励步骤}:如果在前一个阶段你完成了任务,则在此步骤中获得任务奖励。
    \item \term{死亡检索步骤}:移除你当前濒死的随从。详见\nameref{dying}、\nameref{death-creation-step}和\nameref{death-phase}。
    \item \term{光环更新步骤}:更新你所拥有的光环。详见\nameref{aura-update}。
\end{itemize}

\section{濒死}
\label{dying}

\term{濒死}指一个场上的实体被消灭或其当前生命值/耐久度被降至0或更低。\\
被消灭指一个实体受到消灭效果的作用,其\texttt{.TO\_BE\_DESTROYED := 1}。我们也会用\term{濒死状态}和\term{待摧毁状态}来描述这种情况。

\notice 一个实体濒死并不意味着它会被立即送去墓地。它要等到随后的\term{死亡检索步骤}才会被送入墓地。这意味着,一个濒死的实体可能会在后续结算中被救回来。
\example 你操控先入场的\card{炎魔之王拉格纳罗斯},你的对手操控8/8后入场的\card{格鲁尔}。你的回合结束,螺丝对格鲁尔造成8点伤害,然后格鲁尔把自己变成 9/1救了回来。
\example 你操控两个\card{资深档案管理员}。回合结束,管理员A施放\card{扭曲虚空},然后管理员B施放\card{消失}。所有随从被移回手牌,它们不会在场上死亡也不会在手牌中被弃掉。
\notice 沉默不能清除待摧毁状态。因此,把上面例子中的\card{消失}改为\card{群体驱散}不能救回对手的随从。

\section{死亡检索步骤}
\label{death-creation-step}

这是一个阶段间步骤。它按照以下顺序进行:

\begin{itemize}
    \item 游戏检查所有的英雄、随从、武器,并标记濒死的实体。
    \item \term{死亡时扳机}触发,例如\card{火焰狂人}。
    \item 游戏按照入场顺序将所有的濒死实体移入墓地。如果一个英雄被移除,且它是其操控者的英雄,则这个玩家的\texttt{.PLAYSTATE := LOSING}。
    \item 如果游戏检测到并移除了濒死的实体,则进入死亡阶段结算它们的死亡事件;否则进入序列中设定的下一个阶段。
\end{itemize}

\notice 也就是说,实体移除的顺序并不是它们进入濒死的顺序。
\notice 第一步中被标记的实体必定会被移除。也就是说,即使火焰狂人抽一张牌使得某个濒死的随从不再濒死,它也仍然会被移除。
\notice 在极为特殊的情况下,死亡的英雄可能同时为双方的英雄。此时只有它当前的操控者会输掉游戏。

\card{火焰狂人}可以产生一些令人困惑的结算。这是一个例子:、
\example 你操控火焰狂人和\card{小精灵}。你的牌库里只有一张\card{惊奇卡牌}。你使用\card{火焰冲击}消灭小精灵。接下来发生这些事情:

\begin{itemize}
    \item 死亡检索步骤 I.1:标记小精灵为濒死。
    \item 死亡检索步骤 I.2:火焰狂人抽一张牌。抽到惊奇卡牌。
    \begin{itemize}
        \item 惊奇卡牌的抽牌扳机:施放(无关的)法术,进行一次强制死亡。
        \begin{itemize}
            \item 死亡检索步骤 II.1:标记小精灵为濒死。
            \item 死亡检索步骤 II.2:火焰狂人\emph{再}抽一张牌。
        \end{itemize}
        \item 扳机继续结算。你抽一张牌。
    \end{itemize}
\end{itemize}

\noindent 最后一共抽了两张牌。如果你操控更多狂人,会抽更多的牌(而且由于\nameref{self-triggering-protection},其数目不是一个容易计算的数字)。

\section{死亡阶段}
\label{death-phase}

\term{死亡}指一个实体从场上被送去墓地。如果在前一个阶段及其阶段间步骤中有死亡的实体,那么在阶段间步骤结束之后会插入一个死亡阶段。按照前一个\term{阶段}和其后的\term{死亡检索步骤}中被移除的实体的入场顺序,依次结算它们的死亡事件。
\notice 通常情况下实体只在死亡检索步骤中被移除。但也有少数情况下实体在阶段中直接被移除,详见\nameref{move-to-full-zone}。

对于每一个死亡实体的死亡事件,其自身具有的亡语和其他死亡扳机列队结算。
\example 对手操控依次入场的\card{麻风侏儒}、\card{诅咒教派领袖}、\card{发条侏儒}和\card{诈死}。你使用\card{魔爆术}。首先结算麻风侏儒的死亡事件:其亡语对你的英雄造成2点伤害,诅咒教派领袖抽一张牌,然后诈死将一个麻风侏儒加入对手手牌。然后结算发条侏儒的死亡事件:诅咒教派领袖先抽一张牌,然后其亡语将一张零件牌置入对手手牌。
\notice \card{救赎}和\nameref{reborn}在死亡事件中具有特殊优先级。

死亡阶段结束后也会进行阶段间步骤,这可能引发另一个死亡阶段。如此重复直到没有任何实体死亡,才会进入序列设定的下一个阶段。
\example 对手操控两个\card{自爆绵羊}和\card{萨隆铁矿监工},你使用1/1的\card{机械克苏恩}并\card{刺杀}敌方自爆绵羊A。此时你场上没有其他随从且你的手牌和牌库均为空。情况如下:

\begin{itemize}
    \item 法术结算阶段:刺杀将自爆绵羊A标记为待摧毁。
    \item 死亡检索步骤I:移除自爆绵羊A。
    \item 死亡阶段I:自爆绵羊A亡语结算,对自爆绵羊B、监工和机械克苏恩造成2点伤害。
    \item 死亡检索步骤II:移除自爆绵羊B和机械克苏恩。
    \item 死亡阶段II:自爆绵羊B亡语结算,对监工造成2点伤害。然后机械克苏恩消灭敌方英雄。
    \item 死亡检索步骤III:移除敌方英雄和监工。
    \item 死亡阶段III:监工给你召唤\card{自由的矿工}。由于没有实体死亡,进入序列中的下一个阶段,即完成阶段。
    \item 完成阶段:无事发生,序列结束。
    \item 进行胜负裁定,你赢得游戏。
\end{itemize}
\chapter{玩家操作}
\label{player-action}

\term{玩家操作}是玩家在游戏中可以进行的操作。这包括使用一张牌、使用英雄技能、主动攻击、结束回合和投降。

\section{回合结构}
炉石传说的回合分为六个部分,这在游戏中由\texttt{Game.STEP}记录。以你的一个回合为例:
\begin{enumerate}
    \item \term{准备步骤}\texttt{MAIN\_READY}:所有在场实体的在场回合数\texttt{NUM\_TURNS\_IN\_PLAY}加1。恢复你的法力值。将所有「本回合」相关标签归零,如\card{火山幼龙}采用的「本回合死亡随从数」\texttt{Game.NUM\_MINIONS\_KILLED\_THIS\_TURN}、用于确定连击是否触发,以及\card{艾德温·范克里夫}、\card{捕鼠陷阱}等卡牌的效果的「你本回合使用卡牌数」\texttt{Player.NUM\_\allowbreak{}CARDS\_PLAYED\_THIS\_TURN}等。将所有实体的「已攻击次数」标签归零。你的武器变为出鞘状态;对手武器变为入鞘状态。你的奥秘变为不可触发状态;对手奥秘变为可触发状态。
    \item \term{回合开始阶段}\texttt{MAIN\_START\_TRIGGERS}:结算回合开始扳机,包括\card{腐蚀术}、\card{噩梦}等添加的状态、\card{百变泽鲁斯}等卡牌的变形、\card{争强好胜}和\card{死亡暗影}等。
    \item \term{抽牌阶段}\texttt{MAIN\_START}:你抽一张牌。
    \item \term{主时段}\texttt{MAIN\_ACTION}:你可以进行玩家操作,直到选择结束回合或被效果结束回合为止。
    \item \term{回合结束阶段}\texttt{MAIN\_END}:结算回合结束扳机。
    \item \term{清除步骤}\texttt{MAIN\_CLEANUP}:清除所有随从的召唤失调。移除过期的状态,例如\card{叫嚣的中士}或\card{嗜血}所添加的额外攻击力。
\end{enumerate}

\notice 召唤失调是万智牌术语,指随从进入战场的回合不能攻击。在游戏中通过标签\texttt{.JUST\_PLAYED}来记录。
\notice 回合开始、抽牌和回合结束这三个阶段事实上是只包含一个固有阶段的序列。在这三个阶段及之后的死亡阶段(如果有)结束后,会进行胜负裁定。
\example 回合开始时双方都只有1点生命,你的回合开始时\card{鲜血女巫}触发并杀死了你的英雄。游戏结束,你输掉了比赛。你来不及抽到牌库顶的\card{烈焰巨兽}来达成平局。

在两个回合之间也有一个步骤:

\begin{enumerate}
    \item[0.] \term{回合间步骤}\texttt{MAIN\_NEXT}:你的所有手牌入手回合数\texttt{.NUM\_TURNS\_IN\_HAND}加1。若此时没有\card{时空扭曲}等额外回合效果影响,则「当前回合玩家」\texttt{.CURRENT\_PLAYER}切换为你的对手。「游戏总回合数」\texttt{Game.TURN}加1。若游戏总回合数达到90,则游戏立即以平局结束。这意味着一场游戏的回合数上限为89回合。
\end{enumerate}

在酒馆战棋模式中,招募回合和战斗回合将加在一起计算回合数。也就是说,对局将在玩家进行了45次招募和44次战斗之后结束。

\section{使用随从牌}

使用随从牌的序列如下:
\begin{enumerate}
    \item \term{使用阶段}:
    \begin{enumerate}
        \item 你支付费用。\card{海魔钉刺者}等效果在此生效。
        \item 随从进入战场。其区域和位置被设置为对应的值。\texttt{.EXHAUSTED}和\texttt{.JUST\_\allowbreak{}PLAYED}被设置为1。其他相关标签如「本回合使用卡牌数」等进行变动。
        \item 抉择变形和休眠结算。费用状态移除。
        \item \term{使用时步骤}:使用时扳机,例如\card{伊利丹·怒风}、\card{任务达人}、\card{魔能机甲}、\card{无羁元素}、\card{大胖}、\card{人气选手}等列队结算。
        \item \term{召唤时步骤}:召唤时扳机,例如\card{鱼人招潮者}、\card{夜色镇执法官}、\card{饥饿的秃鹫}等列队结算。
        \item 你获得过载。
    \end{enumerate}

    \item \term{结算阶段}:
    \begin{enumerate}
        \item 确定战吼/连击将要触发的次数,\card{布莱恩·铜须}、\card{鲨鱼之灵}、\card{低语元素}的战吼等效果如果满足条件会将下列步骤重复多次。即使这些随从在战吼/连击过程中离开战场,也不会影响这个过程。
        \item 如果随从效果具有目标,指向扳机\card{诺格弗格市长}结算。
        \item 结算战吼/连击/抉择/磁力。
            \exception 抉择变形类随从的抉择由于已结算过,不再处理。
    \end{enumerate}

    \item \term{完成阶段}:
    \begin{enumerate}
        \item \term{召唤后步骤}:召唤后扳机,例如\card{船载火炮}、 \card{飞刀杂耍者}、 \card{公正之剑}、\card{腐化灰熊}、\card{夜色镇议员}等列队结算。
        \item 如果这张牌在序列开始时具有回响,结算回响。
        \item \term{使用后步骤}:使用后扳机,例如腐蚀、\card{镜像实体}、 \card{忏悔}、\card{狙击}、\card{审判}、 \card{变形药水}、\card{顽石元素}、\card{海盗帕奇斯}、\card{低语元素}状态移除、\card{虚空形态}的刷新效果等列队结算。
    \end{enumerate}

\end{enumerate}

如果你仅仅是召唤一个随从(例如\card{战斗号角}、\card{砰砰博士}的战吼、\card{腐面}的受伤扳机等),会进行如下较为简单的流程:
\begin{enumerate}
    \item 移动或创建该实体。休眠结算。
    \item \term{召唤时步骤}
    \item \term{召唤后步骤}
\end{enumerate}

\subsection{随从数量超出上限}

「随从进入战场」步骤不检查随从数量上限。由于支付费用在随从入场之前发生,可以利用这个效果超出场上的随从上限。

\example  你操控四个随从和\card{染病的兀鹫}。你使用\card{海魔钉刺者}和\card{老瞎眼}。首先你受到4点伤害,兀鹫扳机触发召唤一个随从将场上填满。然后老瞎眼入场。此时你操控八个随从。

\subsection{随从在使用过程中离场}

如果你使用的随从牌,或是战吼/连击/抉择指定的目标在结算阶段前离场,其战吼/连击/抉择不会取消,而是以随从的新位置和新数据结算。需要注意,随从进入墓地时会清除伤害和所有状态。
\example 你使用\card{沃金}指定的目标在战吼前被移除,仍然会发生血量交换——因为墓地中的随从数据被重置,你可以获得一个等于目标初始血量的沃金,而不是0血。与此类似,如果沃金在战吼前被移除,它战吼指定的目标会变为2血而不是0血。
\example 你使用的\card{阿古斯防御者}在战吼前被移除,他不会给任何随从嘲讽。详见\nameref{zone}。

如果你使用的随从牌在完成阶段前离场(包括死亡、回手、休眠、磁力、\card{加拉克苏斯大王}的替换英雄等),完成阶段内的多数扳机都不会触发。也就是说,这些扳机都要求随从在场。
\example 你操控\card{布莱恩·铜须}并使用\card{负伤剑圣},剑圣的战吼杀死自己。随后的完成阶段,你的\card{飞刀杂耍者}不会触发,对手的\card{镜像实体}也不会触发。
\exception 部分与随从相关的使用后扳机仍对离场随从生效。这包括\card{全息术士}和\card{伊克斯里德,真菌之王}。
\exception 休眠随从可以触发\card{飞刀杂耍者}、\card{下水道渔人}等卡,但无法触发\card{火焰术士弗洛格尔}。
\exception 部分完成阶段扳机只与「你使用牌这个动作」或「你使用的那张牌」相关,而非「你使用的随从」。它们在随从离场的情况下仍然触发。这包括:\card{军备宝箱}、\card{捕鼠陷阱}和\card{隐秘的智慧}、\card{寒冰克隆}、\card{湿地女王}、\card{探索地下洞穴}、\card{虚空形态}的刷新效果和\card{格林达·鸦羽}。

如果你使用的随从在完成阶段前离场而又进场,完成阶段的扳机可以触发。
\example 你操控\card{飞刀杂耍者}并使用\card{恐惧地狱火}战吼杀死\card{阿努巴尔伏击者}和\card{空灵召唤者}。伏击者将地狱火弹回手牌,空灵再将地狱火召唤出来,飞刀因此触发一次。完成阶段由于地狱火在场,飞刀再触发一次。
\example 你使用\card{唤魔者克鲁尔}。\card{伊利丹·怒风}、\card{飞刀杂耍者}、\card{阿努巴尔伏击者}将克鲁尔在战吼前弹回手牌。克鲁尔战吼再将自己召唤出来,飞刀因此触发一次。完成阶段由于克鲁尔在场,飞刀再触发一次。

\subsection{随从在使用过程中被变形}

如果随从在使用过程中被变形,所有扳机仍然正常触发。

召唤「你使用的随从的复制」的牌,召唤的是那个变形后随从的复制。\card{软泥教授弗洛普}给的也是变形后的牌。
\example 你使用\card{无面操纵者}指定一个随从为目标。敌方\card{镜像实体}、\card{全息术士}和你的\card{伊克斯里德,真菌之王}召唤的都是变形之后的随从,而不是无面本体。
\example 你使用\card{血沼迅猛龙}。敌方\card{变形药水}和镜像实体依次触发,对手得到一个绵羊。
\example 你操控后入场的真菌之王,对手操控先入场的变性药水。你使用血沼迅猛龙。首先敌方变形药水触发,然后你的真菌之王触发,你场上现在有两个绵羊。如果你的真菌之王先入场,那么它先触发,最后你场上有一个绵羊和一个迅猛龙。

\card{湿地女王}的触发条件是:你使用这个随从时该随从为1费,且该随从没有经过变形。
\example 你使用1费的\card{无面操纵者}复制\card{石牙野猪}。湿地女王不触发。
\example 对手先使用\card{变形药水},然后你使用湿地女王和石牙野猪。敌方\card{变形药水}触发,湿地女王不触发。相反,如果你先使用湿地女王,它会在变形药水之前触发。
\notice 即使变形是将该随从变成相同的随从,湿地女王仍不能触发。
\example 将上个例子的石牙野猪换成\card[CS2_tk1]{绵羊}。湿地女王仍然不触发。

\card{探索地下洞穴}的计数所考虑的是变形之前的那个随从。
\example 你使用探索地下洞穴,然后使用\card{无面操纵者}复制\card{血沼迅猛龙},最后使用无面操纵者复制\card{淡水鳄}。探索地下洞穴的进度为2,且记录的是无面操纵者。

\nameref{echo}在\card{变形药水}之前触发,给你的是变形之前的那个随从。

\card{顽石元素}是否触发根据变形前的随从而定。\card{海盗帕奇斯}、\card{空降歹徒}、\card{小型法术红宝石}和\card{奥术守护者卡雷苟斯}则需要变形前后(预检测时与结算时)都满足条件。
\example 你操控一个\card{顽石元素}。你使用\card{无面操纵者}复制非战吼随从。顽石元素触发。
\example 你牌库中有一张\card{海盗帕奇斯}。你使用\card{无面操纵者}复制海盗。帕奇斯不触发。
\example 你牌库中有一张\card{海盗帕奇斯}。你使用\card{南海船工},然后触发敌方\card{变形药水}。帕奇斯不触发。
\example 你操控一个\card{奥术守护者卡雷苟斯}。你使用\card{无面酒客}复制非战吼随从。奥术守护者卡雷苟斯不触发。但如果复制战吼随从。奥术守护者卡雷苟斯触发。

\subsection{随从在使用过程中转移控制权}

如果你使用的随从牌在结算阶段前被对手获得了控制权,其战吼/连击/抉择依然正常结算。
\example 你使用\card{夜刃刺客}但结算阶段前它被敌方\card{希尔瓦娜斯·风行者}的亡语夺取了控制权。结算阶段,夜刃刺客的控制权为对手所有,其战吼对你的英雄造成3点伤害。

如果你使用的随从牌在结算阶段中被对手获得了控制权,其战吼/连击/抉择依然正常结算。
\notice 这一般是通过分为多个步骤执行的战吼,或\card{布莱恩·铜须}等效果来实现。
\example 你操控布莱恩·铜须并使用\card{希拉斯·暗月},选择将暗月交给对手。第二次战吼结算时暗月由对手操控,所以随机选择一个方向。
\example 你使用\card{寒光智者}。其战吼首先为它的控制者,即你抽两张牌。你抽到\card{惊奇卡牌}释放\card{变节}将寒光智者交给了对手。接下来,其战吼为它控制者的对手,即你抽两张牌。

如果你使用的随从牌在完成阶段前被对手获得了控制权,完成阶段内的绝大多数扳机不会触发。
\exception \card{捕鼠陷阱}和\card{隐秘的智慧}依然能够触发。上文中提到的其他能在随从离场时触发的扳机均不能在随从控制权转移的情况下触发。

\section{使用法术牌}

\notice 在游戏中存在着两种不同的\term{施放}。\card{狂野炎术师}的「施放」指「从手牌中使用」;而\card{惊奇套牌}的「施放」指「执行法术效果」。在本规则集中为避免混淆,将「从手牌中使用」称为「使用」,而「执行法术效果」称为「施放」。
\exception \nameref{cast-when-drawn}是一个抽牌扳机。它并不是施放或使用这个法术。
\exception \card{亵渎}、\card{传播瘟疫}、\card{命运骨骰}和\card{永恒之火}等牌中的「施放」实为「重复法术效果」(包括过载)。
\exception \card{白衣幽魂}并不是真的施放了很多\card{心灵之火}。
\exception \card{相位追猎者}和\card{套圈圈}等施放奥秘的牌只是将奥秘置入战场(它并不会消耗风潮的光环)。

使用法术牌的序列如下:
\begin{enumerate}
    \item \term{使用阶段}:
    \begin{enumerate}
        \item 你支付费用。
        \item 如果对手操控\card{法术反制},则阻止这个法术生效。序列中的绝大多数步骤都会被取消,详见\nameref{counter}。
        \item 如果对手操控\card{古神在上},则将此法术替换为另一个费用相同的法术。序列中的绝大多数步骤都会为这个新法术而结算。
        \item 费用状态移除。已测试的包括\card{血色绽放}、\card{古加尔}、\card{肯瑞托法师}、\card{暗金教侍从}、\card{墨水大师索莉娅}、\card{伺机待发}和\card{亡鬼幻象}。此外,\card{暮陨者艾维娜}的光环也在此切换。
        \item 如果这张法术是奥秘或任务,它进入奥秘区;否则它进入战场。其他相关标签如「本回合使用卡牌数」等进行变动。
        \item \term{使用时步骤}:其它使用时扳机,例如\card{伊利丹·怒风}、\card{任务达人}、\card{魔能机甲}、\card{无羁元素}、\card{紫罗兰教师}、\card{法力浮龙}、\card{青蛙之灵}等列队结算。
        \item 你获得过载和双生法术牌。
    \end{enumerate}

    \item \term{结算阶段:}
    \begin{enumerate}
        \item 确定法术执行的次数,\card{伊莱克特拉·风潮}的战吼等效果如果满足条件会将法术描述重复两次,且你会再次获得过载。即使风潮的效果在法术执行过程中失去,也不会影响这个过程。
        \item 如果法术具有目标,指向扳机\card{诺格弗格市长}和\card{扰咒术}列队结算。
        \item 结算法术描述。如果是一个奥秘或任务,没有事情发生。
        \item 如果该法术不是奥秘或任务,它进入墓地。
    \end{enumerate}

    \item \term{完成阶段}:
    \begin{enumerate}
        \item 如果这张牌在序列开始时具有回响,结算回响。
        \item \term{使用后步骤}:使用后扳机,例如\card{法术共鸣}、\card{狂野炎术师}、 \card{火妖}、\card{西风灯神}、\card{普崔塞德教授}、\card{虚空形态}的刷新效果、\card{星界密使}状态移除、\card{伊莱克特拉·风潮}状态移除等列队结算。
    \end{enumerate}
\end{enumerate}

如果你仅仅是施放一个法术(例如\card{尤格-萨隆}、\card{惊奇卡牌}、\card{资深档案管理员}等),会进行如下较为简单的流程:
\begin{enumerate}
    \item 如果该法术是奥秘或任务,它进入奥秘区(无效的奥秘会被展示并送入墓地);否则它进入战场。
    \item 你获得过载和双生法术牌。
    \item 结算法术描述。施放的法术可以享受法术伤害加成。
    \item 如果使用的法术不是奥秘或任务,它进入墓地。
    \item \term{使用后步骤}。这个使用后步骤包括\card{星界密使}与\card{伊莱克特拉·风潮}等卡的状态移除。
\end{enumerate}

\card{西风灯神}和\card{沃拉斯}的效果并不是施放法术。它们的流程仅包括:
\begin{enumerate}
    \item 你获得过载和双生法术牌。
    \item 结算法术描述。西风灯神和沃拉斯的效果可以享受法术伤害加成。
\end{enumerate}


\subsection{目标在使用过程中离场}

与战吼类似,如果你法术指定的目标在结算前离场,法术不会取消,而是以目标的新位置和新数据结算。
\example 你对敌方\card{暮光幼龙}使用\card{死亡缠绕},在死缠结算前暮光龙被弹回手牌。接下来死缠结算时,首先将手牌中的4/1暮光龙打至4/0,然后判定其受致命伤并抽一张牌。注意在手牌中受致命伤的随从并不会真正死去,这是因为死亡检索只会考虑在场的角色。

\subsection{特殊的法术互动}

「某法术再结算一次」指在通常的一次结算之后,你先获得这个法术的过载和双生法术牌,再结算一次它的描述。
\example 你使用\card{伊莱克特拉·风潮}再使用\card{闪电箭}。使用阶段你获得过载1,然后在描述阶段你首先对目标造成3点伤害,再获得过载1,再对该目标造成3点伤害。
\notice \card{星界密使}对整个法术的多次结算都有效,然后在法术结算完成后移除法强状态。

\card{伊莱克特拉·风潮}和\card{星界密使}和法术之间有与众不同的互动:它们不仅能与使用的法术产生互动,而且也能与施放的法术产生互动。施放法术的情况下,它们添加的状态在对应的使用后时机移除,也就是在法术效果结算完之后。
\example \version{13.0}{}你使用风潮,然后装备\card{符文之矛}并攻击。符文之矛发现的法术会结算两次。在此之后你使用\card{闪电箭}。闪电箭不会结算两次。
\example \version{12.2}{}你使用星界密使,然后使用\card{尤格-萨隆的仆从}。副导演施放的法术会获得法术伤害 +2。在这之后你使用\card{寒冰箭}。寒冰箭不会获得法术伤害 +2。
\notice 在之前的版本里,这些战吼的状态只会被使用的法术所消耗。也就是说,如果你使用星界密使然后使用\card{尤格-萨隆},导演施放的所有法术都会受到法强加成。

\card{泽蒂摩}修改结算阶段的结算流程。在通常的一次结算之后,该法术将以相邻的随从为目标再结算一次法术效果,先左再右。
\notice 泽蒂摩和\card{伊莱克特拉·风潮}的互动为:将「结算一次,然后对相邻的随从再结算一次」重复两次。
\notice 泽蒂摩的额外结算也能享受\card{星界密使}状态的临时法强。
\notice 泽蒂摩的效果看似是一个扳机(有闪电符号),但是实质上它是一个光环。也就是说,如果第一次法术结算将泽蒂摩沉默或变形,它仍然会再执行两次法术效果;不过这个时候闪电符号就不会闪亮了。
\example 你从左到右操控\card{血沼迅猛龙}、泽蒂摩和\card{淡水鳄},然后对泽蒂摩施放\card{妖术}。虽然泽蒂摩首先被变形,但是迅猛龙和淡水鳄随后也被变形。
\notice 「以相邻的随从为目标再结算一次法术」的效果不受\card{诺格弗格市长}影响。
\notice 泽蒂摩对施放的法术不会生效。

\section{使用武器牌}

使用武器牌的序列如下:
\begin{enumerate}
    \item \term{使用阶段}:
    \begin{enumerate}
        \item 你支付费用。武器进入战场。费用状态移除。
            \notice 此时武器已经可以触发其扳机,例如\card{公正之剑}和\card{诅咒之刃}的扳机。
        \item 其他相关标签如「本回合使用卡牌数」等进行变动。
        \item \term{使用时步骤}:使用时扳机,例如\card{伊利丹·怒风}、\card{任务达人}、\card{魔能机甲}、\card{人气选手}等列队结算。
        \item 你获得过载。
    \end{enumerate}

    \item \term{结算阶段}:
    \begin{enumerate}
        \item 确定战吼将要触发的次数,\card{布莱恩·铜须}、\card{低语元素}的战吼等效果如果满足条件会将下两个步骤重复多次。即使这些随从在战吼过程中离开战场,也不会影响这个过程。
        \item 如果武器效果具有目标,指向扳机\card{诺格弗格市长}结算。
        \item 结算战吼/连击。
        \item 消灭你的旧武器。「你装备的武器」\texttt{Player.334}被设置为新武器。\card{锈水海盗}和\card{小型法术秘银石}在此时结算。
    \end{enumerate}

    \item \term{完成阶段}:
    \begin{enumerate}
        \item \term{使用后步骤}:使用后扳机,例如\card{瑟拉金之种}、\card{捕鼠陷阱}、\card{虚空形态}的刷新效果等列队结算。
    \end{enumerate}
\end{enumerate}
​
如果你仅仅是装备一把武器(例如\card{升级!}、\card{恩佐斯的副官}、\card{匕首精通}等),会进行如下较为简单的流程:
\begin{enumerate}
    \item 消灭你的旧武器。
    \item 新武器进入战场。\card{锈水海盗}和\card{小型法术秘银石}在此时结算。
    \item 「你装备的武器」被设置为新武器。
\end{enumerate}
​
\subsection{同时持有两把武器}

当你使用武器牌时,在结算阶段替换前你仍装备着旧武器。此时你的新武器和旧武器都在场,它们可以互相影响,或触发它们的扳机。
\example 你在持有\card{圣光的正义}时使用\card{真银圣剑},使用阶段通过\card{伊利丹·怒风}等扳机杀死你的\card{提里奥·弗丁}。\card{灰烬使者}替换圣光的正义。在接下来的结算阶段,真银圣剑替换灰烬使者。
\example 你在持有\card{弑君}使用另一把弑君,使用阶段通过伊利丹等扳机杀死你的\card{南海畸变船长}。其亡语会为旧弑君增加2点攻击。
\example 你在装备\card{公正之剑}时使用另一把公正之剑。使用阶段\card{伊利丹·怒风}触发,两把武器均对召唤的\card[EX1_614t]{埃辛诺斯之焰}生效,使其变为4/3。结算阶段你当前武器变为新公正之剑,此时它仅剩4点耐久。
\example 你在装备公正之剑时使用\card{青玉之爪}。结算阶段青玉之爪发动战吼召唤1/1\card[CFM_712_t01]{青玉魔像},公正之剑触发使其变为2/2。
\example 你在装备\card{诅咒之刃}时使用连击生效的\card{毁灭之刃}并指向你的英雄。结算阶段你的英雄会受到4点伤害。
​
\section{使用英雄牌}

使用英雄牌的序列如下:

\begin{enumerate}
    \item \term{使用阶段}:
    \begin{enumerate}
        \item 你支付费用。英雄牌进入战场。费用状态移除。
        \item 其他相关标签如「本回合使用卡牌数」等进行变动。
        \item \term{使用时步骤}:使用时扳机,例如\card{伊利丹·怒风}、\card{任务达人}、\card{魔能机甲}、\card{人气选手}等列队结算。
        \item 新英雄的最大生命值、当前生命值和护甲被设定为和旧英雄相同,并获得英雄牌上标注的额外护甲。
        \item 移除你的旧英雄技能和旧英雄。将「你的英雄」 \texttt{Player.HERO\_ENTITY} 设置为新英雄。
    \end{enumerate}

    \item \term{结算阶段}:
    \begin{itemize}
        \item 你获得新的英雄技能。
        \item 首先确定战吼将要触发的次数,\card{布莱恩·铜须}、\card{低语元素}的战吼等效果如果满足条件会将下列步骤重复多次。即使这些随从在战吼过程中离开战场,也不会影响这个过程。然后结算战吼/抉择。
    \end{itemize}
    \notice 使用英雄牌的这两个步骤的先后视具体英雄牌有所不同。

    \item \term{完成阶段}:
    \begin{enumerate}
        \item \term{使用后步骤}:使用后扳机,例如\card{瑟拉金之种}、\card{捕鼠陷阱}等列队结算。
    \end{enumerate}
\end{enumerate}

每当你替换英雄时(无论是使用英雄牌或者是其它方式),以下事情发生:
\begin{itemize}
    \item 你不再被\nameref{freeze}。
    \item 你的「本回合已经攻击次数」保留。
    \item 你替换英雄技能。(使用英雄牌时的替换技能已经在描述阶段中描述了)
    \item 你的职业变为新英雄的职业。你的种族变为对应种族。
\end{itemize}

如果你替换成的是\card[EX1_23h]{加拉克苏斯大王}或\card[BRM_027h]{炎魔之王拉格纳罗斯}时,还会额外发生:

\begin{itemize}
    \item 移除你英雄具有的所有状态,例如\card{暗影之刃}或\card{寒冰屏障}添加的。
    \item 你的英雄的护甲与受到的伤害归零;你的最大生命值不变。这与使用英雄牌时不同。
\end{itemize}

\notice \card[BRM_027h]{炎魔之王拉格纳罗斯}\emph{不具有}元素种族。

\example 你操控\card{伊利丹·怒风}和\card{无证药剂师}且具有5点生命。你使用一张英雄牌。首先伊利丹触发,导致无证药剂师将你的生命值降为0,然后你才获得护甲值。因此你的英雄被判定为死亡,你输掉了这盘游戏。

\example 你使用\card{暗影收割者安度因}并通过\card{伊利丹·怒风}等扳机杀死一个友方\card{管理者埃克索图斯},使你的英雄在结算阶段前变为\card[BRM_027h]{炎魔之王拉格纳罗斯}。结算阶段,你的英雄技能\card[BRM_027p]{死吧,虫子!}被\card{虚空形态}替代,然后发动暗影收割者安度因的战吼。而你的英雄仍是炎魔之王拉格纳罗斯。

\subsection{异常地替换英雄}
有几种方式可以「异常地」替换英雄。替换后的英雄是初始血量及上限、初始护甲且没有英雄技能的。这说明,替换英雄这个操作本身仅仅包括新英雄入场、移除旧英雄和英雄技能、修改「你的英雄」三部分。其余部分都是特定的结算过程中添加的额外修饰。
\notice 所有的英雄牌本身就是30血5护甲的实体,因此通过这种异常方式替换的英雄也是30血5护甲。
\example \version{}{11.2} 你对敌方\card{变色龙卡米洛斯}使用\card{埋葬},并在法术结算之前消灭了它。随后你抽到了它,它在手牌中变形成英雄牌。你使用\card{复活术}复活唯一死亡的变色龙。结果你异常替换了英雄。参见\href{https://www.bilibili.com/video/av24248527}{av24248527}。
\example 你对你的\card{阿努巴拉克}使用\card{食肉魔块}。阿努巴拉克移回手牌;你使用它,将它变形成\card{变色龙卡米洛斯}再移回手牌。随后变色龙变形为英雄牌。你消灭食肉魔块。结果你异常替换了英雄。参见\href{https://www.bilibili.com/video/av37117483}{av37117483},7:56处。

\section{战斗}
\label{battle}

当一个角色主动攻击时,会产生如下的序列:
\begin{enumerate}
    \item \term{战斗阶段}:
    \begin{enumerate}
        \item \term{攻击前步骤}:攻击前扳机列队结算。例如\card{爆炸陷阱}、\card{冰冻陷阱}、\card{误导}、\card{毒蛇陷阱}、\card{崇高牺牲}、\card{蒸发}、\card{寒冰护体}、\nameref{forgetful}及类似效果、\card{诺格弗格市长}、\card{叛变}、\card{游荡怪物}、\card{自动防御矩阵}等。
        \item 如果结算完毕后防御者发生了改变,则再进行一次攻击前步骤。重复此步骤直到防御者没有发生改变为止。在额外的步骤中,已经在之前的攻击前步骤触发过的攻击前扳机不能再次触发。
        \item \term{攻击时步骤}:攻击时扳机列队结算。例如\card{真银圣剑}、\card{血吼}、\card{智慧祝福}、\card{窃贼}、\card{真言术:耀}、\card{“鲨鱼”加佐}、\card{收集者沙库尔}、\card{失心农夫}等。
        \item 如果此时攻击者、防御者或任意一方英雄离场、濒死或休眠,则攻击取消,跳过伤害和攻击后步骤。
        \item \term{伤害步骤}:攻击者的潜行移除。然后攻击者对防御者造成伤害同时防御者对攻击者造成伤害。最后结算两者的伤害事件。
        \item 无论攻击是否被取消,攻击者的「本回合已攻击次数」\texttt{.NUM\_ATTACKS\_THIS\_\allowbreak{}TURN}加1。
        \item \term{攻击后步骤}:攻击后扳机,例如\card{捕熊陷阱}、\card{腐树巨人}、\card{符文之矛}等列队结算。
            \notice 如果攻击者是英雄且装备武器,武器失去耐久也与这些扳机共同列队结算。
        \item \card{蜡烛弓}和\card{角斗士的长弓}添加的临时免疫移除。若防御者濒死则攻击者的\card{霜之哀伤}将其记录。
    \end{enumerate}
\end{enumerate}

如果一个角色仅仅是受到某个效果(例如\card{沼泽之王爵德}、\card{狗头人蛮兵}、\card{野兽之心}、\card{群体狂乱}、\card{超级对撞器}和\card{决斗})强制进行了一次攻击,会进行如下较为简单的流程:
\begin{enumerate}
    \item  上文中战斗阶段的所有流程。由于此时处于一个阶段中,在此之后没有阶段间步骤。此外,野兽之心、群体狂乱、超级对撞器和决斗等效果不会使「本回合已攻击次数」加1。
\end{enumerate}

\exception \card{诺格弗格市长}和\card{食人魔勇士穆戈尔}对连续的多次强制战斗只会触发一次。这主要表现在\card{疯狂巨龙死亡之翼}上。

由于额外的攻击前步骤的存在,直观上看攻击前扳机的触发顺序与入场顺序可能不同。
\example 对手操控\card{毒蛇陷阱}和\card{游荡怪物},你攻击敌方英雄。攻击前步骤中游荡怪物触发,将防御者改为新召唤的随从。然后在额外攻击前步骤中,毒蛇陷阱触发。

如果在同一个攻击前步骤中防御者发生了改变,该步骤中的后续扳机仍按原防御者进行结算;那之后才进行一个额外的攻击前步骤,在这个额外步骤中扳机按新防御者进行结算。
\example 对手操控\card{误导}和\card{爆炸陷阱},你攻击敌方英雄。尽管误导先将防御者改为了一个随从,爆炸陷阱仍认为对方英雄受到攻击,因此可以触发。
\example 对手操控\card{崇高牺牲}和\card{自动防御矩阵},你攻击敌方\card{小精灵}。尽管崇高牺牲先将防御者更改为2/1的\card{防御者},自动防御矩阵仍认为小精灵受到攻击,因此为小精灵而触发。小精灵会获得圣盾,而2/1的\card{防御者}成为被攻击目标。
\example 对手操控\card{诺格弗格市长}和\card{自动防御矩阵},你攻击敌方英雄。市长先将防御者更改为\card{小精灵},而自动防御矩阵不属于这一攻击前步骤中的扳机(因为受到了攻击的并不是随从)。在接下来的额外攻击前步骤中自动防御矩阵触发,使小精灵获得圣盾。

\nameref{forgetful}及类似效果如果符合条件但因50\%的概率而没能触发,则它也不能在后续的额外攻击前步骤中触发。
\example 双方操控14个\card{食人魔勇士穆戈尔}。对于每一次战斗而言,穆戈尔触发的数量约为平均值7个。倘若此规则不成立,即没能触发的穆戈尔可以在后续额外攻击前步骤中触发,那么每一次战斗穆戈尔触发的数量会接近为14个。

如果在同一个攻击前步骤中,防御者先从A变为B再变为C,则不会产生以B为防御者的额外攻击前事件。
\example 对手操控\card{游荡怪物}、\card{误导}和\card{毒蛇陷阱},你令你的随从攻击敌方英雄。攻击前步骤中首先游荡怪物触发将防御者改为随从,然后误导触发又将防御者改为你的英雄。在这之后,毒蛇陷阱不会触发。

如果在同一个攻击前步骤中,防御者先从A变为B再变为A,则不会产生额外攻击前事件。
\example 对手操控\card{飞刀杂耍者}、\card{叛变}、\card{游荡怪物}和\card{误导},你令你的唯一随从\card{恐怖的奴隶主}攻击敌方英雄。首先叛变不触发;然后游荡怪物触发召唤随从成为防御者,且飞刀射中奴隶主召唤一个新奴隶主;再之后误导触发将防御者改回为敌方英雄。此时不会产生额外攻击前事件,叛变不能触发。

多数攻击前扳机都有比较复杂的扳机条件,这决定了他们与其他攻击前扳机的互动结果。这部分内容会在\nameref{trigger-cond}中详细说明。

\card{蜡烛弓}与\card{角斗士的长弓}除了具有一个攻击前扳机之外,还具有一个预伤害扳机。它的作用是在你真正获得免疫状态之前防止你受到的所有伤害。

如果攻击者在攻击前步骤中濒死,它可以被攻击步骤中的扳机救回从而完成攻击。
\example 你具有1点生命且对手操控\card{爆炸陷阱},你装备\card{真银圣剑}攻击敌方英雄。首先爆炸陷阱对你造成2点伤害,然后真银圣剑使你恢复2点生命。此时你的生命值为正,攻击可以完成。

除了攻击者/防御者离场或濒死,大多数其他情况都不会导致攻击取消。这包括防御者变成友军,攻击者变成敌人,攻击者变为0攻或攻击者的武器被新武器替换等情况。
\example 对手操控\card{爆炸陷阱}和\card{游荡怪物},你装备\card{蜡烛弓}攻击敌方英雄。首先爆炸陷阱对你的\card{苦痛侍僧}造成2点伤害,其抽到\card{提里奥·弗丁}。\card{人偶大师多里安}和\card{灵魂歌者安布拉}触发使你的武器被\card{灰烬使者}替换。然后游荡怪物触发召唤一个随从成为防御者。在接下来的战斗中,你使用灰烬使者对新的防御者造成5点伤害,且你仍受到蜡烛弓的免疫效果保护。

在伤害步骤中,伤害事件会在双方均造成伤害后结算,先结算攻击者对防御者的伤害事件,再结算防御者对攻击者的伤害事件。
\example 你令\card{暴乱狂战士}攻击对手暴乱狂战士。它们各对对方造成2点伤害,然后它们均变为4攻。
\example 对手操控\card{飞刀杂耍者},你令\card{恐怖的奴隶主}攻击敌方\card{小鬼首领}。首先结算防御者小鬼首领的受伤事件,召唤小鬼触发飞刀射中奴隶主。然后结算攻击者奴隶主的受伤事件,由于此时奴隶主濒死其扳机无法触发。

\notice 「攻击者对防御者造成伤害」这一事件包括「攻击者造成伤害」和「防御者受到伤害」。因此响应「攻击者造成伤害」的扳机和响应「防御者受到伤害」的扳机按顺序触发。「防御者对攻击者造成伤害」事件同理。
\example 你具有2点生命,并令\card{兽人铸甲师}攻击敌方\card{掷斧者}。若铸甲师先入场,则它首先触发,你首先获得护甲而存活;若铸甲师后入场,则掷斧者先触发,你最终死亡。

如果伤害步骤后攻击者、防御者或任意一方英雄濒死,攻击后步骤不会取消。
\example 对手具有1点生命且操控\card{捕熊陷阱}和具有\nameref{lifesteal}的\card{飞刀杂耍者},你令\card{小精灵}攻击对手英雄。攻击后步骤中捕熊触发,飞刀射中小精灵并吸血。最后对手恢复到1点生命而存活。
\exception 如果攻击方英雄离场,其武器耐久度不会减少;如果防御方英雄离场,\card{捕熊陷阱}不会触发。
\example 你装备武器并攻击,触发对手的\card{误导}改为攻击你自己的\card{苦痛侍僧}。苦痛受到伤害抽到\card{假死},受\card{沙德拉斯·月树}影响施放它并触发\card{管理者埃克索图斯}的亡语。你变为大螺丝,武器耐久不会减少。
\example 你的对手挂有捕熊陷阱具有攻击力。你的苦痛侍僧攻击敌方英雄,受到伤害抽到\card{自爆肿胀蝠},因为\card{人偶大师多里安}与\card{灵魂歌者安布拉}的效果召唤并触发亡语;然后对敌方苦痛造成伤害,因为敌方人偶与安布拉的效果召唤并触发管理者的亡语。敌方英雄变身为大螺丝,捕熊不会触发。

\card{符文之矛}施放法术是在额外阶段中执行的,这可以避免战斗中死亡的随从又被法术救活。
\example 对手操控\card{火羽精灵}和\card{愤怒的小鸡},你使用\card{符文之矛}攻击愤怒的小鸡并施放\card{衰变}。愤怒的小鸡立刻死亡,只有火羽精灵会被衰变。
\example 上例中如果你施放的是\card{妖术}且场上没有其他随从,则妖术必定以火羽精灵为目标。

\card{凶恶的雏龙}的进化很晚才实际生效。
\example 你使用凶恶的雏龙并将它打到一血。对手使用\card{火焰结界},然后你令雏龙攻击敌方英雄。在攻击后步骤中首先雏龙进化,你选择 +3 生命值,但这一效果稍后才会执行。接下来火焰结界触发,雏龙濒死。战斗阶段结束,你的雏龙来不及获得 +3 生命值就已经死亡。
\notice 进化效果不能在目标濒死时生效,详见\nameref{adapt}。
\example 将上个例子的雏龙和火焰结界的顺序交换。在攻击后步骤中首先火焰结界触发,雏龙濒死,因此进化不生效。战斗阶段结束,你的雏龙死亡。
\notice \version{}{8.0}发现效果的额外阶段在其他序列中也存在。
\notice \version{}{?}在之前的版本中\card{符文之矛}施放的法术也是在额外阶段中执行的。
\example \version{}{?}对手操控\card{火羽精灵}和\card{愤怒的小鸡},你使用符文之矛攻击愤怒的小鸡并施放\card{衰变}。愤怒的小鸡立刻死亡,只有火羽精灵会被衰变。
\example \version{}{?}战士使用1耐久的符文之矛攻击并施放\card{升级}。符文之矛首先因耐久度用光而摧毁,你只能获得一把1/3的武器。

\subsection{攻击次数}

判断一个角色能否攻击,是通过比较「本回合已攻击次数」与「本回合可攻击次数」来实现的。如果小于则可攻击。

一名角色每进行一次攻击,它的「本回合已攻击次数」会加1。\card{沼泽之王爵德}和\card{狗头人蛮兵}的效果也会使「本回合已攻击次数」加1。「本回合已攻击次数」会在随从区域移动时,及回合间清除步骤中清零;但不会在随从控制权转移时清零。
\example 你令\card{银色指挥官}攻击敌方英雄,然后依次对其使用\card{变节}和\card{精神控制}。它不能再次攻击。
\example 对手操控一个3攻的\card{沼泽之王爵德},你先使用\card{小精灵}后对王爵德使用\card{暗影狂乱}。它不能攻击。
\notice 并非所有由卡牌效果导致的强制战斗都会使「本回合已攻击次数」加1。目前暂时没有明确的规律。
\example 你对一个本回合可进行攻击的随从使用\card{横冲直撞},它在攻击相邻的随从后,仍然可以再次攻击。
\example 你控制一个本回合可进行攻击的随从,你装备\card{引月长弓},并攻击一个随从。它在攻击目标随从后,仍然可以再次攻击。
\example 你使用\card{疯狂巨龙死亡之翼},它攻击数个随从。战吼后,你对他使用\card{火箭靴},它不能攻击。

「本回合可攻击次数」由\term{基本可攻击次数}和\term{额外可攻击次数}两部分组成。
\begin{itemize}
    \item 一个角色的\term{基本可攻击次数}通常为1,\nameref{windfury}、超级风怒和\card{愚者之灾}分别将目标的基本可攻击次数改为2、4 和无穷。
    \item 一个角色的\term{额外可攻击次数}通常为0,\card{巨型沙虫}、\card{贡克,迅猛龙之神}、\card{鞭笞者苏萨斯}等扳机触发时会使目标的额外可攻击次数加1。
\end{itemize}
\example 你控制\card{贡克,迅猛龙之神},持有2耐久的\card{愚者之灾}并使用\card{英勇打击}攻击\card{摩天龙}两次将其消灭。此时由于愚者之灾被摧毁,你的基本攻击次数为 1;贡克使你的额外攻击次数也变为1。本回合已攻击次数为2,不小于本回合可攻击次数。因此你不能再次攻击。

此外,还有一个角色可以攻击还要满足许多其他条件。如不具有召唤失调、不具有「无法攻击」、攻击力不为0、有合法的攻击目标等。但如果是效果导致的强制战斗,那么一般来说只要有合法的攻击目标即可攻击。
\exception \card{决斗}召唤的无法攻击的随从不能进行攻击。

\subsection{强制战斗}
\label{forced-battle}

一些效果可以让随从进行战斗。这类效果可能消耗也可能不消耗随从的攻击次数;有可能考虑冻结也有可能不考虑。一些结果见下表。

\begin{center}
\begin{tabular}{|c|c|c|}
\hline
卡牌 & 是否消耗攻击次数 & 是否适用于冻结角色 \\
\hline
洛萨 & \cmark & \cmark \\
伊利达雷审判官 & \xmark & \cmark \\
\hline
\end{tabular}
\end{center}

\section{使用英雄技能}

使用英雄技能的序列如下:
\begin{enumerate}
    \item \term{结算阶段}:
    \begin{enumerate}
        \item 如果英雄技能具有目标,指向扳机\card{诺格弗格市长}结算。
        \item 结算英雄技能。「你本回合已使用英雄技能次数」加1。
    \end{enumerate}

    \item \term{激励阶段}:
    \begin{enumerate}
        \item 激励扳机,例如\card{低阶侍从}、\card{暗影子嗣}、\card{毒镖陷阱}、\card{寒冰行者}、\card{大胆的吞火者}的状态移除等列队结算。
    \end{enumerate}
\end{enumerate}

\subsection{特殊的英雄技能互动}

\card{寒冰行者}可以在目标离场的情况下触发。这包括被移除、被\card{角斗场主管}回手等等。
\example 你操控\card{寒冰行者}并使用\card{火焰冲击}杀死你的\card{沟渠潜伏者}。其亡语召唤\card{莫拉比}。在接下来的激励阶段,寒冰行者将墓地中的沟渠潜伏者冻结,莫拉比响应这次冻结并将一张沟渠潜伏者的复制加入你的手牌。

\card{龙鹰之灵}修改使用英雄技能的结算流程。在通常的一次结算之后,该英雄技能将以相邻的随从为目标再结算一次技能效果,先左再右。
\notice 龙鹰之灵和\card{寒冰行者}的互动为:「结算一次,然后对左侧随从再结算一次并将其冰冻,然后对右侧随从再结算一次并将其冰冻」。
\notice 龙鹰之灵的额外结算也能享受\card{大胆的吞火者}状态的临时英雄技能增强。
\notice 龙鹰之灵的效果是一个光环。也就是说,如果\card{灵体转化}将龙鹰之灵变形,它仍然会再执行两次英雄技能效果。
\notice 在英雄技能额外结算过程中,由于没有触发时机,\card{诺格弗格市长}不会介入。

\subsection{英雄技能使用次数}

判断英雄技能能否使用,是通过比较「本回合已使用英雄技能次数」与「本回合可使用英雄技能次数」来实现的。如果小于则可使用。

你每使用一次英雄技能,「你本回合已使用英雄技能次数」加1。「本回合已使用英雄技能次数」会在英雄技能被替换时时,及回合间清除步骤中清零。

「本回合可使用英雄技能次数」由\term{基本可使用次数}和\term{额外可使用次数}两部分组成。
\begin{itemize}
    \item 一个技能的\term{基本可使用次数}通常为1,\card{要塞指挥官}和\card{考达拉幼龙}分别将技能的基本可使用次数改为2和无穷。
    \item 一个技能的\term{额外可使用次数}通常为0,\card{大富翁比尔杜}、\card{虚空形态}的刷新效果等扳机触发时会使技能的额外可使用次数加1。当你的技能被替换时,额外可使用次数清零。
\end{itemize}
\example 你操控\card{考达拉幼龙},使用\card{援军}两次。然后你使用\card{芬利·莫格顿爵士}将你的英雄技能替换为\card{图腾召唤}。你还可以使用图腾召唤两次。
\notice 即使替换前后的英雄技能是相同的(如通过\card{杂耍吞法者}),相关标签也会清零。
\notice \card{裁决者图哈特}不替换升级的英雄技能(并非替换成相同的升级技能)。因此,它不能重置你的已升级的英雄技能。

此外,英雄技能可以使用还要满足许多其他条件。如场上没有\card{摧心者}、英雄技能不是被动英雄技能等。
\chapter{顺序和条件}

\section{效果顺序}

效果顺序一般与战吼/法术描述文字中所写出的顺序相同,但也存在不同的情况。

\subsection{双方效果}

描述为「双方各」的效果,一般是效果发动者优先执行。
\example \card{寒光智者}首先为其操控者抽两张牌。
\example \card{决斗}和\card{先祖召唤}首先为法术施放者召唤一个随从。

描述为「双方英雄各」的效果,一般对先入场的英雄首先生效。
\notice 如果双方均未替换过英雄,主玩家的英雄先入场。
\example 双方均未替换过英雄,\card{翡翠掠夺者}和\card{暗影子嗣}首先对主玩家造成伤害。
\example 双方均未替换过英雄,\card{亡灵药剂师}和\card{生命之树}首先为主玩家恢复生命。

\subsection{召唤随从并附加状态的效果}

一些效果在召唤随从的同时为其附加某个状态。事实上这类效果的结算方式可能是「先召唤随从并结算其召唤事件,再附加状态」或「先召唤随从并附加状态,再结算其召唤事件」,究竟采用哪种方式目前暂无明确规律。这导致了以下互动:

如果一个效果「先召唤随从并附加状态,再结算其召唤事件」,那么这个状态可以被\card{卡德加}复制。反之则不行。
\example 卡德加可以复制\card{小鬼妈妈}召唤的恶魔所具有的嘲讽,但不能复制\card{萌芽分裂}召唤的随从所具有的的嘲讽。

如果一个效果「先召唤随从并附加状态,再结算其召唤事件」且召唤的随从具有休眠,则该状态在苏醒时保留。反之则不行。
\example \card{巴内斯}召唤的休眠随从苏醒时仍为1/1,但\card{黑暗预兆}召唤的休眠随从苏醒时不能获得 +3 生命值。

但无论如何,光环更新总是最先进行的。
\example 你具有\card{水晶核心}的效果并使用巴内斯召唤了一个随从。水晶核心立刻为这个随从附加一个名为「\card{晶化}」的状态使其变为4/4,然后因巴内斯的效果变为1/1。
\example 你操控\card{暴风城勇士}并使用巴内斯召唤了一个随从。暴风城勇士立刻为这个随从附加一个 +1/+1的状态,然后因巴内斯为其附加一个变为1/1的状态。但因为暴风城勇士的状态的低优先级,这个随从首先因巴内斯的状态变为1/1,再因暴风城勇士的状态变为2/2。

\notice \card{救赎}等效果并不是添加状态,而是通过修改伤害值\texttt{.DAMAGE}来将当前生命值置为1。这个效果的时间点早于随从享受到场上的光环。
\example 你操控\card{暴风城勇士}并通过救赎复活了你的\card{冰风雪人}。雪人首先因救赎的效果,当前生命变为1,然后受暴风城勇士光环影响变为5/2。
\notice 如果你在暴风城勇士的光环下复制一个场上随从并将其当前生命值置为1(如通过\card{鲜活梦魇}),则复制不会变成2血。

\subsection{反常的结算顺序}

有的效果的描述与实际的流程是相反的。这可能是一个 bug。
\example \card{暗影烈焰}描述为「消灭一个友方随从,对所有敌方随从造成等同于其攻击力的伤害。」,但实际上却刚好相反。这就意味着,如果对方操控\card{飞刀杂耍者}和\card{小鬼首领},你对你的\card{小精灵}使用\card{暗影烈焰}:如果按照描述执行,小精灵会首先被标记为待摧毁,因此飞刀不会射到小精灵。但实际上飞刀可能射到小精灵。
\example \card{神圣愤怒}描述为「抽一张牌,并造成等同于其法力值消耗的伤害。」,但实际上却刚好相反。这保证了神圣愤怒按卡牌在牌库中的费用造成伤害,避免了其他效果的影响。
\example \card{命令怒吼}描述为「在本回合中,你的随从的生命值无法被降到1点以下。抽一张牌。」,但实际上却刚好相反。这导致了如果你刚好抽到了\card{破海者},你场上1血的随从无法因为命令怒吼的效果而存活下来。
\example \card{末日仪式}描述为「消灭一个友方随从。如果你拥有5个或更多随从,召唤一个5/5的恶魔。」,但实际上却刚好相反。这导致了如果你只拥有5个随从,也可以触发末日仪式的效果。以及消灭随从之前没有5个或更多随从,但是消灭随从后拥有了5个或更多随从也不会召唤5/5的恶魔。

卡扎库斯药水的不同组合有不同且确定的结算顺序。详见\card{卡扎库斯}。

\subsection{AoE的结算顺序}

有的AoE按照一个特定的顺序造成伤害,而不是一个群体伤害:
\begin{itemize}
    \item \card{横扫}首先对目标造成伤害,再对其他敌方角色造成群体伤害。
    \item \card{冰锥术}的伤害顺序是「左边,右边,中间」。
    \item 此外所有描述为「对目标及其两侧的随从造成伤害」的效果造成伤害的顺序都是「中间,左边,右边」。包括\card{爆炸射击}、 \card{强风射击}等。
\end{itemize}

一些 AoE 的实际结算顺序如下:
\begin{itemize}
    \item \card{暴风雪}首先对所有敌方随从造成2点伤害,然后冻结所有敌方随从。也就是说,因群体伤害而产生的随从也会被冻结。
    \item 与之相对的是,\card{冰锥术}首先对那几个随从造成伤害,再冻结同样的随从。它不会考虑是否产生了新的随从或是随从之间的位置关系发生了变化。
\end{itemize}

\section{扳机顺序}

前面提到,响应同一个事件的绝大多数扳机按入场顺序触发。实际上,这句话只描述了扳机都在场上的情况。如果扳机位于各个不同区域,情况会有所不同。此外,部分扳机还具有特殊的优先级。

\subsection{不同区域扳机的结算顺序}
\label{trigger-order-in-different-zone}

不同区域的实体具有的扳机可以划分为场上扳机、手牌扳机和牌库扳机三种。
\begin{itemize}
    \item 奥秘、武器、法术和场上的随从具有的扳机都是场上扳机。
    \item 状态具有的扳机都是场上扳机,即使这个状态是结附在手牌或牌库的牌上。
    \item 手牌中的随从具有的扳机是手牌扳机,例如\card{通道爬行者}。
    \item 牌库中的随从具有的扳机是牌库扳机,例如\card{海盗帕奇斯}。
\end{itemize}

当一个事件发生时,所有可用的扳机按照场上、手牌、牌库的顺序触发。
\example 你使用\card{南海船工}。首先场上的\card{冒进的艇长}触发,然后手牌中的\card{空降歹徒}触发,最后牌库中的\card{海盗帕奇斯}触发,不管它们以什么顺序生成,也不管它们在什么时候进入战场/手牌/牌库。

当我们称一个扳机\term{在场}时,实际上是指它位于可触发的区域。如\card{伯瓦尔·弗塔根}在场指在手牌中。

随从自带的扳机(如\card{飞刀杂耍者})的操控者与随从操控者相同,所处区域与随从所处区域相同,入场时间点与随从入场时间点相同。但由状态添加的扳机(如\card{力量的代价})的操控者为效果的使用者,所处区域始终视为场上,入场时间点为状态添加的时间点。
\notice \card{百变泽鲁斯}在未变形时是一个手牌扳机,而变形后是一个被附加了「\card[OG-123e]{变形} 」状态的随从,因而此效果对应的扳机属于场上扳机,入场时间点为上次变形时。其他类似的卡牌也有着相同的机制,包括\card[ICC-827t]{暗影映像} 。
\example 如果你的手中有未变形的\card{百变泽鲁斯}并使用\card{报警机器人},下个回合开始时报警机器人从手牌中召唤出泽鲁斯,它不能变形。而如果你的手中有已变形的泽鲁斯并使用报警机器人,下个回合开始时首先泽鲁斯继续变形,然后报警机器人从手牌中召唤出变形后的随从。
\example 你持有\card{软泥教授弗洛普}并使用\card{灵魂歌者安布拉}和\card{伊克斯里德,真菌之王}。软泥教授在真菌之王的使用后步骤中变形,因此其入场时间点晚于真菌之王(注意已变形的软泥教授是一个场上扳机,因此讨论入场时间点是有必要的)。你接着使用\card{破棺者}。在召唤后步骤中软泥教授尚未变形,因此安布拉不能将其召唤出来。在使用后步骤中首先真菌之王复制一个破棺者并由安布拉触发亡语,而此时软泥教授仍尚未变形,安布拉依然不能将它召唤出来。最后软泥教授变形为破棺者,序列结束。

\subsection{特殊优先级}
\label{special-priority}

部分扳机有特殊优先级,使它们与其他扳机之间不按照入场顺序和区域顺序结算。

\card{救赎}有低优先级,它总是晚于那些没有特殊优先级的场上死亡扳机。
\example 你操控救赎和\card{自爆绵羊}且持有\card{伯瓦尔·弗塔根}.对手杀死你的自爆绵羊。无论使用顺序如何,自爆绵羊的亡语首先触发,然后救赎召唤一个自爆绵羊,最后你的伯瓦尔获得 +1攻击力。

回响牌和\card{不稳定的异变}的消失有高优先级,它总是早于那些没有特殊优先级的回合结束扳机。
\example 你操控\card{基维斯}且持有回响状态下的不稳定的异变。回合结束阶段,首先不稳定的异变消失,然后基维斯为你抽满3张牌。

如果你一回合使用多张秘密通道,回合结束时所有秘密通道会按倒序结算,且以你使用第一张秘密通道之时为时机。
\example 你依次使用\card{唤醒}和秘密通道。回合结束时首先结算唤醒,但此时唤醒牌并不在你手牌中因此不会弃掉。然后秘密通道将那些牌返回你手牌。
\example 你依次使用秘密通道、唤醒和秘密通道。回合结束时首先结算两张秘密通道,将唤醒牌返回你的手牌。然后结算唤醒将那些牌弃掉。
\notice 上面的结算表明,只有你在使用唤醒之前没有使用过秘密通道的情况下,回合结束时唤醒牌才能保留。

\nameref{reborn}有最低优先级。它总是晚于一切其他死亡扳机。
\example 你持有\card{伯瓦尔·弗塔根}并杀死一个友方\card{鱼人木乃伊}。伯瓦尔首先增加1点攻击力,然后鱼人木乃伊才复生。

\nameref{lifesteal}和\nameref{overkill}有低优先级。它们总是晚于一切其他伤害扳机。
\example 你令你的\card{竞技场奴隶主}攻击敌方英雄并被\card{误导}到友方\card{龙蛋}。无论使用顺序如何,龙蛋首先触发召唤一条\card{黑色雏龙},然后竞技场奴隶主触发召唤另外一个竞技场奴隶主。
\notice 如果你使一个超杀法术获得吸血(通过\card{欧米茄灵能者}),超杀先于吸血生效;如果你使一个超杀随从获得吸血,则吸血先于超杀生效。

预伤害扳机之间有固定的触发顺序。详见\nameref{predamage-trigger}。

\subsection{主玩家机制}

\term{主玩家}是指\texttt{Player.PLAYER\_ID = 1}的玩家。相对地,\texttt{Player.PLAYER\_ID = 2}的玩家称为\term{副玩家}。
\begin{itemize}
    \item \texttt{Player.PLAYER\_ID} 这个标签在游戏开始前就已经决定,和先后手无关。
    \item 在官方公告\nameref{rule-update:9.2}中有着Player 1与Player 2的称呼,这似乎是承认了之前的主玩家机制并非一个bug。但目前此机制已被移除。
    \item 在冒险模式下,玩家一方始终为主玩家。
    \item 在早期版本和某个未知版本之后,友谊赛的邀请方永远为主玩家。
\end{itemize}

\version{}{17.4.1}\term{主玩家机制}指当由不同玩家操控,或区域不同的扳机同时响应一个事件时,他们的触发顺序不再由入场顺序决定,而是有着一个固定的顺序,依次为:主玩家场上、主玩家手牌、主玩家牌库、副玩家场上、副玩家手牌、副玩家牌库。
\example 你操控\card{格鲁尔},对手操控\card{克尔苏加德}。在回合结束时,主玩家的随从的扳机先触发,无论这是谁的回合以及两个随从哪个先入场。
\notice 主玩家机制只影响扳机顺序,不影响效果顺序,如:AOE的伤害顺序或死亡阶段各死亡事件的顺序等。
\example 双方各操控\card{瑟玛普拉格}并对换\card{战利品贮藏者}。你的战利品先入场并先结算其死亡步骤,你的战利品亡语和对方的瑟玛普拉格按主玩家顺序结算。然后结算对方贮藏者的死亡步骤,对方战利品和你的瑟玛普拉格按主玩家顺序结算。

在当前的游戏中,主玩家机制已被移除,但依然存在主玩家/副玩家之分。主玩家与副玩家的唯一区别就是初始英雄入场顺序,主玩家早于副玩家。这在某些特定情况下会对游戏结算造成影响。
\example 双方均操控\card{扭曲巨龙泽拉库}且未替换过英雄,你使用\card{翡翠掠夺者}。翡翠掠夺者首先对先入场的英雄(即主玩家的英雄)造成伤害,因此主玩家一方先召唤6/6的龙。

\section{效果条件}
\label{effect-cond}

效果条件指战吼和法术的指向和生效都需要满足一定条件。此外,部分特殊事件也有特殊的条件。

战吼与法术条件又分为\term{结算条件}和\term{指向条件}。部分效果内部还有一定的判断条件。

结算条件与效果的内容一般无关,仅是效果执行所必须满足的外部条件。如「如果你持有龙牌」。\\
部分非指向性战吼需要满足一定的结算条件才能执行。这些条件仅在结算时检测——符合即执行,反之不执行。
\example 对手操控四个随从,你使用\card{布莱恩·铜须}和\card{精神控制技师}。第一次战吼随机获得一个敌方随从的控制权,第二次战吼什么也不做。

部分指向性战吼需要满足一定的结算条件才能执行。这些条件在你从手牌中使用时和结算时均检测——都符合即执行。如果结算时符合但使用时不符合,效果不会执行(因为你当时没有为其选择目标)。如果使用时符合但结算时不符合,效果也不会执行(UI会播放事件执行的动画,但实际上没有任何事情发生)。这类效果包括:「如果你持有龙牌」类、\card{麦迪文的男仆}、\card{阴燃电鳗}、\card{火焰使者}、\card{火焰使者}、\card{塔达拉姆王子}、\card{疾疫使者}和\nameref{combo}等。这个规则最先在「如果你持有龙牌」类指向性战吼上发现,因此也称作\term{龙吼二次检测}。
\notice 连击的条件是「本回合使用卡牌数」标签不小于2。对你而言,你的此标签仅在对方回合结束和你的回合开始之间重置,在你的回合结束和对方回合开始之间不重置。
\notice 你可能认为使用时「上个回合使用过元素」而结算时「上个回合未使用过元素」是不可能的。实际上这是通过伊利丹·怒风等扳机将随从的操控权交给对手实现的。另一种实现的方法是在两次战吼之间把随从的操控权交给对手。如以伤害性战吼指向敌方\card{苦痛侍僧},抽到\card{希尔瓦娜斯·风行者}且被\card{人偶大师多里安}复制,再由\card{灵魂歌者安布拉}触发亡语。
\exception \card{穿刺者戈莫克}仅要求使用时符合条件。这是唯一的例外。

部分法术需要满足一定的结算条件才能从手牌中使用,如\card{绝命乱斗}要求场上有两个或更多随从。这些条件仅在你从手牌中使用时检测,如果你使用的法术在结算时不再符合条件,它不会执行任何效果。
\example 对手操控两个随从,你使用\card{绝命乱斗}。结算时如果对手场上只有一个随从,绝命乱斗不会生效。
\example 场上没有其他随从,你使用\card{尤格-萨隆}施放绝命乱斗。由于不符合条件,绝命乱斗不会生效。
\example 场上没有其他随从且你的牌库只有绝命乱斗一种法术,你的\card{资深档案管理员}在回合结束不会施放任何法术。

\term{指向条件}与效果的内容有关,它要求指向性效果必须选择符合这一条件的目标。如「选择一个友方随从」。\\
你在为指向性战吼与法术选择目标时需要满足一定的\term{指向条件}。这些条件仅在你从手牌中使用时检测,如果结算时目标不再满足条件,战吼/法术依然生效。而当战吼被\card{沙德沃克}重复/法术被施放时,只会选择符合条件的目标,如果没有则不执行任何效果。
\example 你持有龙牌并使用\card{燃棘枪兵}指定了一个受伤的敌方随从。如果结算时目标不再受伤,或目标变成了一个友方随从,战吼依然会消灭它。如果结算时你不再持有龙牌,或枪兵被对手操控且对手不持有龙牌,则战吼不执行。
\example 你对你的野兽使用\card{凶猛狂暴},触发敌方\card{扰咒术}。将三个4/6的\card{扰咒师}洗入你的牌库。

部分效果结算过程中会判断一定条件,以确定是否执行某一事件/执行哪一事件。如「对一个随从造成2点伤害,如果该随从是友方恶魔,则改为使其获得+2/+2」。
\example 在\card{高弗雷勋爵}的战吼中,每次 AOE 后都检测「是否有随从死亡」以决定是否执行下一次 AOE。
\example 你对一个友方恶魔使用\card{恶魔之火}使其+2/+2,你的\card{西风灯神}触发并受到2点伤害。

游戏中的部分操作要求目标随从必须在场且非濒死。已知的例子有\nameref{adapt}和「触发一个随从的亡语」。
\example 你使用\card{饥饿的翼手龙}消灭友方\card{阿诺玛鲁斯},随后阿诺玛鲁斯亡语炸死翼手龙。翼手龙死亡,不会弹出进化选项供你选择。
\example 你对友方\card{阿努巴拉克}使用\card{死金药剂}。第一次亡语阿努巴拉克回手并召唤\card[AT-036t]{蛛魔}。第二次亡语不触发,也不会召唤蛛魔。
\exception \card{九命兽魂}可以在手牌触发随从的亡语。

\section{扳机条件}
\label{trigger-cond}

各种扳机的触发都需要满足一定的扳机条件,包括预检测条件、列队条件和结算条件。就目前而言,准确的判明所有扳机的所有条件几乎是不可能的,因此本节只会列出部分受条件影响很大的扳机(如攻击前步骤中的各个扳机),而其他扳机则会给出一个总体上的结论。

\term{预检测条件}指阶段所包含的步骤中的扳机必须在此步骤开始前的一个更早的时间点满足一定条件(也称为通过\term{预检测})才能触发。固有阶段的各个步骤中的扳机在序列开始时进行预检测;死亡阶段各个死亡事件的扳机在死亡检索步骤开始前进行预检测。
\example 对手操控\card{疯狂的科学家},你使用\card{火元素}的战吼杀死科学家并拉出\card{镜像实体}。在使用后步骤中,镜像实体不会复制火元素,因为它在序列开始时不在场。
\example 你使用\card{空中悍匪}的战吼将一张\card{空降歹徒}置入手牌,在使用后步骤中,空降歹徒不会被召唤到战场。因为它在序列开始时不在手牌。
\example 对手使用\card{烈焰风暴}消灭你依次入场的\card{疯狂的科学家}和其他随从。科学家拉出\card{复制}。在随后任何一个随从的死亡步骤中复制都不会触发,因为它在死亡阶段开始时不在场。
\example 你满场且操控\card{轮回}。对手消灭你的一个随从,轮回不能触发。尽管这可能是一个bug,但证实了死亡扳机的预检测在死亡检索步骤开始前。
\notice 只有阶段所包含的步骤中的扳机和死亡事件的扳机会进行预检测,如果一个扳机只是响应一个非死亡事件,如伤害、治疗、召唤、抽牌等,则不需要预检测。
\example 你使用\card{欧克哈特大师},其战吼召唤的三个随从中,2 攻的随从是\card{飞刀杂耍者}。接下来,它可以为3攻随从的召唤而射出一刀。但在完成阶段欧克哈特大师的召唤后步骤中,它无法为欧克哈特大师的召唤而射出一刀。

\term{列队条件}指扳机在它所响应的事件开始结算时需要满足的条件。
\notice 在绝大多数步骤中,多数扳机的列队条件与预检测条件完全相同,且设计一种情况使扳机在预检测和触发时符合条件而列队时不符合是很困难的。因此可以认为列队条件对这些扳机的触发影响不大。
\notice 在攻击前步骤中,多数扳机没有预检测条件(或者说预检测条件仅要求它们在场)。这样它们能在防御者改变后的额外攻击前步骤中触发。
\example 对手操控先入场的\card{游荡怪物}和\card{误导},你操控后入场的\card{诺格弗格市长}。你令市长攻击敌方英雄,触发游荡和误导后仍然攻击敌方英雄,接下来你的市长不触发。因为市长的扳机在列队时不满足「场上有两个或更多的敌方角色」。
\example 与上个例子相似,但触发游荡怪物和误导后转为攻击你的英雄。由于防御者最终发生了改变,产生了额外攻击前步骤。接下来在以你的英雄为目标的额外攻击前步骤中,由于「场上有两个或更多的敌方角色」,你的市长触发并把防御者重新改为对方英雄或游荡怪物新召唤的随从。
\notice 在以上两个例子中,市长的触发条件为「场上有两个或更多的敌方角色」。如果该条件是预检测条件,则在后一个例子中市长不可能触发(因为预检测时显然只有一个敌方角色);如果该条件仅是结算条件,则市长在前一个例子中应该可以触发(因为攻击前步骤中轮到市长时已经有两个敌方角色了)。因此该条件必定是一个列队条件。

\term{结算条件}指扳机在结算时必须满足才能结算的条件。
\notice 如果结算条件不满足,那扳机的闪电符号根本不会亮起,奥秘也不会揭示。这与扳机结算中的分支情况不同。
\notice 一些扳机或明确或隐含地要求随从不处于濒死状态才能触发。其中明确的如\card{恐怖的奴隶主}、\card{腐面}、\card{发条强盗机器人}等;隐含的包括且不限于\card{达利乌斯·克罗雷}、\card{勇敢的记者}、\card{地穴领主}、\card{波戈蒙斯塔}。
\example 你令你的达利乌斯·克罗雷攻击敌方\card{食人魔法师},克罗雷不能触发以获得+2/+2活下来。
\example \version{7.1?}{}对手操控生命值为1的记者。你对其使用\card{死亡缠绕}。记者受到1点伤害,你抽一张牌,但是记者不触发。
\example \version{11.0}{}你使用\card{冰冷触摸}指向你生命值为1的地穴领主。地穴领主受到1点伤害,你召唤一个水元素,但是地穴领主不触发。

对于绝大多数扳机而言,它们的列队条件和结算条件是相同的。但是对非攻击前步骤的、无需预检测的扳机而言,「列队时条件不满足」而「结算时条件满足」的情况是很少见的。对于需要预检测的扳机而言,「预检测时条件满足」、「列队时条件不满足」而「结算时条件满足」的情况也是很少见的。因此,我们很少讨论非攻击前步骤扳机的列队条件。以下是这种情况的一个例子:
\example 你操控先入场的\card{人偶大师多里安}并使用\card{死亡缠绕}指向对手后入场的1血\card{勇敢的记者}。死亡缠绕抽到\card{舞动之剑},人偶大师召唤复制并触发你的\card{灵魂歌者安布拉}。对手抽到一张\card{恩佐斯的子嗣}。对手的人偶大师和灵魂歌者安布拉触发并使濒死的勇敢的记者变为 4/1救了回来。尽管勇敢的记者当前不处于濒死状态,但它在列队时处于濒死状态,因此其扳机无法触发。
\example 对手操控\card{镜像实体}和\card{寒冰克隆}并持有8张手牌。你使用\card{寒光智者}。对手镜像实体触发召唤一个寒光智者的复制,然后对手\card{飞刀杂耍者}射中你的\card{小鬼首领}召唤一个小鬼。你的飞刀杂耍者触发射中对手的\card{咆哮魔}。现在对手有9张手牌。但寒冰克隆不触发,因为在使用后步骤开始时你的手牌是满的。

绝大多数奥秘在无法产生任何效果,或可能起到副作用时,会保持隐藏。
\example 你持有十张手牌且操控\card{寒冰克隆}。对手使用一张随从牌,你的寒冰克隆不会触发。
\example 你手中没有随从牌且操控\card{军备宝箱}。对手使用一张随从牌,你的军备宝箱不会触发。
\example 对手令他的随从攻击你的\card{提里奥·弗丁}。弗丁已经具有圣盾,你的\card{自动防御矩阵}不会触发。
\example 你操控\card{狙击}和\card{忏悔}。对手使用\card{小精灵},狙击对小精灵造成致命伤,而忏悔不触发。
\exception \card{隐秘的智慧}在你拥有10张手牌时依然会触发。这可能是一个bug。

\subsection{攻击前步骤扳机条件}

攻击前步骤中的扳机大多具有较复杂的条件,且其触发与否对战斗的最终结果有较大影响。本部分详细说明这些条件:
\notice 部分条件在卡牌描述中被明确指出(如\card{冰冻陷阱}要求「攻击者是一个随从」),此处不再重复。

\begin{itemize}
    \item \card{冰冻陷阱}的条件为「攻击者在场且未濒死」。
    \item \card{误导}的条件为「攻击者在场且未濒死」和「除攻击者和防御者外,场上至少有一个未濒死,且非免疫的其他角色」。
    \item \card{爆炸陷阱}没有任何条件。
    \item \card{毒蛇陷阱}的条件为「场地未满」。
    \item \card{游荡怪物}的条件为「场地未满」。
    \item \card{崇高牺牲}的条件为「场地未满」。
    \item \card{自动防御矩阵}的条件为「场地未满」和「防御者没有圣盾」。
    \item \card{蒸发}的结算条件为「攻击者在场且未濒死」。
    \item \card{寒冰护体}没有任何条件。
    \item \card{裂魂残像}的结算条件为「场地未满」。
    \item \card{叛变}的条件为「攻击者在场且未濒死」和「攻击者两侧至少有一个未濒死,且非免疫的随从」。
    \item \nameref{forgetful}、\card{食人魔勇士穆戈尔}和\card{诺格弗格市长}的条件为「场上有两个或更多的敌方角色」。
\end{itemize}

\section{随机目标选取条件}

部分效果和扳机需要随机选取目标生效,它们在选择目标时也要满足一定的目标选取条件。

随机伤害效果和扳机会忽略濒死角色,但不忽略免疫角色。
\example 你操控七个\card{炎魔之王拉格纳罗斯},对手空场且具有30点生命。你的回合结束阶段,前四个大螺丝会轰击敌方英雄,而后三个大螺丝不会产生效果。
\example 你操控\card{炎魔之王拉格纳罗斯}而对手操控\card{玛尔加尼斯}。你的回合结束阶段,大螺丝可能会轰击具有免疫的敌方英雄。
\example 你操控一个具有\nameref{poisonous}的\card{飞刀杂耍者}而对手操控三个\card{冰风雪人}。你使用\card{作战动员},不会有任何一个雪人受到两次飞刀。

随机治疗效果和扳机会忽略未受伤角色,但不忽略免疫角色。
\example 你操控七个未受伤的随从且具有18点生命。你使用\card{治疗之雨},所有治疗全部会分配给你的英雄。
\example 你操控\card{光耀之主拉格纳罗斯}且所有友方角色均未受伤。你的回合结束阶段,光螺丝不会触发它的扳机。

随机选择防御者的效果忽略濒死角色和免疫角色。
\example 场上只有一个你操控的\card{玛尔加尼斯}。你令玛尔加尼斯攻击敌方英雄,对手的\card{误导}不触发。

随机增益效果不会忽略濒死角色。
\example 对手操控两个\card{小精灵}和\card{复仇}。你令你后入场的\card{自爆绵羊}攻击一个敌方小精灵,其亡语对另一个小精灵造成致命伤害。此时\card{复仇}触发,濒死的小精灵受到增益效果变为4/1存活。

此外的随机效果一般都会忽略濒死角色。这包括且不限于\card{希尔瓦娜斯·风行者}的随机获得控制权效果、\card{阿努巴尔伏击者}的随机回手效果、\card{狂奔科多兽}和\card{虚空碾压者}的随机摧毁效果、\card{误导}和\card{叛变}的随机选择防御者效果。

还有许多法术牌的效果是忽略濒死角色的。这可以在\card{伊莱克特拉·风潮}生效时,或回合结束时\card{资深档案管理员}连续释放多个法术时体现出来。目前测试过的有:
\begin{itemize}
    \item \card{麦迪文的残影}和\card{幻觉药水}不会给你濒死随从的复制。
    \item 濒死随从不会获得\card{绝命乱斗}的胜利。(需要进一步测试)
\end{itemize}

\chapter{其他机制}

\section{生命值与攻击力}

本节中「生命值」也可以指武器的耐久。

一个实体的生命上限由\texttt{.HEALTH} 记录,而伤害值由\texttt{.DAMAGE} 记录。生命由前者减去后者得到,它在游戏中并不直接存储。对于武器而言,它的耐久上限由\texttt{.DURABILITY} 记录。

当一个实体增加生命时,其生命上限增加;伤害值不变。因此生命值也等量增加。
\example 你对\card{北郡牧师}使用\card{真言术:盾},其生命上限和生命值均为5。

\notice \card{神圣之灵}实际上使目标随从的生命上限增加等于生命值的数值。
\example 你对4/3 的\card{负伤剑圣}使用\card{神圣之灵},其生命上限增加3,为10,生命值也增加3,为6。

当一个实体被设定生命时,其生命上限变为设定数值;伤害值清零,因此生命值也变为设定的数值。互换攻击力和生命值的效果实际上将该实体的生命值和攻击力设置为效果结算时的固定值。
\example 你使用\card{疯狂的炼金师}指定4/3 的\card{负伤剑圣},其生命上限和生命值均变为4。

\notice 「生命设定」效果导致实体生命增加不是治疗;「生命设定」效果导致实体生命减少不是伤害。
\example 你使用\card{希望守护者阿玛拉},你的生命上限和生命值均变为40。你手牌中\card{开心的食尸鬼}不会减至0 费。

\notice 若你的生命上限大于15 点,\card{阿莱克丝塔萨}不会使你的生命上限减为15 点;而若你的生命上限大于40 点,\card{希望守护者阿玛拉}\emph{会}使你的生命上限减为40 点。

当一个实体被沉默时,若沉默使实体生命上限增加,则其伤害值不变;若沉默使实体生命上限减少,则其伤害值也等量减少,但不会低于0。
\example 你对\card{伊瑟拉}使用\card{黑暗裁决}和\card{奥术射击},然后将其沉默。沉默使伊瑟拉的生命上限由3增加到12,因此其生命值也增加9,变为10。
\example 你对伊瑟拉使用\card{圣殿执行者}和奥术射击,然后将其沉默。沉默使伊瑟拉的生命上限由15 降低到12,而其伤害值为2 小于3。因此伤害值变为0。

实体的攻击力在计算过程中可以为负。如果计算得到一个为负数的攻击力,它将被调整为0。
\example 你操控一个0/2 的\card{活动假人}。对手使用\card{虚弱诅咒}。接下来你的回合你对活动假人使用\card{力量祝福},活动假人变为1 攻。
\example 你装备一把已经攻击随从7 次的0 攻\card{血吼},然后使用\card{英勇打击}再攻击随从。攻击后血吼的攻击力减为 -1。你下回合再次使用英勇打击时,你依然造成4 点伤害而不是3 点。

实体的攻击力、生命值、伤害和英雄的护甲值上限为2147483647点(即$2^{31}-1$,十六进制表示为\texttt{0x7FFFFFFF})。如果某效果将使得攻击力或生命值溢出,则它不会生效。溢出的伤害值和护甲值会变为0。
\example 你对一个$2^{30}$攻的随从使用\card{受祝福的勇士},其攻击变为$2^{31}$点,显示为0。然后你对其使用\card{缩小射线工程师},其攻击力变为$2^{31}-2$点,显示为$2^{31}-2$。
\example 你对一个1431655765 攻(十六进制表示为\texttt{0x55555555})的随从连续使用多张受祝福的勇士,随从的攻击力总是在0 和一个极大数字之间反复变化。
\example 你控制\card{布莱恩·铜须}并使用\card{吞噬者穆坦努斯}吃掉2张对手牌中攻击和生命都是$2^{31}-1$点的随从,吞噬者穆坦努斯属性值先变成0/0 ,然后变成2/2。
\history \version{}{21.8} 在之前的版本中,攻击力和生命值可以溢出,溢出之后这个数值变为0。

当你复制一个攻击力为负数的随从时,你实际上得到的是一个攻击力为0的复制体。
\example \version{}{21.8} 你对一个$2^{31}$攻的随从使用\card{无面操纵者},两个随从均显示为0攻。对手使用\card{虚弱诅咒}。原本的随从变为$2^{31}-2$攻,而复制体依然显示为0攻(实际上是$-2$攻)。
\example \version{}{21.8} 你对一个$-2$攻的随从使用无面操纵者,两个随从均显示为0攻。然后你使用\card{嗜血}。原本的随从变为1攻,而复制体则变为3攻。

\section{光环更新}
\label{aura-update}

\term{光环更新}指游戏检查是否有新光环入场、是否有旧光环离场。事实上光环并非入场即生效、离场即失效。找到游戏中所有的光环更新时机并非易事,目前已经确定的光环更新时机有(包括费用光环,不包括\card[ICC_829p]{天启四骑士}的消灭效果):
\begin{itemize}
    \item \term{阶段间步骤}中的\term{光环更新步骤}
    \item 实体被创建时
    \item 实体入场时
    \item 实体被变形时
    \item 实体解除休眠时
\end{itemize}

\notice 费用光环的相关内容详见\nameref{cost}。
\example 你使用\card{唤魔者克鲁尔}召唤手牌中的\card{无证药剂师}和\card{玛尔加尼斯}。玛尔加尼斯的免疫光环在被召唤时立刻生效,你不会受到5点伤害。
\example 对手操控\card{变形药水}和\card{爆炸符文}。你使用玛尔加尼斯。被变形后玛尔加尼斯的免疫光环立刻失效,你的英雄会受到爆炸符文的5点伤害。
\example 你的玛尔加尼斯在回合开始时苏醒,接下来,你的英雄不会受到\card{鲜血女巫}的伤害。

已确定没有光环更新的情况包括:
\begin{itemize}
    \item 随从控制权转移时
    \item 实体被移回手牌、抽牌、造成伤害等
    \item 阶段结束后,死亡检索步骤之前
\end{itemize}

\example 你的\card{希尔瓦娜斯·风行者}和对手的\card{麻风侏儒}对撞。女王的亡语首先偷走对手的\card{玛尔加尼斯},但麻风侏儒的亡语依然对你造成2点伤害。
\example 你使用\card{地狱烈焰}依次杀死你的\card{自爆绵羊}、\card{希尔瓦娜斯·风行者},并将你的\card{索瑞森大帝}打成2血。在接下来的死亡阶段,自爆绵羊的亡语将大帝打成0血,然后女王的亡语偷走对手的\card{暴风城勇士}。光环不会更新,你的大帝还是会死掉。如果调换自爆绵羊和女王的入场顺序,结果没有变化。

\vspace{\parskip}
\example 对手操控\card{蛛魔拆解者},你使用\card{始生保护者},抽到\card{远古诅咒}。由于抽到牌并不会进行光环更新,始生保护者召唤一个随机的4费随从。
\example 对手操控\card{蛛魔拆解者},你使用\card{始生保护者},抽到\card{巨龙的蜡烛}。巨龙的蜡烛召唤蜡烛巨龙导致光环更新,此时巨龙的蜡烛仍然在手牌中,因此它变为7费。最后始生保护者召唤一个7费随从。
\example 对手操控\card{蛛魔拆解者},你使用\card{始生保护者},抽到\card{巨龙的蜡烛},蜡烛又抽到第二张蜡烛。召唤第二个巨龙的时候会进行第二次光环更新,此时第一张蜡烛已经离开手牌,不再受蛛魔光环影响。最后始生保护者召唤一个5费随从。

\subsection{尚未解决的问题}
事实上,关于光环更新仍有许多问题有待进一步验证:

\subsubsection{所有光环的更新时机是否相同?}
上文的例子中仅测试了\card{玛尔加尼斯}、\card{暴风城勇士}等较为简单的光环,因此我们无法断定其他光环的表现也相同。事实上,\card[ICC_829p]{天启四骑士}的消灭效果就是一个典型的不相同的示例。

\example 你操控三个天启骑士,且第四个天启骑士在你的回合开始时苏醒。此时光环不会更新,对手英雄不会被消灭。使用英雄技能、发动攻击依然不会导致对手英雄被消灭。但如果你召唤一个随从或使用一张牌,光环会更新,天启四骑士会消灭敌方英雄。

尽管我们并不能完全了解天启四骑士的消灭机制,但显然它与之前玛尔加尼斯与苏醒的互动并不相同。

\subsubsection{阶段间步骤中的光环更新是否总会发生?}
在前面天启四骑士的例子中,我们提到了使用英雄技能、发动攻击不会导致对手英雄被消灭。
\begin{itemize}
    \item 这是否意味着使用技能、战斗序列中的阶段间步骤中没有光环更新?
    \item 这是否意味着在某种特定情况下的阶段间步骤中没有光环更新?
\end{itemize}

尽管最有可能的答案是天启四骑士的光环更新时机与其他不同,但我们依然无法肯定地说以上问题的答案全部为否定。事实上,我们认为阶段间步骤包括光环更新是基于以下实例:
\example \version{18.0}{}你操控\card{暴风城勇士}、\card{救赎}和受到3点伤害变为5/3的\card{冰风雪人}。对手杀死你的暴风城勇士,雪人变为4/3。接下来救赎召唤一个新的暴风城勇士,雪人变为5/4。

这个互动在之前的版本是这样的:
\example \version{}{18.0?}你操控\card{暴风城勇士}、\card{救赎}和受到3点伤害变为5/3的\card{冰风雪人}。对手杀死你的暴风城勇士,接下来救赎召唤一个新的暴风城勇士。在这个过程中没有光环更新,暴风城勇士的光环始终被认为在场。因此雪人保持5/3。

\subsection{状态更新}
与光环不同,当一个状态被添加到一个实体的时候,一般而言状态总是及时更新的。但在之前的版本中,一个实体获得自身扳机所添加的同名状态时,只有第一次会立刻更新。我们将这个bug称为\term{同名状态更新延迟}或\term{自buff bug}。

\example \version{}{9.4.0}回合结束时,如果你的\card{炎魔之王拉格纳罗斯}首先将对手的8/8\card{格鲁尔}打成0血,格鲁尔的扳机触发但无法把自己变为9/1存活。在当前版本下,格鲁尔可以变成9/1活下来。
\example \version{}{17.4.1}你操控多个2/4的\card{暴乱狂战士}并使用\card{圣光炸弹}。第一个暴乱狂战士受到2点伤害,之后的每个暴乱狂战士都只受到3点伤害。在当前版本下,之后的暴乱狂战士受到的伤害递增。

在当前版本下,所有已知的受自buff bug影响的互动均已修复,且很有可能是整体性的机制修改,而非针对个别卡牌间互动的单独修改。

\section{消灭/摧毁/弃牌/烧毁卡牌}

\term{消灭}随从或英雄、\term{摧毁}武器、\term*{弃}一张\term*{牌}\termindex{弃牌}和\term*{烧毁}一张\term*{牌}\termindex{烧毁牌}实际上是同一个效果在目标实体处于不同位置的时候产生的不同效果。
\notice 造成伤害的效果不是消灭。因此,对手牌中的实体造成伤害并不会使得它被弃掉。

\begin{itemize}
    \item 消灭和摧毁:实体应在战场上或位于奥秘区。实体的\texttt{.TO\_BE\_DESTROYED := 1} 。这个实体将在死亡检索中视为死亡并被移除。详见\nameref{death-creation-step}。
    \item 弃牌:实体应位于手牌。将这张牌从手牌移动到墓地,然后产生弃牌事件。
        \notice 当你执行「弃多张牌」的效果时,首先弃掉所有牌,然后依次结算每一张弃牌的弃牌扳机。
        \example 你操控\card{玛克扎尔的小鬼}并使用\card{末日守卫}。首先你弃两张牌,然后玛克扎尔小鬼使你抽两张牌。
        \notice 被弃掉的牌的弃牌扳机总是在最后结算,因为这属于手牌扳机。详见\nameref{trigger-order-in-different-zone}。
        \example 你操控六个随从,其中包括\card{人偶大师多里安}、\card{玛克扎尔的小鬼}和\card{咆哮魔}。咆哮魔受到伤害使你弃掉\card{镀银魔像}。玛克扎尔的小鬼首先触发并抽到\card{死亡之翼}。人偶大师触发并召唤一个死亡之翼的1/1的复制填满你的场地。镀银魔像无法召唤自己。
    \item 烧毁卡牌:实体应位于牌库。将这张牌从牌库移动到墓地。
\end{itemize}

一个消灭效果作用到手牌的例子是:
\example 你对友方\card{镀银魔像}使用\card{恶魔之箭}。你的\card{紫罗兰教师}触发召唤1/1,再触发\card{飞刀杂耍者}射中对方\card{自爆绵羊},绵羊击杀友方\card{阿努巴尔伏击者}将镀银魔像弹回手牌。结算时你弃掉镀银魔像并召唤它。

一个烧毁效果作用到场上的例子是:
\example 你操控\card{灵魂歌者安布拉}和四个其它随从,并使用\card{亡者大军}。首先召唤\card{灰熊守护者}I,它触发亡语召唤牌库中的已经被\card{班纳布斯}变为零费的灰熊守护者II。接下来需要召唤/烧毁的卡牌是就灰熊守护者II,但由于你已经满场,亡者大军将其烧毁,即消灭了灰熊守护者II。
\notice 上面的例子表明了亡者大军的结算是在效果一开始就选定了牌库顶的五张牌,然后根据「这张牌的类型」和「你是否已经满场」依次决定将它们烧毁还是召唤到场上。类似地,如果例子中的两个灰熊守护者换成\card{疯狂的科学家}和\card{镜像实体},镜像实体也会被亡者大军烧毁,在动画上表现为奥秘被摧毁(与\card{照明弹}动画类似)。

\section{抽牌}
\term{抽牌}即将牌库顶的牌移动到手牌。

当你执行「抽一张牌」的效果前,游戏首先检查你的牌库是否为空。若是,则改为获得一次疲劳。再检查你的手牌是否已满。若是,则改为烧毁一张牌。
\notice 疲劳不属于抽牌。烧毁卡牌既不属于抽牌,也不属于弃牌。相应的扳机均不会触发。

当你执行「抽多张牌」的效果时,游戏首先计算总共抽牌数,然后执行对应次数的「抽一张牌」的效果。
\notice 这意味着当所有抽牌扳机结算完毕后,才执行下一次抽牌。
\example 你有9张手牌且操控\card{克洛玛古斯}。你使用\card{寒光智者}。抽第一张牌时,克洛玛古斯触发并填满你的手牌。第二张牌会被烧毁。
\notice 这意味着抽牌效果可以嵌套在抽牌效果里。
\example 你使用\card{怨灵之书},第一张抽到\card{地雷}。地雷生效并对你造成5点伤害,然后抽到\card{烈焰风暴}。烈焰风暴是由地雷的效果抽到的,因此不会被弃掉。然后怨灵之书再结算第二、第三张抽牌。
\example 你使用处于流放位置的\card{古尔丹之颅},第一张抽到\card{灵魂残片}。残片生效为你恢复2点生命值,然后抽到古尔丹之颅。古尔丹之颅是由灵魂残片抽到的。因此,它不会被减费。

当你抽牌时,若多个抽牌扳机在场,它们按顺序结算。
\notice 一张卡牌的抽到时效果属于手牌扳机,但它总是早于其他场上抽牌扳机。
\example \version{}{某版本} 你操控\card{人偶大师多里安}并抽到\card{破海者}。人偶大师首先触发并召唤一个1/1的复制,然后破海者的效果触发对其造成1点伤害。
\example \version{某版本}{} 你操控人偶大师多里安并抽到破海者。破海者效果首先触发对友方随从造成1点伤害,然后人偶大师触发召唤一个1/1的破海者。
\example \version{}{17.4.1}副玩家的你操控2血的\card{勇敢的记者}。对手抽到\card{烈焰巨兽}。烈焰巨兽首先触发并对记者造成致命伤,然后记者无法触发其扳机。

描述为「抽一张牌,并执行E」的效果,会在抽牌结算完毕后才执行E,无论E是否与抽到的这张牌有关。
\example 你操控\card{人偶大师多里安}并使用\card{侏儒实验技师}抽到\card{死亡之翼}。人偶大师触发并召唤一个死亡之翼的1/1的复制,然后侏儒实验技师才会将死亡之翼变为\card[GVG_092t]{小鸡}。
\example 你使用\card{狗头人图书管理员}抽到\card[GVG_056t]{地雷}。地雷生效并对你造成10点伤害,然后抽到\card{小型法术紫水晶}。接下来狗头人的战吼对你的英雄造成2点伤害,你的紫水晶可以升级。

\section{变形}

当一个实体被变形,它的\texttt{.cardId} 变为新的 ID,其余的属性变为新实体的对应属性。如果是变形成另一个实体的复制,则遵从复制的相关规则。变形不是死亡,因此不触发死亡扳机,也不会增加死亡记录。变形也不是召唤,因此不触发任何召唤时或召唤后扳机。
\notice \version{}{11.2}曾经的召唤回溯会对非法术变形触发。在召唤回溯被消除并修改为正式的召唤时扳机之后,它不再与变形互动。详见\nameref{rule-update:11.2}。

\example 你操控\card{血沼迅猛龙}、\card{诅咒教派领袖}、\card{夜色镇执法官}、\card{飞刀杂耍者},然后对迅猛龙使用\card{变形术}。迅猛龙被变形为绵羊,然后什么也不发生。

\subsection{替换}

当一个实体或一些实体被替换时,他们将被移除,然后在相应的区域创建新的实体,如\card{窃魂者阿扎莉娜}。替换与变形的区别在于替换改变\texttt{.EntityID} ,而变形不改变。这意味着一张牌被替换后可以被\card{魔网操控者}减费而变形不能。
\notice 一些效果被描述为替换,但实际上是变形。如\card{黄金狗头人}。
\example 你用\card{黄金猿}得到的传说随从可以被魔网操控者减费,但用黄金狗头人得到的就不行。

\subsection{历史}

\version{}{11.2}曾经的变形与现在不同。那个时候的变形是将原实体除外,并在原位置放一个新的实体。因此,所有指定该随从为目标的效果都不能对新的随从生效。详见\nameref{rule-update:11.2}。
\notice 此时为了解决变形与使用后扳机与召唤后扳机的互动问题,在一个随从战吼变形或者被\card{变形药水}变形之后,会引入一个完成阶段,对新随从触发。此时,如果该随从是战吼变形,那么这个额外的完成阶段与战吼之间是没有死亡结算的。

\section{休眠}
当一个实体进入休眠状态时,它保留自身具有的所有状态和标签。当它苏醒时,这些状态和标签不改变。与变形类似,休眠不是死亡,因此不触发死亡扳机,也不会增加死亡记录。苏醒也不是召唤,因此不触发任何召唤时或召唤后扳机。此外,随从苏醒时会获得「召唤失调」。

\example 你使用\card{暴风城骑士}并令其攻击对手英雄。然后使用\card{玛维·影歌}使暴风城骑士休眠。在暴风城骑士苏醒的回合,它不能再发动攻击。
\example 你操控两个\card{飞刀杂耍者}且使用玛维·影歌休眠先入场的飞刀。在飞刀苏醒之后,它依然比后入场的飞刀先触发扳机。
\exception 具有「亡语:进入休眠状态」的随从,在苏醒时召唤自己一个新的复制,且不保留任何状态或标签。这包括\card{“尸魔花”瑟拉金}和\card{卢森巴克}。

\section{移动}
\label{move}

当一个随从从其他区域移动到场上,或在场上控制权转移时,一般会获得召唤失调。
\exception \card{暗影狂乱}、\card{疯狂药水}和\card{变节}不会使随从获得召唤失调。

\subsection{移动到已满的区域}
\label{move-to-full-zone}

当一个实体尝试移动到一个已满的区域时,一般会发生:

\begin{itemize}
    \item 从牌库移动到场上(如\card{死亡领主}等)时,取消移动。
        \exception 根据卡牌描述,\card{瓦里安·乌瑞恩}改为抽牌,\card{亡者大军}改为烧毁该牌。
    \item 从手牌移动到场上(如\card{先祖召唤}等)时,改为移动到墓地。它不是弃牌。
        \notice 通常的满场移动,例如\card{卑劣的脏鼠}会在满场时取消移动,这看起来是满场移动的默认情况。但下述例子证明了并非如此,它应该是在移动之前做了预先检测。
        \example 你操控\card{扭曲巨龙泽拉库}和其它五个随从,并使用被\card{戈霍恩,鲜血之神}抽上来的暗影猎手沃金。在沃金进场前你先失去生命,导致场上有七个随从。然后沃金进场并结算战吼,将一个随从移回手牌。手牌中即将进场的随从进入墓地而不是留在手牌不动。
    \item 在场上的随从控制权转移(如\card{精神控制技师}等)时,改为消灭该随从。
    \example 你操控六个随从,其中包括\card{布莱恩·铜须}。对手操控五个随从。你使用精神控制技师依次偷取对方后入场的\card{战利品贮藏者}和先入场的\card{发条侏儒},它们被直接移除。接下来的死亡阶段,对手先获得一张零件牌,再抽一张牌。
\end{itemize}

\begin{itemize}
    \item 从牌库移动到手牌(如「抽一张牌」等)时,改为烧毁该牌。
    \item 从牌库移动到对手手牌(如\card{死亡之握}等)时,改为烧毁该牌。
    \item 从场上移动到手牌(如\card{消失}等)时,改为消灭该随从。
    \item 在手牌控制权转移(如裂心者伊露希亚等)时,改为移动到墓地。它不是弃牌。
        \example 对手因\card{瓦迪瑞斯·邪噬}的效果拥有12张手牌。你使用裂心者伊露希亚。你只能获得对手的前10张手牌,另外的两张会被移动到墓地。尽管在动画上显示为弃牌,你和对手的\card{小鬼骑士}均不触发。
\end{itemize}

\begin{itemize}
    \item 从场上移动到牌库(如\card{埋葬}等)时,改为消灭该随从。
    \item 从手牌移动到牌库(如\card{情势反转}等)时,改为移动到墓地。它不是弃牌。
\end{itemize}

\begin{itemize}
    \item 从牌库移动到奥秘区(如\card{疯狂的科学家}等)时,取消移动。
    \item 在奥秘区控制权转移(如\card{科赞秘术师}等)时,改为直接摧毁该奥秘。
\end{itemize}

\subsection{伪区域移动}

有一些描述为区域移动的效果实际上并不是区域移动,例如:

\begin{itemize}
    \item 死亡扳机,这包括:
    \begin{itemize}
        \item 随从/武器自身的区域移动类亡语,例如\card{鼬鼠挖掘工}、\card{玛洛恩}、\card{骷髅骑士}、\card{派烙斯}、\card{弑君}等;
            \exception \card{阿努巴拉克}和\card{莫瑞甘博士}的亡语是真区域移动。
        \item 死亡触发的某些奥秘,例如\card{战术撤离}和\card{诈死};
        \item \card{灵魂回响}添加的亡语;
    \end{itemize}
    \item 一些特定的法术,例如\card{回收}和\card{极恶之咒}。
\end{itemize}

伪区域移动实际上是将原实体除外并在指定位置创建一个复制(可以通过\card{魔网操控者}证实)。它们被特殊处理以保证与真正的区域移动类似,例如\card{瑞文戴尔男爵}不会使伪区域移动类死亡扳机触发两次。
\exception \card{鼬鼠挖掘工}的亡语格外不同。它首先将自己洗入对手牌库,然后再将自己移回你的除外区,最后向对手牌库中洗入一张鼬鼠。这可能是为了动画能够正确显示鼬鼠洗入对方牌库。
\example 双方都操控\card{强能雷象},你的鼬鼠死亡。首先鼬鼠将自己洗入对面牌库,此时\emph{对方的}雷象触发。然后它除外自身,并将一张复制洗入对方的牌库。此时它的操控者是你,所以你的雷象也触发。最后一共触发了两次。

\notice 伪区域移动与直接复制不同,它将原实体移除。这使它们与\card{装死}正确互动。
\notice 一些看起来是「起死回生」的效果,例如\card{救赎}或\card{殊死一搏},是召唤原随从的一个复制。它不是任何形式的区域移动。

当一个随从的死亡步骤中有多个区域移动类死亡扳机时,仅有第一个会生效。
\notice 这包括真区域移动、伪区域移动和休眠类效果。
\example 你操控具有\card{丛林之魂}的\card{“尸魔花”瑟拉金}和\card{诈死}。对手消灭你的尸魔花。在接下来的死亡阶段,尸魔花首先进入休眠,然后召唤一个\card[EX1_158t]{树人}。诈死不会触发。

\section{复制}

无论是「使另一个实体变形为某实体的复制」还是「创建一个新实体,这个实体是某实体的复制」,它们都遵循相同的规则。

源实体的属性都会被复制,除了:
\begin{itemize}
    \item 影响该实体的光环不会复制。
        \example 你的对手操控一个\card{暴风城勇士}和一个1血的\card{血沼迅猛龙}。你使用\card{无面操纵者}复制迅猛龙。复制的迅猛龙是0血,然后死亡。
    \item 目标实体的操控者、区域及位置不会复制,而是由该复制效果指定。
    \item 新实体进入战场的时间不会被复制。这意味着\card{无面操纵者}复制之后仍然有召唤失调。
\end{itemize}

源实体的状态是否被复制遵循移动/复制状态规则。复制的新状态附加在目标实体上。
\notice 复制的状态的操控者仍然是原操控者。
\example 对手对其\card{血沼迅猛龙}使用\card{智慧祝福}。迅猛龙攻击时,对手抽牌。你使用\card{无面操纵者}目标迅猛龙。无面的迅猛龙攻击时,仍是对手抽牌,因为状态的操控者没有改变。
\example \version{}{17.4.1}副玩家你手中的\card{变色龙卡米洛斯}在你的回合开始时变形成为对手的已变形的\card{变形卷轴}的复制。在你回合开始时,变色龙身上有变形卷轴的继续变形和变色龙自己的继续变形。其中变形卷轴的继续变形的操控者是作为主玩家的对手,因此它先触发,变成一张随机法师法术牌。接下来,由于变形清除状态,变色龙的自身的「继续变形」被清除。在接下来的每个回合,变色龙都只会按照变形卷轴的规则继续变形。如果你是主玩家,那变色龙的变形会先触发。在接下来的每个回合,变色龙依然按照自身的规则继续变形。
\notice \version{12.0?}{14.0?}在上述例子中,虽然那个状态的操控者是对手,它仍然是在你的回合开始时变形。但是在旧的版本中,这个状态是在\term{状态操控者}的回合开始时变形的。也就是说,旧版本的上述例子中,如果你直接结束回合,变形卷轴的继续变形会在对手回合开始触发并覆盖变色龙的继续变形,而这与主玩家就没有关系了。这个例子于 12.0 版本初次发现,于 14.0 版本发现已经被修改。
\notice \version{17.4.1}{}在当前版本下,由于变形卷轴「继续变形」的状态入场早于变色龙「继续变形」的状态,变色龙在接下来的每个回合中总会按照变形卷轴的规则继续变形。

\subsection{移动/复制状态规则}

在不同的区域移动时,实体是否保留状态遵循以下的规则:
\begin{itemize}
    \item 当实体从牌库移动到手牌、手牌移动到场上、牌库移动到场上时,保留所有状态;
    \item 当实体从手牌移动到牌库、场上移动到手牌、场上移动到牌库,以及从任何区域进入或离开墓地时,不保留任何状态。
        \exception 交易一张牌会保留所有状态。参见\card{tradeable}。
    \item 当一个「入场时休眠」的实体从手牌移动到场上、牌库移动到场上时,进入休眠状态。
\end{itemize}

\notice 当实体从手牌或牌库到场上时,所有的费用状态都会被移除。这保证了实体与\nameref{evolve-and-devolve}的互动正确。
\notice \version{}{18.0?}在之前的版本中,当实体从手牌移动到场上时,沉默效果会被移除(保证其战吼和本身具有的亡语可用),伤害会被清除(保证随从入场时满血)所有附加的亡语也会被移除(\card{瓦兰奈尔}除外)。但在当前的版本中,这些都不会被移除或清除。
\exception 从手牌移动到墓地可能并不会清除所有状态。如果你通过某种手段使手牌中的骨网之卵获得\card{唤尸者}附加的亡语。它在被弃掉时会触发唤尸者的亡语召唤一个自身的复制。然而这个例子并不能完全证实上面的观点:另一种可能的解释是骨网之卵在被弃掉时会记录自身拥有哪些亡语。目前尚无其他卡牌可以证实/证伪这个观点。注意\card{镀银魔像}实质上是召唤另一个镀银魔像,而不是自身的复制。

在不同的区域之间复制时,实体是否复制状态遵循以下的规则,这包括某实体成为复制和创建一个复制的情况:
\begin{itemize}
    \item 当实体从一个区域复制到相同区域时,保留所有状态;
    \item 当实体从一个区域复制到不同区域时,状态保留与否与移动时状态保留与否相同。
    \item 当一个「入场时休眠」的实体从手牌复制到场上、牌库复制到场上时,进入休眠状态。
    \item 当一个「入场时休眠」的实体从场上复制到场上时,不进入休眠状态。
\end{itemize}
\exception \card{塔达拉姆王子}和\card{阴暗的人影}复制「入场时休眠」的实体时,进入休眠状态。

\section{伤害与治疗}
\label{damage-healing}

伤害与治疗的结算流程如下:
\begin{enumerate}
    \item 改变伤害量或治疗量的效果生效。
    \begin{enumerate}
        \item 使伤害量增加的效果首先生效,如\nameref{spell-damage}、\card{英雄之魂}等。
        \item 使伤害量或治疗量加倍的效果其次生效,如\card{先知维伦}、\card{发条机器人}、\card{水晶工匠坎格尔}等。
    \end{enumerate}
    \item \card{奥金尼灵魂祭司}等效果生效,将治疗改为伤害。
    \item 若伤害量为0或目标角色具有\nameref{immune},防止此伤害。
    \item \term{预伤害扳机}列队结算。
    \item 若目标具有\nameref{divine-shield},则移除圣盾并将伤害量改为0。
    \item 伤害减少实体的护甲或生命值减少;治疗增加实体的当前生命值。
    \item 若受伤害的实体受\card{命令怒吼}效果影响,则改变伤害量使实体受伤后的生命值不小于1。
    \item 若伤害减少了目标的护甲或生命值,\term{伤害扳机}列队结算;若治疗使目标的生命值改变,\term{治疗扳机}列队结算。
\end{enumerate}

\notice 在游戏中有「每当你受到伤害」和「在你受到伤害后」两种伤害扳机的描述。它们实际上没有区别,按照入场顺序列队结算。
\notice \version{}{14.2}在之前的版本中,\card{奥金尼灵魂祭司}等效果早于改变伤害量或治疗量的效果。这导致了治疗被转化为伤害后无法受到\card{水晶工匠坎格尔}加成。

\card{小鬼爆破}召唤\card[GVG_045t]{小鬼}的数量和\card{岩浆爆裂}召唤\card{熔岩元素}的数量和在一开始就决定好了,与实际造成的伤害无关。注意若小鬼爆破的伤害被免疫防止或圣盾改为0时不召唤小鬼,但岩浆爆裂无论如何都召唤元素。
\example 你使用\card{命令怒吼},并对你1血的\card{苦痛侍僧}使用小鬼爆破。小鬼爆破的伤害量为4点。但因命令怒吼的效果,苦痛侍僧实际生命值没有减少,因此你不能抽一张牌。接下来,你召唤4个小鬼。
\example 你使用命令怒吼,并令你2血的\card{闪金镇步兵}攻击对手\card{血虫}。血虫只能吸1点血。

\subsection{预伤害扳机}
\label{predamage-trigger}

在伤害生效前,部分扳机可以改变伤害。这些扳机被称作\term{预伤害扳机}。

响应英雄受到伤害的预伤害扳机结算顺序如下:
\begin{enumerate}
    \item 改变受伤目标的扳机\card{博尔夫·碎盾}结算,并将此伤害改为随从伤害。
        \notice 碎盾触发后受伤目标不再是你的英雄,其他的英雄预伤害扳机都不能再触发,而随从预伤害扳机可以触发。若你操控多个碎盾,仅有最先入场的会生效。
    \item 改变伤害量的扳机\card{复活的铠甲}、\card{诅咒之刃}和\card{虚触侍从}列队结算。
    \item \card{埃辛诺斯壁垒}触发。
    \item \card{寒冰屏障}触发。
\end{enumerate}

响应随从受到伤害的预伤害扳机结算顺序如下:
\begin{enumerate}
    \item 改变受伤目标的扳机\card{钳嘴龟盾卫}结算。
    \item 改变伤害量的扳机\card{莫尔葛工匠}结算。
    \item 改变伤害量的扳机\card{月牙}结算。
\end{enumerate}

一次伤害可以被数个钳嘴龟盾卫多次转移,但每个钳嘴龟盾卫只能对一次伤害触发一次。直观上,如果场上连续存在多个钳嘴龟盾卫,伤害会传导到最左或最右的盾卫身上。
\notice 在一次伤害的传导中,若目标随从两侧都有盾卫且均未触发过,先入场的会生效。
\example 你操控从左到右依次入场的七个钳嘴龟盾卫,按入场顺序记为A、B、C、D、E、F 和G。对手使用\card{魔爆术}。
\begin{itemize}
    \item A将受到的伤害依次传导到B、A。
    \item B将受到的伤害依次传导到A、B、C、D、E、F、G。
    \item C将受到的伤害依次传导到B、A。
    \item D将受到的伤害依次传导到C、B、A。
    \item E将受到的伤害依次传导到D、C、B、A。
    \item F将受到的伤害依次传导到E、D、C、B、A。
    \item G将受到的伤害依次传导到 F、E、D、C、B、A。
\end{itemize}

\section{死亡记录}

尽管游戏中存在墓地区,但复活效果并不是将墓地区的卡牌移回场上,其实质上是检测死亡记录,并召唤死亡记录中相应随从的复制。相关规则如下:

当一个实体被移除时,死亡记录会记录此次死亡事件,包括相应随从的卡牌编号\texttt{.CardID} 、控制者和回合数。
\notice 如果一个实体因爆牌、弃牌等效果进入墓地区,它不会被加入死亡记录。

如果一个实体死亡多次,每一次死亡事件都可以被加入死亡记录。
\example 你使用\card{阿努巴拉克}后对手将其消灭,如此重复三次。你使用\card{恩佐斯},其战吼复活三个阿努巴拉克。

任何效果都不会将死亡记录中的死亡事件删去。这包括「此随从被复活效果选中」、「此随从最终不在墓地区」等。
\example 你的死亡记录中仅包括一个\card{大法师瓦格斯}。你使用\card{复活术}复活大法师。回合结束阶段,大法师施放的复活术还可以复活一个大法师。
\example 你的死亡记录中仅包括一个\card{克苏恩}。你操控\card{布莱恩·铜须}并使用\card{厄运召唤者},其战吼将两个克苏恩洗回你的牌库。
\example \version{}{旧版本} 你的\card{阿努巴拉克}死亡,其亡语将它移回手牌。你使用复活术依然可以召唤一个阿努巴拉克。

除了\card{克尔苏加德}按入场顺序复活随从外,其余复活效果都在符合条件的随从中随机选择。

复活效果只检测死亡记录中随从的原始状态。
\example 你对你的\card{小精灵}使用\card{剑龙骑术},然后使用\card{食肉魔块}吃掉小精灵。对手对你的模块使用\card{灵魂虹吸},模块亡语召唤两个1/1的小精灵,然后对手使用\card{扭曲虚空}清场。接下来你使用\card{恩佐斯},其战吼只能复活一个模块,且这个模块的亡语不能召唤任何随从。

不同随从可以具有相同的卡牌编号,对于这些随从,死亡记录会记录更详细的内容以正确复活。
\example 你使用\card{合成僵尸兽}选择的第一个野兽和其组成的僵尸兽具有相同的卡牌编号。当僵尸兽死亡时,死亡记录会记录组成它的两种野兽\texttt{.MODULAR\_ENTITY\_PART\_1}与\texttt{.MODU\-LAR\_ENTITY\_PART\_2} 。
\example \card[ICC_047t]{命运织网蛛}抉择后的两种不同形态具有相同的卡牌编号\texttt{.CardID = ICC\_047t} 。当命运织网蛛死亡时,死亡记录会记录它的秘密抉择选项\texttt{.HIDDEN\_CHOICE} 。
\notice 抉择前的\card{命运织网蛛}卡牌编号\texttt{.CardID = ICC\_047} ;抉择同时生效的\card[ICC_047t2]{命运织网蛛}卡牌编号\texttt{.CardID = ICC\_047t2} 。他们均是不同的实体。
\example 身材不同的\card{老虎}和被教会不同法术的\card{德鲁斯瓦恐魔}也具有相同的卡牌编号。

具有相同卡牌编号的不同随从会被\card{小型法术钻石}视为相同随从。他们不能被同时复活。

\section{法力值消耗}
\label{cost}

当多个效果改变一张牌的法力值消耗时,它们按入场顺序排列。

状态类效果(如\card{碎枝}、\card{露娜的口袋银河}等)的入场时间点为此状态添加到目标卡牌上的时间点。
\notice 部分效果并不是直接向卡牌添加状态来实现的,这类效果也符合本规则。如\card{洛欧塞布}的战吼向对手添加「本回合你的法术增加5费」,这一效果影响对手手中所有法术。它的入场时间是洛欧塞布战吼的时机。
\example 对手操控\card{法术共鸣}。你依次使用洛欧塞布和\card{寒冰箭}。接下来对手的回合里,对手法术共鸣获得的寒冰箭为0费。
\exception \card{异教低阶牧师}的表现与洛欧塞布不一致。
\example 对手操控\card{法术共鸣}。你依次使用异教低阶牧师和\card{寒冰箭}。接下来对手的回合里,对手法术共鸣获得的寒冰箭为1费。

光环类效果(包括\card{艾维娜}等持续型光环和\card{暮陨者艾维娜}等消耗型光环)改变一张牌的费用实质上也是通过添加一个费用改变状态来实现的。其状态的入场时间点如下:
\begin{itemize}
    \item 当一个具有费用光环的随从入场时,它会为手牌中相应的卡牌添加费用改变状态。此状态的入场时间点即为具有光环的随从的入场时间点。
        \example 你持有被\card{索瑞森大帝}减为2费的\card{炎爆术}并使用纳迦海巫。炎爆术变为5费。回合结束索瑞森大帝的效果触发,将炎爆术减为4费。对手杀死你的纳迦海巫,你的炎爆术变为1费。
    \item 当一个具有消耗型费用光环的随从刷新其光环时,它会为手牌中相应的卡牌添加费用改变状态。此状态的入场时间点即为光环刷新的时间点。其中:
    \begin{itemize}
        \item 暮陨者艾维娜在回合开始时刷新光环。即使你本回合没有使用牌,暮陨者的效果也会在回合结束时失效,在你下个回合开始时刷新。
        \item \card{小个子召唤师}和\card{卡雷苟斯}在回合开始时刷新光环。如果你本回合没有使用牌,回合结束时小个子召唤师和卡雷苟斯的效果\emph{不会}失效,并且在你的下回合开始时,小个子召唤师和卡雷苟斯就\emph{不会}刷新光环。
    \end{itemize}
        \example 你操控暮陨者艾维娜并消灭你的\card{沟渠潜伏者}。其亡语召唤了纳迦海巫。此时你没有使用过任何卡牌,你的所有卡牌均变为5费。下个回合开始时,暮陨者的效果刷新,你的所有卡牌变为0费。接下来你抽到一张死亡之翼。暮陨者和纳迦按入场顺序为其添加状态,因此死亡之翼为5费。
        \example 你操控小个子召唤师并消灭你的沟渠潜伏者。其亡语召唤了纳迦海巫。此时你没有使用过任何随从,你的所有随从均变为5费。下个回合开始时,小个子召唤师的效果\emph{不}刷新,此时你手牌中的所有随从仍为5费。你使用\card{小精灵}。再下个回合开始时,小个子召唤师的效果刷新,此时你手牌中的所有随从变为4费。
        \notice \version{}{?}在之前的版本里,卡雷苟斯没有对应的费用改变状态,其效果的入场时间点始终为卡雷苟斯入场时。

    \item 当一张牌进入手牌时,场上所有相应的光环为其添加费用改变状态。此状态的入场时间点即为这张牌进入手牌时。如果场上有多个光环为其添加费用改变状态,它们按光环具有者的入场顺序添加。
        \example 你使用\card{纳迦海巫}和露娜的口袋银河,然后从牌库里抽到一张\card{死亡之翼}。纳迦海巫的效果在你抽到死亡之翼时才生效,因而晚于口袋银河。死亡之翼最终为5费。
    \item 更准确地讲,当一张牌进入手牌时,由光环所添加的费用改变状态的入场时间点为这张牌进入你手牌后下一次光环更新的时间点。
        \example 你操控\card{影舞者索尼娅}和\card{机械跃迁者}(无论入场顺序),然后消灭你的\card{机械异种蝎}。在接下来的死亡阶段,你首先获得一个机械蝎的复制并因索妮娅的效果变为1费,然后在接下来的光环更新中跃迁使其变为0费。
        \example 你操控\card{凯尔萨斯·逐日者}且使用本回合第二张法术自然研习。在法术的使用阶段,凯尔萨斯刷新其光环,你手牌中的法术变为1费。自然研习先发现一张法术牌,再为你手牌中所有的法术牌减1费。因此,你之前的法术变为0费,新发现的法术减1费。法术结算阶段结束,在光环更新步骤中凯尔萨斯为你新发现的法术添加状态使其变为1费。因此,你之前的法术为0费,新发现的法术为1费。你送掉场上的\card{战利品贮藏者}并抽到一张法术牌。在光环更新步骤中凯尔萨斯和自然研习按顺序为其添加状态,因此新抽到的法术也为0费。
        \example 你操控艾维娜和\card{人偶大师多里安},并使用\card{热情的探险家}抽到一张死亡之翼。人偶大师的效果首先生效,召唤一个1/1的复制。此时光环更新,艾维娜的效果使死亡之翼变为1费。最后,探险家的战吼使死亡之翼变为5费。如果没有人偶大师多里安,死亡之翼首先会因热情的探险家变为5费,然后在接下来的光环更新中变为1费。
        \example 将上述例子中的人偶大师多里安换成\card{克洛玛古斯},那么热情的探险家抽到死亡之翼时,克洛玛古斯的效果首先生效,将一张死亡之翼的复制加入你的手牌。此时光环更新,艾维娜的效果使两张死亡之翼都变为1费。最后,探险家的战吼使前一个死亡之翼变为5费。
\end{itemize}

\version{}{?}在之前的版本中,\card{召唤传送门}的效果具有最高优先级,先于一切其他改变费用的效果。在当前版本中,它已经没有特殊优先级。
\example 无论你以何顺序使用召唤传送门和\card{机械跃迁者},你手牌中的\card{蜘蛛坦克}都是0费。在当前版本中,蜘蛛坦克的费用取决于入场顺序。

\card{异教低阶牧师}、\card{安娜科德拉}等牌具有最低优先级,他们将在其它费用效果结算后生效。

大部分自减费光环也具有最低优先级。
\example 你具有25点生命。你使用\card{热情的探险家}抽到一张\card{熔核巨人}。巨人首先因探险家效果变为5费,然后因其自减费效果变为0费。
\notice \card{蛛魔先知}和\card{荒野骑士}不是自减费光环。
\exception 某些卡牌的自减费光环不具有最低优先级。

当一个效果提及卡牌的费用时,绝大多数情况下取实际费用。
\example 你依次使用\card{“践踏者”班纳布斯}和\card{“丛林猎人”赫米特}。你牌库中所有随从牌都会被摧毁。
\exception \card{游荡恶鬼}、\card{巨龙召唤者奥兰纳}和\card{灰贤鹦鹉}取卡牌的原始费用。

与攻击力类似,如果费用的最终计算结果为负数,它将被调整为0。
\example 你手牌中有一个被\card{索瑞森大帝}减了五回合费用的\card{小精灵}。你抽到\card{雪鳍企鹅},然后使用\card{模拟幻影}。哪个随从被复制是不确定的。

\section{法力水晶}
双方玩家各拥有一系列与法力水晶相关的标签:
\begin{itemize}
    \item \term{最大法力上限}\texttt{.MAXRESOURCES}:你的法力上限可以达到的最大值,通常为10。在部分冒险或乱斗中可能不为10。
    \item \term{当前法力上限}\texttt{.MAXRESOURCES}:你当前的法力上限,你每个回合开始时增加1,但不超过最大法力上限。
    \item \term{当前已消耗法力值}\texttt{.RESOURCES\_USED}:本回合你因使用卡牌或上回合的过载消耗掉的法力值。
    \item \term{临时法力值}\texttt{.TEMP\_RESOURCES}:本回合你临时获得的法力值,如通过\card{激活}。
    \item \term{下回合过载值}\texttt{.OVERLOAD\_OWED}:将在下个回合生效的过载。
    \item \term{本回合过载值}\texttt{.OVERLOAD\_LOCKED}:本回合你具有的过载。
\end{itemize}
\notice 「当前法力值」不是一个标签,而是通过「当前法力上限」+「临时法力值」+「当前已消耗法力值」计算得到的。

这些标签在以下情况中变化:
\begin{itemize}
    \item 在你的回合开始时,将你的「当前法力上限」增加 1(不超过最大法力上限),然后将你的「当前已消耗法力值」和「本回合过载值」置为「下回合过载值」,再将「下回合过载值」清零。
    \item 在你的回合结束时,将你的「临时法力值」清零。
    \item 当你获得一个满的法力水晶时,若你的「当前法力上限」未达到「最大法力上限」,则你的「当前法力上限」增加1点,「当前已消耗法力值」不变。若达到,则你的「当前已消耗法力值」减少1点但不低于0,「当前法力上限」不变。如果「当前法力上限」与「临时法力值」之和超过「最大法力上限」,则「临时法力值」要相应地减少1点,使得它们之和不超过,然后若此时你的「当前已消耗法力值」不为0,则减少1点。动画上看来,一个临时水晶被填入了一个空的真实水晶槽。
        \example 你在6费回合使用\card{滋养}。你的当前法力上限增加为8,而当前已消耗法力值保持6不变。因此你的法力栏显示为2/8。
        \example 你在10费回合已经消耗掉了9点法力水晶,然后你使用一张0费的\card{生物计划}。你的当前已消耗法力值减少2点,你的法力栏显示为3/10。
        \example 你在7费回合已经消耗掉了6点法力水晶,你使用三张激活使你的临时法力值变为3,此时你的当前法力值为$7+3-6=4$点。然后你使用一张0费的\card{生物计划}。你的法力上限变为9,它与临时法力值之和超过了10。因此你的临时法力值减2点变为1。相应的,你的当前已消耗法力值减少2点,因此你的当前法力值为$9+1-4=6$点。从动画上来看,你法力栏最右侧的两个临时水晶被填入了两个空的真实水晶槽。
        \history \version{}{15.4.0?}在之前的版本中,当「当前法力上限」与「临时法力值」之和超过「最大法力上限」时,多出的临时法力值直接消失而你的当前已消耗法力值不会减少。例如在上述例子中,你的临时法力值减2点变为1。而你的当前已消耗法力值不减少,因此你的当前法力值仍为$9+1-6=4$点。从动画上来看,你法力栏最右侧的两个临时水晶像是被挤掉了一样。
    \item 当你获得一个空的法力水晶时,你的「当前法力上限」和「当前已消耗法力值」均增加1点。
        \example 你在3费回合使用\card{野性成长}。你的当前法力上限和当前已消耗法力值均增加1。因此你的法力栏显示为0/4。
    \item 当你获得一个临时法力水晶的时候,如果「当前法力上限」与「临时法力值」之和未达到「最大法力上限」,你的「临时法力值」增加1点。若达到,则你的「当前已消耗法力值」减少1点,但不低于0。
        \example 你在9费回合已经消耗掉了9点法力水晶,并使用两张激活。你首先获得1点临时法力值,然后当前已消耗法力值减少1点,因此你的法力栏显示为2/9。回合结束时,你的临时法力值清零,你的法力栏显示为1/9。
        \history \version{15.4.0?}{18.2.0}在之前的某一段时间中,获得临时水晶的逻辑是:当你试图获得一个临时水晶时,若你「当前已消耗法力值」不为0,则减少1点;否则才获得一个临时水晶。例如你在6费回合依次使用\card{林地守护者欧穆}和\card{激活},你的法力栏只能变为6/6而非7/6。
    \item 当你失去一个法力水晶时,你的「当前法力上限」和「当前已消耗法力值」均减少1点,但「当前已消耗法力值」不低于0。
        \example 你在10费回合使用一张0费的\card{恶魔卫士},你的当前法力上限减少1点,因此你的法力栏显示为9/9。
        \example 你在10费回合已经消耗掉了1点法力水晶,并使用一张0费的\card{恶魔卫士},你的当前法力上限和当前已消耗法力值均减少1点,因此你的法力栏显示为9/9。
    \item 当你复原你的法力水晶时,你的「当前已消耗法力值」减少相应的数值,但不低于0。
        \notice \card{遗忘之王库恩}的效果实质上是复原10个法力水晶,即使你当前法力上限尚未达到10。
        \example 你在9费回合过载了25点法力水晶,并使用两张0费的遗忘之王库恩。你的当前已消耗法力值减少20点,为5。因此你的法力栏显示为4/9。
    \item 当你的「当前法力上限」与「临时法力值」之和达到「最大法力上限」时,某些卡牌会产生不同的结算。
    \begin{itemize}
        \item  野性成长、\card{妙手空空}、\card{星界沟通}和过度生长改为获得\card{法力过剩(CS2 013t)}。
            \example 你在8费回合获得了2点临时法力值,并使用一张0费的野性成长。你得到一张法力过剩。
        \item 其他获得空的法力水晶的牌改为什么都不做,如\card{贪婪的林精}。它们不会挤掉你的临时法力值。
            \example 你在8费回合获得了2点临时法力值,并消灭了你的贪婪的林精。什么也不会发生。
    \end{itemize}
\end{itemize}

\section{额外回合}
目前游戏中有两个获得额外回合的手段:时空扭曲和坦普卢斯。
\begin{itemize}
    \item \card{时空扭曲}的效果是在下个回合之前插入一个额外的己方回合。
    \item \card{坦普卢斯}的效果是在下个回合之前插入四个额外回合,首先是两个对手回合,然后是两个己方回合。
\end{itemize}

\notice 在下面的示例中,对手回合记作「E」,己方回合记作「O」。
\example 你先使用坦普卢斯然后使用时空扭曲,接下来的五个额外回合是 O-E-E-O-O。如果调换顺序,先使用时空扭曲后使用坦普卢斯,则接下来的五个额外回合是 E-E-O-O-O。
\example 你使用坦普卢斯,然后对手在其第一个额外回合中也使用坦普卢斯。接下来的七个额外回合是 O-O-E-E-E-O-O。

\notice 绝大多数改变对手下个回合英雄技能或某些卡牌法力值消耗的效果只生效一个回合,即在对手下个回合开始时生效,结束时失效。这包括\card{洛欧塞布}、\card{责难}、\card{持枪恶霸}、\card{异教低阶牧师}。
\example 对手法力上限为10,且持有唯一一张10费的\card{精神控制}。你依次使用洛欧塞布、坦普卢斯和\card{混乱凝视者}。混乱凝视者的战吼发动时,对手手中精神控制仍为10费,因此会被腐化。在对手的第一个额外回合中,精神控制变为15费无法使用,并在回合结束时丢弃。在对手的第二个额外回合中,洛欧塞布才失效。
\exception \card{米尔豪斯·法力风暴}和\card{破坏者}会在连续的多个对手回合中持续生效。此外,米尔豪斯·法力风暴的效果在使用后立刻生效,而不是等到对手下个回合开始才生效。

\section{动画显示}
当玩家进行一个操作时,游戏首先将这个操作的全部内容在服务器结算完毕,然后慢慢地在客户端播放动画。除非投降,游戏会展示所有玩家操作的结果,如召唤一个随从,翻转英雄技能,翻转回合结束按钮等。一般而言,在所有动画播放完毕后,你才能得知一个操作的最终结果,但在某些情况下,你可以在动画播放完毕之前得知一部分结果:

\example 你在4费回合使用\card{银月城传送门}后发现你手中的\card{发条侏儒}变绿。你可以断定银月城传送门召唤了\card{机械跃迁者}。
\example 你使用\card{尤格-萨隆}后,回合结束按钮变为黄色且无法按下。你可以确定这局比赛结束了,尤格-萨隆的战吼杀死了某一方的英雄。
\example 你使用\card{年轻的酒仙}回手一个随从,它在变成卡牌前也会绿色高亮。如果你去操作它,游戏会把它从你手中打出去,这是一个有效的节约时间的方法。
\example 你在抽卡时就听到“收工了”,你知道这回合你只能空过。

如果玩家操作导致的结算量过大,则游戏可能卡顿,甚至使玩家掉线。在重新连接至对局后,客户端直接将操作的最终结果显示给玩家。
\example 你通过\card{送葬者安德提卡}的亡语向牌库中洗入60张牌,游戏会直接卡住不动。在短暂的等待后,游戏会提示「你的对战由于断线而中断」并重新连接。之后你可以看到你的牌库中现在已经洗入了60张牌。

玩家也可以在一个较长的动画播放时手动关闭客户端并重启,或断开网络重连。重新连接至对局后动画不会继续播放,而是直接将最终结果显示给玩家。
\example 在酒馆战棋中,如果你在战斗回合的一开始就关闭客户端并重启,则你会直接进入招募回合。这可以使你充分利用招募回合的时间。

%\section{手牌调度}

%手牌调度<ref>\texttt{Game.STEP = BEGIN_MULLIGAN} </ref>

\section{职业}

目前游戏中有10种不同的职业,分别是:恶魔猎手、德鲁伊、猎人、法师、圣骑士、牧师、潜行者、萨满祭司、术士、战士。
\notice 梦境牌\texttt{.CLASS = DREAM} 的职业虽然不为中立,但不被算作职业卡牌。\card{虚灵商人}不能使这些牌减费,使用这些牌也不会给你的\card{幽灵弯刀}增加耐久度。然而死亡骑士牌\texttt{.CLASS = DEATHKNIGHT} 却被虚灵商人、幽灵弯刀等牌视为职业牌。

「随机获得/变为你对手职业的卡牌」、「随机获得/变为你的职业卡牌」与中立英雄互动时会产生不同的效果,具体效果如下:
\begin{itemize}
    \item 「随机获得你对手职业的卡」效果:\card{吹嘘海盗}、\card{幽暗城商贩}、\card{剽窃}、\card{收集者沙库尔}和\card{闪狐}获得的是\card{幸运币}。
        \exception \card{学术剽窃}和\card{搜索}获得的是随机中立职业卡。
        \exception \card{奈法利安}获得的是两张\card{扫尾}。
    \item 「随机获得你的职业的卡」效果:\card{深蓝系咒师}获得的是随机职业的卡(职业可以不同)。
        \exception \card{黑岩法术抄写员}获得的是两张\card{幸运币}。
    \item 「随机变为你对手职业的卡」效果:\card{莉莉安·沃斯}不会生效。
    \item 「随机变为你的职业的卡」效果:\card{龙眠净化者}会将其变成随机中立卡。
\end{itemize}

\chapter{规则细节}
\label{rule-detail}

\setcounter{tocdepth}{2}
\section{关键字效果}
\label{keyword}

% 简单关键字
\subsection{嘲讽}
\label{taunt}

当一个角色选择攻击目标时,如果有敌人具有\term{嘲讽},则攻击目标必须为某个具有嘲讽的角色。
\notice 英雄也可以具有嘲讽。例如卡拉赞冒险模式的馆长。

如果一个随从又具有嘲讽又具有潜行或免疫,则嘲讽失效。

\subsection{圣盾}
\label{divine-shield}

当一个具有圣盾的角色将受到伤害时,将伤害值改为0且该角色失去圣盾。详见\nameref{damage-healing}。

\subsection{吸血}
\label{lifesteal}

\version{10.0}{} 吸血虽然是一个伤害扳机,但是它对群体伤害具有与其它扳机不同的响应方式:它在所有其它被单次伤害触发的扳机全部结算完毕之后触发一次,为你的英雄恢复等于总伤害量的生命值。
\example 你的对手操控五个\card{小精灵},你使用\card{灵魂鞭笞}。鞭笞对所有小精灵造成1点伤害,然后你恢复5点生命值。如果换成\card{欧米茄灵能者}和\card{圣光炸弹},则每造成一次伤害就回一次血。
\notice 在 10.0 版本之前,群体伤害的吸血与普通的伤害扳机表现形式是相同的:连续触发多次,每次为你的英雄恢复等同于单次伤害的生命。

\subsection{剧毒}
\label{poisonous}

\term{剧毒}是一个伤害扳机。它等价于「在该实体对一个随从造成伤害后,消灭该随从」。

\subsection{风怒}
\label{windfury}

具有风怒的角色每回合可以攻击两次。
\notice \term{超级风怒}是风怒的变种。具有超级风怒的角色每回合可以攻击四次。

\subsection{冻结}
\label{freeze}

被冻结的随从不能攻击。在回合结束后,所有被冻结且具有攻击次数的随从不再被冻结。
\example 你使用\card{铁鬃灰熊}和\card{狂暴的狼人}并将它们冻结。在回合结束后,狼人解冻,而灰熊仍然保持冻结。
\example 你使用\card{洛萨},对手将它冻结。在回合结束时洛萨触发效果。由于洛萨的效果会消耗其攻击次数,它在回合结束后不会解冻。

\subsection{免疫}
\label{immune}

免疫防止该实体将受到的伤害或将减少的耐久。免疫角色不能成为敌方的目标。
\example 双方场上没有随从且英雄都具有免疫。你使用\card{燃烧权杖},所有\card{炎爆术}都只会以你的英雄为目标。如果双方英雄具有的是「无法成为法术或英雄技能的目标」,所有炎爆术都无法生效。
\notice 免疫不等于无敌。免疫仍然会受到消灭效果的影响。

\subsection{抽到时施放}
\label{cast-when-drawn}

当你抽到具有\term{抽到时施放}的牌时,你会直接施放它然后抽一张牌。
\notice 抽到时施放实际为抽牌扳机,而并非施放这张法术本身。因此它不会消耗\card{日蚀}等牌的效果。
\exception \card{沙德拉斯·月树}为法术添加的抽到时施放会施放一个法术,也会消耗日蚀的效果。

如果在一张具有抽到时施放的牌本身效果结算完毕之后,有一方英雄处于濒死状态或已离场,则不会再抽一张牌。
\example 你的生命值为4,抽到一张\card{炸弹}。你不会再抽一张牌,因此也无法抽到接下来的\card{灵魂残片}来救回自己。
\example 你和对手的生命值都为1,你的牌库中只有一张\card{流血}。你抽到流血对对方造成2点伤害。你不会再抽一张牌,因此你赢得此盘对局而不是以平局结束。

% 带参数的简单关键字
\subsection{法术伤害}
\label{spell-damage}

法术伤害增加法术造成的伤害。有的卡牌具有派系法术伤害,这类法术伤害只会增加特定派系法术的伤害。

一些法术免疫法术伤害。这包括:
\begin{itemize}
    \item 对自己造成负面效果且不可控的法术。这包括\card{诅咒}、\card{地雷}、\card{远古诅咒}、\card{炸弹}、\card{堕落之血}等牌。
        \notice 如果法术的伤害是可以自行控制的,那它们通常会受到法术伤害影响,例如\card{亡者复生}、\card{黑暗附体}。
        \exception \card{湮灭}也不受法术伤害影响。
    \item 随从战吼的选项。这些牌逻辑上是随从施放的效果。这包括\card{毁灭}、\card{火之祈咒}、\card{气之祈咒}等牌。
\end{itemize}

还有一些法术在实现上免疫法术伤害;这是由于它们的特殊机制导致的。法术伤害在这些牌的具体效果中被单独处理。这包括:
\begin{itemize}
    \item 飞弹类法术。这是为了让法术伤害增加飞弹数量而非每发飞弹的伤害。这包括\card{奥术飞弹}、\card{复仇之怒}、\card{恐怖丧钟}、\card{狂乱传染}、\card{火山喷发}、\card{燃烬风暴}、\card{克苏恩面具}、\card{克苏恩之眼}、\card{雷区挑战}、\card{噬灵疫病}等牌。
        \notice \card{治疗之雨}也属于这一类牌。尽管它与法术伤害不能互动,但是它可以受到\card{先知维纶}的影响。
        \exception \card{强能奥术飞弹}总是发射固定数量的飞弹。它正常受法术伤害影响。
        \notice 由于\card{噬灵疫病}不受法术伤害影响,因此它受到先知维伦影响时仅仅能翻倍伤害而无法翻倍治疗。
    \item 转移过量伤害的法术。这是为了避免过量伤害受到两次法术伤害的影响。这包括\card{爆炸符文}、\card{火球滚滚}、\card{燃烧}、\card{穿刺射击}、\card{不稳定的暗影震爆}等牌。
\end{itemize}

此外,例如\card{背叛}、\card{末日回旋镖}这类法术,其伤害并非由法术造成,因此也不受法术伤害影响。

% 复杂关键字
\subsection{连击}
\label{combo}

必须在使用这张牌之前使用过其它牌才能结算的效果。

某些牌可能同时具有战吼和连击。这些牌实际上只有战吼,该战吼会根据之前有无使用过其它的牌改变效果。
\example 你操控\card{布莱恩·铜须},并使用\card{毁灭之刃}。尽管铜须并不影响连击效果,毁灭之刃仍然会造成两次2点伤害。

\subsection{流放}
\label{outcast}

当你从手牌最左或最右使用这张牌的时候会产生的效果。
\notice 与其它流放牌不同,\card{眼棱}的流放效果是一个在流放位生效的自身费用光环,而非结算阶段的效果。
\notice 与\nameref{combo}类似,同时具有战吼和流放的牌实际上仅仅具有战吼,该战吼在结算过程中因该牌是否在流放位具有不同的效果。

流放只有在你从手牌中使用这张牌的时候可以触发。
\example 你使用了流放位的\card{幽灵视觉}。你的\card{大法师瓦格斯}和对手的\card{永恒巨龙姆诺兹多}重复这个法术时均不能流放。
\example 对手使用了\card{隐秘破坏者}拉出你唯一一张手牌幽灵视觉。它不能流放。

\subsection{发现}
\label{discover}

从三张牌中选择一张。在没有额外说明的情况下,选择的牌将置入你的手牌。
\notice 发现也可能发现英雄技能。例如\card{芬利·莫格顿爵士}、\card{沙漠爵士芬利}、\card{偷师学艺}等等。
\exception \card{迈拉·腐泉}、\card{九命兽魂}、\card{战歌驯兽师}和\card{墓园召唤}在将发现的牌置入你的手牌同时还有额外的操作。这与其它卡牌,例如\card{永恒奴役}等不同。
\notice 如果这张牌是从你牌库中发现的,例如\card{追踪术}、\card{暗中生长}或战歌驯兽师,则这个置入手牌的动作属于抽牌,将正常触发抽牌扳机。

发现效果分为如下几种类型:
\begin{itemize}
    \item 发现范围固定。例如\card{织法巨龙玛里苟斯}、\card{沙漠爵士芬利}。
    \item 发现范围为游戏内的固定范围。例如\card{拉祖尔女士}(对方手牌)、\card{追踪术}(你的牌库)。
    \item 发现范围为可收集牌(可能带有条件限制),但限定了职业。例如\card{荆棘帮蟊贼}、\card{盗取武器}、\card{魔杖窃贼}。
    \item 发现范围为可收集牌,且未限定职业。例如\card{深渊探险}、\card{导师火心}、\card{复苏}等等。这类发现的限定范围为你的职业卡牌与中立牌。
        \notice 描述为「任意职业」的牌不属于未限定职业,例如\card{钥匙守护者艾芙瑞}。
\end{itemize}

如果一个发现效果为上述类型的后两者,且限定了某个职业(例如\card{幻觉}、\card{扭曲学识}或\card{虚灵巫师},则还有以下例外情况:
\notice 如果限定的职业为中立,或限定职业后发现范围中没有没有可发现的牌,则会使用发现牌本身的职业加上中立卡牌进行发现。如果发现牌本身为中立牌,则会随机选择一个职业加上中立卡牌进行发现。
\example 牧师使用\card{套圈圈}发现的是法师奥秘;使用\card{奥秘图纸}发现的是猎人奥秘;使用\card{水文学家}发现的是圣骑士奥秘。潜行者使用这三张牌均发现潜行者奥秘。
\history \version{}{KNC} 在狗头人版本已实装但还未上线的时间内,潜行者拥有三张尚未实装的奥秘。此时使用水文学家不会发现任何牌。
\example 你的英雄为\card{炎魔之王拉格纳罗斯}。你使用\card{虚灵巫师}。你会发现法师法术牌。
\example 你的英雄为\card{炎魔之王拉格纳罗斯}。你使用\card{奥能水母}。你会发现来自同一职业的法术牌。
\example 你的英雄为\card{炎魔之王拉格纳罗斯}。你的对手使用\card{幻觉}。他会发现潜行者职业牌。
\example 牧师使用\card{先到先得}发现的是萨满职业牌。
\exception 尽管德鲁伊有一张职业过载牌\card{雷霆绽放},但德鲁伊使用先到先得仍然可以发现萨满职业牌。

% 版本关键字
\subsection{反制和失效}
\label{counter}

目前游戏中只有\card{法术反制}一张可收集牌拥有反制这个关键字。但除此之外,如果你装备一把冰冠堡垒冒险中的\card[ICCA08_020]{霜之哀伤},你的所有手牌处于失效状态。这为我们测试法术以外的牌失效的情况提供了仅有的手段。反制实质上是将那个法术设置为失效状态,所以本节中将一视同仁地处理。

如果一张牌失效,它的\texttt{.CANT\_PLAY := 1} 。然后,这张牌的法术效果/战吼/连击等等不结算,且该序列中的所有扳机不能触发。
\notice 这不是在说「任何事情都不会发生」—— 在法术反制触发之前的那些步骤自然可以正常进行。这包括所有费用状态的自我移除(已测试的有\card{肯瑞托法师}、\card{暗金教侍从}、\card{血色绽放}、\card{古加尔}、\card{亡鬼幻象}、\card{伺机待发}和\card{墨水大师索莉娅})。此外,\card{暮陨者艾维娜}与\card{卡雷苟斯}的光环也会切换。\card{伊莱克特拉·风潮}和\card{星界密使}的状态移除是在完成阶段,因此会被法术反制阻止。
\exception 失效的\card{黑暗之主}可以休眠。在它休眠之后,由于它已经离场,失效状态被清除,它的战吼可以正常结算。
\exception \version{17.2.1}{} 即使一张法术被\card{法术反制}反制,它仍然可以增加\card{学习龙语}的进度。这很可能是有意为之。

如果一张牌失效,它将会被送去墓地(时机尚不明确,但看起来是在相当早的时候)。这包括这张牌是随从或武器的极为特殊的情况(它们的亡语可以触发)。此外,你装备着\card[ICCA08_020]{霜之哀伤}使用武器并不会替换你的武器。新的武器会被直接摧毁送去墓地,你仍然装备霜之哀伤且仍然具有攻击力。但是由于显示bug,看起来你像是没有装备武器。

如果一张牌失效,所有计数器不会增加计数。
\example 当你使用一张法术且被反制后,\card{墓园恐魔}、\card{奥术统御者}、\card{奥术巨人}、\card{没电的铁皮人}、\card{大法师瓦格斯}、\card{暗金教水晶侍女}、\card{法力飓风}、\card{疾疫使者}、\card{卡格瓦,青蛙之神}、\card{巨龙召唤者奥兰纳}、\card{尤格-萨隆}、\card{祖尔金}、\card{黑曜石碎片}、\card{苔丝·格雷迈恩}都将像你并未使用过这张法术一样地处理它们的效果。
\exception 即使你使用了一张被反制的法术,你手牌中所有\nameref{combo}牌也都可以被激活。这与其它计数器显著不同。

\subsection{进化}
\label{adapt}

\term{进化}一个随从指,从下列十项效果中随机选择三项,你从中选择一项。该随从获得该效果。进化可选择的效果包括:\\
嘲讽、圣盾、风怒、剧毒、+1/+1、+3攻击力、+3生命值、获得「亡语:召唤两个1/1的孢子」、潜行直到下回合、获得「无法成为法术或英雄技能的目标」。

如果你是在酒馆战棋中进化,则改为从上述十项的前八项中随机选择。

如果进化的随从不在场上或濒死,则什么都不会发生。

\subsection{回响}
\label{echo}

在你使用一张回响牌后,将一张该牌的复制加入你的手牌。这个复制在回合结束时会移除。

具有回响状态的牌的法力值消耗不能低于1点。这个效果将会覆盖其余任何费用效果。

\card{不稳定的异变}、\card{女巫杂酿}和\card{沼泽女巫的召唤}不是回响。他们的「回响牌」的法力值消耗可以低于1点。
\notice 但在一个与\card{古神在上}的互动bug中,它们的表现与回响一致。

\subsection{超杀}
\label{overkill}

超杀是伤害扳机。与吸血类似,超杀也只会在所有其它伤害扳机结算完成之后触发一次。
\example 你令你的\card{黑心票贩}攻击对手的\card{苦痛侍僧}。无论入场顺序,苦痛侍僧先抽牌,然后黑心票贩抽牌。
\example 你令你的先入场的黑心票贩攻击对手1血的\card{兽人铸甲师},且对手控制一个后入场的\card{铸甲师}。无论入场顺序,首先结算「黑心票贩对兽人铸甲师造成伤害」,铸甲师给对手1点护甲,然后黑心票贩抽牌。接下来结算「兽人铸甲师对黑心票贩造成伤害」,兽人铸甲师给对手2点护甲。

超杀触发与否只与造成伤害时目标的血量是否为负数有关。如果目标在造成伤害后,超杀触发前又被救回0血或以上,超杀依然触发。
\example 你操控两个3血\card{咆哮魔}和一个1血\card{小鬼骑士},并使用\card{冲击波}。冲击波对所有随从造成2点伤害,然后咆哮魔触发你弃两张牌并把小鬼骑士变成5/1,然后你获得一张随机法师法术。

\subsection{复生}
\label{reborn}

在一个具有复生的随从死亡后,会召唤一个它的复制,但该复制生命值为1且不具有复生。

\subsection{可交易}
\label{tradeable}

你可以在你的回合中支付1点费用将\term{可交易}随从洗回你的牌库并抽一张牌,这称为\term{交易}。这是一个主动效果,且不属于使用一张牌。可交易牌首先将本身洗回牌库,但交易抽到的牌不会是这张牌本身。

可交易牌在交易的时候保留所有增益。
\notice \version{21.0}{21.3}可交易牌在进行其它区域移动时也保留所有增益。

\section{非关键字效果}

\subsection{健忘}
\label{forgetful}

一些牌具有「有50\%的概率攻击错误的敌人」。这包括\card{食人魔步兵}、\card{食人魔忍者}、\card{穆戈尔的勇士}、\card{砂槌萨满祭司}、\card{食人魔战槌}、\card{食人魔勇士穆戈尔}和\card{乐观的食人魔}。此外,\card{莫什奥格播报员}也会让攻击它的角色享受类似效果。

如果该50\%概率检测没有命中,则该扳机根本不会触发。其闪电符号也不会亮起。

\subsection{异变和衰变}
\label{evolve-and-devolve}

在本节中,\term{异变}指「随机将某随从变形为法力值消耗增加N点的随从」这一效果。相对的,\term{衰变}指「随机将某随从变形为法力值消耗减少N点的随从」。在当前的对战模式中,包含异变的牌有\card{异变}、\card{异变之主}、\card{死亡先知萨尔}(及\card{灵体转化})、\card{不稳定的异变}、\card{突变}和\card{女巫跟班}。包含衰变的牌有\card{衰变}和\card{衰变飞弹}。此外,\card{终极巫毒}与此机制有相关联之处,在此一并讨论。

\version{16.0}{} 当你异变一个「不存在所需法力值消耗的随从」的随从时,会变形为法力值消耗尽可能高(不会低于原始费用,也不会超出所需的费用)的随从。相对的,当你衰变一个零费随从时,会变形为随机的零费随从。类似地,如果\card{侏儒变形师}变形一个不存在对应费用的随从,则该随从会重新变形为自身。
\example 你操控只有1点生命值的普通\card{熔核巨人}。你对其使用金色的\card{突变}。它被变形成为一个满血的金色熔核巨人。
\example 你操控\card{雪怒巨人}。你使用\card{死亡先知萨尔}。它被变形成为一个随机的12费随从。
\example 你操控\card{猛虎之灵},然后依次使用\card{古加尔}和一个23费的法术。猛虎之灵为你召唤一个23费的猛虎。你将其打到1点生命值,然后对其使用\card{突变}。猛虎未被变形成随机的20费或25费随从,而是变形为一个原始状态的猛虎(尽管猛虎并不是可收集随从)。
\notice 在之前的版本中,这些随从不会被变形。
\notice 终极巫毒的修改要更早。它的早期修改与异变修改的不同见下。

如果你在满手牌时令一个被贴了\card{终极巫毒}的随从攻击,且该随从触发\card{冰冻陷阱}导致回手(然后由于你满手牌被爆掉),终极巫毒会认为该随从的法力值消耗是已经被冰冻陷阱增加之后的数值。
\example 你有十张手牌,然后令一个贴有终极巫毒的\card{血沼迅猛龙}攻击。触发敌方冰冻陷阱之后它死亡,并召唤一个随机5费随从。
\example \version{早期版本}{} 你有十张手牌,然后令一个贴有终极巫毒的\card{机械克苏恩}攻击。触发敌方冰冻陷阱之后它死亡,并召唤一个随机12费随从。
\example \version{早期版本}{16.0} 你有十张手牌,然后令一个贴有终极巫毒的\card{山岭巨人}攻击。触发敌方冰冻陷阱之后它死亡,并什么都不召唤(而不会召唤一个12费随从,这是早期修改与异变16.0修改的不同之处)。
\example \version{16.0}{} 你有十张手牌,然后令一个贴有终极巫毒的\card{山岭巨人}攻击。触发敌方冰冻陷阱之后它死亡,并召唤一个随机12费随从。

\notice 大多数类似的效果不会作这种修正。
\example 你操控\card{召唤石},并通过\card{古加尔}施放了一个13费的法术。召唤石不会为你召唤随从。
\exception \version{早期版本}{} \card{鲁莽试验}现在与之不同。它也遵循与异变类似的规则,会召唤费用尽可能高的随从。
\example 你操控三个\card{玛里苟斯},并使用鲁莽试验。尽管并不存在17费的随从,鲁莽试验为你召唤了两个随机的12费随从。

\setcounter{tocdepth}{1}
\section{扳机所响应的事件}

游戏中的大多数扳机所响应的事件都是明确的,例如「在你召唤一个随从后」响应召唤后事件,「每当一个随从获得治疗」响应治疗事件。但是有些扳机由于文本所限不能准确地描述自己所响应的是什么事件。本节将对其进行说明。

\subsection{任务和法术石}
安戈洛任务:
\begin{itemize}
    \item 使用时扳机:\card{火羽之心}
    \item 使用后扳机:\card{湿地女王}、\card{打开时空之门}、\card{最后的水晶龙}、\card{探索地下洞穴}
    \item 召唤后扳机:\card{丛林巨兽}、\card{唤醒造物者}、\card{鱼人总动员}
    \item 弃牌扳机:\card{拉卡利献祭}
\end{itemize}

奥丹姆任务:
\begin{itemize}
    \item 回合结束扳机:\card{发掘潜力}
    \item 使用时扳机:\card{洗劫天空殿}
    \item 使用后扳机:\card{制作木乃伊}、\card{腐化水源}
    \item 召唤后扳机:\card{打开宝库}
    \item 攻击后扳机:\card{侵入系统}
    \item 置入手牌扳机:\card{劫掠集市}
    \item 抽牌扳机:\card{最最伟大的考古学}
    \item 治疗扳机:\card{激活方尖碑}
\end{itemize}

暴风城任务线:
\begin{itemize}
    \item 使用时扳机:\card{挺身而出}
    \item 使用后扳机:\card{寻求指引}、\card{探查内鬼}、\card{巫师的计策}、\card{号令元素}、\card{开进码头}
    \item 攻击力增加扳机:\card{游园迷梦}
    \item 伤害扳机:\card{保卫矮人区}、\card{恶魔之种}
    \item 抽牌扳机:\card{一决胜负}
\end{itemize}

支线任务:
\begin{itemize}
    \item 回合开始扳机:\card{庇护}
    \item 使用时扳机:\card{人多势众}
    \item 使用后扳机:\card{学习龙语}、\card{元素盟军}
    \item 召唤后扳机:\card{正义感召}、\card{扫清道路}
    \item 攻击后扳机:\card{保护甲板}
    \item 激励扳机:\card{病毒增援}
\end{itemize}

法术石:
\begin{itemize}
    \item 使用时扳机:\card{小型法术黑曜石}
    \item 使用后扳机:\card{小型法术翡翠}、\card{小型法术红宝石}、\card{小型法术钻石}
    \item 伤害扳机:\card{小型法术紫水晶}
    \item 治疗扳机:\card{小型法术珍珠}
    \item 获得护甲扳机:\card{小型法术玉石}
    \item 过载扳机:\card{小型法术蓝宝石}
\end{itemize}

\subsection{即时变形效果}

\card{百变泽鲁斯}、\card{熔岩之刃}、\card{变形卷轴}、\card{变色龙卡米洛斯}和\card{班德斯莫什}在\emph{你的}回合开始阶段触发。它们具有类似的机制。此外,使用时扳机\card{暗影映像}及使用后扳机\card{软泥教授弗洛普}也采用类似的机制:\\
如果一个即时变形卡牌不具有「变形特效」,那它的自身扳机在你的回合开始触发,属于手牌扳机。这个扳机将它变形并添加一个继续变形的状态。继续变形的状态也是一个回合开始扳机,但是它是一个场上扳机(因为状态都在场上)。这会带来一些顺序问题。
\example 你操控一个\card{报警机器人}。你手牌中的未变形的百变泽鲁斯总是晚于机器人触发;而已变形的百变泽鲁斯和报警机器人的顺序则取决于变形发生的时间与报警机器人入场时间的先后。

狼人牌、\card{红色按钮}与阴谋牌在\emph{你的}回合结束触发。其中,狼人牌的扳机不能被\card{达卡莱附魔师}加倍。

\section{扳机自触发保护}
\label{self-triggering-protection}

如果某个扳机A在自身的结算中又产生了能触发它的事件,在该事件过程中A不能被触发。某些扳机在这次触发完全结算完毕后,会补偿进行跳过的结算。目前已知的有如下几个例子:

\example 你操控两个\card{苦痛侍僧}A和B。你对苦痛A造成1点伤害,A 触发并抽到一张\card{烈焰巨兽},对两者都造成伤害。但由于这是在A的扳机结算过程中,A不能再次触发。于是B触发之后A再(补偿)触发,触发顺序是 A-B-A 而非 A-A-B。
\example 你操控一个\card{灵魂歌者安布拉},并使用\card{暮光召唤}。在你召唤第一个随从后,安布拉立即触发其亡语;然后才召唤第二个随从。
\example 你操控一个\card{灵魂歌者安布拉},并使用\card{食肉魔块}消灭你操控的一个亡语随从。安布拉立即触发食肉魔块的亡语;但在召唤两个亡语随从之间并没有插入安布拉的触发。在两个随从均召唤完毕之后,安布拉连续触发这两个随从的亡语。如果你还操控一个\card{送葬者},送葬者的扳机会在两个随从之间触发。
\example 你操控一个灵魂歌者安布拉,使用\card{暗言术:灭}消灭你操控的9/9\card{血骨傀儡}I。注意血骨傀儡的亡语实质上先召唤一个原身材的血骨傀儡,再使其-1/-1。血骨傀儡I首先召唤一个9/9的血骨傀儡II,安布拉立即触发其亡语再召唤一个9/9的血骨傀儡III。由于此时处于安布拉扳机结算过程中,安布拉不能再次触发。血骨傀儡III的身材减为8/8。紧接着,后续召唤血骨傀儡IV并将身材减为7/7,血骨傀儡V则是6/6,紧接着是5/5和4/4。最后,血骨傀儡II的身材减为8/8。最终你场上六个血骨傀儡的身材依次是:8/8,8/8,7/7,6/6,5/5,4/4。
\example 类似的,你操控一个灵魂歌者安布拉并使用9/9的血骨傀儡,最终你场上六个血骨傀儡的身材依次是:9/9,8/8,7/7,6/6,5/5,4/4。
\notice 上面两个例子相似之处在于:你都是试图通过安布拉以外的效果(一个是亡语,一个是从手牌使用)召唤一个9/9的血骨傀儡,只不过第一个例子中你在召唤完战吼还要将它的身材 -1/-1。所以最终场面的唯一区别就是第一个血骨傀儡是8/8还是9/9。
\example 对手操控\card{以眼还眼},你使用\card{枯萎化身塔姆辛},并对自己造成伤害。塔姆辛的扳机触发对对手造成伤害,触发以眼还眼。此时塔姆辛的扳机不能再次触发,最后你受到伤害。如果你使用了两个塔姆辛,则另一个扳机可以触发,最终对手受到伤害。

下面是一个较为复杂的例子,也是扳机自触发保护最初被发现时的情况。参见\href{https://www.youtube.com/watch?v=DowBB0GhGnA}{这里}\\
在这个例子中,盗贼操控4个\card{苦痛侍僧}而牧师操控6个。盗贼牌库中有6张\card{烈焰巨兽}而牧师有12张。现在盗贼的回合开始,抽到一张烈焰巨兽。\\
最终,盗贼抽了6张烈焰巨兽后受到 3-61 的疲劳伤害,总计$16*4+1$次抽牌;而牧师抽到10张烈焰巨兽,爆掉2张,受到 6-89 的疲劳伤害,总计$16*6$次抽牌。苦痛侍僧触发的总次数是正确的。\\
苦痛侍僧的触发顺序如下(按入场顺序编号为 0-9):
\begin{center}
    \texttt{
        0123456789\\
        999999\\
        88888889\\
        7777777789\\
        666666666789\\
        55555555556789\\
        4444444444456789\\
        333333333333456789\\
        22222222222223456789\\
        1111111111111123456789\\
        000000000000000123456789
    }
\end{center}

\section{对战开始时扳机的顺序}

所有的\term{对战开始时}牌均包含两个相同的扳机:一个在手牌触发,一个在牌库触发。对战开始时扳机的顺序按照如下规则:

\begin{enumerate}
    \item 主玩家的所有扳机先触发,然后是副玩家的。这可以理解为是主玩家的所有实体先入场的直接结论。
    \item 手牌扳机先于牌库扳机。
    \item 对牌库中的各扳机,按如下顺序触发:
    \begin{enumerate}
        \item 费用低的先触发。费用相同的则按随机顺序触发。
        \item  如果一张牌因为起手换牌进入牌库,它在最后触发,无视费用顺序。
    \end{enumerate}
\end{enumerate}

\example 你起始牌库中有\card{黑暗主教本尼迪塔斯}和\card{克苏恩,破碎之劫}。黑暗主教先触发。
\example 你起手有黑暗主教,将它换掉,牌库中有克苏恩。克苏恩先触发。但由于黑暗主教只作预检测,它仍然可以触发。
\example 你起始牌库中有黑暗主教和\card{玛克扎尔王子}。它们按随机顺序触发。

但手牌中多个对战开始时扳机,以及同时换入牌库的多个对战开始时扳机的触发顺序仍然是不清楚的。

\section{叠加状态}

\version{}{17.4.1} \term{可叠加的}状态具有如下特点:如果一个实体已经具有了该状态,那么它将要再次获得同类状态的时候,并不会获得一个新的状态,而是改为修改已有的那个状态。这看起来像是「状态并没有立即生效,而是一直推迟到光环更新时才生效」。这句话可能较难理解,请比对下面的例子帮助理解。
此外,当你将鼠标放在该随从上时,大图下方的状态列表也只会显示一个状态。
\example 典型例子包括\card{暴乱狂战士}、\card{加兹瑞拉}、\card{漂浮观察者}等。

\example 你操控\card{索瑞森大帝}。在回合结束时,那些上回合已经减过费的牌看起来明显「晚减费」。

\example 你操控\card{血沼迅猛龙}和3/4的\card{暴乱狂战士}(尚未触发过效果),然后使用\card{圣光炸弹}。迅猛龙对自己造成伤害立即触发暴乱狂,然后暴乱狂对自己造成4点伤害。
\example 你操控迅猛龙和3/4的暴乱狂战士(已经触发过效果),然后使用圣光炸弹。迅猛龙对自己造成伤害立即触发暴乱狂,但是由于暴乱狂已经拥有一个同类状态,它仍然保持3攻。因此它对自己造成3点伤害,在炸弹结算完毕后(实际上是结算阶段结束后),暴乱狂变为4攻。

\example 你操控6/2的\card{漂浮观察者}(尚未触发过效果),并攻击操控\card{爆炸陷阱}的对手。爆炸触发对漂浮和你造成2点伤害,然后漂浮触发变成8/2。战斗继续,漂浮对敌方英雄造成8点伤害。
\example 你操控6/2的漂浮观察者(已经触发过效果),并攻击操控爆炸陷阱的对手。爆炸触发对漂浮和你造成2点伤害,然后漂浮触发,但是由于漂浮已经有一个同类状态,它保持6/0。战斗被跳过。在战斗阶段结束后,漂浮变为8/2。
\example 你操控6/2的漂浮观察者(已经触发过效果)和\card{龙蛋},并攻击操控爆炸陷阱的对手。爆炸触发对漂浮、龙蛋和你造成2点伤害,然后漂浮触发,但是由于漂浮已经有一个同类状态,它保持6/0。随后龙蛋触发召唤一个雏龙,光环更新,漂浮变成8/2。战斗继续,漂浮对敌方英雄造成8点伤害。

\section{强制死亡}

通常情况下,一个阶段之内不会进行死亡检索与死亡结算。但一些牌会在结算中进行死亡检索与死亡结算,这称作\term{强制死亡}。部分牌是因为它们先消灭随从再召唤随从,为了防止召唤的随从被挤掉,必须使用强制死亡;部分牌是为了防止连续多个法术产生反直觉的结果。

到贫瘠之地版本为止,包含强制死亡的效果有:
\begin{itemize}
    \item 先消灭随从再在原地召唤新随从的效果或类似效果,包括\card{剧毒之种}、\card{转生}、\card{米米尔隆的头部}、\card{饥饿的翼手龙}、\card{咒术师的召唤}、\card{至暗时刻}、\card{卑劣的回收者}、\card{教导主任加丁}、\card{仇恨之轮};
    \item 多次伤害,包括\card{亵渎}、\card{高弗雷勋爵}、\card{地震术}、\card{献祭光环}、\card{大地崩陷}、\card{深水炸弹}、\card{燃烧权杖}、\card{永恒之火};
    \item 防止连续结算多个法术或战吼产生反直觉的结果,包括\card{尤格-萨隆}、\card{惊奇卡牌}、\card{莱妮莎·炎伤}、\card{苔丝·格雷迈恩}、\card{沙德沃克}、\card{祖尔金}、\card{尤格-萨隆的谜之匣}、\card{永恒巨龙姆诺兹多}、\card{神秘魔盒}、\card{杰斯·织暗}、\card{大魔导师安东尼达斯}的每次法术/战吼之后,以及\card{尤格-萨隆的仆从}和\card{隐秘破坏者}战吼施放法术后。通过下文所提到的卡德加的bug可以证明,这些强制死亡即使在没有铜须的情况下依然存在。
    \item 酒馆战棋中的每次攻击后。但由随从效果导致的攻击后没有强制死亡(指海盗无赖召唤的\card{空中海盗})。
\end{itemize}

当进行强制死亡时,首先进行所有正常的阶段间步骤,然后如果满足随从死亡的条件,进行一个死亡阶段。循环如此直到没有随从死亡为止。实际上,这与一般的阶段间发生的事情完全相同。
\example 你操控两个\card{鬼灵爬行者}并使用\card{剧毒之种}。所有鬼灵爬行者被消灭,然后强制死亡:移除所有鬼灵爬行者,召唤四个\card{鬼灵蜘蛛}。最后召唤两个树人。
\example 你操控六个\card{可靠的灯泡}和\card{米米尔隆的头部},然后使用\card{丛林之魂}。你的下个回合开始时,米米尔隆的头部触发,消灭所有随从并移除,然后触发它们的亡语,召唤七个\card[EX1_158t]{树人}。最后由于满场,\card{V-07-TR-0N}不会召唤。

\notice 一个有关\card{卡德加}的结算表明,强制死亡似乎被认为是效果的一部分,这与一般的死亡结算不同。参见\href{https://www.bilibili.com/video/av64643539}{这里}。\\
\card{校长克尔苏加德}也有类似的结算。
\example 你操控卡德加,你的对手操控\card{鱼人木乃伊}。你对鱼人木乃伊施放\card{火球术},它召唤一个鱼人。
\example 你操控卡德加,你的对手操控鱼人木乃伊。你对鱼人木乃伊施放\card{转生},它召唤一个鱼人之后,卡德加会触发再召唤一个鱼人。

\subsection{历史}

\version{Naxx}{GvG} \card{克尔苏加德}在回合结束召唤死亡的随从之前会进行一次强制死亡。这就意味着,被\card{炎魔之王拉格纳罗斯}消灭的克总可以复活自己。
\version{4.2}{4.3} 强制死亡中的死亡阶段被移除。这就意味着,\card{自爆绵羊}和\card{剧毒之种}的互动被改为:消灭所有随从,移除所有随从,召唤等量的2/2,自爆绵羊的亡语结算。
\version{}{KNC} 强制死亡中,在死亡阶段结束之后没有光环更新步骤。这导致\card{食腐土狼}和\card{亵渎}之间产生了难以理解的结算。
\version{}{TWW} \card{惊奇卡牌}之后没有强制死亡。
\version{}{11.2} 在早期版本中,强制死亡只包含一次死亡检索和死亡阶段;在这个死亡阶段中产生的新的濒死不会处理。在当前的版本中,强制死亡会一直进行到没有任何濒死实体为止。这和一般的死亡结算完全一致。

\section{回合时间}

玩家的回合长度为75秒。
\begin{itemize}
    \item 每个玩家的第一个回合长度为45秒,第二个回合长度为55秒。
    \item 冒险模式中,玩家无回合时间限制。
    \item 当\card{诺兹多姆}在场时,玩家的回合长度为20秒(不是15秒)。如果是冒险模式,玩家的回合长度为75秒。
\end{itemize}
\chapter{规则更改}
\label{rule-update}

\section{早期的规则更改}
\label{rule-update:early}

在某个早期版本中,强制死亡仅包括死亡步骤,不包括随后的死亡阶段。
\example 你操控\card{自爆绵羊},然后使用\card{剧毒之种}。绵羊被强制死亡移除,但是并没有结算亡语。在树人召唤之后,结算绵羊的亡语。这通常可以消灭所有随从。

在某个早期版本中,法术的结算阶段和完成阶段之间没有阶段间步骤(即它们实际上是一个阶段)。
\example 你操控\card{狂野炎术师},然后使用\card{极恶之咒}。虽然炎术师处于濒死状态,但是仍然触发了效果。

\section{9.2}
\label{rule-update:9.2}

9.2更新的主要内容是引入了预检测机制。官方说明详见\href{https://hs.blizzard.cn/article/16/11199}{9.2机制更新}。
\notice 官方说明中对于许多名词的定义与规则集不同,如「序列」等。

\version{}{9.2}固有阶段的各个步骤中的扳机,只要在列队前在场且合法即可列队和触发。
\example 对手使用\card{火元素}消灭你的\card{疯狂的科学家},其亡语拉出\card{镜像实体}。镜像实体在结算阶段后的死亡阶段入场,这早于完成阶段的使用后步骤,因此可以触发并召唤一个火元素。
\example 你使用\card{控心术}召唤了一个\card{狂野炎术师}。炎术师在结算阶段中入场,这早于完成阶段的使用后步骤,因此可以触发并对所有随从造成1点伤害。
\example 你操控\card{西风灯神},并使用\card{疯狂药水}控制了一个对手随从。西风灯神在完成阶段的使用后步骤中结算,而在这之前被控制的随从已经成为了一个友方随从,符合西风灯神的条件。因此西风灯神可以触发并获得冲锋。

\version{}{9.2}死亡阶段各个死亡事件的扳机,只要在那个死亡事件列队前在场且合法即可列队和触发。
\example 对手使用\card{扭曲虚空}消灭你的\card{疯狂的科学家}和\card{冰风雪人}。首先结算疯狂的科学家的死亡事件,其亡语拉出\card{救赎}。然后结算冰风雪人的死亡事件,由于救赎在冰风雪人的死亡事件结算开始前就在场,它可以触发并召唤一个4/1的雪人。
\example 你的手中没有龙牌,且对手使用\card{扭曲虚空}消灭你的\card{战利品贮藏者}和\card{冰喉}。首先结算战利品贮藏者的死亡事件,其亡语抽到一张龙。然后结算龙的死亡事件,由于在冰风雪人的死亡事件结算开始前你已持有龙牌,其亡语可以触发并对所有随从造成3点伤害。
\example 你操控\card{诅咒教派领袖}和\card{冰喉}且手中没有龙牌。对手使用\card{火球术}消灭你的\card{冰喉}。死亡阶段结算冰喉的死亡事件。诅咒教派领袖首先触发并抽到一张龙,然后结算冰喉的亡语,由于在冰喉的死亡事件结算开始前你并不持有龙牌,其亡语不能触发。

\version{}{9.2}\term{主玩家机制}在9.2版本之前为:当由不同玩家操控,或区域不同的扳机同时响应一个事件时,他们按主玩家场上、主玩家手牌、主玩家牌库、副玩家场上、副玩家手牌、副玩家牌库依次列队并结算。由于主玩家机制,一个较晚队列中的扳机如果在一个较早队列结算完毕后变得合法,即可列队和触发。
\example 对手消灭你的\card{战利品贮藏者},其亡语(场上扳机)抽到\card{伯瓦尔·弗塔根}。由于伯瓦尔在手牌扳机列队前变的合法,他可以触发并获得 +1 攻击力。

\section{11.2}
\label{rule-update:11.2}

变形

\section{17.4.1}
\label{rule-update:17.4.1}

主玩家机制被移除。
\chapter{其他}

\section{「扳机」一词的翻译}

「扳机」一词是从英文版进阶规则集的「trigger」一词直译而来,用于表示因某事件发生而触发的效果。在使用规则集的过程中,有很多人向我们提出这个翻译难以理解。实际上,我们使用该翻译主要有以下原因:
\begin{itemize}
    \item 明确。英文trigger一词是明确的(虽然会与表示触发的动词trigger混淆,但是这问题不大)。扳机一词虽然稍微难以理解,但是它不会与其它的用词产生混淆。
    \item 简洁。使用「扳机」而非「触发器」或「触发效果」的主要原因是该词长度较短,且不具有可分解的结构。因此「治疗扳机」相比「治疗触发器」或「治疗触发效果」更加流畅且不易产生歧义。
\end{itemize}
当然,这并不意味着我们找到了指示这个概念的最好的词。如果你有更好的翻译,欢迎向我们提出。

\appendix
\chapter{卡牌列表}

\section{具有强制死亡效果的牌}
\label{appendix:forced-death}

\subsection{先消灭再在原地召唤新随从的效果或类似效果}

\begin{center}
\begin{tabularx}{\linewidth}{*{3}{X}}
    \card{剧毒之种} & \card{转生} & \card{米米尔隆的头部} \\
    \card{饥饿的翼手龙} & \card{咒术师的召唤} & \card{至暗时刻} \\
    \card{卑劣的回收者} & \card{教导主任加丁} & \card{仇恨之轮} \\
    \card{献祭召唤者} & \card{被亵渎的墓园} &
\end{tabularx}
\end{center}

\subsection{多段伤害}

\begin{center}
\begin{tabularx}{\linewidth}{*{3}{X}}
    \card{亵渎} & \card{高弗雷勋爵} & \card{地震术} \\
    \card{献祭光环} & \card{大地崩陷} & \card{深水炸弹} \\
    \card{燃烧权杖} & \card{永恒之火} & \card{哥利亚,斯尼德的杰作} \\
    \card{话痨奥术师} & \card{倒刺捕网} \\
\end{tabularx}
\end{center}

\subsection{效果中结算其它法术或战吼}

\begin{center}
\begin{tabularx}{\linewidth}{*{3}{X}}
    \card{尤格-萨隆} & \card{尤格-萨隆的仆从} & \card{惊奇卡牌} \\
    \card{莱妮莎·炎伤} & \card{苔丝·格雷迈恩} & \card{沙德沃克} \\
    \card{祖尔金} & \card{隐秘破坏者} & \card{尤格-萨隆的谜之匣} \\
    \card{永恒巨龙姆诺兹多} & \card{神秘魔盒} & \card{杰斯·织暗} \\
    \card{大魔导师安东尼达斯} & \card{魔导师晨拥} & \card{珍藏私货} \\
    \card{大法师的符文} & \card{怒脊附魔师} &
\end{tabularx}
\end{center}

\section{不受法术伤害影响的法术}
\label{appendix:spell-ignore-spell-damage}

\subsection{对自己造成伤害且不可控的法术}

\begin{center}
\begin{tabularx}{\linewidth}{*{3}{X}}
    \card{诅咒} & \card{地雷} & \card{远古诅咒} \\
    \card{炸弹} & \card{堕落之血} & \card{深渊诅咒}
\end{tabularx}
\end{center}

\subsection{其它方式造成伤害的法术}

\begin{center}
\begin{tabularx}{\linewidth}{*{3}{X}}
    \card{背叛} & \card{末日回旋镖} & \card{痛苦}
\end{tabularx}
\end{center}

\subsection{飞弹类法术}

\begin{center}
\begin{tabularx}{\linewidth}{*{3}{X}}
    \card{奥术飞弹} & \card{复仇之怒} & \card{恐怖丧钟} \\
    \card{狂乱传染} & \card{火山喷发} & \card{治疗之雨} \\
    \card{燃烬风暴} & \card{克苏恩面具} & \card{克苏恩之眼} \\
    \card{雷区挑战} & \card{噬灵疫病} & \card{邪恶入骨} \\
    \card{别站在火里!} & \card{深海低语}
\end{tabularx}
\end{center}

\subsection{转移过量伤害的法术}

\begin{center}
\begin{tabularx}{\linewidth}{*{3}{X}}
    \card{爆炸符文} & \card{火球滚滚} & \card{燃烧} \\
    \card{穿刺射击} & \card{不稳定的暗影震爆} & \card{奥术溢爆}
\end{tabularx}
\end{center}

\section{伪区域移动}
\label{appendix:fake-move}

\subsection{死亡}

\begin{center}
\begin{tabularx}{\linewidth}{*{3}{X}}
    \card{玛洛恩} & \card{骷髅骑士} & \card{鼬鼠挖掘工} \\
    \card{战术撤离} & \card{派烙斯} & \card{灵魂回响} \\
    \card{弑君} & \card{诈死} & \card{永恒祭司}
\end{tabularx}
\end{center}

\subsection{其它}

\begin{center}
\begin{tabularx}{\linewidth}{*{3}{X}}
    \card{回收} & \card{极恶之咒} &
\end{tabularx}
\end{center}

\backmatter
\printindex[termindex]
\setglossarystyle{tree}
\printglossary[title=卡牌效果列表]

\immediate\closeout\missingcard

\end{document}