\chapter{其他机制}

\section{生命值与攻击力}

本节中「生命值」也可以指武器的耐久。

一个实体的生命上限由\texttt{.HEALTH} 记录,而伤害值由\texttt{.DAMAGE} 记录。生命由前者减去后者得到,它在游戏中并不直接存储。对于武器而言,它的耐久上限由\texttt{.DURABILITY} 记录。

当一个实体增加生命时,其生命上限增加;伤害值不变。因此生命值也等量增加。
\example 你对\card{北郡牧师}使用\card{真言术:盾},其生命上限和生命值均为5。

\notice \card{神圣之灵}实际上使目标随从的生命上限增加等于生命值的数值。
\example 你对4/3 的\card{负伤剑圣}使用\card{神圣之灵},其生命上限增加3,为10,生命值也增加3,为6。

当一个实体被设定生命时,其生命上限变为设定数值;伤害值清零,因此生命值也变为设定的数值。互换攻击力和生命值的效果实际上将该实体的生命值和攻击力设置为效果结算时的固定值。
\example 你使用\card{疯狂的炼金师}指定4/3 的\card{负伤剑圣},其生命上限和生命值均变为4。

\notice 「生命设定」效果导致实体生命增加不是治疗;「生命设定」效果导致实体生命减少不是伤害。
\example 你使用\card{希望守护者阿玛拉},你的生命上限和生命值均变为40。你手牌中\card{开心的食尸鬼}不会减至0 费。

\notice 若你的生命上限大于15 点,\card{阿莱克丝塔萨}不会使你的生命上限减为15 点;而若你的生命上限大于40 点,\card{希望守护者阿玛拉}\emph{会}使你的生命上限减为40 点。

当一个实体被沉默时,若沉默使实体生命上限增加,则其伤害值不变;若沉默使实体生命上限减少,则其伤害值也等量减少,但不会低于0。
\example 你对\card{伊瑟拉}使用\card{黑暗裁决}和\card{奥术射击},然后将其沉默。沉默使伊瑟拉的生命上限由3增加到12,因此其生命值也增加9,变为10。
\example 你对伊瑟拉使用\card{圣殿执行者}和奥术射击,然后将其沉默。沉默使伊瑟拉的生命上限由15 降低到12,而其伤害值为2 小于3。因此伤害值变为0。

实体的攻击力在计算过程中可以为负。攻击为负的实体在战斗时不会使对方回血。攻击力为负的武器在与你的英雄攻击力叠加时视为0。
\example 你操控一个0/2 的\card{活动假人}。对手使用\card{虚弱诅咒}。接下来你的回合你对活动假人使用\card{力量祝福},活动假人变为1 攻。
\example 你装备一把已经攻击随从7 次的0 攻\card{血吼},然后使用\card{英勇打击}再攻击随从。攻击后血吼的攻击力减为 -1。你下回合再次使用英勇打击时,你依然造成4 点伤害而不是3 点。

实体的攻击力、生命值、受伤值和英雄的护甲值上限为2147483647 点(即$2^31-1$,十六进制表示为\texttt{0x7FFFFFFF})。超过这个数值会被视为负数。攻击力溢出的实体显示为0 攻;生命值溢出的实体显示为0 血并在下个死亡检索步骤中被视为死亡;受伤值溢出的实体受伤值清零并表现为恢复满血;护甲值溢出时立即清零。
\example 你对一个$2^30$攻的随从使用\card{受祝福的勇士},其攻击变为$2^31$点,显示为0。然后你对其使用\card{缩小射线工程师},其攻击力变为$2^31-2$点,显示为$2^31-2$。
\example 你对一个1431655765 攻(十六进制表示为\texttt{0x55555555})的随从连续使用多张受祝福的勇士,随从的攻击力总是在0 和一个极大数字之间反复变化。
\example 你控制\card{布莱恩·铜须}并使用\card{吞噬者穆坦努斯}吃掉2张对手牌中攻击和生命都是$2^31-1$点的随从,吞噬者穆坦努斯属性值先变成0/0 ,然后变成2/2。

当你复制一个攻击力为负数的随从时,你实际上得到的是一个攻击力为0的复制体。
\example 你对一个$2^31$攻的随从使用\card{无面操纵者},两个随从均显示为0攻。对手使用\card{虚弱诅咒}。原本的随从变为$2^31-2$攻,而复制体依然显示为0攻(实际上是$-2$攻)。
\example 你对一个$-2$攻的随从使用无面操纵者,两个随从均显示为0攻。然后你使用\card{嗜血}。原本的随从变为1攻,而复制体则变为3攻。

\section{光环更新}
\label{aura-update}

\term{光环更新}指游戏检查是否有新光环入场、是否有旧光环离场。事实上光环并非入场即生效、离场即失效。找到游戏中所有的光环更新时机并非易事,目前已经确定的光环更新时机有(包括费用光环,不包括\card[ICC_829p]{天启四骑士}的消灭效果):
\begin{itemize}
    \item \term{阶段间步骤}中的\term{光环更新步骤}
    \item 实体被创建时
    \item 实体入场时
    \item 实体被变形时
    \item 实体解除休眠时
\end{itemize}

\notice 费用光环的相关内容详见\nameref{cost}。
\example 你使用\card{唤魔者克鲁尔}召唤手牌中的\card{无证药剂师}和\card{玛尔加尼斯}。玛尔加尼斯的免疫光环在被召唤时立刻生效,你不会受到5点伤害。
\example 对手操控\card{变形药水}和\card{爆炸符文}。你使用玛尔加尼斯。被变形后玛尔加尼斯的免疫光环立刻失效,你的英雄会受到爆炸符文的5点伤害。
\example 你的玛尔加尼斯在回合开始时苏醒,接下来,你的英雄不会受到\card{鲜血女巫}的伤害。

已确定没有光环更新的情况包括:
\begin{itemize}
    \item 随从控制权转移时
    \item 实体被移回手牌、抽牌、造成伤害等
    \item 阶段结束后,死亡检索步骤之前
\end{itemize}

\example 你的\card{希尔瓦娜斯·风行者}和对手的\card{麻风侏儒}对撞。女王的亡语首先偷走对手的\card{玛尔加尼斯},但麻风侏儒的亡语依然对你造成2点伤害。
\example 你使用\card{地狱烈焰}依次杀死你的\card{自爆绵羊}、\card{希尔瓦娜斯·风行者},并将你的\card{索瑞森大帝}打成2血。在接下来的死亡阶段,自爆绵羊的亡语将大帝打成0血,然后女王的亡语偷走对手的\card{暴风城勇士}。光环不会更新,你的大帝还是会死掉。如果调换自爆绵羊和女王的入场顺序,结果没有变化。

\subsection{尚未解决的问题}
事实上,关于光环更新仍有许多问题有待进一步验证:

\subsubsection{所有光环的更新时机是否相同?}
上文的例子中仅测试了\card{玛尔加尼斯}、\card{暴风城勇士}等较为简单的光环,因此我们无法断定其他光环的表现也相同。事实上,\card[ICC_829p]{天启四骑士}的消灭效果就是一个典型的不相同的示例。

\example 你操控三个天启骑士,且第四个天启骑士在你的回合开始时苏醒。此时光环不会更新,对手英雄不会被消灭。使用英雄技能、发动攻击依然不会导致对手英雄被消灭。但如果你召唤一个随从或使用一张牌,光环会更新,天启四骑士会消灭敌方英雄。

尽管我们并不能完全了解天启四骑士的消灭机制,但显然它与之前玛尔加尼斯与苏醒的互动并不相同。

\subsubsection{阶段间步骤中的光环更新是否总会发生?}
在前面天启四骑士的例子中,我们提到了使用英雄技能、发动攻击不会导致对手英雄被消灭。
\begin{itemize}
    \item 这是否意味着使用技能、战斗序列中的阶段间步骤中没有光环更新?
    \item 这是否意味着在某种特定情况下的阶段间步骤中没有光环更新?
\end{itemize}

尽管最有可能的答案是天启四骑士的光环更新时机与其他不同,但我们依然无法肯定地说以上问题的答案全部为否定。事实上,我们认为阶段间步骤包括光环更新是基于以下实例:
\example \version{18.0}{}你操控\card{暴风城勇士}、\card{救赎}和受到3点伤害变为5/3的\card{冰风雪人}。对手杀死你的暴风城勇士,雪人变为4/3。接下来救赎召唤一个新的暴风城勇士,雪人变为5/4。

这个互动在之前的版本是这样的:
\example \version{}{18.0?}你操控\card{暴风城勇士}、\card{救赎}和受到3点伤害变为5/3的\card{冰风雪人}。对手杀死你的暴风城勇士,接下来救赎召唤一个新的暴风城勇士。在这个过程中没有光环更新,暴风城勇士的光环始终被认为在场。因此雪人保持5/3。

\subsection{状态更新}
与光环不同,当一个状态被添加到一个实体的时候,一般而言状态总是及时更新的。但在之前的版本中,一个实体获得自身扳机所添加的同名状态时,只有第一次会立刻更新。我们将这个bug称为\term{同名状态更新延迟}或\term{自buff bug}。

\example \version{}{9.4.0}回合结束时,如果你的\card{炎魔之王拉格纳罗斯}首先将对手的8/8\card{格鲁尔}打成0血,格鲁尔的扳机触发但无法把自己变为9/1存活。在当前版本下,格鲁尔可以变成9/1活下来。
\example \version{}{17.4.1}你操控多个2/4的\card{暴乱狂战士}并使用\card{圣光炸弹}。第一个暴乱狂战士受到2点伤害,之后的每个暴乱狂战士都只受到3点伤害。在当前版本下,之后的暴乱狂战士受到的伤害递增。

在当前版本下,所有已知的受自buff bug影响的互动均已修复,且很有可能是整体性的机制修改,而非针对个别卡牌间互动的单独修改。

\section{消灭/摧毁/弃牌/烧毁卡牌}

\term{消灭}随从或英雄、\term{摧毁}武器、\term{弃}一张\term{牌}和\term{烧毁}一张\term{牌}实际上是同一个效果在目标实体处于不同位置的时候产生的不同效果。
\notice 造成伤害的效果不是消灭。因此,对手牌中的实体造成伤害并不会使得它被弃掉。

\begin{itemize}
    \item 消灭和摧毁:实体应在战场上或位于奥秘区。实体的\texttt{.TO\_BE\_DESTROYED := 1} 。这个实体将在死亡检索中视为死亡并被移除。详见\nameref{death-creation-step}。
    \item 弃牌:实体应位于手牌。将这张牌从手牌移动到墓地,然后产生弃牌事件。
        \notice 当你执行「弃多张牌」的效果时,首先弃掉所有牌,然后依次结算每一张弃牌的弃牌扳机。
        \example 你操控\card{玛克扎尔的小鬼}并使用\card{末日守卫}。首先你弃两张牌,然后玛克扎尔小鬼使你抽两张牌。
        \notice 被弃掉的牌的弃牌扳机总是在最后结算,因为这属于手牌扳机。详见\nameref{trigger-order-in-different-zone}。
        \example 你操控六个随从,其中包括\card{人偶大师多里安}、\card{玛克扎尔的小鬼}和\card{咆哮魔}。咆哮魔受到伤害使你弃掉\card{镀银魔像}。玛克扎尔的小鬼首先触发并抽到\card{死亡之翼}。人偶大师触发并召唤一个死亡之翼的1/1的复制填满你的场地。镀银魔像无法召唤自己。
    \item 烧毁卡牌:实体应位于牌库。将这张牌从牌库移动到墓地。
\end{itemize}



一个消灭效果作用到手牌的例子是:
\example 你对友方\card{镀银魔像}使用\card{恶魔之箭}。你的\card{紫罗兰教师}触发召唤1/1,再触发\card{飞刀杂耍者}射中对方\card{自爆绵羊},绵羊击杀友方\card{阿努巴尔伏击者}将镀银魔像弹回手牌。结算时你弃掉镀银魔像并召唤它。

一个烧毁效果作用到场上的例子是:
\example 你操控\card{灵魂歌者安布拉}和四个其它随从,并使用\card{亡者大军}。首先召唤\card{灰熊守护者}I,它触发亡语召唤牌库中的已经被\card{班纳布斯}变为零费的灰熊守护者II。接下来需要召唤/烧毁的卡牌是就灰熊守护者II,但由于你已经满场,亡者大军将其烧毁,即消灭了灰熊守护者II。
\notice 上面的例子表明了亡者大军的结算是在效果一开始就选定了牌库顶的五张牌,然后根据「这张牌的类型」和「你是否已经满场」依次决定将它们烧毁还是召唤到场上。类似地,如果例子中的两个灰熊守护者换成\card{疯狂的科学家}和\card{镜像实体},镜像实体也会被亡者大军烧毁,在动画上表现为奥秘被摧毁(与\card{照明弹}动画类似)。

\section{抽牌}
\term{抽牌}即将牌库顶的牌移动到手牌。

当你执行「抽一张牌」的效果前,游戏首先检查你的牌库是否为空。若是,则改为获得一次疲劳。再检查你的手牌是否已满。若是,则改为烧毁一张牌。
\notice 疲劳不属于抽牌。烧毁卡牌既不属于抽牌,也不属于弃牌。相应的扳机均不会触发。

当你执行「抽多张牌」的效果时,游戏首先计算总共抽牌数,然后执行对应次数的「抽一张牌」的效果。
\notice 这意味着当所有抽牌扳机结算完毕后,才执行下一次抽牌。
\example 你有9张手牌且操控\card{克洛玛古斯}。你使用\card{寒光智者}。抽第一张牌时,克洛玛古斯触发并填满你的手牌。第二张牌会被烧毁。
\notice 这意味着抽牌效果可以嵌套在抽牌效果里。
\example 你使用\card{怨灵之书},第一张抽到\card{地雷}。地雷生效并对你造成5点伤害,然后抽到\card{烈焰风暴}。烈焰风暴是由地雷的效果抽到的,因此不会被弃掉。然后怨灵之书再结算第二、第三张抽牌。
\example 你使用处于流放位置的\card{古尔丹之颅},第一张抽到\card{灵魂残片}。残片生效为你恢复2点生命值,然后抽到古尔丹之颅。古尔丹之颅是由灵魂残片抽到的。因此,它不会被减费。

当你抽牌时,若多个抽牌扳机在场,它们按顺序结算。
\notice 一张卡牌的抽到时效果属于手牌扳机,它总是晚于其他场上抽牌扳机。
\example 你操控\card{人偶大师多里安}并抽到\card{破海者}。人偶大师首先触发并召唤一个1/1的复制,然后破海者的效果触发对其造成1点伤害。
\example \version{}{17.4.1}副玩家的你操控2血的\card{勇敢的记者}。对手抽到\card{烈焰巨兽}。烈焰巨兽首先触发并对记者造成致命伤,然后记者无法触发其扳机。在当前版本下,勇敢的记者作为场上扳机总是早于烈焰巨兽触发。

描述为「抽一张牌,并执行E」的效果,会在抽牌结算完毕后才执行E,无论E是否与抽到的这张牌有关。
\example 你操控\card{人偶大师多里安}并使用\card{侏儒实验技师}抽到\card{死亡之翼}。人偶大师触发并召唤一个死亡之翼的1/1的复制,然后侏儒实验技师才会将死亡之翼变为\card[GVG_092t]{小鸡}。
\example 你使用\card{狗头人图书管理员}抽到\card[GVG_056t]{地雷}。地雷生效并对你造成10点伤害,然后抽到\card{小型法术紫水晶}。接下来狗头人的战吼对你的英雄造成2点伤害,你的紫水晶可以升级。

\section{变形}

当一个实体被变形,它的\texttt{.cardId} 变为新的 ID,其余的属性变为新实体的对应属性。如果是变形成另一个实体的复制,则遵从复制的相关规则。变形不是死亡,因此不触发死亡扳机,也不会增加死亡记录。变形也不是召唤,因此不触发任何召唤时或召唤后扳机。
\notice \version{}{11.2}曾经的召唤回溯会对非法术变形触发。在召唤回溯被消除并修改为正式的召唤时扳机之后,它不再与变形互动。详见\nameref{rule-update:11.2}。

\example 你操控\card{血沼迅猛龙}、\card{诅咒教派领袖}、\card{夜色镇执法官}、\card{飞刀杂耍者},然后对迅猛龙使用\card{变形术}。迅猛龙被变形为绵羊,然后什么也不发生。

\subsection{替换}

当一个实体或一些实体被替换时,他们将被移除,然后在相应的区域创建新的实体,如\card{窃魂者阿扎莉娜}。替换与变形的区别在于替换改变\texttt{.EntityID} ,而变形不改变。这意味着一张牌被替换后可以被\card{魔网操控者}减费而变形不能。
\notice 一些效果被描述为替换,但实际上是变形。如\card{黄金狗头人}。
\example 你用\card{黄金猿}得到的传说随从可以被魔网操控者减费,但用黄金狗头人得到的就不行。

\subsection{历史}

\version{}{11.2}曾经的变形与现在不同。那个时候的变形是将原实体除外,并在原位置放一个新的实体。因此,所有指定该随从为目标的效果都不能对新的随从生效。详见\nameref{rule-update:11.2}。
\notice 此时为了解决变形与使用后扳机与召唤后扳机的互动问题,在一个随从战吼变形或者被\card{变形药水}变形之后,会引入一个完成阶段,对新随从触发。此时,如果该随从是战吼变形,那么这个额外的完成阶段与战吼之间是没有死亡结算的。

\section{休眠}
当一个实体进入休眠状态时,它保留自身具有的所有状态和标签。当它苏醒时,这些状态和标签不改变。与变形类似,休眠不是死亡,因此不触发死亡扳机,也不会增加死亡记录。苏醒也不是召唤,因此不触发任何召唤时或召唤后扳机。此外,随从苏醒时会获得「召唤失调」。

\example 你使用\card{暴风城骑士}并令其攻击对手英雄。然后使用\card{玛维·影歌}使暴风城骑士休眠。在暴风城骑士苏醒的回合,它不能再发动攻击。
\example 你操控两个\card{飞刀杂耍者}且使用玛维·影歌休眠先入场的飞刀。在飞刀苏醒之后,它依然比后入场的飞刀先触发扳机。
\exception 具有「亡语:进入休眠状态」的随从,在苏醒时召唤自己一个新的复制,且不保留任何状态或标签。这包括\card{“尸魔花”瑟拉金}和\card{卢森巴克}。

\section{移动}
\label{move}

当一个随从从其他区域移动到场上,或在场上控制权转移时,一般会获得召唤失调。
\exception \card{暗影狂乱}、\card{疯狂药水}和\card{变节}不会使随从获得召唤失调。

\subsection{移动到已满的区域}
\label{move-to-full-zone}

当一个实体尝试移动到一个已满的区域时,一般会发生:

\begin{itemize}
    \item 从牌库移动到场上(如\card{死亡领主}等)时,取消移动。
        \exception 根据卡牌描述,\card{瓦里安·乌瑞恩}改为抽牌,\card{亡者大军}改为烧毁该牌。
    \item 从手牌移动到场上(如\card{先祖召唤}等)时,改为移动到墓地。它不是弃牌。
        \notice 通常的满场移动,例如\card{卑劣的脏鼠}会在满场时取消移动,这看起来是满场移动的默认情况。但下述例子证明了并非如此,它应该是在移动之前做了预先检测。
        \example 你操控\card{扭曲巨龙泽拉库}和其它五个随从,并使用被\card{戈霍恩,鲜血之神}抽上来的暗影猎手沃金。在沃金进场前你先失去生命,导致场上有七个随从。然后沃金进场并结算战吼,将一个随从移回手牌。手牌中即将进场的随从进入墓地而不是留在手牌不动。
    \item 在场上的随从控制权转移(如\card{精神控制技师}等)时,改为消灭该随从。
    \example 你操控六个随从,其中包括\card{布莱恩·铜须}。对手操控五个随从。你使用精神控制技师依次偷取对方后入场的\card{战利品贮藏者}和先入场的\card{发条侏儒},它们被直接移除。接下来的死亡阶段,对手先获得一张零件牌,再抽一张牌。
\end{itemize}

\begin{itemize}
    \item 从牌库移动到手牌(如「抽一张牌」等)时,改为烧毁该牌。
    \item 从牌库移动到对手手牌(如\card{死亡之握}等)时,改为烧毁该牌。
    \item 从场上移动到手牌(如\card{消失}等)时,改为消灭该随从。
    \item 在手牌控制权转移(如裂心者伊露希亚等)时,改为移动到墓地。它不是弃牌。
        \example 对手因\card{瓦迪瑞斯·邪噬}的效果拥有12张手牌。你使用裂心者伊露希亚。你只能获得对手的前10张手牌,另外的两张会被移动到墓地。尽管在动画上显示为弃牌,你和对手的\card{小鬼骑士}均不触发。
\end{itemize}

\begin{itemize}
    \item 从场上移动到牌库(如\card{埋葬}等)时,改为消灭该随从。
    \item 从手牌移动到牌库(如\card{情势反转}等)时,改为移动到墓地。它不是弃牌。
\end{itemize}

\begin{itemize}
    \item 从牌库移动到奥秘区(如\card{疯狂的科学家}等)时,取消移动。
    \item 在奥秘区控制权转移(如\card{科赞秘术师}等)时,改为直接摧毁该奥秘。
\end{itemize}

\subsection{伪区域移动}

有一些描述为区域移动的效果实际上并不是区域移动,例如:

\begin{itemize}
    \item 死亡扳机,这包括:
    \begin{itemize}
        \item 随从/武器自身的区域移动类亡语,例如\card{鼬鼠挖掘工}、\card{玛洛恩}、\card{骷髅骑士}、\card{派烙斯}、\card{弑君}等;
            \exception \card{阿努巴拉克}和\card{莫瑞甘博士}的亡语是真区域移动。
        \item 死亡触发的某些奥秘,例如\card{战术撤离}和\card{诈死};
        \item \card{灵魂回响}添加的亡语;
    \end{itemize}
    \item 一些特定的法术,例如\card{回收}和\card{极恶之咒}。
\end{itemize}

伪区域移动实际上是将原实体除外并在指定位置创建一个复制(可以通过\card{魔网操控者}证实)。它们被特殊处理以保证与真正的区域移动类似,例如\card{瑞文戴尔男爵}不会使伪区域移动类死亡扳机触发两次。
\exception \card{鼬鼠挖掘工}的亡语格外不同。它首先将自己洗入对手牌库,然后再将自己移回你的除外区,最后向对手牌库中洗入一张鼬鼠。这可能是为了动画能够正确显示鼬鼠洗入对方牌库。
\example 双方都操控\card{强能雷象},你的鼬鼠死亡。首先鼬鼠将自己洗入对面牌库,此时\term{对方的}雷象触发。然后它除外自身,并将一张复制洗入对方的牌库。此时它的操控者是你,所以你的雷象也触发。最后一共触发了两次。

\notice 伪区域移动与直接复制不同,它将原实体移除。这使它们与\card{装死}正确互动。
\notice 一些看起来是「起死回生」的效果,例如\card{救赎}或\card{殊死一搏},是召唤原随从的一个复制。它不是任何形式的区域移动。

当一个随从的死亡步骤中有多个区域移动类死亡扳机时,仅有第一个会生效。
\notice 这包括真区域移动、伪区域移动和休眠类效果。
\example 你操控具有\card{丛林之魂}的\card{“尸魔花”瑟拉金}和\card{诈死}。对手消灭你的尸魔花。在接下来的死亡阶段,尸魔花首先进入休眠,然后召唤一个\card[EX1_158t]{树人}。诈死不会触发。

\section{复制}

无论是「使另一个实体变形为某实体的复制」还是「创建一个新实体,这个实体是某实体的复制」,它们都遵循相同的规则。

源实体的属性都会被复制,除了:
\begin{itemize}
    \item 影响该实体的光环不会复制。
        \example 你的对手操控一个\card{暴风城勇士}和一个1血的\card{血沼迅猛龙}。你使用\card{无面操纵者}复制迅猛龙。复制的迅猛龙是0血,然后死亡。
    \item 目标实体的操控者、区域及位置不会复制,而是由该复制效果指定。
    \item 新实体进入战场的时间不会被复制。这意味着\card{无面操纵者}复制之后仍然有召唤失调。
\end{itemize}

源实体的状态是否被复制遵循移动/复制状态规则。复制的新状态附加在目标实体上。
\notice 复制的状态的操控者仍然是原操控者。
\example 对手对其\card{血沼迅猛龙}使用\card{智慧祝福}。迅猛龙攻击时,对手抽牌。你使用\card{无面操纵者}目标迅猛龙。无面的迅猛龙攻击时,仍是对手抽牌,因为状态的操控者没有改变。
\example \version{}{17.4.1}副玩家你手中的\card{变色龙卡米洛斯}在你的回合开始时变形成为对手的已变形的\card{变形卷轴}的复制。在你回合开始时,变色龙身上有变形卷轴的继续变形和变色龙自己的继续变形。其中变形卷轴的继续变形的操控者是作为主玩家的对手,因此它先触发,变成一张随机法师法术牌。接下来,由于变形清除状态,变色龙的自身的「继续变形」被清除。在接下来的每个回合,变色龙都只会按照变形卷轴的规则继续变形。如果你是主玩家,那变色龙的变形会先触发。在接下来的每个回合,变色龙依然按照自身的规则继续变形。
\notice \version{12.0?}{14.0?}在上述例子中,虽然那个状态的操控者是对手,它仍然是在你的回合开始时变形。但是在旧的版本中,这个状态是在\term{状态操控者}的回合开始时变形的。也就是说,旧版本的上述例子中,如果你直接结束回合,变形卷轴的继续变形会在对手回合开始触发并覆盖变色龙的继续变形,而这与主玩家就没有关系了。这个例子于 12.0 版本初次发现,于 14.0 版本发现已经被修改。
\notice \version{17.4.1}{}在当前版本下,由于变形卷轴「继续变形」的状态入场早于变色龙「继续变形」的状态,变色龙在接下来的每个回合中总会按照变形卷轴的规则继续变形。

\subsection{移动/复制状态规则}

在不同的区域移动时,实体是否保留状态遵循以下的规则:
\begin{itemize}
    \item 当实体从牌库移动到手牌、手牌移动到场上、牌库移动到场上时,保留所有状态;
    \item 当实体从手牌移动到牌库、场上移动到手牌、场上移动到牌库,以及从任何区域进入或离开墓地时,不保留任何状态。
        \exception 交易一张牌会保留所有状态。参见\card{tradeable}。
    \item 当一个「入场时休眠」的实体从手牌移动到场上、牌库移动到场上时,进入休眠状态。
\end{itemize}

\notice 当实体从手牌或牌库到场上时,所有的费用状态都会被移除。这保证了实体与\nameref{evolve-and-devolve}的互动正确。
\notice \version{}{18.0?}在之前的版本中,当实体从手牌移动到场上时,沉默效果会被移除(保证其战吼和本身具有的亡语可用),伤害会被清除(保证随从入场时满血)所有附加的亡语也会被移除(\card{瓦兰奈尔}除外)。但在当前的版本中,这些都不会被移除或清除。
\exception 从手牌移动到墓地可能并不会清除所有状态。如果你通过某种手段使手牌中的骨网之卵获得\card{唤尸者}附加的亡语。它在被弃掉时会触发唤尸者的亡语召唤一个自身的复制。然而这个例子并不能完全证实上面的观点:另一种可能的解释是骨网之卵在被弃掉时会记录自身拥有哪些亡语。目前尚无其他卡牌可以证实/证伪这个观点。注意\card{镀银魔像}实质上是召唤另一个镀银魔像,而不是自身的复制。

在不同的区域之间复制时,实体是否复制状态遵循以下的规则,这包括某实体成为复制和创建一个复制的情况:
\begin{itemize}
    \item 当实体从一个区域复制到相同区域时,保留所有状态;
    \item 当实体从一个区域复制到不同区域时,状态保留与否与移动时状态保留与否相同。
    \item 当一个「入场时休眠」的实体从手牌复制到场上、牌库复制到场上时,进入休眠状态。
    \item 当一个「入场时休眠」的实体从场上复制到场上时,不进入休眠状态。
\end{itemize}
\exception \card{塔达拉姆王子}和\card{阴暗的人影}复制「入场时休眠」的实体时,进入休眠状态。

\section{伤害与治疗}
\label{damage-healing}

伤害与治疗的结算流程如下:
\begin{enumerate}
    \item 改变伤害量或治疗量的效果生效。
    \begin{enumerate}
        \item 使伤害量增加的效果首先生效,如\nameref{spell-damage}、\card{英雄之魂}等。
        \item 使伤害量或治疗量加倍的效果其次生效,如\card{先知维伦}、\card{发条机器人}、\card{水晶工匠坎格尔}等。
    \end{enumerate}
    \item \card{奥金尼灵魂祭司}等效果生效,将治疗改为伤害。
    \item 若伤害量为0或目标角色具有\nameref{immune},防止此伤害。
    \item \term{预伤害扳机}列队结算。
    \item 若目标具有\nameref{divine-shield},则移除圣盾并将伤害量改为0。
    \item 伤害减少实体的护甲或生命值减少;治疗增加实体的当前生命值。
    \item 若受伤害的实体受\card{命令怒吼}效果影响,则改变伤害量使实体受伤后的生命值不小于1。
    \item 若伤害减少了目标的护甲或生命值,\term{伤害扳机}列队结算;若治疗使目标的生命值改变,\term{治疗扳机}列队结算。
\end{enumerate}

\notice 在游戏中有「每当你受到伤害」和「在你受到伤害后」两种伤害扳机的描述。它们实际上没有区别,按照入场顺序列队结算。
\notice \version{}{14.2}在之前的版本中,\card{奥金尼灵魂祭司}等效果早于改变伤害量或治疗量的效果。这导致了治疗被转化为伤害后无法受到\card{水晶工匠坎格尔}加成。

\card{小鬼爆破}召唤\card[GVG_045t]{小鬼}的数量和\card{岩浆爆裂}召唤\card{熔岩元素}的数量和在一开始就决定好了,与实际造成的伤害无关。注意若小鬼爆破的伤害被免疫防止或圣盾改为0时不召唤小鬼,但岩浆爆裂无论如何都召唤元素。
\example 你使用\card{命令怒吼},并对你1血的\card{苦痛侍僧}使用小鬼爆破。小鬼爆破的伤害量为4点。但因命令怒吼的效果,苦痛侍僧实际生命值没有减少,因此你不能抽一张牌。接下来,你召唤4个小鬼。
\example 你使用命令怒吼,并令你2血的\card{闪金镇步兵}攻击对手\card{血虫}。血虫只能吸1点血。

\subsection{预伤害扳机}
\label{predamage-trigger}

在伤害生效前,部分扳机可以改变伤害。这些扳机被称作\term{预伤害扳机}。

响应英雄受到伤害的预伤害扳机结算顺序如下:
\begin{enumerate}
    \item 改变受伤目标的扳机\card{博尔夫·碎盾}结算,并将此伤害改为随从伤害。
        \notice 碎盾触发后受伤目标不再是你的英雄,其他的英雄预伤害扳机都不能再触发,而随从预伤害扳机可以触发。若你操控多个碎盾,仅有最先入场的会生效。
    \item 改变伤害量的扳机\card{复活的铠甲}、\card{诅咒之刃}和\card{虚触侍从}列队结算。
    \item \card{埃辛诺斯壁垒}触发。
    \item \card{寒冰屏障}触发。
\end{enumerate}

响应随从受到伤害的预伤害扳机结算顺序如下:
\begin{enumerate}
    \item 改变受伤目标的扳机\card{钳嘴龟盾卫}结算。
    \item 改变伤害量的扳机\card{月牙}结算。
    \item 改变伤害量的扳机\card{莫尔葛工匠}结算。
\end{enumerate}
\notice 复活的铠甲和诅咒之刃按入场顺序结算,但月牙总是早于莫尔葛工匠触发。这可能是一个bug。

一次伤害可以被数个钳嘴龟盾卫多次转移,但每个钳嘴龟盾卫只能对一次伤害触发一次。直观上,如果场上连续存在多个钳嘴龟盾卫,伤害会传导到最左或最右的盾卫身上。
\notice 在一次伤害的传导中,若目标随从两侧都有盾卫且均未触发过,先入场的会生效。
\example 你操控从左到右依次入场的七个钳嘴龟盾卫,按入场顺序记为A、B、C、D、E、F 和G。对手使用\card{魔爆术}。
\begin{itemize}
    \item A将受到的伤害依次传导到B、A。
    \item B将受到的伤害依次传导到A、B、C、D、E、F、G。
    \item C将受到的伤害依次传导到B、A。
    \item D将受到的伤害依次传导到C、B、A。
    \item E将受到的伤害依次传导到D、C、B、A。
    \item F将受到的伤害依次传导到E、D、C、B、A。
    \item G将受到的伤害依次传导到 F、E、D、C、B、A。
\end{itemize}

\section{死亡记录}

尽管游戏中存在墓地区,但复活效果并不是将墓地区的卡牌移回场上,其实质上是检测死亡记录,并召唤死亡记录中相应随从的复制。相关规则如下:

当一个实体被移除时,死亡记录会记录此次死亡事件,包括相应随从的卡牌编号\texttt{.CardID} 、控制者和回合数。
\notice 如果一个实体因爆牌、弃牌等效果进入墓地区,它不会被加入死亡记录。

如果一个实体死亡多次,每一次死亡事件都可以被加入死亡记录。
\example 你使用\card{阿努巴拉克}后对手将其消灭,如此重复三次。你使用\card{恩佐斯},其战吼复活三个阿努巴拉克。

任何效果都不会将死亡记录中的死亡事件删去。这包括「此随从被复活效果选中」、「此随从最终不在墓地区」等。
\example 你的死亡记录中仅包括一个\card{大法师瓦格斯}。你使用\card{复活术}复活大法师。回合结束阶段,大法师施放的复活术还可以复活一个大法师。
\example 你的死亡记录中仅包括一个\card{克苏恩}。你操控\card{布莱恩·铜须}并使用\card{厄运召唤者},其战吼将两个克苏恩洗回你的牌库。
\example \version{}{旧版本} 你的\card{阿努巴拉克}死亡,其亡语将它移回手牌。你使用复活术依然可以召唤一个阿努巴拉克。

除了\card{克尔苏加德}按入场顺序复活随从外,其余复活效果都在符合条件的随从中随机选择。

复活效果只检测死亡记录中随从的原始状态。
\example 你对你的\card{小精灵}使用\card{剑龙骑术},然后使用\card{食肉魔块}吃掉小精灵。对手对你的模块使用\card{灵魂虹吸},模块亡语召唤两个1/1的小精灵,然后对手使用\card{扭曲虚空}清场。接下来你使用\card{恩佐斯},其战吼只能复活一个模块,且这个模块的亡语不能召唤任何随从。

不同随从可以具有相同的卡牌编号,对于这些随从,死亡记录会记录更详细的内容以正确复活。
\example 你使用\card{合成僵尸兽}选择的第一个野兽和其组成的僵尸兽具有相同的卡牌编号。当僵尸兽死亡时,死亡记录会记录组成它的两种野兽\texttt{.MODULAR\_ENTITY\_PART\_1}与\texttt{.MODULAR\_\allowbreak ENTITY\_PART\_2} 。
\example \card[ICC_047t]{命运织网蛛}抉择后的两种不同形态具有相同的卡牌编号\texttt{.CardID = ICC\_047t} 。当命运织网蛛死亡时,死亡记录会记录它的秘密抉择选项\texttt{.HIDDEN\_CHOICE} 。
\notice 抉择前的\card{命运织网蛛}卡牌编号\texttt{.CardID = ICC\_047} ;抉择同时生效的\card[ICC_047t2]{命运织网蛛}卡牌编号\texttt{.CardID = ICC\_047t2} 。他们均是不同的实体。
\example 身材不同的\card{老虎}和被教会不同法术的\card{德鲁斯瓦恐魔}也具有相同的卡牌编号。

具有相同卡牌编号的不同随从会被\card{小型法术钻石}视为相同随从。他们不能被同时复活。

\section{法力值消耗}
\label{cost}

当多个效果改变一张牌的法力值消耗时,它们按入场顺序排列。

状态类效果(如\card{碎枝}、\card{露娜的口袋银河}等)的入场时间点为此状态添加到目标卡牌上的时间点。
\notice 部分效果并不是直接向卡牌添加状态来实现的,这类效果也符合本规则。如\card{洛欧塞布}的战吼向对手添加「本回合你的法术增加5费」,这一效果影响对手手中所有法术。它的入场时间是洛欧塞布战吼的时机。
\example 对手操控\card{法术共鸣}。你依次使用洛欧塞布和\card{寒冰箭}。接下来对手的回合里,对手法术共鸣获得的寒冰箭为0费。
\exception \card{异教低阶牧师}的表现与洛欧塞布不一致。
\example 对手操控\card{法术共鸣}。你依次使用异教低阶牧师和\card{寒冰箭}。接下来对手的回合里,对手法术共鸣获得的寒冰箭为1费。

光环类效果(包括\card{艾维娜}等持续型光环和\card{暮陨者艾维娜}等消耗型光环)改变一张牌的费用实质上也是通过添加一个费用改变状态来实现的。其状态的入场时间点如下:
\begin{itemize}
    \item 当一个具有费用光环的随从入场时,它会为手牌中相应的卡牌添加费用改变状态。此状态的入场时间点即为具有光环的随从的入场时间点。
        \example 你持有被\card{索瑞森大帝}减为2费的\card{炎爆术}并使用纳迦海巫。炎爆术变为5费。回合结束索瑞森大帝的效果触发,将炎爆术减为4费。对手杀死你的纳迦海巫,你的炎爆术变为1费。
    \item 当一个具有消耗型费用光环的随从刷新其光环时,它会为手牌中相应的卡牌添加费用改变状态。此状态的入场时间点即为光环刷新的时间点。其中:
    \begin{itemize}
        \item 暮陨者艾维娜在回合开始时刷新光环。即使你本回合没有使用牌,暮陨者的效果也会在回合结束时失效,在你下个回合开始时刷新。
        \item \card{小个子召唤师}和\card{卡雷苟斯}在回合开始时刷新光环。如果你本回合没有使用牌,回合结束时小个子召唤师和卡雷苟斯的效果\term{不会}失效。这样在你的下回合开始时,小个子召唤师和卡雷苟斯就\term{不会}刷新光环。
    \end{itemize}
        \example 你操控暮陨者艾维娜并消灭你的\card{沟渠潜伏者}。其亡语召唤了纳迦海巫。此时你没有使用过任何卡牌,你的所有卡牌均变为5费。下个回合开始时,暮陨者的效果刷新,你的所有卡牌变为0费。接下来你抽到一张死亡之翼。暮陨者和纳迦按入场顺序为其添加状态,因此死亡之翼为5费。
        \example 你操控小个子召唤师并消灭你的沟渠潜伏者。其亡语召唤了纳迦海巫。此时你没有使用过任何随从,你的所有随从均变为5费。下个回合开始时,小个子召唤师的效果\term{不}刷新,此时你手牌中的所有随从仍为5费。你使用\card{小精灵}。再下个回合开始时,小个子召唤师的效果刷新,此时你手牌中的所有随从变为4费。
        \notice \version{}{?}在之前的版本里,卡雷苟斯没有对应的费用改变状态,其效果的入场时间点始终为卡雷苟斯入场时。

    \item 当一张牌进入手牌时,场上所有相应的光环为其添加费用改变状态。此状态的入场时间点即为这张牌进入手牌时。如果场上有多个光环为其添加费用改变状态,它们按光环具有者的入场顺序添加。
        \example 你使用\card{纳迦海巫}和露娜的口袋银河,然后从牌库里抽到一张\card{死亡之翼}。纳迦海巫的效果在你抽到死亡之翼时才生效,因而晚于口袋银河。死亡之翼最终为5费。
    \item 更准确地讲,当一张牌进入手牌时,由光环所添加的费用改变状态的入场时间点为这张牌进入你手牌后下一次光环更新的时间点。
        \example 你操控\card{影舞者索尼娅}和\card{机械跃迁者}(无论入场顺序),然后消灭你的\card{机械异种蝎}。在接下来的死亡阶段,你首先获得一个机械蝎的复制并因索妮娅的效果变为1费,然后在接下来的光环更新中跃迁使其变为0费。
        \example 你操控\card{凯尔萨斯·逐日者}且使用本回合第二张法术自然研习。在法术的使用阶段,凯尔萨斯刷新其光环,你手牌中的法术变为1费。自然研习先发现一张法术牌,再为你手牌中所有的法术牌减1费。因此,你之前的法术变为0费,新发现的法术减1费。法术结算阶段结束,在光环更新步骤中凯尔萨斯为你新发现的法术添加状态使其变为1费。因此,你之前的法术为0费,新发现的法术为1费。你送掉场上的\card{战利品贮藏者}并抽到一张法术牌。在光环更新步骤中凯尔萨斯和自然研习按顺序为其添加状态,因此新抽到的法术也为0费。
        \example 你操控艾维娜和\card{人偶大师多里安},并使用\card{热情的探险家}抽到一张死亡之翼。人偶大师的效果首先生效,召唤一个1/1的复制。此时光环更新,艾维娜的效果使死亡之翼变为1费。最后,探险家的战吼使死亡之翼变为5费。如果没有人偶大师多里安,死亡之翼首先会因热情的探险家变为5费,然后在接下来的光环更新中变为1费。
        \example 将上述例子中的人偶大师多里安换成\card{克洛玛古斯},那么热情的探险家抽到死亡之翼时,克洛玛古斯的效果首先生效,将一张死亡之翼的复制加入你的手牌。此时光环更新,艾维娜的效果使两张死亡之翼都变为1费。最后,探险家的战吼使前一个死亡之翼变为5费。
\end{itemize}

\version{}{?}在之前的版本中,\card{召唤传送门}的效果具有最高优先级,先于一切其他改变费用的效果。在当前版本中,它已经没有特殊优先级。
\example 无论你以何顺序使用召唤传送门和\card{机械跃迁者},你手牌中的\card{蜘蛛坦克}都是0费。在当前版本中,蜘蛛坦克的费用取决于入场顺序。

\card{异教低阶牧师}、\card{安娜科德拉}等牌具有最低优先级,他们将在其它费用效果结算后生效。

大部分自减费光环也具有最低优先级。
\example 你具有25点生命。你使用\card{热情的探险家}抽到一张\card{熔核巨人}。巨人首先因探险家效果变为5费,然后因其自减费效果变为0费。
\notice \card{蛛魔先知}和\card{荒野骑士}不是自减费光环。
\exception \card{沼泽射线}、\card{铁木树皮}等卡在受到设定费用效果之后,它们的自减费不能生效。

当一个效果提及卡牌的费用时,绝大多数情况下取实际费用。
\example 你依次使用\card{“践踏者”班纳布斯}和\card{“丛林猎人”赫米特}。你牌库中所有随从牌都会被摧毁。
\exception \card{游荡恶鬼}和\card{巨龙召唤者奥兰纳}取卡牌的原始费用。

与攻击力类似,如果费用的最终计算结果为负数,它将被调整为0。
\example 你手牌中有一个被\card{索瑞森大帝}减了五回合费用的\card{小精灵}。你抽到\card{雪鳍企鹅},然后使用\card{模拟幻影}。哪个随从被复制是不确定的。

\section{法力水晶}
双方玩家各拥有一系列与法力水晶相关的标签:
\begin{itemize}
    \item \term{最大法力上限}\texttt{.MAXRESOURCES}:你的法力上限可以达到的最大值,通常为10。在部分冒险或乱斗中可能不为10。
    \item \term{当前法力上限}\texttt{.MAXRESOURCES}:你当前的法力上限,你每个回合开始时增加1,但不超过最大法力上限。
    \item \term{当前已消耗法力值}\texttt{.RESOURCES\_USED}:本回合你因使用卡牌或上回合的过载消耗掉的法力值。
    \item \term{临时法力值}\texttt{.TEMP\_RESOURCES}:本回合你临时获得的法力值,如通过\card{激活}。
    \item \term{下回合过载值}\texttt{.OVERLOAD\_OWED}:将在下个回合生效的过载。
    \item \term{本回合过载值}\texttt{.OVERLOAD\_LOCKED}:本回合你具有的过载。
\end{itemize}
\notice 「当前法力值」不是一个标签,而是通过「当前法力上限」+「临时法力值」+「当前已消耗法力值」计算得到的。

这些标签在以下情况中变化:
\begin{itemize}
    \item 在你的回合开始时,将你的「当前法力上限」增加 1(不超过最大法力上限),然后将你的「当前已消耗法力值」和「本回合过载值」置为「下回合过载值」,再将「下回合过载值」清零。
    \item 在你的回合结束时,将你的「临时法力值」清零。
    \item 当你获得一个满的法力水晶时,若你的「当前法力上限」未达到「最大法力上限」,则你的「当前法力上限」增加1点,「当前已消耗法力值」不变。若达到,则你的「当前已消耗法力值」减少1点但不低于0,「当前法力上限」不变。如果「当前法力上限」与「临时法力值」之和超过「最大法力上限」,则「临时法力值」要相应地减少1点,使得它们之和不超过,然后若此时你的「当前已消耗法力值」不为0,则减少1点。动画上看来,一个临时水晶被填入了一个空的真实水晶槽。
        \example 你在6费回合使用\card{滋养}。你的当前法力上限增加为8,而当前已消耗法力值保持6不变。因此你的法力栏显示为2/8。
        \example 你在10费回合已经消耗掉了9点法力水晶,然后你使用一张0费的\card{生物计划}。你的当前已消耗法力值减少2点,你的法力栏显示为3/10。
        \example 你在7费回合已经消耗掉了6点法力水晶,你使用三张激活使你的临时法力值变为3,此时你的当前法力值为$7+3-6=4$点。然后你使用一张0费的\card{生物计划}。你的法力上限变为9,它与临时法力值之和超过了10。因此你的临时法力值减2点变为1。相应的,你的当前已消耗法力值减少2点,因此你的当前法力值为$9+1-4=6$点。从动画上来看,你法力栏最右侧的两个临时水晶被填入了两个空的真实水晶槽。
        \history \version{}{15.4.0?}在之前的版本中,当「当前法力上限」与「临时法力值」之和超过「最大法力上限」时,多出的临时法力值直接消失而你的当前已消耗法力值不会减少。例如在上述例子中,你的临时法力值减2点变为1。而你的当前已消耗法力值不减少,因此你的当前法力值仍为$9+1-6=4$点。从动画上来看,你法力栏最右侧的两个临时水晶像是被挤掉了一样。
    \item 当你获得一个空的法力水晶时,你的「当前法力上限」和「当前已消耗法力值」均增加1点。
        \example 你在3费回合使用\card{野性成长}。你的当前法力上限和当前已消耗法力值均增加1。因此你的法力栏显示为0/4。
    \item 当你获得一个临时法力水晶的时候,如果「当前法力上限」与「临时法力值」之和未达到「最大法力上限」,你的「临时法力值」增加1点。若达到,则你的「当前已消耗法力值」减少1点,但不低于0。
        \example 你在9费回合已经消耗掉了9点法力水晶,并使用两张激活。你首先获得1点临时法力值,然后当前已消耗法力值减少1点,因此你的法力栏显示为2/9。回合结束时,你的临时法力值清零,你的法力栏显示为1/9。
        \history \version{15.4.0?}{18.2.0}在之前的某一段时间中,获得临时水晶的逻辑是:当你试图获得一个临时水晶时,若你「当前已消耗法力值」不为0,则减少1点;否则才获得一个临时水晶。例如你在6费回合依次使用\card{林地守护者欧穆}和\card{激活},你的法力栏只能变为6/6而非7/6。
    \item 当你失去一个法力水晶时,你的「当前法力上限」和「当前已消耗法力值」均减少1点,但「当前已消耗法力值」不低于0。
        \example 你在10费回合使用一张0费的\card{恶魔卫士},你的当前法力上限减少1点,因此你的法力栏显示为9/9。
        \example 你在10费回合已经消耗掉了1点法力水晶,并使用一张0费的\card{恶魔卫士},你的当前法力上限和当前已消耗法力值均减少1点,因此你的法力栏显示为9/9。
    \item 当你复原你的法力水晶时,你的「当前已消耗法力值」减少相应的数值,但不低于0。
        \notice \card{遗忘之王库恩}的效果实质上是复原10个法力水晶,即使你当前法力上限尚未达到10。
        \example 你在9费回合过载了25点法力水晶,并使用两张0费的遗忘之王库恩。你的当前已消耗法力值减少20点,为5。因此你的法力栏显示为4/9。
    \item 当你的「当前法力上限」与「临时法力值」之和达到「最大法力上限」时,某些卡牌会产生不同的结算。
    \begin{itemize}
        \item  野性成长、\card{妙手空空}、\card{星界沟通}和过度生长改为获得\card{法力过剩(CS2 013t)}。
            \example 你在8费回合获得了2点临时法力值,并使用一张0费的野性成长。你得到一张法力过剩。
        \item 其他获得空的法力水晶的牌改为什么都不做,如\card{贪婪的林精}。它们不会挤掉你的临时法力值。
            \example 你在8费回合获得了2点临时法力值,并消灭了你的贪婪的林精。什么也不会发生。
    \end{itemize}
\end{itemize}

\section{额外回合}
目前游戏中有两个获得额外回合的手段:时空扭曲和坦普卢斯。
\begin{itemize}
    \item \card{时空扭曲}的效果是在下个回合之前插入一个额外的己方回合。
    \item \card{坦普卢斯}的效果是在下个回合之前插入四个额外回合,首先是两个对手回合,然后是两个己方回合。
\end{itemize}

\notice 在下面的示例中,对手回合记作「E」,己方回合记作「O」。
\example 你先使用坦普卢斯然后使用时空扭曲,接下来的五个额外回合是 O-E-E-O-O。如果调换顺序,先使用时空扭曲后使用坦普卢斯,则接下来的五个额外回合是 E-E-O-O-O。
\example 你使用坦普卢斯,然后对手在其第一个额外回合中也使用坦普卢斯。接下来的七个额外回合是 O-O-E-E-E-O-O。

\notice 绝大多数改变对手下个回合英雄技能或某些卡牌法力值消耗的效果只生效一个回合,即在对手下个回合开始时生效,结束时失效。这包括\card{洛欧塞布}、\card{责难}、\card{持枪恶霸}、\card{异教低阶牧师}。
\example 对手法力上限为10,且持有唯一一张10费的\card{精神控制}。你依次使用洛欧塞布、坦普卢斯和\card{混乱凝视者}。混乱凝视者的战吼发动时,对手手中精神控制仍为10费,因此会被腐化。在对手的第一个额外回合中,精神控制变为15费无法使用,并在回合结束时丢弃。在对手的第二个额外回合中,洛欧塞布才失效。
\exception \card{米尔豪斯·法力风暴}和\card{破坏者}会在连续的多个对手回合中持续生效。此外,米尔豪斯·法力风暴的效果在使用后立刻生效,而不是等到对手下个回合开始才生效。

\section{动画显示}
当玩家进行一个操作时,游戏首先将这个操作的全部内容在服务器结算完毕,然后慢慢地在客户端播放动画。除非投降,游戏会展示所有玩家操作的结果,如召唤一个随从,翻转英雄技能,翻转回合结束按钮等。一般而言,在所有动画播放完毕后,你才能得知一个操作的最终结果,但在某些情况下,你可以在动画播放完毕之前得知一部分结果:

\example 你在4费回合使用\card{银月城传送门}后发现你手中的\card{发条侏儒}变绿。你可以断定银月城传送门召唤了\card{机械跃迁者}。
\example 你使用\card{尤格-萨隆}后,回合结束按钮变为黄色且无法按下。你可以确定这局比赛结束了,尤格-萨隆的战吼杀死了某一方的英雄。
\example 你使用\card{年轻的酒仙}回手一个随从,它在变成卡牌前也会绿色高亮。如果你去操作它,游戏会把它从你手中打出去,这是一个有效的节约时间的方法。
\example 你在抽卡时就听到“收工了”,你知道这回合你只能空过。

如果玩家操作导致的结算量过大,则游戏可能卡顿,甚至使玩家掉线。在重新连接至对局后,客户端直接将操作的最终结果显示给玩家。
\example 你通过\card{送葬者安德提卡}的亡语向牌库中洗入60张牌,游戏会直接卡住不动。在短暂的等待后,游戏会提示「你的对战由于断线而中断」并重新连接。之后你可以看到你的牌库中现在已经洗入了60张牌。

玩家也可以在一个较长的动画播放时手动关闭客户端并重启,或断开网络重连。重新连接至对局后动画不会继续播放,而是直接将最终结果显示给玩家。
\example 在酒馆战棋中,如果你在战斗回合的一开始就关闭客户端并重启,则你会直接进入招募回合。这可以使你充分利用招募回合的时间。

%\section{手牌调度}

%手牌调度<ref>\texttt{Game.STEP = BEGIN_MULLIGAN} </ref>

\section{职业}

目前游戏中有10种不同的职业,分别是:恶魔猎手、德鲁伊、猎人、法师、圣骑士、牧师、潜行者、萨满祭司、术士、战士。
\notice 梦境牌\texttt{.CLASS = DREAM} 的职业虽然不为中立,但不被算作职业卡牌。\card{虚灵商人}不能使这些牌减费,使用这些牌也不会给你的\card{幽灵弯刀}增加耐久度。然而死亡骑士牌\texttt{.CLASS = DEATHKNIGHT} 却被虚灵商人、幽灵弯刀等牌视为职业牌。

「随机获得/变为你对手职业的卡牌」、「随机获得/变为你的职业卡牌」与中立英雄互动时会产生不同的效果,具体效果如下:
\begin{itemize}
    \item 「随机获得你对手职业的卡」效果:\card{吹嘘海盗}、\card{幽暗城商贩}、\card{剽窃}、\card{收集者沙库尔}和\card{闪狐}获得的是\card{幸运币}。
        \exception \card{学术剽窃}和\card{搜索}获得的是随机中立职业卡。
        \exception \card{奈法利安}获得的是两张\card{扫尾}。
    \item 「随机获得你的职业的卡」效果:\card{深蓝系咒师}获得的是随机职业的卡(职业可以不同)。
        \exception \card{黑岩法术抄写员}获得的是两张\card{幸运币}。
    \item 「随机变为你对手职业的卡」效果:\card{莉莉安·沃斯}不会生效。
    \item 「随机变为你的职业的卡」效果:\card{龙眠净化者}会将其变成随机中立卡。
\end{itemize}
