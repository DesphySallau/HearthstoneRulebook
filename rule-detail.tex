\chapter{规则细节}
\label{rule-detail}

\setcounter{tocdepth}{2}
\section{关键字效果}
\label{keyword}

% 简单关键字
\subsection{嘲讽}
\label{taunt}

当一个角色选择攻击目标时,如果有敌人具有\term{嘲讽},则攻击目标必须为某个具有嘲讽的角色。
\notice 英雄也可以具有嘲讽。例如卡拉赞冒险模式的馆长。

如果一个随从又具有嘲讽又具有潜行或免疫,则嘲讽失效。

\subsection{圣盾}
\label{divine-shield}

当一个具有圣盾的角色将受到伤害时,将伤害值改为0且该角色失去圣盾。详见\nameref{damage-healing}。

\subsection{吸血}
\label{lifesteal}

\version{10.0}{} 吸血虽然是一个伤害扳机,但是它对群体伤害具有与其它扳机不同的响应方式:它在所有其它被单次伤害触发的扳机全部结算完毕之后触发一次,为你的英雄恢复等于总伤害量的生命值。
\example 你的对手操控五个\card{小精灵},你使用\card{灵魂鞭笞}。鞭笞对所有小精灵造成1点伤害,然后你恢复5点生命值。如果换成\card{欧米茄灵能者}和\card{圣光炸弹},则每造成一次伤害就回一次血。
\notice 在 10.0 版本之前,群体伤害的吸血与普通的伤害扳机表现形式是相同的:连续触发多次,每次为你的英雄恢复等同于单次伤害的生命。

\subsection{剧毒}
\label{poisonous}

\term{剧毒}是一个伤害扳机。它等价于「在该实体对一个随从造成伤害后,消灭该随从」。

\subsection{风怒}
\label{windfury}

具有风怒的角色每回合可以攻击两次。
\notice \term{超级风怒}是风怒的变种。具有超级风怒的角色每回合可以攻击四次。

\subsection{冻结}
\label{freeze}

被冻结的随从不能攻击。在回合结束后,所有被冻结且具有攻击次数的随从不再被冻结。
\example 你使用\card{铁鬃灰熊}和\card{狂暴的狼人}并将它们冻结。在回合结束后,狼人解冻,而灰熊仍然保持冻结。
\example 你使用\card{洛萨},对手将它冻结。在回合结束时洛萨触发效果。由于洛萨的效果会消耗其攻击次数,它在回合结束后不会解冻。

\subsection{免疫}
\label{immune}

免疫防止该实体将受到的伤害或将减少的耐久。免疫角色不能成为敌方的目标。
\example 双方场上没有随从且英雄都具有免疫。你使用\card{燃烧权杖},所有\card{炎爆术}都只会以你的英雄为目标。如果双方英雄具有的是「无法成为法术或英雄技能的目标」,所有炎爆术都无法生效。
\notice 免疫不等于无敌。免疫仍然会受到消灭效果的影响。

\subsection{抽到时施放}
\label{cast-when-drawn}

当你抽到具有\term{抽到时施放}的牌时,你会直接施放它然后抽一张牌。
\notice 抽到时施放实际为抽牌扳机,而并非施放这张法术本身。因此它不会消耗\card{日蚀}等牌的效果。
\exception \card{沙德拉斯·月树}为法术添加的抽到时施放会施放一个法术,也会消耗日蚀的效果。

如果在一张具有抽到时施放的牌本身效果结算完毕之后,有一方英雄处于濒死状态或已离场,则不会再抽一张牌。
\example 你的生命值为4,抽到一张\card{炸弹}。你不会再抽一张牌,因此也无法抽到接下来的\card{灵魂残片}来救回自己。
\example 你和对手的生命值都为1,你的牌库中只有一张\card{流血}。你抽到流血对对方造成2点伤害。你不会再抽一张牌,因此你赢得此盘对局而不是以平局结束。

% 带参数的简单关键字
\subsection{法术伤害}
\label{spell-damage}

法术伤害增加法术造成的伤害。有的卡牌具有派系法术伤害,这类法术伤害只会增加特定派系法术的伤害。

一些法术免疫法术伤害。这包括:
\begin{itemize}
    \item 对自己造成负面效果且不可控的法术。这包括\card{诅咒}、\card{地雷}、\card{远古诅咒}、\card{炸弹}、\card{堕落之血}等牌。
        \notice 如果法术的伤害是可以自行控制的,那它们通常会受到法术伤害影响,例如\card{亡者复生}、\card{黑暗附体}。
        \exception \card{湮灭}也不受法术伤害影响。
    \item 随从战吼的选项。这些牌逻辑上是随从施放的效果。这包括\card{毁灭}、\card{火之祈咒}、\card{气之祈咒}等牌。
\end{itemize}

还有一些法术在实现上免疫法术伤害;这是由于它们的特殊机制导致的。法术伤害在这些牌的具体效果中被单独处理。这包括:
\begin{itemize}
    \item 飞弹类法术。这是为了让法术伤害增加飞弹数量而非每发飞弹的伤害。这包括\card{奥术飞弹}、\card{复仇之怒}、\card{恐怖丧钟}、\card{狂乱传染}、\card{火山喷发}、\card{燃烬风暴}、\card{克苏恩面具}、\card{克苏恩之眼}、\card{雷区挑战}、\card{噬灵疫病}等牌。
        \notice \card{治疗之雨}也属于这一类牌。尽管它与法术伤害不能互动,但是它可以受到\card{先知维纶}的影响。
        \exception \card{强能奥术飞弹}总是发射固定数量的飞弹。它正常受法术伤害影响。
        \notice 由于\card{噬灵疫病}不受法术伤害影响,因此它受到先知维伦影响时仅仅能翻倍伤害而无法翻倍治疗。
    \item 转移过量伤害的法术。这是为了避免过量伤害受到两次法术伤害的影响。这包括\card{爆炸符文}、\card{火球滚滚}、\card{燃烧}、\card{穿刺射击}、\card{不稳定的暗影震爆}等牌。
\end{itemize}

此外,例如\card{背叛}、\card{末日回旋镖}这类法术,其伤害并非由法术造成,因此也不受法术伤害影响。

% 复杂关键字
\subsection{连击}
\label{combo}

必须在使用这张牌之前使用过其它牌才能结算的效果。

某些牌可能同时具有战吼和连击。这些牌实际上只有战吼,该战吼会根据之前有无使用过其它的牌改变效果。
\example 你操控\card{布莱恩·铜须},并使用\card{毁灭之刃}。尽管铜须并不影响连击效果,毁灭之刃仍然会造成两次2点伤害。

\subsection{流放}
\label{outcast}

当你从手牌最左或最右使用这张牌的时候会产生的效果。
\notice 与其它流放牌不同,\card{眼棱}的流放效果是一个在流放位生效的自身费用光环,而非结算阶段的效果。
\notice 与\nameref{combo}类似,同时具有战吼和流放的牌实际上仅仅具有战吼,该战吼在结算过程中因该牌是否在流放位具有不同的效果。

流放只有在你从手牌中使用这张牌的时候可以触发。
\example 你使用了流放位的\card{幽灵视觉}。你的\card{大法师瓦格斯}和对手的\card{永恒巨龙姆诺兹多}重复这个法术时均不能流放。
\example 对手使用了\card{隐秘破坏者}拉出你唯一一张手牌幽灵视觉。它不能流放。

\subsection{发现}
\label{discover}

从三张牌中选择一张。在没有额外说明的情况下,选择的牌将置入你的手牌。
\notice 发现也可能发现英雄技能。例如\card{芬利·莫格顿爵士}、\card{沙漠爵士芬利}、\card{偷师学艺}等等。
\exception \card{迈拉·腐泉}、\card{九命兽魂}、\card{战歌驯兽师}和\card{墓园召唤}在将发现的牌置入你的手牌同时还有额外的操作。这与其它卡牌,例如\card{永恒奴役}等不同。
\notice 如果这张牌是从你牌库中发现的,例如\card{追踪术}、\card{暗中生长}或战歌驯兽师,则这个置入手牌的动作属于抽牌,将正常触发抽牌扳机。

发现效果分为如下几种类型:
\begin{itemize}
    \item 发现范围固定。例如\card{织法巨龙玛里苟斯}、\card{沙漠爵士芬利}。
    \item 发现范围为游戏内的固定范围。例如\card{拉祖尔女士}(对方手牌)、\card{追踪术}(你的牌库)。
    \item 发现范围为可收集牌(可能带有条件限制),但限定了职业。例如\card{荆棘帮蟊贼}、\card{盗取武器}、\card{魔杖窃贼}。
    \item 发现范围为可收集牌,且未限定职业。例如\card{深渊探险}、\card{导师火心}、\card{复苏}等等。这类发现的限定范围为你的职业卡牌与中立牌。
        \notice 描述为「任意职业」的牌不属于未限定职业,例如\card{钥匙守护者艾芙瑞}。
\end{itemize}

如果一个发现效果为上述类型的后两者,且限定了某个职业(例如\card{幻觉}、\card{扭曲学识}或\card{虚灵巫师},则还有以下例外情况:
\notice 如果限定的职业为中立,或限定职业后发现范围中没有没有可发现的牌,则会使用发现牌本身的职业加上中立卡牌进行发现。如果发现牌本身为中立牌,则会随机选择一个职业加上中立卡牌进行发现。
\example 牧师使用\card{套圈圈}发现的是法师奥秘;使用\card{奥秘图纸}发现的是猎人奥秘;使用\card{水文学家}发现的是圣骑士奥秘。潜行者使用这三张牌均发现潜行者奥秘。
\history \version{}{KNC} 在狗头人版本已实装但还未上线的时间内,潜行者拥有三张尚未实装的奥秘。此时使用水文学家不会发现任何牌。
\example 你的英雄为\card{炎魔之王拉格纳罗斯}。你使用\card{虚灵巫师}。你会发现法师法术牌。
\example 你的英雄为\card{炎魔之王拉格纳罗斯}。你使用\card{奥能水母}。你会发现来自同一职业的法术牌。
\example 你的英雄为\card{炎魔之王拉格纳罗斯}。你的对手使用\card{幻觉}。他会发现潜行者职业牌。
\example 牧师使用\card{先到先得}发现的是萨满职业牌。
\exception 尽管德鲁伊有一张职业过载牌\card{雷霆绽放},但德鲁伊使用先到先得仍然可以发现萨满职业牌。

% 版本关键字
\subsection{反制和失效}
\label{counter}

目前游戏中只有\card{法术反制}一张可收集牌拥有反制这个关键字。但除此之外,如果你装备一把冰冠堡垒冒险中的\card[ICCA08_020]{霜之哀伤},你的所有手牌处于失效状态。这为我们测试法术以外的牌失效的情况提供了仅有的手段。反制实质上是将那个法术设置为失效状态,所以本节中将一视同仁地处理。

如果一张牌失效,它的\texttt{.CANT\_PLAY := 1} 。然后,这张牌的法术效果/战吼/连击等等不结算,且该序列中的所有扳机不能触发。
\notice 这不是在说「任何事情都不会发生」—— 在法术反制触发之前的那些步骤自然可以正常进行。这包括所有费用状态的自我移除(已测试的有\card{肯瑞托法师}、\card{暗金教侍从}、\card{血色绽放}、\card{古加尔}、\card{亡鬼幻象}、\card{伺机待发}和\card{墨水大师索莉娅})。此外,\card{暮陨者艾维娜}与\card{卡雷苟斯}的光环也会切换。\card{伊莱克特拉·风潮}和\card{星界密使}的状态移除是在完成阶段,因此会被法术反制阻止。
\exception 失效的\card{黑暗之主}可以休眠。在它休眠之后,由于它已经离场,失效状态被清除,它的战吼可以正常结算。
\exception \version{17.2.1}{} 即使一张法术被\card{法术反制}反制,它仍然可以增加\card{学习龙语}的进度。这很可能是有意为之。

如果一张牌失效,它将会被送去墓地(时机尚不明确,但看起来是在相当早的时候)。这包括这张牌是随从或武器的极为特殊的情况(它们的亡语可以触发)。此外,你装备着\card[ICCA08_020]{霜之哀伤}使用武器并不会替换你的武器。新的武器会被直接摧毁送去墓地,你仍然装备霜之哀伤且仍然具有攻击力。但是由于显示bug,看起来你像是没有装备武器。

如果一张牌失效,所有计数器不会增加计数。
\example 当你使用一张法术且被反制后,\card{墓园恐魔}、\card{奥术统御者}、\card{奥术巨人}、\card{没电的铁皮人}、\card{大法师瓦格斯}、\card{暗金教水晶侍女}、\card{法力飓风}、\card{疾疫使者}、\card{卡格瓦,青蛙之神}、\card{巨龙召唤者奥兰纳}、\card{尤格-萨隆}、\card{祖尔金}、\card{黑曜石碎片}、\card{苔丝·格雷迈恩}都将像你并未使用过这张法术一样地处理它们的效果。
\exception 即使你使用了一张被反制的法术,你手牌中所有\nameref{combo}牌也都可以被激活。这与其它计数器显著不同。

\subsection{进化}
\label{adapt}

\term{进化}一个随从指,从下列十项效果中随机选择三项,你从中选择一项。该随从获得该效果。进化可选择的效果包括:\\
嘲讽、圣盾、风怒、剧毒、+1/+1、+3攻击力、+3生命值、获得「亡语:召唤两个1/1的孢子」、潜行直到下回合、获得「无法成为法术或英雄技能的目标」。

如果你是在酒馆战棋中进化,则改为从上述十项的前八项中随机选择。

如果进化的随从不在场上或濒死,则什么都不会发生。

\subsection{回响}
\label{echo}

在你使用一张回响牌后,将一张该牌的复制加入你的手牌。这个复制在回合结束时会移除。

具有回响状态的牌的法力值消耗不能低于1点。这个效果将会覆盖其余任何费用效果。

\card{不稳定的异变}、\card{女巫杂酿}和\card{沼泽女巫的召唤}不是回响。他们的「回响牌」的法力值消耗可以低于1点。
\notice 但在一个与\card{古神在上}的互动bug中,它们的表现与回响一致。

\subsection{超杀}
\label{overkill}

超杀是伤害扳机。与吸血类似,超杀也只会在所有其它伤害扳机结算完成之后触发一次。
\example 你令你的\card{黑心票贩}攻击对手的\card{苦痛侍僧}。无论入场顺序,苦痛侍僧先抽牌,然后黑心票贩抽牌。
\example 你令你的先入场的黑心票贩攻击对手1血的\card{兽人铸甲师},且对手控制一个后入场的\card{铸甲师}。无论入场顺序,首先结算「黑心票贩对兽人铸甲师造成伤害」,铸甲师给对手1点护甲,然后黑心票贩抽牌。接下来结算「兽人铸甲师对黑心票贩造成伤害」,兽人铸甲师给对手2点护甲。

超杀触发与否只与造成伤害时目标的血量是否为负数有关。如果目标在造成伤害后,超杀触发前又被救回0血或以上,超杀依然触发。
\example 你操控两个3血\card{咆哮魔}和一个1血\card{小鬼骑士},并使用\card{冲击波}。冲击波对所有随从造成2点伤害,然后咆哮魔触发你弃两张牌并把小鬼骑士变成5/1,然后你获得一张随机法师法术。

\subsection{复生}
\label{reborn}

在一个具有复生的随从死亡后,会召唤一个它的复制,但该复制生命值为1且不具有复生。

\subsection{可交易}
\label{tradeable}

你可以在你的回合中支付1点费用将\term{可交易}随从洗回你的牌库并抽一张牌,这称为\term{交易}。这是一个主动效果,且不属于使用一张牌。可交易牌首先将本身洗回牌库,但交易抽到的牌不会是这张牌本身。

可交易牌在交易的时候保留所有增益。
\notice \version{21.0}{21.3}可交易牌在进行其它区域移动时也保留所有增益。

\section{非关键字效果}

\subsection{健忘}
\label{forgetful}

一些牌具有「有50\%的概率攻击错误的敌人」。这包括\card{食人魔步兵}、\card{食人魔忍者}、\card{穆戈尔的勇士}、\card{砂槌萨满祭司}、\card{食人魔战槌}、\card{食人魔勇士穆戈尔}和\card{乐观的食人魔}。此外,\card{莫什奥格播报员}也会让攻击它的角色享受类似效果。

如果该50\%概率检测没有命中,则该扳机根本不会触发。其闪电符号也不会亮起。

\subsection{异变和衰变}
\label{evolve-and-devolve}

在本节中,\term{异变}指「随机将某随从变形为法力值消耗增加N点的随从」这一效果。相对的,\term{衰变}指「随机将某随从变形为法力值消耗减少N点的随从」。在当前的对战模式中,包含异变的牌有\card{异变}、\card{异变之主}、\card{死亡先知萨尔}(及\card{灵体转化})、\card{不稳定的异变}、\card{突变}和\card{女巫跟班}。包含衰变的牌有\card{衰变}和\card{衰变飞弹}。此外,\card{终极巫毒}与此机制有相关联之处,在此一并讨论。

\version{16.0}{} 当你异变一个「不存在所需法力值消耗的随从」的随从时,会变形为法力值消耗尽可能高(不会低于原始费用,也不会超出所需的费用)的随从。相对的,当你衰变一个零费随从时,会变形为随机的零费随从。类似地,如果\card{侏儒变形师}变形一个不存在对应费用的随从,则该随从会重新变形为自身。
\example 你操控只有1点生命值的普通\card{熔核巨人}。你对其使用金色的\card{突变}。它被变形成为一个满血的金色熔核巨人。
\example 你操控\card{雪怒巨人}。你使用\card{死亡先知萨尔}。它被变形成为一个随机的12费随从。
\example 你操控\card{猛虎之灵},然后依次使用\card{古加尔}和一个23费的法术。猛虎之灵为你召唤一个23费的猛虎。你将其打到1点生命值,然后对其使用\card{突变}。猛虎未被变形成随机的20费或25费随从,而是变形为一个原始状态的猛虎(尽管猛虎并不是可收集随从)。
\notice 在之前的版本中,这些随从不会被变形。
\notice 终极巫毒的修改要更早。它的早期修改与异变修改的不同见下。

如果你在满手牌时令一个被贴了\card{终极巫毒}的随从攻击,且该随从触发\card{冰冻陷阱}导致回手(然后由于你满手牌被爆掉),终极巫毒会认为该随从的法力值消耗是已经被冰冻陷阱增加之后的数值。
\example 你有十张手牌,然后令一个贴有终极巫毒的\card{血沼迅猛龙}攻击。触发敌方冰冻陷阱之后它死亡,并召唤一个随机5费随从。
\example \version{早期版本}{} 你有十张手牌,然后令一个贴有终极巫毒的\card{机械克苏恩}攻击。触发敌方冰冻陷阱之后它死亡,并召唤一个随机12费随从。
\example \version{早期版本}{16.0} 你有十张手牌,然后令一个贴有终极巫毒的\card{山岭巨人}攻击。触发敌方冰冻陷阱之后它死亡,并什么都不召唤(而不会召唤一个12费随从,这是早期修改与异变16.0修改的不同之处)。
\example \version{16.0}{} 你有十张手牌,然后令一个贴有终极巫毒的\card{山岭巨人}攻击。触发敌方冰冻陷阱之后它死亡,并召唤一个随机12费随从。

\notice 大多数类似的效果不会作这种修正。
\example 你操控\card{召唤石},并通过\card{古加尔}施放了一个13费的法术。召唤石不会为你召唤随从。
\exception \version{早期版本}{} \card{鲁莽试验}现在与之不同。它也遵循与异变类似的规则,会召唤费用尽可能高的随从。
\example 你操控三个\card{玛里苟斯},并使用鲁莽试验。尽管并不存在17费的随从,鲁莽试验为你召唤了两个随机的12费随从。

\setcounter{tocdepth}{1}
\section{扳机所响应的事件}

游戏中的大多数扳机所响应的事件都是明确的,例如「在你召唤一个随从后」响应召唤后事件,「每当一个随从获得治疗」响应治疗事件。但是有些扳机由于文本所限不能准确地描述自己所响应的是什么事件。本节将对其进行说明。

\subsection{任务和法术石}
安戈洛任务:
\begin{itemize}
    \item 使用时扳机:\card{火羽之心}
    \item 使用后扳机:\card{湿地女王}、\card{打开时空之门}、\card{最后的水晶龙}、\card{探索地下洞穴}
    \item 召唤后扳机:\card{丛林巨兽}、\card{唤醒造物者}、\card{鱼人总动员}
    \item 弃牌扳机:\card{拉卡利献祭}
\end{itemize}

奥丹姆任务:
\begin{itemize}
    \item 回合结束扳机:\card{发掘潜力}
    \item 使用时扳机:\card{洗劫天空殿}
    \item 使用后扳机:\card{制作木乃伊}、\card{腐化水源}
    \item 召唤后扳机:\card{打开宝库}
    \item 攻击后扳机:\card{侵入系统}
    \item 置入手牌扳机:\card{劫掠集市}
    \item 抽牌扳机:\card{最最伟大的考古学}
    \item 治疗扳机:\card{激活方尖碑}
\end{itemize}

暴风城任务线:
\begin{itemize}
    \item 使用时扳机:\card{挺身而出}
    \item 使用后扳机:\card{寻求指引}、\card{探查内鬼}、\card{巫师的计策}、\card{号令元素}、\card{开进码头}
    \item 攻击力增加扳机:\card{游园迷梦}
    \item 伤害扳机:\card{保卫矮人区}、\card{恶魔之种}
    \item 抽牌扳机:\card{一决胜负}
\end{itemize}

支线任务:
\begin{itemize}
    \item 回合开始扳机:\card{庇护}
    \item 使用时扳机:\card{人多势众}
    \item 使用后扳机:\card{学习龙语}、\card{元素盟军}
    \item 召唤后扳机:\card{正义感召}、\card{扫清道路}
    \item 攻击后扳机:\card{保护甲板}
    \item 激励扳机:\card{病毒增援}
\end{itemize}

法术石:
\begin{itemize}
    \item 使用时扳机:\card{小型法术黑曜石}
    \item 使用后扳机:\card{小型法术翡翠}、\card{小型法术红宝石}、\card{小型法术钻石}
    \item 伤害扳机:\card{小型法术紫水晶}
    \item 治疗扳机:\card{小型法术珍珠}
    \item 获得护甲扳机:\card{小型法术玉石}
    \item 过载扳机:\card{小型法术蓝宝石}
\end{itemize}

\subsection{即时变形效果}

\card{百变泽鲁斯}、\card{熔岩之刃}、\card{变形卷轴}、\card{变色龙卡米洛斯}和\card{班德斯莫什}在\emph{你的}回合开始阶段触发。它们具有类似的机制。此外,使用时扳机\card{暗影映像}及使用后扳机\card{软泥教授弗洛普}也采用类似的机制:\\
如果一个即时变形卡牌不具有「变形特效」,那它的自身扳机在你的回合开始触发,属于手牌扳机。这个扳机将它变形并添加一个继续变形的状态。继续变形的状态也是一个回合开始扳机,但是它是一个场上扳机(因为状态都在场上)。这会带来一些顺序问题。
\example 你操控一个\card{报警机器人}。你手牌中的未变形的百变泽鲁斯总是晚于机器人触发;而已变形的百变泽鲁斯和报警机器人的顺序则取决于变形发生的时间与报警机器人入场时间的先后。

狼人牌、\card{红色按钮}与阴谋牌在\emph{你的}回合结束触发。其中,狼人牌的扳机不能被\card{达卡莱附魔师}加倍。

\section{扳机自触发保护}
\label{self-triggering-protection}

如果某个扳机A在自身的结算中又产生了能触发它的事件,在该事件过程中A不能被触发。某些扳机在这次触发完全结算完毕后,会补偿进行跳过的结算。目前已知的有如下几个例子:

\example 你操控两个\card{苦痛侍僧}A和B。你对苦痛A造成1点伤害,A 触发并抽到一张\card{烈焰巨兽},对两者都造成伤害。但由于这是在A的扳机结算过程中,A不能再次触发。于是B触发之后A再(补偿)触发,触发顺序是 A-B-A 而非 A-A-B。
\example 你操控一个\card{灵魂歌者安布拉},并使用\card{暮光召唤}。在你召唤第一个随从后,安布拉立即触发其亡语;然后才召唤第二个随从。
\example 你操控一个\card{灵魂歌者安布拉},并使用\card{食肉魔块}消灭你操控的一个亡语随从。安布拉立即触发食肉魔块的亡语;但在召唤两个亡语随从之间并没有插入安布拉的触发。在两个随从均召唤完毕之后,安布拉连续触发这两个随从的亡语。如果你还操控一个\card{送葬者},送葬者的扳机会在两个随从之间触发。
\example 你操控一个灵魂歌者安布拉,使用\card{暗言术:灭}消灭你操控的9/9\card{血骨傀儡}I。注意血骨傀儡的亡语实质上先召唤一个原身材的血骨傀儡,再使其-1/-1。血骨傀儡I首先召唤一个9/9的血骨傀儡II,安布拉立即触发其亡语再召唤一个9/9的血骨傀儡III。由于此时处于安布拉扳机结算过程中,安布拉不能再次触发。血骨傀儡III的身材减为8/8。紧接着,后续召唤血骨傀儡IV并将身材减为7/7,血骨傀儡V则是6/6,紧接着是5/5和4/4。最后,血骨傀儡II的身材减为8/8。最终你场上六个血骨傀儡的身材依次是:8/8,8/8,7/7,6/6,5/5,4/4。
\example 类似的,你操控一个灵魂歌者安布拉并使用9/9的血骨傀儡,最终你场上六个血骨傀儡的身材依次是:9/9,8/8,7/7,6/6,5/5,4/4。
\notice 上面两个例子相似之处在于:你都是试图通过安布拉以外的效果(一个是亡语,一个是从手牌使用)召唤一个9/9的血骨傀儡,只不过第一个例子中你在召唤完战吼还要将它的身材 -1/-1。所以最终场面的唯一区别就是第一个血骨傀儡是8/8还是9/9。
\example 对手操控\card{以眼还眼},你使用\card{枯萎化身塔姆辛},并对自己造成伤害。塔姆辛的扳机触发对对手造成伤害,触发以眼还眼。此时塔姆辛的扳机不能再次触发,最后你受到伤害。如果你使用了两个塔姆辛,则另一个扳机可以触发,最终对手受到伤害。

下面是一个较为复杂的例子,也是扳机自触发保护最初被发现时的情况。参见\href{https://www.youtube.com/watch?v=DowBB0GhGnA}{这里}\\
在这个例子中,盗贼操控4个\card{苦痛侍僧}而牧师操控6个。盗贼牌库中有6张\card{烈焰巨兽}而牧师有12张。现在盗贼的回合开始,抽到一张烈焰巨兽。\\
最终,盗贼抽了6张烈焰巨兽后受到 3-61 的疲劳伤害,总计$16*4+1$次抽牌;而牧师抽到10张烈焰巨兽,爆掉2张,受到 6-89 的疲劳伤害,总计$16*6$次抽牌。苦痛侍僧触发的总次数是正确的。\\
苦痛侍僧的触发顺序如下(按入场顺序编号为 0-9):
\begin{center}
    \texttt{
        0123456789\\
        999999\\
        88888889\\
        7777777789\\
        666666666789\\
        55555555556789\\
        4444444444456789\\
        333333333333456789\\
        22222222222223456789\\
        1111111111111123456789\\
        000000000000000123456789
    }
\end{center}

\section{对战开始时扳机的顺序}

所有的\term{对战开始时}牌均包含两个相同的扳机:一个在手牌触发,一个在牌库触发。对战开始时扳机的顺序按照如下规则:

\begin{enumerate}
    \item 主玩家的所有扳机先触发,然后是副玩家的。这可以理解为是主玩家的所有实体先入场的直接结论。
    \item 手牌扳机先于牌库扳机。
    \item 对牌库中的各扳机,按如下顺序触发:
    \begin{enumerate}
        \item 费用低的先触发。费用相同的则按随机顺序触发。
        \item  如果一张牌因为起手换牌进入牌库,它在最后触发,无视费用顺序。
    \end{enumerate}
\end{enumerate}

\example 你起始牌库中有\card{黑暗主教本尼迪塔斯}和\card{克苏恩,破碎之劫}。黑暗主教先触发。
\example 你起手有黑暗主教,将它换掉,牌库中有克苏恩。克苏恩先触发。但由于黑暗主教只作预检测,它仍然可以触发。
\example 你起始牌库中有黑暗主教和\card{玛克扎尔王子}。它们按随机顺序触发。

但手牌中多个对战开始时扳机,以及同时换入牌库的多个对战开始时扳机的触发顺序仍然是不清楚的。

\section{叠加状态}

\version{}{17.4.1} \term{可叠加的}状态具有如下特点:如果一个实体已经具有了该状态,那么它将要再次获得同类状态的时候,并不会获得一个新的状态,而是改为修改已有的那个状态。这看起来像是「状态并没有立即生效,而是一直推迟到光环更新时才生效」。这句话可能较难理解,请比对下面的例子帮助理解。
此外,当你将鼠标放在该随从上时,大图下方的状态列表也只会显示一个状态。
\example 典型例子包括\card{暴乱狂战士}、\card{加兹瑞拉}、\card{漂浮观察者}等。

\example 你操控\card{索瑞森大帝}。在回合结束时,那些上回合已经减过费的牌看起来明显「晚减费」。

\example 你操控\card{血沼迅猛龙}和3/4的\card{暴乱狂战士}(尚未触发过效果),然后使用\card{圣光炸弹}。迅猛龙对自己造成伤害立即触发暴乱狂,然后暴乱狂对自己造成4点伤害。
\example 你操控迅猛龙和3/4的暴乱狂战士(已经触发过效果),然后使用圣光炸弹。迅猛龙对自己造成伤害立即触发暴乱狂,但是由于暴乱狂已经拥有一个同类状态,它仍然保持3攻。因此它对自己造成3点伤害,在炸弹结算完毕后(实际上是结算阶段结束后),暴乱狂变为4攻。

\example 你操控6/2的\card{漂浮观察者}(尚未触发过效果),并攻击操控\card{爆炸陷阱}的对手。爆炸触发对漂浮和你造成2点伤害,然后漂浮触发变成8/2。战斗继续,漂浮对敌方英雄造成8点伤害。
\example 你操控6/2的漂浮观察者(已经触发过效果),并攻击操控爆炸陷阱的对手。爆炸触发对漂浮和你造成2点伤害,然后漂浮触发,但是由于漂浮已经有一个同类状态,它保持6/0。战斗被跳过。在战斗阶段结束后,漂浮变为8/2。
\example 你操控6/2的漂浮观察者(已经触发过效果)和\card{龙蛋},并攻击操控爆炸陷阱的对手。爆炸触发对漂浮、龙蛋和你造成2点伤害,然后漂浮触发,但是由于漂浮已经有一个同类状态,它保持6/0。随后龙蛋触发召唤一个雏龙,光环更新,漂浮变成8/2。战斗继续,漂浮对敌方英雄造成8点伤害。

\section{强制死亡}

通常情况下,一个阶段之内不会进行死亡检索与死亡结算。但一些牌会在结算中进行死亡检索与死亡结算,这称作\term{强制死亡}。部分牌是因为它们先消灭随从再召唤随从,为了防止召唤的随从被挤掉,必须使用强制死亡;部分牌是为了防止连续多个法术产生反直觉的结果。

到贫瘠之地版本为止,包含强制死亡的效果有:
\begin{itemize}
    \item 先消灭随从再在原地召唤新随从的效果或类似效果,包括\card{剧毒之种}、\card{转生}、\card{米米尔隆的头部}、\card{饥饿的翼手龙}、\card{咒术师的召唤}、\card{至暗时刻}、\card{卑劣的回收者}、\card{教导主任加丁}、\card{仇恨之轮};
    \item 多次伤害,包括\card{亵渎}、\card{高弗雷勋爵}、\card{地震术}、\card{献祭光环}、\card{大地崩陷}、\card{深水炸弹}、\card{燃烧权杖}、\card{永恒之火};
    \item 防止连续结算多个法术或战吼产生反直觉的结果,包括\card{尤格-萨隆}、\card{惊奇卡牌}、\card{莱妮莎·炎伤}、\card{苔丝·格雷迈恩}、\card{沙德沃克}、\card{祖尔金}、\card{尤格-萨隆的谜之匣}、\card{永恒巨龙姆诺兹多}、\card{神秘魔盒}、\card{杰斯·织暗}、\card{大魔导师安东尼达斯}的每次法术/战吼之后,以及\card{尤格-萨隆的仆从}和\card{隐秘破坏者}战吼施放法术后。通过下文所提到的卡德加的bug可以证明,这些强制死亡即使在没有铜须的情况下依然存在。
    \item 酒馆战棋中的每次攻击后。但由随从效果导致的攻击后没有强制死亡(指海盗无赖召唤的\card{空中海盗})。
\end{itemize}

当进行强制死亡时,首先进行所有正常的阶段间步骤,然后如果满足随从死亡的条件,进行一个死亡阶段。循环如此直到没有随从死亡为止。实际上,这与一般的阶段间发生的事情完全相同。
\example 你操控两个\card{鬼灵爬行者}并使用\card{剧毒之种}。所有鬼灵爬行者被消灭,然后强制死亡:移除所有鬼灵爬行者,召唤四个\card{鬼灵蜘蛛}。最后召唤两个树人。
\example 你操控六个\card{可靠的灯泡}和\card{米米尔隆的头部},然后使用\card{丛林之魂}。你的下个回合开始时,米米尔隆的头部触发,消灭所有随从并移除,然后触发它们的亡语,召唤七个\card[EX1_158t]{树人}。最后由于满场,\card{V-07-TR-0N}不会召唤。

\notice 一个有关\card{卡德加}的结算表明,强制死亡似乎被认为是效果的一部分,这与一般的死亡结算不同。参见\href{https://www.bilibili.com/video/av64643539}{这里}。\\
\card{校长克尔苏加德}也有类似的结算。
\example 你操控卡德加,你的对手操控\card{鱼人木乃伊}。你对鱼人木乃伊施放\card{火球术},它召唤一个鱼人。
\example 你操控卡德加,你的对手操控鱼人木乃伊。你对鱼人木乃伊施放\card{转生},它召唤一个鱼人之后,卡德加会触发再召唤一个鱼人。

\subsection{历史}

\version{Naxx}{GvG} \card{克尔苏加德}在回合结束召唤死亡的随从之前会进行一次强制死亡。这就意味着,被\card{炎魔之王拉格纳罗斯}消灭的克总可以复活自己。
\version{4.2}{4.3} 强制死亡中的死亡阶段被移除。这就意味着,\card{自爆绵羊}和\card{剧毒之种}的互动被改为:消灭所有随从,移除所有随从,召唤等量的2/2,自爆绵羊的亡语结算。
\version{}{KNC} 强制死亡中,在死亡阶段结束之后没有光环更新步骤。这导致\card{食腐土狼}和\card{亵渎}之间产生了难以理解的结算。
\version{}{TWW} \card{惊奇卡牌}之后没有强制死亡。
\version{}{11.2} 在早期版本中,强制死亡只包含一次死亡检索和死亡阶段;在这个死亡阶段中产生的新的濒死不会处理。在当前的版本中,强制死亡会一直进行到没有任何濒死实体为止。这和一般的死亡结算完全一致。

\section{回合时间}

玩家的回合长度为75秒。
\begin{itemize}
    \item 每个玩家的第一个回合长度为45秒,第二个回合长度为55秒。
    \item 冒险模式中,玩家无回合时间限制。
    \item 当\card{诺兹多姆}在场时,玩家的回合长度为20秒(不是15秒)。如果是冒险模式,玩家的回合长度为75秒。
\end{itemize}